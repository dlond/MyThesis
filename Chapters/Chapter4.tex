%!TEX root = ../Thesis.tex
% Chapter 4

\chapter{K\"ulshammer's Second Problem}
\label{Chapter4}
\lhead{Chapter 4. \emph{K\"ulshammer's Second Problem}}

Let $G$ be a connected reductive algebraic group over an algebraically closed field $k$ with $\mathrm{char}(k) = p$. Let $P$ be a parabolic subgroup of $G$, with Levi subgroup $L$ and unipotent radical $V$. We have $P = V \rtimes L$, and we denote by $\pi^L$ the canonical projection
\begin{eqnarray*} \pi^L:P \rightarrow L. \end{eqnarray*}

Since $L$ normalizes $V$ we have an action by group automorphisms of $L$ on $V$ given by
\begin{eqnarray} \label{laction} l \cdot v = lvl^{-1}, \end{eqnarray}
for $l \in L, v \in V$.

Let $K$ be a linear algebraic group, $H$ a connected reductive algebraic group, and $\Gamma$ a finite group.
As pointed out in the Introduction, K\"ulshammer's second question has positive answer for $\Gamma, G$ so long as $p$ is good for $G$. We wish to determine whether there exists a reductive counterexample to K\"ulshammer's second question.


Generally speaking, we will be concerned with homomorphisms from $K \rightarrow G$. Where possible, we will work in this general setting and state things in terms of $K$. When we want to work with finite $K$ or connected reductive $K$, we will replace $K$ with $\Gamma$ or $H$, respectively.

\begin{definition} Let $X,Y$ be groups. Define
\begin{eqnarray*} \mathrm{Hom}(X, Y) = \{ \rho : X \rightarrow Y \,|\, \rho \textrm{ is a homomorphism}\}. 
\end{eqnarray*}
\end{definition}

For example, K\"ulshammer's second question as originally stated concerns $G$-conjugacy classes of $\mathrm{Hom}(\Gamma, G)$ and $\mathrm{Hom}(\Gamma_p, G)$, where $\Gamma_p$ is a Sylow $p$-subgroup of $\Gamma$.


A consequence of the following Theorem \cite[Theorem 1.2]{martin2003reductive} is that if there is a reductive counterexample to K\"ulshammer's second question then the infinitely many $G$-conjugacy classes of $\mathrm{Hom}(\Gamma, G)$ must be of non-$G$-completely reducible homomorphisms.

\begin{theorem} \label{thm:finiteGCR} There are only finitely many $G$-conjugacy classes of $G$-completely reducible homomorphisms from $\Gamma\rightarrow G$.
\end{theorem}


The following general result shows that finiteness of $G$-conjugacy classes of $\mathrm{Hom}(K, P)$ carries over to $P$-conjugacy classes. 
\begin{lemma} Let $R \subset \mathrm{Hom}(K, P)$. Then $R$ is contained in a finite union of $G$-conjugacy classes if and only if it is contained in a finite union of $P$-conjugacy classes.
  \label{lem:GPconj}
\end{lemma}
\begin{proof}
	Let $\rho, \varsigma \in R$ such that $\rho$ and $\varsigma$ lie in the same $G$-conjugacy class of $R$. Then there exists $g\in G$ such that
	\begin{eqnarray} \label{gclass}
		g \rho(x) g^{-1} = \varsigma(x),
	\end{eqnarray}
for all $x \in K$.
	
	Let $Q = gPg^{-1}$. By Equation \ref{gclass}, $\varsigma(K) \subset Q$. Hence $\varsigma(K) \subset P \cap Q$.
	
	Let $T$ be a maximal torus of $G$ contained in $P\cap Q$. Since $T$ and $gTg^{-1}$ are maximal tori of $Q$ they must be $Q$-conjugate, so there exists $q\in Q$ such that
	\begin{eqnarray} \label{GPconj:conjT}
		qTq^{-1} = gTg^{-1}.
	\end{eqnarray}
	Then there exists $p\in P$ such that $q = gpg^{-1}$, and Equation \ref{GPconj:conjT} becomes
	\begin{eqnarray*}
		gpg^{-1}Tgp^{-1}g^{-1} &=& gTg^{-1} \\
		\Rightarrow pg^{-1}Tgp^{-1} &=& T.
	\end{eqnarray*}
	Therefore $gp^{-1} \in N_G(T)$. 

	Fix a finite set $N \subset N_G(T)$ of coset representatives for the Weyl group $W = N_G(T)/T$ and let $n \in N, t \in T$ such that
	\begin{eqnarray} \label{GPconj:g1}
		gp^{-1} = nt.
	\end{eqnarray}
	Let \begin{eqnarray} \label{GPconj:g2} p' = p^{-1}t^{-1} \in P. \end{eqnarray} Then by Equations \ref{gclass}, \ref{GPconj:g1} and \ref{GPconj:g2},
	\begin{eqnarray*}
		\rho(x) &=& g^{-1} \varsigma(x) g\\
		&=& (p'n^{-1}) \varsigma(x) np'^{-1},
	\end{eqnarray*}
	for all $x \in K$.
	This shows 
	%\begin{eqnarray*}
	%	R \cap \left(G\cdot \varsigma \right) \subset \bigcup_{n \in N} P \cdot(n^{-1} \cdot \varsigma).
	%\end{eqnarray*}
	%
	%That is,
	that a $G$-conjugacy class of $R$ is contained in a union of at most $|N| = |W|$ $P$-conjugacy classes.

	Therefore, if $R$ is contained in a finite union of $G$-conjugacy classes then it is contained in a finite union of $P$-conjugacy classes. The converse is trivial.
\end{proof}


\section{The 1-Cohomology}
Let $\sigma \in \mathrm{Hom}(K, L)$. Then via the action of $L$ on $V$ given in Equation \ref{laction}, we have an action of $K$ on $V$ given by
\begin{eqnarray} \label{haction} x \cdot v = \sigma(x) \cdot v, \end{eqnarray}
for $x \in K, v \in V$.


\begin{definition} Let $\rho \in \mathrm{Hom}(K, P)$. We associate with $\rho$ the map $\rho^L \in \mathrm{Hom}(K, L)$ defined by
\begin{eqnarray*} \rho^L = \pi^L \circ \rho, \end{eqnarray*}
and the map $\alpha_\rho: K \rightarrow V$ defined by
\begin{eqnarray} \alpha_\rho(x) = \rho(x)\rho^L(x^{-1}). \end{eqnarray}
\end{definition}

Hence given a $\rho \in \mathrm{Hom}(K, P)$ we have an action of $K$ on $V$ via $\rho^L$ using Equation \ref{haction}, and a map $\alpha_\rho:K \rightarrow V$. We show that with this action the map $\alpha_\rho$ is a 1-cocycle from $K \rightarrow V$. To this end, let $x_1, x_2 \in K$.

Then,
\begin{eqnarray*} \alpha_\rho(x_1x_2) &=& \alpha_\rho(x_1x_2) \rho^L(x_1x_2) \left(\rho^L(x_1x_2)\right)^{-1} \\
	&=& \rho(x_1x_2) \left(\rho^L(x_1x_2)\right)^{-1} \\
	&=& \rho(x_1) \rho(x_2) \left(\rho^L(x_1x_2)\right)^{-1} \\
	&=& \alpha_\rho(x_1) \rho^L(x_1) \alpha_\rho(x_2) \rho^L(x_2) \left(\rho^L(x_1x_2)\right)^{-1} \\
	&=& \alpha_\rho(x_1) \rho^L(x_1) \alpha_\rho(x_2) \rho^L(x_1^{-1}) \rho^L(x_1) \rho^L(x_2) \left(\rho^L(x_1x_2)\right)^{-1} \\
	&=& \alpha_\rho(x_1) \left(x_1 \cdot \alpha_\rho(x_2)\right). 
\end{eqnarray*}

Therefore $\alpha_\rho$ satisfies the 1-cocycle condition in Equation \ref{eqn:na_z}.

\begin{definition} Let $\sigma \in \mathrm{Hom}(K, L)$. We denote by
\begin{eqnarray*} Z^1(K, V)_\sigma \end{eqnarray*}
the set of 1-cocycles from $K \rightarrow V$ where $K$ acts on $V$ via $\sigma$.
Likewise, denote by
\begin{eqnarray*} H^1(K, V)_\sigma \end{eqnarray*}
the 1-cohomology obtained from $Z^1(K, V)_\sigma$ under the equivalence relation in Equation \ref{eqn:h_equiv}. Denote by $\psi$ the canonical projection
\begin{eqnarray*} \psi : Z^1(K, V)_\sigma \rightarrow H^1(K, V)_\sigma. \end{eqnarray*}
\end{definition} 

\begin{definition} Let $\sigma \in \mathrm{Hom}(K, L)$ and define
\begin{eqnarray*} \mathrm{Hom}(K, P)_\sigma = \{ \rho \in \mathrm{Hom}(K, P) \,|\, \rho^L = \sigma\}. \end{eqnarray*}
More generally, if $R \subset \mathrm{Hom}(K, P)$ define
\begin{eqnarray*} R_\sigma = \{ \rho \in R \,|\, \rho^L = \sigma \}. \end{eqnarray*}
\end{definition}

By construction, each $\rho \in \mathrm{Hom}(K, P)_\sigma$ yields a 1-cocycle $\alpha_\rho \in Z^1(K, V)_\sigma$.
Conversely given a 1-cocycle $\alpha \in Z^1(K, V)_\sigma$ we can construct a map $\rho: K \rightarrow P$ defined by
\begin{eqnarray*}
\rho(x) = \alpha(x)\sigma(x),
\end{eqnarray*}
for all $x \in K$. This construction is a homomorphism from $K \rightarrow P$. For take $x_1, x_2 \in K$, then
\begin{eqnarray*}
  \rho(x_1 x_2) &=& \alpha(x_1 x_2) \sigma(x_1 x_2) \\
  &=& \alpha(x_1)(x_1 \cdot \alpha(x_2)) \sigma(x_1) \sigma(x_2) \\
  &=& \alpha(x_1) \sigma(x_1) \alpha(x_2) \sigma(x_1)^{-1} \sigma(x_1) \sigma(x_2) \\
  &=& \alpha(x_1) \sigma(x_1) \alpha(x_2) \sigma(x_2) \\
  &=& \rho(x_1) \rho(x_2).
\end{eqnarray*}

Hence the following Lemma.

\begin{lemma}
  Let $\sigma \in \mathrm{Hom}(K, L)$. The map
\begin{eqnarray*} z: Hom(K, P)_{\sigma} \rightarrow Z^1(K, V)_\sigma, \end{eqnarray*}
defined by
\begin{eqnarray*} z(\rho)(x) = \rho(x)\sigma(x^{-1}), \end{eqnarray*}
for all $\rho \in \mathrm{Hom}(K, P)_\sigma$ and all $x \in K$, is a bijection.
\label{lem:hom_z1}
\end{lemma}
\begin{proof}
We have previously shown that $z$ is onto and has inverse defined by
\begin{eqnarray*} z^{-1}(\alpha)(x) = \alpha(x)\sigma(x), \end{eqnarray*}
for all $\alpha \in Z^1(K, V)_\sigma$ and all $x \in K$, so there is nothing to prove.
\end{proof}

%Henceforth we do no harm to use the otherwise suggestive notation ``$\alpha_\rho \in Z^1(K, V)_\sigma$'', meaning an element of $Z^1(K, V)_\sigma$ with corresponding element $\rho \in \mathrm{Hom}(K, P)_\sigma$ given by $\rho = z^{-1}(\alpha_\rho)$.

Since $\mathrm{Hom}(K, P)_\sigma$ is stable under conjugation by elements of $V$ we can consider $V$-conjugacy classes of $\mathrm{Hom}(K, P)_\sigma$.
\begin{definition} Denote by $\mathrm{Hom}(K, P)_\sigma / V$ the collection of $V$-conjugacy classes of $\mathrm{Hom}(K, P)_\sigma$, and denote by $\phi$ the canonical projection,
\begin{eqnarray*} \phi : \mathrm{Hom}(K, P)_\sigma \rightarrow \mathrm{Hom}(K, P)_\sigma / V. \end{eqnarray*}
\end{definition}

In fact, we show that $z$ descends to give a bijection from $\mathrm{Hom}(K, P)_\sigma / V \rightarrow H^1(K, V)_\sigma$.

\begin{lemma}
For $\rho \in \mathrm{Hom}(K, P)_\sigma$, define
\begin{eqnarray*}
h(V \cdot \rho) = \psi(z)(\rho).
\end{eqnarray*}
Then $h$ is a well-defined bijection from $\mathrm{Hom}(K, P)_\sigma / V \rightarrow H^1(K, P)_\sigma$. Moreover, the following diagram commutes:
  \begin{eqnarray*}
    \xymatrix{
    Hom(K, P)_{\sigma} \ar[r]^z \ar[d]_\phi & Z^{1}(K, V)_\sigma \ar[d]^\psi \\
    Hom(K, P)_{\sigma}/V \ar[r]^h & H^{1}(K, V)_\sigma.
    }
  \end{eqnarray*}
  \label{lem:v_h1}
\end{lemma}
\begin{proof}  
Let $\rho, \varsigma \in \mathrm{Hom}(K, P)_\sigma$ such that $\phi(\rho) = \phi(\varsigma)$. Then there exists $v \in V$ such that
\begin{eqnarray*} \varsigma(x) =  v\rho(x)v^{-1}, \end{eqnarray*}
for all $x \in K$. Furthermore, for all $x \in K$
\begin{eqnarray*}
z(\varsigma)(x) % &=& \alpha_\varsigma(x) \\
&=& \varsigma(x)\sigma(x^{-1}) \\
&=& v \rho(x) v^{-1} \sigma(x^{-1}) \\
&=& v \rho(x) \sigma(x^{-1})\sigma(x) v^{-1} \sigma(x^{-1}) \\
&=& v \rho(x) \sigma(x^{-1}) \left(x \cdot v^{-1}\right) \\
%&=& v \alpha_\rho(x) \left(x \cdot v^{-1}\right) \\
&=& v \left(z(\rho)(x)\right) \left(x \cdot v^{-1}\right).
\end{eqnarray*}
This shows that $z(\rho)$ and $z(\varsigma)$ satisfy the equivalence relation in Equation \ref{eqn:h_equiv}. Therefore $\psi\left(z(\rho)\right) = \psi\left(z(\varsigma)\right)$ and so $h$ is well-defined.

Since $z$ and $\psi$ are onto, $h$ is onto. We show $h$ is one-to-one.

Let $\alpha, \beta \in Z^1(K, V)_\sigma$ such that $\psi(\alpha) = \psi(\beta)$. Then there exists $v \in V$ such that
\begin{eqnarray*} \beta(x) = v \alpha(x) \left(x \cdot v^{-1}\right), \end{eqnarray*}
for all $x \in K$. Then, for all $x \in K$
\begin{eqnarray*}
z^{-1}(\beta)(x) 
&=& \beta(x)\sigma(x) \\
&=& v \alpha(x) \left(x \cdot v^{-1}\right) \sigma(x) \\
&=& v \alpha(x) \sigma(x) v^{-1} \sigma(x^{-1}) \sigma(x) \\
&=& v \alpha(x) \sigma(x) v^{-1} \\
&=& v z^{-1}(\alpha)(x) v^{-1}.
\end{eqnarray*}
This shows that $\phi\left(z^{-1}(\alpha)\right) = \phi\left(z^{-1}(\beta)\right)$ and therefore $h$ is one-to-one. This completes the proof.
\end{proof}

Let $p \in P$, and let $v \in V, l \in L$ such that $p = vl$. We can conjugate $\rho \in \mathrm{Hom}(K, P)_\sigma$ by $p$ to yield an element of $\mathrm{Hom}(K, P)_{l \cdot \sigma}$. For
\begin{eqnarray*}% \left(p \cdot \rho(x)\right)^L
\pi^L\left(p \rho(x) p^{-1}\right)% \\
&=& \pi^L\left(p\right) \pi^L\left(\rho(x)\right) \pi^L\left(p^{-1}\right) \\
&=& l \sigma(x) l^{-1},
\end{eqnarray*}
for all $x \in K$. Likewise, conjugating $\alpha \in Z^1(K, V)_\sigma$ by $p$ yields an element of $Z^1(K, V)_{l \cdot \sigma}$. In fact, we have a map $V \rightarrow V$ defined by
\begin{eqnarray*} v \mapsto lvl^{-1}, \end{eqnarray*}
for all $v \in V$ which yields a map of 1-cohomomologies $H^1(K, V)_\sigma \rightarrow H^1(K, V)_{l \cdot \sigma}$ as in Definition TODO.
$\mathcal{H}(f, g):H^1(K, V) \rightarrow H^1(K', V')$.
Furthermore, if $l \in C_L\left(\sigma(K)\right)$ then $H^1(K, V)_\sigma$ and $H^1(K, V)_{l \cdot \sigma}$ are equivalent as 1-cohomologies
\begin{eqnarray*}
  (z \cdot \sigma) (\gamma) &=& z \sigma(\gamma) z^{-1} \\
  (z \cdot \alpha_\sigma) (\gamma) &=&  z \alpha_\sigma(\gamma) z^{-1},
\end{eqnarray*}
and $h$ is $Z(L)^\circ$-equivariant:
\begin{eqnarray*}
  h(z \cdot \sigma)(\gamma) &=&  z \sigma(\gamma) z^{-1} \rho_L(\gamma)^{-1} \\
  &=& z \sigma(\gamma) \rho_L(\gamma)^{-1} z^{-1} \\
  &=& (z \cdot h(\sigma))(\gamma),
\end{eqnarray*}
that is
\begin{eqnarray}
  \alpha_{z \cdot \sigma} &=& z \cdot \alpha_\sigma,
  \label{eqn:h_z_equivar}
\end{eqnarray}
for all $\alpha_\sigma \in Z^1(\Gamma, \rho_L, V)$ and all $z \in Z(L)^\circ$.

We show that the $Z(L)^\circ$-action on $Hom(\Gamma, P)_{\rho_L}$ and $Z^1(\Gamma, \rho_L, V)$ descends to give a $Z(L)^\circ$-action on $Hom(\Gamma, P)_{\rho_L}/V$ and $H^1(\Gamma, \rho_L, V)$, respectively. The actions will be well-defined as a consequence of the fact that $L$ normalizes $V$.

Let $z \in Z(L)^\circ$ and $\overline{\sigma} \in Hom(\Gamma, P)_{\rho_L}/V$. We define the $Z(L)^\circ$-action on $Hom(\Gamma, P)_{\rho_L}/V$ so that the projection $Hom(\Gamma, P)_{\rho_L} \rightarrow Hom(\Gamma, P)_{\rho_L}/V$ is $Z(L)^\circ$-equivariant:
\begin{eqnarray}
  z \cdot \overline{\sigma} = \overline{z \cdot \sigma}.
  \label{eqn:z_act_rv}
\end{eqnarray}

Suppose $\overline{\sigma} = \overline{\tau}$. Then there is $v \in V$ such that $\sigma = v \cdot \tau \in Hom(\Gamma, P)_{\rho_L}$, and $v' \in V$ such that $zv = v' z$. Therefore
\begin{eqnarray*}
  \overline{z \cdot \sigma} &=& \overline{zv \cdot \tau} \\
  &=& \overline{v' z \cdot \tau} \\
  &=& \overline{z \cdot \tau}.
\end{eqnarray*}

Hence the $Z(L)^\circ$-action on $Hom(\Gamma, P)_{\rho_L}/V$ in Equation \ref{eqn:z_act_rv} is well-defined.

Similarly, the $Z(L)^\circ$-action on $H^1(\Gamma, \rho_L, V)$ is defined so that the projection to the 1-cohomology $Z^1(\Gamma, \rho_L, V) \rightarrow H^1(\Gamma, \rho_L, V)$ is $Z(L)^\circ$-equivariant:
\begin{eqnarray}
  z \cdot [\alpha_\sigma] = [z \cdot \alpha_\sigma].
  \label{eqn:z_act_h}
\end{eqnarray}

Suppose $[\alpha_\sigma] = [\alpha_\tau]$. Then there is $v \in V$ such that for all $\gamma \in \Gamma$,
\begin{eqnarray*}
  \alpha_\sigma(\gamma) = v \alpha_\tau(\gamma) (\gamma \cdot v^{-1}),
\end{eqnarray*}
and there is $v' \in V$ such that $zv = v' z$. Therefore, for all $\gamma \in \Gamma$
\begin{eqnarray*}
	(z \cdot \alpha_\sigma)(\gamma) 
  &=& z v \alpha_\tau(\gamma) (\gamma \cdot v^{-1}) z^{-1} \\
  &=& z v \alpha_\tau(\gamma) \rho_L(\gamma) v^{-1} \rho_L(\gamma)^{-1} z^{-1} \\
  &=& v' z \alpha_\tau(\gamma) z^{-1} \rho_L(\gamma) v'^{-1} \rho_L(\gamma)^{-1} \\
  &=& v' z \alpha_\tau(\gamma) z^{-1} (\gamma \cdot v'^{-1}) \\
  &=& v' ((z \cdot \alpha_\tau)(\gamma)) (\gamma \cdot v'^{-1}).
\end{eqnarray*}

Therefore $[z \cdot \alpha_\sigma] = [z \cdot \alpha_\tau]$ and the $Z(L)^\circ$-action on $H^1(\Gamma, \rho_L, V)$ in Equation \ref{eqn:z_act_h} is well-defined.

Since $h$ is $Z(L)^\circ$-equivariant, it follows that $\overline{h}$ is also:
\begin{eqnarray*}
  \overline{h}(z \cdot \overline{\sigma}) &=& \overline{h}(\overline{z \cdot \sigma}) \\
  &=& [\alpha_{z \cdot \sigma}] \\
  &=& [z \cdot \alpha_\sigma] \qquad (\textrm{Equation }\ref{eqn:h_z_equivar}) \\
  &=& z \cdot [\alpha_\sigma] \\
  &=& z \cdot \overline{h}(\overline{\sigma}).
\end{eqnarray*}

In summary we have a well-defined $Z(L)^\circ$-action on $Hom(\Gamma, P)_{\rho_L}/V$ and $H^1(\Gamma, \rho_L, V)$, and a $Z(L)^\circ$-equivariant bijection 
\begin{eqnarray*}
  \overline{h}:Hom(\Gamma, P)_{\rho_L}/V \rightarrow H^1(\Gamma, \rho_L, V).
\end{eqnarray*}

Hence the following Lemma:

\begin{lemma}
  The bijection $h: Hom(\Gamma, P)_{\rho_L} \rightarrow Z^1(\Gamma, \rho_L, V)$ gives rise to a bijection
  \begin{eqnarray*}
    \tilde{h}: Hom(\Gamma, P)_{\rho_L}/VZ(L)^\circ \rightarrow H^1(\Gamma, \rho_L, V)/Z(L)^\circ.
  \end{eqnarray*}
  \label{lem:vzl_h1zl}
\end{lemma}
\begin{proof}
  It remains to show that $\left(Hom(\Gamma, P)_{\rho_L}/V\right)/Z(L)^\circ \simeq Hom(\Gamma, P)_{\rho_L}/VZ(L)^\circ$.

  Denote by $\pi_Z$ the canonical projection from $Hom(\Gamma, P)_{\rho_L}/V \rightarrow \left( Hom(\Gamma, P)_{\rho_L}/V \right)/Z(L)^\circ$ and $\tilde{\sigma}$ the element of $Hom(\Gamma, P)_{\rho_L}/VZ(L)^\circ$ with representative $\sigma \in Hom(\Gamma, P)_{\rho_L}$. We show that the map
  \begin{eqnarray*}
    \varphi: (Hom(\Gamma, P)_{\rho_L}/V)/Z(L)^\circ \rightarrow Hom(\Gamma, P)_{\rho_L}/VZ(L)^\circ,
  \end{eqnarray*}
  defined by
  \begin{eqnarray*}
    \varphi(\pi_Z(\overline{\sigma})) = \tilde{\sigma},
  \end{eqnarray*}
  is well defined and a bijection. To this end consider the following statements:
  \begin{eqnarray*}
    && \pi_Z(\overline{\sigma}) = \pi_Z(\overline{\tau}) \\
    &\Longleftrightarrow& \textrm{there exists } z \in Z(L)^\circ \textrm{ such that } \overline{\sigma} = z \cdot \overline{\tau} \\
    &\Longleftrightarrow& \textrm{there exists } z \in Z(L)^\circ \textrm{ such that } \overline{\sigma} = \overline{z \cdot \tau} \qquad (\textrm{Equation } \ref{eqn:z_act_h}) \\
    &\Longleftrightarrow& \textrm{there exists } v \in V, z \in Z(L)^\circ \textrm{ such that } \sigma = v \cdot (z \cdot \tau) \\
    &\Longleftrightarrow& \textrm{there exists } v \in V, z \in Z(L)^\circ \textrm{ such that } \sigma = (vz) \cdot \tau \\
    &\Longleftrightarrow& \tilde{\sigma} = \tilde{\tau}.
  \end{eqnarray*}
  Hence the forward direction implies that $\varphi$ is well-defined and the reverse direction implies that $\varphi$ is injective. Surjectivity follows from the fact that the canonical projections:
  \begin{eqnarray*}
    Hom(\Gamma, P)_{\rho_L} \rightarrow Hom(\Gamma, P)_{\rho_L}/V \rightarrow (Hom(\Gamma, P)_{\rho_L}/V)/Z(L)^\circ, 
  \end{eqnarray*}
  and
  \begin{eqnarray*}
    Hom(\Gamma, P)_{\rho_L} \rightarrow Hom(\Gamma, P)_{\rho_L}/VZ(L)^\circ, 
  \end{eqnarray*}
  are surjective.
\end{proof}

Let $R = \{ \rho_\lambda : \Gamma \rightarrow G \,|\, \lambda \in \Lambda \}$ be a collection of representations indexed by the set $\Lambda$. For $\rho \in R$, we say that a parabolic subgroup $P$ of $G$ is $\rho$-minimal if $P$ is minimal among the parabolic subgroups of $G$ that contain $\rho(\Gamma)$. For a parabolic subgroup $P < G$ define
\begin{eqnarray*}
  R_P = \{ \rho \in R \,|\, P \textrm{ is } \rho\textrm{-minimal} \}.
\end{eqnarray*}

Since there are only finitely many $G$-conjugacy classes of parabolic subgroups of $G$ (a standard result, e.g. \cite[Theorem 30.1(a)]{humphreys1975linear}), we can choose a finite set of representative parabolic subgroups $\{Q_i\}_{i = 0}^n$ of $G$ such that every parabolic subgroup $P < G$ is $G$-conjugate to precisely one $Q_i$. Every $\rho \in R$ has a minimal parabolic subgroup and so there exists an element of each $G$-conjugacy class in $R$ with minimal parabolic $Q_i$ for some $i$, hence
\begin{eqnarray}
  G \cdot R = \bigcup_i G \cdot R_{Q_i}.
  \label{eqn:gr_gqi}
\end{eqnarray}

Furthermore, fix a particular $Q_i$ with Levi subgroup $M_i$. Since $Q_i$ is minimal for each $\rho \in R_{Q_i}$, $\rho_{M_i}$ (Equation \ref{eqn:proj_l}) is $M_i$-irreducible \cite[Lemma 6.2(ii)]{bate2005geometric}.

For an $M_i$-irreducible representation $\sigma : \Gamma \rightarrow M_i$ define
\begin{eqnarray*}
  R_{\sigma} = \{ \rho \in R_{Q_i} \,|\, \rho_{M_i} = \sigma \}.
\end{eqnarray*}

Since there are only finitely many $M_i$-conjugacy classes of $M_i$-irreducible representations $\Gamma \rightarrow M_i$ (Theorem \ref{thm:finiteGCR}), we can choose a finite set of representative $M_i$-irreducible representations $\{\sigma_i^j : \Gamma \rightarrow M_i \}_{j=0}^{n_i}$, such that every $M_i$-irreducible representation $\Gamma \rightarrow M_i$ is $M_i$-conjugate to precisely one $\sigma_i^j$. Every $M_i$-conjugacy class in $R_{Q_i}$ has an element $\rho$ such that $\rho_{M_i} = \sigma_i^j$ for some $j$, hence
\begin{eqnarray*}
  M_i \cdot R_{Q_i} = \bigcup_j M_i \cdot R_{\sigma_i^j},
\end{eqnarray*}
and therefore
\begin{eqnarray}
  G \cdot R = \bigcup_i G \cdot R_{Q_i} = \bigcup_i \bigcup_j G \cdot R_{\sigma_i^j}.
  \label{eqn:gr_grsigma}
\end{eqnarray}

Fix an $M_i$-irreducible representation $\sigma: \Gamma \rightarrow M_i$. We have a map from $Hom(\Gamma, P_i)_\sigma \rightarrow Hom(\Gamma, P_i)_\sigma / V_i Z(M_i)^\circ$ given by the canonical projection and a map $\tilde{h}: Hom(\Gamma, P_i)_\sigma / V_i Z(M_i)^\circ \rightarrow H^1(\Gamma, \sigma, V_i)/Z(M_i)^\circ$. We define the map
\begin{eqnarray*}
  \mathcal{H}: Hom(\Gamma, P_i)_\sigma \rightarrow H^1(\Gamma, \sigma, V_i) / Z(M_i)^\circ
\end{eqnarray*}
to be the composition of the above canonical projection with $\tilde{h}$. That is, $\mathcal{H}(\rho) = \tilde{h}(\tilde{\rho})$ for all $\rho \in Hom(\Gamma, P_i)_\sigma$, where $\tilde{\rho}$ is the projection of $\rho$ to $Hom(\Gamma, P_i)_\sigma / V_i Z(M_i)^\circ$.

We note that each subset $R_{\sigma_i^j} \subset R_{Q_i}$ is a $VZ(M_i)^\circ$-stable subset of $Hom(\Gamma, Q_i)_{\sigma_i^j}$, so it makes sense to calculate
\begin{eqnarray*}
  \mathcal{H}(R_{\sigma_i^j}) \subset H^1(\Gamma, \sigma_i^j, V_i) / Z(M_i)^\circ.
\end{eqnarray*}

\begin{lemma}
  Let $R_P = \{\rho_\lambda:\Gamma\rightarrow P\,|\,\lambda\in\Lambda\}$ be a collection of representations such that $P$ $\rho_\lambda$-minimal for each $\rho_\lambda$.
  
  The following statements are equivalent:
  \begin{itemize}
    \item[(i)] $R_P$ is contained in a finite union of $P$-conjugacy classes.
    \item[(ii)] For each irreducible representation $\sigma:\Gamma\rightarrow L$, $R_{\sigma}$ is contained in a finite union of $VZ(L)^\circ$-conjugacy classes.
    \item[(iii)] For each irreducible representation $\sigma:\Gamma\rightarrow L$,
      \begin{eqnarray*}
	\mathcal{H}(R_{\sigma}) \subset H^{1}(\Gamma,\sigma,V)/Z(L)^\circ
      \end{eqnarray*}
      is finite.
  \end{itemize}
  \label{lem:p_h1}
\end{lemma}
\begin{proof}\quad

  $(i) \Rightarrow (ii)$ Assume $R_P$ is contained in a finite union of $P$-conjugacy classes and fix an irreducible representation $\sigma : \Gamma \rightarrow L$. Then $R_{\sigma}$ is contained a finite union of $P$-conjugacy classes. Take $\rho \in R_{\sigma}$ and suppose that $p \cdot \rho \in R_{\sigma}$ for some $p \in P$. Writing $p = vl$ for some $v \in V$ and some $l \in L$, $(vl) \cdot \rho \in R_{\sigma}$ implies that in fact $l \in C_L(\sigma(\Gamma))$. Furthermore, since $\sigma$ is irreducible it follows that $C_L(\sigma(\Gamma))/Z(L)^\circ$ is finite \cite[Lemma 6.2]{martin2003reductive}, so we can choose a finite set $\{c_1, \ldots, c_m\}$ of coset representatives for $Z(L)^\circ\backslash C_L(\sigma(\Gamma))$. Therefore
  \begin{eqnarray*}
    R_{\sigma} \cap (P \cdot \rho) \subset \bigcup_{i = 1}^{m} VZ(L)^\circ \cdot \left( c_i \cdot \rho \right).
  \end{eqnarray*}
  Since $R_{\sigma}$ is contained in a finite number of $P$-conjugacy classes, we are done.

  $(ii) \Rightarrow (i)$ Assume that for each irreducible representation $\sigma : \Gamma \rightarrow L$, $R_{\sigma}$ is contained in a finite union of $VZ(L)^\circ$-conjugacy classes, so for each $\sigma$ there is a finite set $\Phi^\sigma \subset R_P$ such that
  \begin{eqnarray*}
    R_\sigma \subset \bigcup_{\phi \in \Phi^\sigma} VZ(L)^\circ \cdot \phi.
  \end{eqnarray*}
  We do no harm to assume that $R_P = P \cdot R_P$. Denote by $Hom(\Gamma, L)_{irr}$ the collection of all irreducible representations from $\Gamma \rightarrow L$. By Theorem \ref{thm:finiteGCR} we can choose a finite set $\Sigma \subset Hom(\Gamma, L)_{irr}$ such that
  \begin{eqnarray*}
    Hom(\Gamma, L)_{irr} = \bigcup_{\sigma \in \Sigma} L \cdot \sigma.
  \end{eqnarray*}
  For each $\sigma \in \Sigma$ define 
  \begin{eqnarray*}
    R_{L \cdot \sigma} = \bigcup_{l \in L} R_{l \cdot \sigma}.
  \end{eqnarray*}
  Since $P$ is minimal for each $\rho \in R_P$, $\rho_L$ is $L$-irreducible. Hence
  \begin{eqnarray*}
    R_P = \bigcup_{\sigma \in \Sigma} R_{L \cdot \sigma}.
  \end{eqnarray*}

  Suppose $\rho \in R_{L \cdot \sigma}$. Then there exists $l \in L$ such that $\rho_L = l \cdot \sigma$, so that $l^{-1} \cdot \rho_L = \sigma$. Since we assumed $R_P = P \cdot R_P$, $l^{-1} \cdot \rho \in R_P$ and therefore $l^{-1} \cdot \rho \in R_\sigma$. Conversely if $\rho \in L \cdot R_\sigma$ then $l \cdot \rho \in R_{L \cdot \sigma}$ for some $l \in L$. Hence
  \begin{eqnarray*}
    R_{L \cdot \sigma} = L \cdot R_\sigma.
  \end{eqnarray*}

  Therefore
  \begin{eqnarray*}
    R_P = \bigcup_{\sigma \in \Sigma} L \cdot R_\sigma \subset \bigcup_{\sigma \in \Sigma}\bigcup_{\phi \in \Phi^\sigma} LVZ(L)^\circ \cdot \phi = \bigcup_{\sigma \in \Sigma}\bigcup_{\phi \in \Phi^\sigma} P \cdot \phi.
  \end{eqnarray*}

  $(ii) \Leftrightarrow (iii)$ This follows directly from the fact that $\tilde{h}$ is a bijection (Lemma \ref{lem:vzl_h1zl}).

\end{proof}

We are now ready to state precisely the connection with $G$-conjugacy classes of representations and the 1-cohomology.

\begin{theorem}
  Let $R=\{\rho_\lambda:\Gamma\rightarrow G\,|\,\lambda \in \Lambda\}$ be a collection of representations indexed by the set $\Lambda$. 
  
  Suppose $R = G \cdot R$. Then $R$ is a finite union of $G$-conjugacy classes if and only if for each $i, j$ the subset $\mathcal{H}(R_{\sigma_i^j}) \subset H^1(\Gamma, \sigma_i^j, V_i) / Z(M_i)^\circ$ is finite.
  \label{thm:g_h1}
\end{theorem}
\begin{proof}
  Assume $R$ is a finite union of $G$-conjugacy classes. Then for each $Q_i$, $R_{Q_i}$ is contained in a finite union of $G$-conjugacy classes. By Lemma \ref{lem:GPconj} $R_{Q_i}$ is contained in a finite union of $Q_i$-conjugacy classes, and by Lemma \ref{lem:p_h1} $\tilde{h}(R_{\sigma_i^j}/VZ(M_i)^\circ)$ is finite for each $j$.

  Conversely, suppose that for each $i$, for each $j$, $\tilde{h}(R_{\sigma_i^j}/VZ(M_i)^\circ)$ is finite. Then by Lemma \ref{lem:p_h1}, each $R_{Q_i}$ is contained in a finite union of $Q_i$-conjugacy classes. Since
  \begin{eqnarray*}
    G \cdot R = \bigcup_i G \cdot R_{Q_i} \qquad (\textrm{Equation }\ref{eqn:gr_gqi})
  \end{eqnarray*}
  we are done.
\end{proof}

\begin{theorem}
  Let $G$ be an algebraic group over an algebraically closed field $k$ of characteristic $p$, $\Gamma$ a finite group and $\Gamma_p < \Gamma$ a Sylow $p$-subgroup. Define $M_i < Q_i < G$ and $\sigma_i^{j}:\Gamma \rightarrow_{irr} M_{i}$ as above. Let $\iota$ be the inclusion map $\iota : \Gamma_p \rightarrow \Gamma$ and for each $i,j$ let $Z^1(\iota), H^1(\iota)$ be the corresponding 1-cocycle and 1-cohomology restriction maps
  \begin{eqnarray*}
    Z^1(\iota) : Z^1(\Gamma, \sigma_i^j, V_i) \rightarrow Z^1(\Gamma_p, \sigma_i^j, V_i),
  \end{eqnarray*}
  and
  \begin{eqnarray*}
    H^1(\iota) : H^1(\Gamma, \sigma_i^j, V_i) \rightarrow H^1(\Gamma_p, \sigma_i^j, V_i).
  \end{eqnarray*}
  
  The answer to K\"ulshammer's second question for $\Gamma, G$ is positive only if $H^1(\iota)$ is injective for each $i,j$.
  \label{thm:k2_h1}
\end{theorem}
\begin{proof}
  Fix a representation $\rho_0: \Gamma_p \rightarrow G$ and let $X = G \cdot \rho_0$. For a map $\varphi$ from $\Gamma$ denote by $\varphi^p$ its restriction to $\Gamma_p$. Let
  \begin{eqnarray*}
    R &=&  \{ \rho \in \textrm{Hom}(\Gamma, G) \,|\, \rho^p \in X\}, \\
    R_{Q_i} &=&  \{ \rho \in R \,|\, Q_i \textrm{ is } \rho \textrm{-minimal} \}, \\
    R_{\sigma_i^j} &=&  \{ \rho \in R_{Q_i} \,|\, \rho_{M_i} = \sigma_i^j \},
  \end{eqnarray*}
  as done previously.
  
  Fix $g \in G$ and consider the set $R_j^p$ defined
  \begin{eqnarray*}
    R_j^p = \{ \rho \in R_{\sigma_i^j} \,|\, \rho^p = g \cdot \rho_0 \}.
  \end{eqnarray*}

  Then
  \begin{eqnarray}
    R_{\sigma_i^j} \subset G \cdot R_j^p.
    \label{eqn:rs_grp}
  \end{eqnarray}

  Assume $H^1(\iota)$ is injective. Since $Z^1(\iota)(h(\rho))$ is equal to a fixed 1-cocycle for all $\rho \in R_j^p$ the image of $R_j^p$ in $H^1(\Gamma_p, \sigma_i^j, V_i)$ is finite, hence the image of $R_j^p$ in $H^1(\Gamma, \sigma_i^j, V_i)$ is finite. 
  
  Therefore, the image of $R_j^p$ in $H^1(\Gamma, \sigma_i^j, V_i)/Z(M_i)^\circ$ is finite. By Lemma \ref{lem:p_h1}, $R_j^p$ is contained in a finite union of $Q_i$-conjugacy classes, hence by Equation \ref{eqn:rs_grp}, $R_{\sigma_i^j}$ is contained in a finite union of $G$-conjugacy classes.

  So by Equation \ref{eqn:gr_grsigma}, $G \cdot R = R$ is contained in a finite union of $G$-conjugacy classes. Therefore the answer to K\"ulshammer's second question for $\Gamma, G$ is positive.
\end{proof}

\section{A Non-Reductive Counterexample}
In \cite{slodowy1997two} a counterexample to K\"ulshammer's second question is presented for a closed field $k$ of characteristic $p = 2$ and a non-reductive algebraic group $G$.
\begin{example} Let $Q$ be the algebraic group isomorphic to the affine space $\mathbf{A}^3$ with the group multiplication law:
\begin{eqnarray*}
	\left(\begin{matrix} u_1 \\ u_2 \\ u_3 \end{matrix}\right) \times
	\left(\begin{matrix} v_1 \\ v_2 \\ v_3 \end{matrix}\right) &=&
	\left(\begin{matrix} u_1 + v_1 \\ u_2 + v_2 \\ u_3 + v_3 + u_1v_1 + u_2v_2 + u_1v_2 \end{matrix}\right).
\end{eqnarray*}
Let $\Gamma = \langle \sigma, \tau | \sigma^3 = \tau^2 = 1, \tau\sigma\tau = \sigma^2 \rangle$ and $\Gamma_2 = \langle \tau \rangle$, a Sylow 2-subgroup of $\Gamma$. $\Gamma$ acts on $Q$ via
\begin{eqnarray*}
	\tau \cdot \left(\begin{matrix} u_1 \\ u_2 \\ u_3 \end{matrix} \right) &=&
	\left(\begin{matrix} u_2 \\ u_1 \\ u_3 + u_1^2 + u_2^2 + u_1u_2 \end{matrix} \right) \\
	\sigma \cdot \left(\begin{matrix} u_1 \\ u_2 \\ u_3 \end{matrix} \right) &=&
	\left(\begin{matrix} u_2 \\ u_1 + u_2 \\ u_3 \end{matrix} \right).
\end{eqnarray*}
Let $G = Q \rtimes \Gamma$ and fix the representation $\rho:\Gamma_2 \rightarrow G$ defined by the natural inclusion $\Gamma_2 \rightarrow \Gamma \rightarrow G$. Then there are infinitely many pairwise $G$-conjugate classes of extensions of $\rho$ to representations of $\Gamma$ into $G$ \cite[Appendix]{slodowy1997two}.
\label{eg:non_red}
\end{example}
\begin{proof}
	Our proof will be way of a 1-cohomology calculation. Choose a 1-cocycle $\alpha \in Z^1(\Gamma, Q)$ such that $\alpha|_{\langle \sigma \rangle} = 1$. Let
	\begin{eqnarray*}
		\alpha(\tau) = \left(\begin{matrix} u_1 \\ u_2 \\ u_3 \end{matrix} \right),
	\end{eqnarray*}
	for some $u_1, u_2, u_3 \in k$. Since $\tau$ is an involution we have
	\begin{eqnarray*}
		1 = \alpha(\tau^2) &=& \alpha(\tau) \times \tau\cdot\alpha(\tau)\\
		&=& \left(\begin{matrix} u_1 \\ u_2 \\ u_3\end{matrix} \right) \times 
		\left(\begin{matrix} u_2 \\ u_1 \\ u_3 + u_1^2 + u_2^2 + u_1u_2\end{matrix} \right) \\
		&=& \left(\begin{matrix} u_1 + u_2 \\ u_1 + u_2 \\ 2u_3 + 2u_1^2 + u_2^2 + 3u_1u_2\end{matrix} \right)\\
		&=& \left(\begin{matrix} u_1 + u_2 \\ u_1 + u_2 \\ u_2^2 + u_1u_2\end{matrix} \right).
	\end{eqnarray*}
	This shows $u_1 = u_2$, so
	\begin{eqnarray*}
		\alpha(\tau) = \left(\begin{matrix} u_1 \\ u_1 \\ u_3\end{matrix} \right).
	\end{eqnarray*}
	Furthermore, as $\tau\sigma\tau = \sigma^2$ we obtain
	\begin{eqnarray*}
		1 = \alpha(\sigma^2) &=& \alpha(\tau\sigma\tau) \\
		&=& \alpha(\tau) \times \tau\cdot\alpha(\sigma\tau)\\
		&=& \alpha(\tau) \times \tau\cdot\alpha(\sigma) \times \tau\sigma\cdot\alpha(\tau) \\
		&=& \alpha(\tau) \times \tau\sigma\cdot\alpha(\tau) \\
		&=& \left(\begin{matrix} u_1 \\ u_1 \\ u_3\end{matrix} \right) \times
		\tau\sigma\cdot\left(\begin{matrix} u_1 \\ u_1 \\ u_3\end{matrix} \right)\\
		&=& \left(\begin{matrix} u_1 \\ u_1 \\ u_3\end{matrix} \right) \times
		\tau\cdot\left(\begin{matrix} u_1 \\ 0 \\ u_3\end{matrix} \right)\\
		&=& \left(\begin{matrix} u_1 \\ u_1 \\ u_3\end{matrix} \right) \times
		\left(\begin{matrix} 0 \\ u_1 \\ u_3 + u_1^2\end{matrix} \right)\\
		&=& \left(\begin{matrix} u_1 \\ 0 \\ 2u_3 + 3u_1^2\end{matrix} \right)\\
		&=& \left(\begin{matrix} u_1 \\ 0 \\ u_1^2\end{matrix} \right).
	\end{eqnarray*}
	Therefore $u_1 = 0$. Hence a typical 1-cocycle that is trivial on $\langle \sigma \rangle$ satisfies
	\begin{eqnarray*}
		\alpha_u(\tau) = \left(\begin{matrix} 0 \\ 0 \\ u \end{matrix} \right),\qquad (u\in k).
	\end{eqnarray*}
	This is a necessary condition. To show it is sufficient one can apply \cite[Proposition 2]{martin2004nonab} which involves looking at the presentation of $\Gamma$.
	Now we calculate the class $[\alpha_u] \in H^1(\Gamma, Q)$. Suppose $[\alpha_v] = [\alpha_u]$. Then
	there is a $q\in Q$ fixed under the action of $\sigma$, that is of the form
	\begin{eqnarray*}
		q = \left(\begin{matrix} 0 \\ 0 \\ \lambda\end{matrix}\right),
	\end{eqnarray*}
	such that $\alpha_v(\gamma) = q\times\alpha_u(\gamma)\times\gamma\cdot q^{-1}$. In particular, for $\gamma = \tau$
	\begin{eqnarray*}
		\left(\begin{matrix} 0 \\ 0 \\ v\end{matrix}\right) &=&
		\left(\begin{matrix} 0 \\ 0 \\ \lambda\end{matrix}\right) \times
		\left(\begin{matrix} 0 \\ 0 \\ u\end{matrix}\right) \times
		\tau\cdot\left(\begin{matrix} 0 \\ 0 \\ \lambda\end{matrix}\right)\\
		&=&
		\left(\begin{matrix} 0 \\ 0 \\ \lambda\end{matrix}\right) \times
		\left(\begin{matrix} 0 \\ 0 \\ u\end{matrix}\right) \times
		\left(\begin{matrix} 0 \\ 0 \\ \lambda\end{matrix}\right)\\
		&=&
		\left(\begin{matrix} 0 \\ 0 \\ \lambda\end{matrix}\right) \times
		\left(\begin{matrix} 0 \\ 0 \\ u + \lambda\end{matrix}\right) \\
		&=&
		\left(\begin{matrix} 0 \\ 0 \\ u\end{matrix}\right).
	\end{eqnarray*}
	Hence only if $u=v$ are two 1-cocycles of the particular form in the same class, and therefore $H^1(\Gamma, Q)$ is infinite.
\end{proof}

In view of Theorem \ref{thm:k2_h1} one could show that the map $H^1(\Gamma, Q) \rightarrow H^1(\Gamma_p, Q)$ is not injective. Furthermore, it is natural to ask whether Example \ref{eg:non_red} leads to a reductive counterexample to K\"ulshammer's second question, although we can quickly verify that the answer is ``not immediately''. For suppose there was a reductive group with unipotent radical \emph{containing} the multiplication law:
\begin{eqnarray*}
	&&\ldots \epsilon_\alpha(u_\alpha) \ldots \epsilon_\beta(u_\beta) \ldots \epsilon_\gamma(u_\gamma) \times
	\ldots \epsilon_\alpha(v_\alpha) \ldots \epsilon_\beta(v_\beta) \ldots \epsilon_\gamma(v_\gamma)\\
	&&=
	\ldots \epsilon_\alpha(u_\alpha + v_\alpha) \ldots \epsilon_\beta(u_\beta + v_\beta) \ldots \epsilon_\gamma(u_\gamma + v_\gamma + u_\alpha v_\alpha + u_\beta v_\beta + u_\alpha v_\beta).
\end{eqnarray*}
Then setting $u_\delta = v_\delta = 0$ whenever $\delta \neq \alpha$ gives
\begin{eqnarray*}
	\epsilon_\alpha(u_\alpha) \times \epsilon_\alpha(v_\alpha) &=& \epsilon_\alpha(u_\alpha + v_\alpha) \epsilon_\gamma(u_\alpha v_\alpha),
\end{eqnarray*}
which is absurd.


