%!TEX root = ../Thesis.tex
% Chapter 4

\chapter{K\"ulshammer's Second Question}
\label{Chapter4}
\lhead{Chapter 4. \emph{K\"ulshammer's Second Question}}

\begin{definition} Let $X,Y$ be groups. Define
\begin{align*} \mathrm{Hom}(X, Y) = \{ \rho : X \rightarrow Y \,|\, \rho \textrm{ is a homomorphism}\}. 
\end{align*}
\end{definition}

For example, K\"ulshammer's second question as originally stated concerns $G$-conjugacy classes of $\mathrm{Hom}(\Gamma, G)$ and $\mathrm{Hom}(\Gamma_p, G)$, where $\Gamma_p$ is a Sylow $p$-subgroup of $\Gamma$.

As pointed out in the Introduction, K\"ulshammer's second question has positive answer for $\Gamma, G$ so long as $p$ is good for $G$. We wish to determine whether there exists a reductive counterexample to K\"ulshammer's second question.
A consequence of the following Theorem \cite[Theorem 1.2]{martin2003reductive} is that if there is a reductive counterexample to K\"ulshammer's second question then the infinitely many $G$-conjugacy classes of $\mathrm{Hom}(\Gamma, G)$ must be of non-$G$-completely reducible homomorphisms.

\begin{theorem} \label{thm:finiteGCR} There are only finitely many $G$-conjugacy classes of $G$-completely reducible homomorphisms from $\Gamma\rightarrow G$.
\end{theorem}
Let $G$ be a connected reductive algebraic group over an algebraically closed field $k$ with $\mathrm{char}(k) = p$. Let $P$ be a parabolic subgroup of $G$, with Levi subgroup $L$ and unipotent radical $V$. We have $P = V \rtimes L$, and we denote by $\pi^L$ the canonical projection
\begin{align*} \pi^L:P \rightarrow L. \end{align*}

Since $L$ normalizes $V$ we have an action by group automorphisms of $L$ on $V$ given by
\begin{align} \label{laction} l \cdot v = lvl^{-1}, \end{align}
for $l \in L, v \in V$.

Let $K$ be a linear algebraic group, $H$ a connected reductive algebraic group, and $\Gamma$ a finite group.
Generally speaking, we will be concerned with homomorphisms from $K \rightarrow G$. Where possible, we will work in this general setting and state things in terms of $K$. When we must specifically work with finite $K$ or connected reductive $K$, we will replace $K$ with $\Gamma$ or $H$, respectively.

The following general result shows that finiteness of $G$-conjugacy classes of $\mathrm{Hom}(K, P)$ carries over to $P$-conjugacy classes. 
\begin{lemma} Let $R \subset \mathrm{Hom}(K, P)$. Then $R$ is contained in a finite union of $G$-conjugacy classes if and only if it is contained in a finite union of $P$-conjugacy classes.
  \label{lem:GPconj}
\end{lemma}
\begin{proof}
	Let $\rho, \varsigma \in R$ such that $\rho$ and $\varsigma$ lie in the same $G$-conjugacy class of $R$. Then there exists $g\in G$ such that
	\begin{align*}
		g \rho(x) g^{-1} = \varsigma(x),
	\end{align*}
for all $x \in K$.
	
	Let $Q = gPg^{-1}$, hence $\varsigma(K) \subset P \cap Q$.
	Let $T$ be a maximal torus of $G$ contained in $P\cap Q$. Since $T$ and $gTg^{-1}$ are maximal tori of $Q$ they must be $Q$-conjugate, so there exists $q\in Q$ such that
	\begin{align*}
		qTq^{-1} = gTg^{-1}.
	\end{align*}
	Then there exists $r\in P$ such that $q = grg^{-1}$, so
	\begin{align*}
		grg^{-1}Tgr^{-1}g^{-1} &= gTg^{-1} \\
		\Rightarrow rg^{-1}Tgr^{-1} &= T.
	\end{align*}
	Therefore $gr^{-1} \in N_G(T)$. 

	Fix a finite set $N \subset N_G(T)$ of coset representatives for the Weyl group $W = N_G(T)/T$ and let $n \in N, t \in T$ such that
	\begin{align*}
		gr^{-1} = nt.
	\end{align*}
	Let $q' = r^{-1}t^{-1}$ so $q' \in P$. Then
	\begin{align*}
		\rho(x) &= g^{-1} \varsigma(x) g\\
		&= (q'n^{-1}) \varsigma(x) nq'^{-1},
	\end{align*}
	for all $x \in K$.
	This shows 
	%\begin{align*}
	%	R \cap \left(G\cdot \varsigma \right) \subset \bigcup_{n \in N} P \cdot(n^{-1} \cdot \varsigma).
	%\end{align*}
	%
	%That is,
	that a $G$-conjugacy class of $R$ is contained in a union of at most $|N| = |W|$ $P$-conjugacy classes.

	Therefore, if $R$ is contained in a finite union of $G$-conjugacy classes then it is contained in a finite union of $P$-conjugacy classes. The converse is trivial.
\end{proof}


\section{The Application of the 1-Cohomology}
Let $\sigma \in \mathrm{Hom}(K, L)$. Then via the action of $L$ on $V$ given in Equation \ref{laction}, we have an action of $K$ on $V$ given by
\begin{align} \label{haction} x \cdot v = \sigma(x) \cdot v, \end{align}
for $x \in K, v \in V$.

\begin{definition} Let $\rho \in \mathrm{Hom}(K, P)$. We associate with $\rho$ the map $\rho^L \in \mathrm{Hom}(K, L)$ defined by
%\begin{align*}
$\rho^L = \pi^L \circ \rho.$
%\end{align*}
\end{definition}

Let $\rho \in \mathrm{Hom}(K, P)$ and define the map $\alpha_\rho: K \rightarrow V$ by
\begin{align}\label{rho:alpha}
\alpha_\rho(x) = \rho(x)\rho^L(x^{-1}).
\end{align}
We have an action of $K$ on $V$ via $\rho^L$ using Equation \ref{haction}, and we will show that with this action the map $\alpha_\rho$ is a 1-cocycle from $K \rightarrow V$. To this end, let $x, y \in K$.
Then,
\begin{align*} \alpha_\rho(xy) &= \alpha_\rho(xy) \rho^L(xy) \left(\rho^L(xy)\right)^{-1} \\
	&= \rho(xy) \left(\rho^L(xy)\right)^{-1} \\
	&= \rho(x) \rho(y) \left(\rho^L(xy)\right)^{-1} \\
	&= \alpha_\rho(x) \rho^L(x) \alpha_\rho(y) \rho^L(y) \left(\rho^L(xy)\right)^{-1} \\
	&= \alpha_\rho(x) \rho^L(x) \alpha_\rho(y) \rho^L(x^{-1}) \rho^L(x) \rho^L(y) \left(\rho^L(xy)\right)^{-1} \\
	&= \alpha_\rho(x) \left(x \cdot \alpha_\rho(y)\right). 
\end{align*}

Therefore $\alpha_\rho$ satisfies the 1-cocycle condition in Equation \ref{eqn:na_z}.

\begin{definition}\label{h1sigma} Let $\sigma \in \mathrm{Hom}(K, L)$. We denote by
\begin{align*} Z^1(K, V)_\sigma \end{align*}
the set of 1-cocycles from $K \rightarrow V$ where $K$ acts on $V$ via $\sigma$.
Likewise, denote by
\begin{align*} H^1(K, V)_\sigma \end{align*}
the 1-cohomology obtained from $Z^1(K, V)_\sigma$ under the equivalence relation in Equation \ref{eqn:h_equiv}. Denote by $\psi$ the canonical projection
\begin{align*} \psi : Z^1(K, V)_\sigma \rightarrow H^1(K, V)_\sigma. \end{align*}
\end{definition} 

\begin{definition} Let $\sigma \in \mathrm{Hom}(K, L)$ and define
\begin{align*} \mathrm{Hom}(K, P)_\sigma = \{ \rho \in \mathrm{Hom}(K, P) \,|\, \rho^L = \sigma\}. \end{align*}
More generally, if $R \subset \mathrm{Hom}(K, P)$ define
\begin{align*} R_\sigma = \{ \rho \in R \,|\, \rho^L = \sigma \}. \end{align*}
\end{definition}

By construction, each $\rho \in \mathrm{Hom}(K, P)_\sigma$ yields a 1-cocycle $\alpha_\rho \in Z^1(K, V)_\sigma$ using Equation \ref{rho:alpha}.
Conversely given a 1-cocycle $\alpha \in Z^1(K, V)_\sigma$ we can construct a map $\rho: K \rightarrow P$ defined by
\begin{align}\label{alpha:rho}
\rho(x) = \alpha(x)\sigma(x),
\end{align}
for all $x \in K$. This construction is a homomorphism from $K \rightarrow P$. For take $x, y \in K$, then
\begin{align*}
  \rho(x y) &= \alpha(x y) \sigma(x y) \\
  &= \alpha(x)(x \cdot \alpha(y)) \sigma(x) \sigma(y) \\
  &= \alpha(x) \sigma(x) \alpha(y) \sigma(x)^{-1} \sigma(x) \sigma(y) \\
  &= \alpha(x) \sigma(x) \alpha(y) \sigma(y) \\
  &= \rho(x) \rho(y).
\end{align*}

We point out that the constructions in Equations \ref{rho:alpha} and \ref{alpha:rho} are inverses of each other, and formalize the preceding in the following Lemma.

\begin{lemma}
  Let $\sigma \in \mathrm{Hom}(K, L)$. The map
\begin{align*} z: \mathrm{Hom}(K, P)_{\sigma} \rightarrow Z^1(K, V)_\sigma, \end{align*}
defined by
\begin{align*} z(\rho)(x) = \rho(x)\sigma(x^{-1}), \end{align*}
for all $\rho \in \mathrm{Hom}(K, P)_\sigma$ and all $x \in K$, is a bijection.
\label{lem:hom_z1}
\end{lemma}
\begin{proof}
We have previously shown that $z$ is onto and has inverse defined by
\begin{align*} z^{-1}(\alpha)(x) = \alpha(x)\sigma(x), \end{align*}
for all $\alpha \in Z^1(K, V)_\sigma$ and all $x \in K$, so there is nothing to prove.
\end{proof}

%Henceforth we do no harm to use the otherwise suggestive notation ``$\alpha_\rho \in Z^1(K, V)_\sigma$'', meaning an element of $Z^1(K, V)_\sigma$ with corresponding element $\rho \in \mathrm{Hom}(K, P)_\sigma$ given by $\rho = z^{-1}(\alpha_\rho)$.

Since $\mathrm{Hom}(K, P)_\sigma$ is stable under conjugation by elements of $V$ we can consider $V$-conjugacy classes of $\mathrm{Hom}(K, P)_\sigma$.
\begin{definition} Denote by $\mathrm{Hom}(K, P)_\sigma / V$ the collection of $V$-conjugacy classes of $\mathrm{Hom}(K, P)_\sigma$, and denote by $\phi$ the canonical projection,
\begin{align*} \phi : \mathrm{Hom}(K, P)_\sigma \rightarrow \mathrm{Hom}(K, P)_\sigma / V. \end{align*}
\end{definition}

In fact, we show that $z$ descends to give a bijection from $\mathrm{Hom}(K, P)_\sigma / V \rightarrow H^1(K, V)_\sigma$.

\begin{lemma} \label{maph}
For $\rho \in \mathrm{Hom}(K, P)_\sigma$, define
\begin{align*}
h(\phi( \rho)) = \psi(z)(\rho).
\end{align*}
Then $h$ is a well-defined bijection from $\mathrm{Hom}(K, P)_\sigma / V \rightarrow H^1(K, P)_\sigma$. Moreover, the following diagram commutes:
  \begin{align*}
    \xymatrix{
    \mathrm{Hom}(K, P)_{\sigma} \ar[r]^z \ar[d]_\phi & Z^{1}(K, V)_\sigma \ar[d]^\psi \\
    \mathrm{Hom}(K, P)_{\sigma}/V \ar[r]^h & H^{1}(K, V)_\sigma.
    }
  \end{align*}
  \label{lem:v_h1}
\end{lemma}
\begin{proof}  
Let $\rho, \varsigma \in \mathrm{Hom}(K, P)_\sigma$ such that $\phi(\rho) = \phi(\varsigma)$. Then there exists $v \in V$ such that
\begin{align*} \varsigma(x) =  v\rho(x)v^{-1}, \end{align*}
for all $x \in K$. Furthermore, for all $x \in K$
\begin{align*}
z(\varsigma)(x) % &= \alpha_\varsigma(x) \\
&= \varsigma(x)\sigma(x^{-1}) \\
&= v \rho(x) v^{-1} \sigma(x^{-1}) \\
&= v \rho(x) \sigma(x^{-1})\sigma(x) v^{-1} \sigma(x^{-1}) \\
&= v \rho(x) \sigma(x^{-1}) \left(x \cdot v^{-1}\right) \\
%&= v \alpha_\rho(x) \left(x \cdot v^{-1}\right) \\
&= v \left(z(\rho)(x)\right) \left(x \cdot v^{-1}\right).
\end{align*}
This shows that $z(\rho)$ and $z(\varsigma)$ satisfy the equivalence relation in Equation \ref{eqn:h_equiv}. Therefore $\psi\left(z(\rho)\right) = \psi\left(z(\varsigma)\right)$ and so $h$ is well-defined.

Since $z$ and $\psi$ are onto, $h$ is onto. We show $h$ is one-to-one.

Let $\alpha, \beta \in Z^1(K, V)_\sigma$ such that $\psi(\alpha) = \psi(\beta)$. Then there exists $v \in V$ such that
\begin{align*} \beta(x) = v \alpha(x) \left(x \cdot v^{-1}\right), \end{align*}
for all $x \in K$. Then
\begin{align*}
z^{-1}(\beta)(x) 
&= \beta(x)\sigma(x) \\
&= v \alpha(x) \left(x \cdot v^{-1}\right) \sigma(x) \\
&= v \alpha(x) \sigma(x) v^{-1} \sigma(x^{-1}) \sigma(x) \\
&= v \alpha(x) \sigma(x) v^{-1} \\
&= v z^{-1}(\alpha)(x) v^{-1},
\end{align*}
for all $x \in K$.
This shows that $\phi\left(z^{-1}(\alpha)\right) = \phi\left(z^{-1}(\beta)\right)$ and therefore $h$ is one-to-one. This completes the proof.
\end{proof}

Let $q \in P$, and let $v \in V, l \in L$ such that $q = vl$. We can conjugate $\rho \in \mathrm{Hom}(K, P)_\sigma$ by $q$ to yield an element of $\mathrm{Hom}(K, P)_{l \cdot \sigma}$. For
\begin{align*}% \left(p \cdot \rho(x)\right)^L
\pi^L\left(q \rho(x) q^{-1}\right)% \\
&= \pi^L\left(q\right) \pi^L\left(\rho(x)\right) \pi^L\left(q^{-1}\right) \\
&= l \sigma(x) l^{-1},
\end{align*}
for all $x \in K$, so $q \cdot \rho \in \mathrm{Hom}(K, P)_{l\cdot \sigma}$.

Using Definition \ref{h1sigma}, we can define $Z^1(K, V)_{l \cdot \sigma}$ to be the collection of 1-cocycles from $K \rightarrow V$ where $K$ acts on $V$ via $l \cdot \sigma$. The formula for the action is then given by
\begin{align*}
	x.v = l \sigma(x) l^{-1} v l \sigma(x^{-1}) l^{-1},
\end{align*}
for all $x \in K, v \in V$. Evidently $Z^1(K, V)_{l \cdot \sigma} = z\left(\mathrm{Hom}(K, P)_{l \cdot \sigma}\right)$. The reason for the new notation for the action above will become apparent presently, as we show that we have a map of 1-cohomomologies $H^1(K, V)_\sigma \rightarrow H^1(K, V)_{l \cdot \sigma}$.

	We show that we can apply Lemma \ref{h1maps}. To this end, let $K' = K$, $V' = V$ and let $\zeta = \mathrm{id}:K \rightarrow K$.

Let $\xi = \xi_l: V \rightarrow V'$ be defined by	$\xi_l(v) = lvl^{-1}$ for all $v \in V$. Now we show that $\xi_l$ satisfies condition (d) of the Lemma. Let $x \in K, v \in V$. Then
	\begin{align*}
		x . \xi_l(v) &= x . \left(lvl^{-1}\right) \\
		&= l\sigma(x)l^{-1} \left(lvl^{-1}\right) l\sigma(x^{-1})l^{-1} \\
		&= l\sigma(x)v\sigma(x^{-1})l^{-1} \\
		&= l\left(x \cdot v \right)l^{-1} \\
		&= \xi_l \left(x \cdot v \right).
	\end{align*}
	Therefore, by Lemma \ref{h1maps} the map $Z^1(\zeta, \xi_l):Z^1(K, V)_\sigma \rightarrow Z^1(K, V)_{l \cdot \sigma}$ defined by 
	\begin{align*}
		Z^1(\zeta, \xi_l)(\alpha) = \xi_l \circ \alpha,
	\end{align*}
	for all $\alpha \in Z^1(K, V)_\sigma$,
	descends to give a well-defined map of 1-cohomologies
	\begin{align*}
		H^1(\zeta, \xi_l):H^1(K, V)_\sigma \rightarrow H^1(K, V)_{l \cdot \sigma},
	\end{align*}
	defined by
	\begin{align*}
		H^1(\zeta, \xi_l)\left(\psi(\alpha)\right) = \psi'\left(Z^1(\zeta, \xi_l)(\alpha)\right),
	\end{align*}
	for all $\alpha \in Z^1(K, V)_\sigma$,
	where $\psi'$ is the canonical projection from $Z^1(K, V)_{l \cdot \sigma} \rightarrow H^1(K, V)_{l \cdot \sigma}$.

	Evidently, if $l \in C_L\left(\sigma(K)\right)$ then $\mathrm{Hom}(K, P)_\sigma$ is stable under conjugation by $vl$ and $H^1(\zeta, \xi_l)$ maps $H^1(K, V)_{\sigma}$ into itself.

	\begin{lemma}\label{lem:vcl} We have an action of $C_L(\sigma(K))$ on $\mathrm{Hom}(K, V)_\sigma / V$ defined by
		\begin{align} \label{vcl.hom}
			c \cdot \phi(\rho) = \phi\left(\xi_c \circ \rho \right),
		\end{align}
		for $c \in C_L(\sigma(K)), \rho \in \mathrm{Hom}(K, V)_\sigma, x \in K$.
		
		Similarly, we have an action of $C_L(\sigma(K))$ on $H^1(K, V)_\sigma$ defined by
		\begin{align} \label{cl.h1}
			c \cdot \psi(\alpha)
			&= H^1(\zeta, \xi_c)\left(\psi(\alpha)\right), 
		\end{align}
		for $c \in C_L(\sigma(K)), \alpha \in Z^1(K, V)_\sigma$.

	Moreover,
	\begin{align}\label{eqn:caction}
		h(c\cdot\phi(\rho)) = c \cdot h(\phi(\rho)).
	\end{align}
	\end{lemma}
\begin{proof}
	First we show Equation \ref{vcl.hom} defines a well-defined map. Let $\rho, \varsigma \in \mathrm{Hom}(K, V)_\sigma$ such that $\phi(\rho) = \phi(\varsigma)$. Then there exists $v \in V$ such that
	\begin{align*} \varsigma(x) = v\rho(x)v^{-1}, \end{align*}
		for all $x \in K$. Now let $c \in C_L(\sigma(K))$. Since $L$ normalizes $V$ there exists $w \in V$ such that $cv = wc$. Then
		\begin{align*}
			c \varsigma(x)c^{-1}
			&= cv \rho(x) v^{-1}c^{-1} \\
			&= wc \rho(x) c^{-1}w^{-1} \\
			&= c \rho(x) c^{-1},
		\end{align*}
		for all $x \in K$. Therefore $\phi(\xi_c \circ \varsigma) = \phi(\xi_c \circ \rho)$ and hence the map defined by Equation \ref{vcl.hom} is well-defined. It is clear that the group action axioms are satisfied, so Equation \ref{vcl.hom} defines an action of $C_L(\sigma(K))$ on $\mathrm{Hom}(K, P)_\sigma$.

		Equation \ref{cl.h1} is well-defined thanks to Lemma \ref{h1maps}. If $e$ is the identity element of $C_L(\sigma(K))$ then the map $H^1(\zeta, \xi_e) = \mathrm{id}:H^1(K, V)_\sigma \rightarrow H^1(K, V)_\sigma$. Finally, let $c_1, c_2 \in C_L(\sigma(K))$. Then
		\begin{align*}
			H^1(\zeta, \xi_{c_1})\left(H^1(\zeta, \xi_{c_2})(\psi(\alpha))\right)
			&= H^1(\zeta, \xi_{c_1})\left(\psi(Z^1(\zeta, \xi_{c_2})(\alpha))\right) \\
			&= H^1(\zeta, \xi_{c_1})\left(\psi(c_2 \cdot \alpha)\right) \\
			&= \psi\left( Z^1(\zeta, \xi_{c_1})(c_2 \cdot \alpha)\right) \\
			&= \psi\left( c_1 \cdot (c_2 \cdot \alpha)\right) \\
			&= \psi\left( c_1c_2 \cdot \alpha\right) \\
			&= \psi\left( Z^1(\zeta, \xi_{c_1c_2})(\alpha)\right) \\
			&= H^1(\zeta, \xi_{c_1c_2})\left(\psi(\alpha)\right).
		\end{align*}
		This shows that Equation \ref{cl.h1} defines an action of $C_L(\sigma(K))$ on $H^1(K, V)_\sigma$.

		To prove Equation \ref{eqn:caction} first notice that $\left(\xi_c \circ \rho\right)^L = \rho^L$, hence
		\begin{align*}
			(z(\xi_c \circ \rho))(x) &= c\rho(x)c^{-1}\rho^L(x^{-1}) \\
				&= c\rho(x)\rho^L(x^{-1})c^{-1} \\
				&= \left(\xi_c ( z(\rho))\right)(x),
		\end{align*}
		for all $x \in K$. Therefore $z\left(\xi_c \circ \rho\right) = \xi_c\left(z(\rho)\right)$, and so
		\begin{align*}
			h(c \cdot \phi(\rho)) &= h(\phi(\xi_c \circ \rho)) \\
			&= \psi(z(\xi_c\circ\rho)) \\
			&= \psi(\xi_c(z(\rho))) \\
			&= \psi\left(Z^1(\zeta, \xi_c)(z(\rho))\right) \\
			&= H^1(\zeta, \xi_c)(\psi(z(\rho))) \\
			&= c \cdot \psi(z(\rho)) \\
			&= c \cdot h(\phi(\rho)).
		\end{align*}
\end{proof}
	\begin{definition}
		Denote by $\mathrm{Hom}(K, P)_\sigma/VC_L(\sigma)$ the collection of $C_L(\sigma(K))$-conjugacy classes of $\mathrm{Hom}(K, P)_\sigma/V$, equivalently the collection of $VC_L(\sigma(K))$-conjugacy classes of $\mathrm{Hom}(K, P)_\sigma$, and denote by $\widetilde{\phi}$ the canonical projection,
		\begin{align*}
			\widetilde{\phi}: \mathrm{Hom}(K, P)_\sigma/V \rightarrow \mathrm{Hom}(K, P)_\sigma/VC_L(\sigma).
		\end{align*}
		Similarly, denote by $H^1(K, V)_\sigma/C_L(\sigma)$ the set of orbits of $H^1(K, V)_\sigma$ under the action defined in Equation \ref{cl.h1}, and denote by $\widetilde{\psi}$ the canonical projection
		\begin{align*}
			\widetilde{\psi}:H^1(K, V)_\sigma \rightarrow H^1(K, V)_\sigma/C_L(\sigma).
		\end{align*}
	Define $\Phi:\mathrm{Hom}(K, P)_\sigma \rightarrow \mathrm{Hom}(K, P)_\sigma/VC_L(\sigma)$ by
	\begin{align*}
		\Phi = \widetilde{\phi} \circ \phi,
	\end{align*}
	and define $\Psi:Z^1(K, V)_\sigma \rightarrow H^1(K, V)_\sigma/C_L(\sigma)$ by
	\begin{align*}
		\Psi = \widetilde{\psi} \circ \psi.
	\end{align*}
	\end{definition}

	We have the following Lemma, which shows that the map $h$ descends to give a bijection from $\mathrm{Hom}(K, P)_\sigma /VC_L(\sigma) \rightarrow H^1(K, V)_{\sigma}/C_L(\sigma)$.

\begin{lemma}
	For $\rho \in \mathrm{Hom}(K, P)_\sigma$, define
	\begin{align*}
		\widetilde{h}(\Phi(\rho)) = \widetilde{\psi}(h(\phi(\rho))).
	\end{align*}
	Then $\widetilde{h}$ is a well-defined bijection from
	\begin{align*}
		\mathrm{Hom}(K, P)_\sigma / VC_L(\sigma) \rightarrow H^1(K, V)_\sigma/C_L(\sigma).
	\end{align*}
	Moreover, the following diagram commutes:
  \begin{align*}
    \xymatrix{
		\mathrm{Hom}(K, P)_{\sigma}/V \ar[r]^h \ar[d]_{\widetilde{\phi}} & H^{1}(K, V)_\sigma \ar[d]^{\widetilde{\psi}} \\
		\mathrm{Hom}(K, P)_{\sigma}/VC_L(\sigma) \ar[r]^{\widetilde{h}} & H^{1}(K, V)_\sigma/C_L(\sigma).
    }
  \end{align*}
\end{lemma}
\begin{proof}
Let $\rho, \varsigma \in \mathrm{Hom}(K, V)_\sigma$ such that $\Phi(\rho) = \Phi(\varsigma)$. Then there exists $c \in C_L(\sigma)$ such that $\phi(\varsigma) = c\cdot\phi(\rho)$, and we have
\begin{align*}
	h(\phi(\varsigma)) &= h(c \cdot\phi(\rho)) \\
	&= c \cdot h(\phi(\rho))\quad(\textrm{Lemma }\ref{lem:vcl}). 
\end{align*}
Hence $\widetilde{\psi}(h(\phi(\rho))) = \widetilde{\psi}(h(\phi(\varsigma)))$ and so $\widetilde{h}$ is well-defined.

Let $\alpha, \beta \in Z^1(K, V)_\sigma$ such that $\Psi(\alpha) = \Psi(\beta)$. Then there exists $c \in C_L(\sigma)$ such that $\psi(\beta) = c \cdot \psi(\alpha)$, and since $h$ is a bijection, $h^{-1}(\psi(\beta)) = h^{-1}(c \cdot \psi(\alpha)) = c \cdot h^{-1}(\psi(\alpha))$. This shows $\widetilde{h}$ is injective. Since $h$ and $\widetilde{\psi}$ are surjective, $\widetilde{h}$ is surjective. Therefore $\widetilde{h}$ is a bijection.
\end{proof}

\begin{lemma} Let $K' < K'$, let $\zeta$ the inclusion of $K'$ in $K$, and let $\xi$ be the identity map on $V$. Then the map $H^1(\zeta, \xi):H^1(K, V)_\sigma \rightarrow H^1(K', V)_{\sigma\circ\zeta}$ descends to give a well-defined map
	\begin{align*}
		\widetilde{H^1}(\zeta, \xi): H^1(K, V)_\sigma/C_L(\sigma) \rightarrow H^1(K', V)_{\sigma\circ\zeta}/C_L(\sigma\circ \zeta),
	\end{align*}
defined by
	\begin{align*}
		\widetilde{H^1}(\zeta, \xi)(\Psi(\alpha)) = \widetilde{\psi'}\left(H^1(\zeta, \xi)(\psi(\alpha))\right).
	\end{align*}
Moreover, the following diagram commutes:
  \begin{align*}
    \xymatrix@C=60pt{
		H^1(K, V)_\sigma \ar[r]^{H^1(\zeta, \xi)} \ar[d]_{\widetilde{\psi}} & H^1(K', V)_{\sigma\circ\zeta} \ar[d]^{\widetilde{\psi'}} \\
		H^1(K, V)_\sigma/C_L(\sigma) \ar[r]^{\widetilde{H^1}(\zeta, \xi)} & {H^1(K', V)_{\sigma\circ\zeta}/C_L(\sigma\circ\zeta)}.
    }
  \end{align*}
\end{lemma}
\begin{proof}
	Let $\alpha, \beta \in Z^1(K, V)_\sigma$ such that $\Psi(\alpha) = \Psi(\beta)$. Then there exists $c \in C_L(\sigma)$ such that $\psi(\beta) = c \cdot \psi(\alpha)$. Since $H^1(K, V)_\sigma \subset H^1(K', V)_\sigma$ and $C_L(\sigma) < C_L(\sigma\circ\zeta)$, we have
	\begin{align*}
		H^1(\zeta, \xi)\left(\psi(\beta)\right) = c \cdot H^1(\zeta, \xi)\left(\psi(\beta)\right),
	\end{align*}
	and so $\Psi'(\alpha) = \Psi'(\beta)$. Therefore $\widetilde{H^1}(\zeta, \xi)$ is well-defined.
\end{proof}

\begin{definition} We define
	\begin{align*}
		\mathrm{Hom}(K, P)^L = \{\rho^L\,|\,\rho \in \mathrm{Hom}(K, P)\}.
	\end{align*}
	More generally, when $R \subset \mathrm{Hom}(K, P)$ we define
	\begin{align*}
		R^L = \{\rho^L\,|\,\rho \in R\}.
	\end{align*}
\end{definition}

\begin{lemma} \label{pr:lrl} Let $R \subset \mathrm{Hom}(K, P)$. Suppose $R = P \cdot \rho$ for some $\rho \in R$. Then $R^L = L\cdot \rho^L$.
	More generally, if $R = P \cdot R$ then $R^L = L \cdot R^L$.
\end{lemma}
\begin{proof}
	Let $\rho \in R$. Let $q \in P, v \in V, l \in L$ such that $q = vl$. Since $q\cdot\rho \in R$, and
	\begin{align*}
		(q \cdot \rho)^L(x) &= (q\rho(x)q^{-1})^L \\
			&= (vl\rho(x)l^{-1}v^{-1})^L \\
			&= \pi^L(v)\pi^L(l)\pi^L(\rho(x))\pi^L\left(l^{-1}\right)\pi^L\left(v^{-1}\right) \\
			&= l \rho^L(x) l^{-1},
	\end{align*}
	then $(q \cdot \rho)^L = l \cdot \rho^L \in R^L$. This shows $R^L \subset L \cdot R^L$. Conversely, let $l \in L$. Then $l \cdot \rho \in R$, so $(l \cdot \rho)^L = l\cdot \rho^L \in R^L$. Therefore $L\cdot R^L \subset R^L$ and so $R^L = L \cdot R^L$.
\end{proof}

\begin{lemma} \label{rsigma:vcl} Let $R \subset \mathrm{Hom}(K, P)$ and suppose that $R = P \cdot R$. Then for all $\sigma \in \mathrm{Hom}(K, L)$,
	\begin{align*}
		R_\sigma \cap P \cdot \rho = \left(VC_L(\sigma)\right) \cdot \rho,
	\end{align*}
	for all $\rho \in R_\sigma$.
\end{lemma}
\begin{proof}
	Let $\sigma \in \mathrm{Hom}(K, L)$ and choose $\rho \in R_\sigma$. Suppose there exists $q \in P$ such that $q \cdot \rho \in R_\sigma$, and let $v \in V, l \in L$ such that $q = vl$. Then
	\begin{align*}
		q \cdot \rho \in R_\sigma &\Leftrightarrow (vl) \cdot \rho \in R_\sigma \\
		&\Leftrightarrow \left( (vl) \cdot \rho\right)^L = \sigma \\
		&\Leftrightarrow l\cdot \rho^L = \sigma \\
		&\Leftrightarrow l \in C_L(\sigma).
	\end{align*}
	This shows that $R_\sigma \cap P \cdot \rho \subset \left(VC_L(\sigma)\right) \cdot \rho$. The reverse inclusion follows since $R = P \cdot R$ and $R_\sigma$ is stable under conjugation by $V$ and $C_L(\sigma)$.
\end{proof}

\begin{theorem}\label{r:finite_p} Let $R \subset \mathrm{Hom}(K, P)$ and suppose $P \cdot R = R$. Then $R$ is a finite union of $P$-conjugacy classes if and only if
	\begin{itemize}
		\item[(i)] $R^L$ is a finite union of $L$-conjugacy classes, and
		\item[(ii)] for each $\sigma \in \mathrm{Hom}(K, L)$, $(\widetilde{h} \circ \Phi)(R_\sigma)$ is finite in $H^1(K, V)_\sigma/C_L(\sigma)$.
	\end{itemize}
\end{theorem}
\begin{remark} Conditions (i) and (ii) are equivalent to 
	\begin{itemize}
		\item[(i$'$)] $R_\sigma = \emptyset$ for all but finitely many $L$-conjugacy classes of $\sigma \in \mathrm{Hom}(K, L)$, and
		\item[(ii$'$)] for each $\sigma \in \mathrm{Hom}(\Gamma, L)$, $R_\sigma$ is a finite union of $VC_L(\sigma)$-conjugacy classes,
	\end{itemize}
	respectively. We obtain (ii) $\Leftrightarrow$ (ii$'$) by appealing to the bijection $\widetilde{h}$, while (i) $\Leftrightarrow$ (i$'$) is self-evident.
\end{remark}
\begin{proof}
Suppose $R$ is a finite union of $P$-conjugacy classes, so there exists a finite set $\mathcal{P} \subset \mathrm{Hom}(\Gamma, P)$ such that
\begin{align*} R = \bigcup_{\rho \in \mathcal{P}} P \cdot \rho \end{align*}
Lemma \ref{pr:lrl} shows that (i) holds. We have
\begin{align*} R_\sigma &= R_\sigma \cap R \\
&= R_\sigma \cap \big( \bigcup_{\rho \in \mathcal{P}} P \cdot \rho \,\,\big) \\
&= \bigcup_{\rho \in \mathcal{P}} \left( R_\sigma \cap P \cdot \rho \right).
\end{align*}
Then by Lemma \ref{rsigma:vcl}
\begin{align*} R_\sigma = \bigcup_{\rho \in \mathcal{P}} (VC_L(\sigma)) \cdot \rho. \end{align*}
Hence (ii$'$), and therefore (ii), holds. This proves the forward direction.


Conversely, suppose (i) and (ii) hold and let $\sigma \in \mathrm{Hom}(K, L)$. By (ii$'$) there exists a finite set $\mathcal{Q} \subset \mathrm{Hom}(\Gamma, P)$ such that
\begin{align*}
	R_\sigma &= \bigcup_{\rho \in \mathcal{Q}} (VC_L(\sigma)) \cdot \rho.
\end{align*}
Applying Lemma \ref{rsigma:vcl}, we get
\begin{align*}
	R_\sigma &= \bigcup_{\rho \in \mathcal{Q}} \left( R_\sigma \cap P \cdot \rho \right) \\
	&= R_\sigma \cap \big( \bigcup_{\rho \in \mathcal{Q}} P \cdot \rho \,\,\big).
\end{align*}
Hence $R_\sigma$ is contained in a finite union of $P$-conjugacy classes.
Define
\begin{align*}
L \cdot R_\sigma = \{L \cdot \rho \,|\, \rho \in R_\sigma\},
\end{align*}
so $L \cdot R_\sigma$ is contained in a finite union of $P$-conjugacy classes.

Now let $\rho \in R$. By (i) there exists a finite set $\mathcal{S} \subset \mathrm{Hom}(\Gamma, L)$ such that
\begin{align*}
R^L = \bigcup_{\tau \in \mathcal{S}} L \cdot \tau.
\end{align*}

Then there exists $l \in L$, $\tau \in \mathcal{S}$ such that $\rho^L = l \cdot \tau$. Hence $l^{-1} \cdot \rho^L = \tau$, so $\rho \in L \cdot R_\tau$. This shows that $R$ is contained in a finite union of $P$-conjugacy classes. The reverse inclusion is satisfied since $R = P \cdot R$, and this completes the proof.
\end{proof}

\begin{definition} Let $K' < K$ and let $\zeta$ be the inclusion of $K'$ in $K$. Define
	\begin{align*}
		\mathrm{Hom}(K, P)^\zeta = \{ \rho\circ\zeta \,|\, \rho \in \mathrm{Hom}(K, P)\} \subset \mathrm{Hom}(K', P).
	\end{align*}
	More generally, if $R \subset \mathrm{Hom}(K, P)$ then define
	\begin{align*}
		R^\zeta = \{\rho\circ\zeta \,|\, \rho \in R\}.
	\end{align*}
\end{definition}

\begin{theorem}\label{main_thm} Let $K'<K$, let $\zeta$ be the inclusion of $K'$ in $K$, and let $\xi$ be the identity map on $V$. Let $R \subset \mathrm{Hom}(K, P)$ such that $R = P \cdot R$. Suppose
	\begin{itemize}
		\item[(i)] $R^L$ is a finite union of $L$-conjugacy classes,
		\item[(ii)] for all $\sigma \in \mathrm{Hom}(K, L)$ such that $R_\sigma \neq \emptyset$, the map
			\begin{align*}
				\widetilde{H^1}(\zeta, \xi):H^1(K, V)_\sigma/C_L(\sigma) \rightarrow H^1(K', V)_{\sigma\circ\zeta}/C_L(\sigma\circ\zeta),
			\end{align*}
			has finite fibres, and
		\item[(iii)] $R^\zeta$ is a finite union of $P$-conjugacy classes.
	\end{itemize}
	Then $R$ is a finite union of $P$-conjugacy classes.
\end{theorem}
\begin{remark}
	Since $R = P \cdot R$, then by Lemma \ref{pr:lrl} $R^L$ is already a union of $L$-conjugacy classes. The point of (i) is that the union is finite.
\end{remark}
\begin{proof}
	Denote the maps
	\begin{align*}
		\mathcal{Z}(\zeta):&\mathrm{Hom}(K, P)_\sigma \rightarrow \mathrm{Hom}(K', P)_{\sigma\circ\zeta}, \\
		\mathcal{H}(\zeta):&\mathrm{Hom}(K, P)_\sigma/V \rightarrow \mathrm{Hom}(K', P)_{\sigma\circ\zeta}/V, \\
		\widetilde{\mathcal{H}}(\zeta):&\mathrm{Hom}(K, P)_\sigma/VC_L(\sigma) \rightarrow \mathrm{Hom}(K', P)_{\sigma\circ\zeta}/VC_L(\sigma\circ\zeta,
	\end{align*}
	defined by
	\begin{align*}
		\mathcal{Z}(\zeta)(\rho) &= \rho\circ\zeta, \\
		\mathcal{H}(\zeta)(\phi(\rho)) &= \phi'(\rho\circ\zeta), \\
		\widetilde{\mathcal{H}}(\zeta)(\Phi(\rho)) &= \Phi'(\rho\circ\zeta).
	\end{align*}
It is clear that the maps $\mathcal{H}(\zeta), \widetilde{\mathcal{H}}(\zeta)$ are well-defined.
The situation is summarised by the following commutative diagram.
\par\nobreak
	{\small
	\setlength{\abovedisplayskip}{6pt}
	\setlength{\belowdisplayskip}{\abovedisplayskip}
	\setlength{\abovedisplayshortskip}{3pt}
	\setlength{\belowdisplayshortskip}{3pt}
	\begin{align*}
		\xymatrix@R=40pt{
			\mathrm{Hom}(K, P)_\sigma \ar[r]|-{z} \ar[d]^{\phi} \ar[dddr]|-{\mathcal{Z}(\zeta)} & Z^1(K, V)_\sigma \ar[d]_{\psi} \ar[dddr]|-{Z^1(\zeta, \xi)} & \\
			\mathrm{Hom}(K, P)_\sigma/V \ar[r]|-{h} \ar[d]^{\widetilde{\phi}} \ar[dddr]|-{\mathcal{H}(\zeta)} & H^1(K, P)_\sigma \ar[d]_{\widetilde{\psi}} \ar[dddr]|-{H^1(\zeta, \xi)} & \\
			\mathrm{Hom}(K, P)_\sigma/VC_L(\sigma) \ar[r]|-{\widetilde{h}} \ar[dddr]|-{\widetilde{\mathcal{H}}(\zeta)} & H^1(K, P)_\sigma/C_L(\sigma) \ar[dddr]|-{\widetilde{H^1}(\zeta, \xi)} & \\
			& \mathrm{Hom}(K', P)_{\sigma\circ\zeta} \ar[r]|-{z'} \ar[d]^{\phi'} & Z^1(K', V)_\sigma \ar[d]_{\psi'} \\
			& \mathrm{Hom}(K', P)_{\sigma\circ\zeta}/V \ar[r]|-{h'} \ar[d]^{\widetilde{\phi'}} & H^1(K', V)_{\sigma\circ\zeta} \ar[d]_{\widetilde{\psi'}} \\
			& \mathrm{Hom}(K', P)_{\sigma\circ\zeta}/VC_L(\sigma\circ\zeta) \ar[r]|-{\widetilde{h'}} & H^1(K', V)_{\sigma\circ\zeta}/C_L(\sigma\circ\zeta) \\
		}
	\end{align*}
	}%
\par
Since $P \cdot R = R$, it follows that $P\cdot R^\zeta = R^\zeta$. By definition $R^\zeta \subset \mathrm{Hom}(K', V)$ and by (iii) $R^\zeta$ is a finite union of $P$-conjugacy classes. Therefore by Theorem \ref{r:finite_p}
\begin{itemize}
	\item[(iv)] $\left(R^\zeta\right)^L$ is a finite union of $L$-conjugacy classes, and
	\item[(v)] for each $\sigma \in \mathrm{Hom}(K', L)$, $(\widetilde{h'} \circ \Phi')\left(\left(R^\zeta\right)_\sigma\right)$ is finite in $H^1(K', V)_{\sigma\circ\zeta}/C_L(\sigma\circ\zeta)$.
\end{itemize}

Let $\sigma \in \mathrm{Hom}(K, L)$ and suppose $R_\sigma \neq \emptyset$. Since $\sigma\circ\zeta \in \mathrm{Hom}(K', L)$, $(\widetilde{h'} \circ \Phi')\left((R^\zeta)_{\sigma\circ\zeta}\right)$ is finite by (v). Furthermore, since $R_\sigma \neq \emptyset$,
\begin{align*}
\widetilde{H^1}(\zeta, \xi)^{-1}\left((\widetilde{h'} \circ \Phi')\left((R^\zeta)_{\sigma\circ\zeta}\right)\right)
\end{align*}
is finite by (ii). But 
\begin{align*}
(\widetilde{h} \circ \Phi)\left(R_\sigma)\right) \subset \widetilde{H^1}(\zeta, \xi)^{-1}\left((\widetilde{h'} \circ \Phi')\left((R^\zeta)_{\sigma\circ\zeta}\right)\right),
\end{align*}
so $(\widetilde{h} \circ \Phi)\left(R_\sigma)\right)$ is finite.

Clearly $(\widetilde{h} \circ \Phi)\left(R_\sigma)\right)$ is finite if $R_\sigma = \emptyset$. Then, together with (i), Theorem \ref{r:finite_p} states that $R$ is a finite union of $P$-conjugacy classes.
\end{proof}

\section{A Non-Reductive Counterexample}
In \cite{slodowy1997two} a counterexample to K\"ulshammer's second question is presented for a closed field $k$ of characteristic $p = 2$ and a non-reductive algebraic group $G$.
\begin{example} Let $Q$ be the algebraic group isomorphic to the affine space $\mathbf{A}^3$ with the group multiplication law:
\begin{align*}
	\left(\begin{matrix} u_1 \\ u_2 \\ u_3 \end{matrix}\right) \times
	\left(\begin{matrix} v_1 \\ v_2 \\ v_3 \end{matrix}\right) &=
	\left(\begin{matrix} u_1 + v_1 \\ u_2 + v_2 \\ u_3 + v_3 + u_1v_1 + u_2v_2 + u_1v_2 \end{matrix}\right).
\end{align*}
Let $\Gamma = \langle \sigma, \tau | \sigma^3 = \tau^2 = 1, \tau\sigma\tau = \sigma^2 \rangle$ and $\Gamma_2 = \langle \tau \rangle$, a Sylow 2-subgroup of $\Gamma$. $\Gamma$ acts on $Q$ via
\begin{align*}
	\tau \cdot \left(\begin{matrix} u_1 \\ u_2 \\ u_3 \end{matrix} \right) &=
	\left(\begin{matrix} u_2 \\ u_1 \\ u_3 + u_1^2 + u_2^2 + u_1u_2 \end{matrix} \right) \\
	\sigma \cdot \left(\begin{matrix} u_1 \\ u_2 \\ u_3 \end{matrix} \right) &=
	\left(\begin{matrix} u_2 \\ u_1 + u_2 \\ u_3 \end{matrix} \right).
\end{align*}
Let $G = Q \rtimes \Gamma$ and fix the representation $\rho:\Gamma_2 \rightarrow G$ defined by the natural inclusion $\Gamma_2 \rightarrow \Gamma \rightarrow G$. Then there are infinitely many pairwise $G$-conjugate classes of extensions of $\rho$ to representations of $\Gamma$ into $G$ \cite[Appendix]{slodowy1997two}.
\label{eg:non_red}
\end{example}
\begin{proof}
	Our proof will be way of a 1-cohomology calculation. Choose a 1-cocycle $\alpha \in Z^1(\Gamma, Q)$ such that $\alpha|_{\langle \sigma \rangle} = 1$. Let
	\begin{align*}
		\alpha(\tau) = \left(\begin{matrix} u_1 \\ u_2 \\ u_3 \end{matrix} \right),
	\end{align*}
	for some $u_1, u_2, u_3 \in k$. Since $\tau$ is an involution we have
	\begin{align*}
		1 = \alpha(\tau^2) &= \alpha(\tau) \times \tau\cdot\alpha(\tau)\\
		&= \left(\begin{matrix} u_1 \\ u_2 \\ u_3\end{matrix} \right) \times 
		\left(\begin{matrix} u_2 \\ u_1 \\ u_3 + u_1^2 + u_2^2 + u_1u_2\end{matrix} \right) \\
		&= \left(\begin{matrix} u_1 + u_2 \\ u_1 + u_2 \\ 2u_3 + 2u_1^2 + u_2^2 + 3u_1u_2\end{matrix} \right)\\
		&= \left(\begin{matrix} u_1 + u_2 \\ u_1 + u_2 \\ u_2^2 + u_1u_2\end{matrix} \right).
	\end{align*}
	This shows $u_1 = u_2$, so
	\begin{align*}
		\alpha(\tau) = \left(\begin{matrix} u_1 \\ u_1 \\ u_3\end{matrix} \right).
	\end{align*}
	Furthermore, as $\tau\sigma\tau = \sigma^2$ we obtain
	\begin{align*}
		1 = \alpha(\sigma^2) &= \alpha(\tau\sigma\tau) \\
		&= \alpha(\tau) \times \tau\cdot\alpha(\sigma\tau)\\
		&= \alpha(\tau) \times \tau\cdot\alpha(\sigma) \times \tau\sigma\cdot\alpha(\tau) \\
		&= \alpha(\tau) \times \tau\sigma\cdot\alpha(\tau) \\
		&= \left(\begin{matrix} u_1 \\ u_1 \\ u_3\end{matrix} \right) \times
		\tau\sigma\cdot\left(\begin{matrix} u_1 \\ u_1 \\ u_3\end{matrix} \right)\\
		&= \left(\begin{matrix} u_1 \\ u_1 \\ u_3\end{matrix} \right) \times
		\tau\cdot\left(\begin{matrix} u_1 \\ 0 \\ u_3\end{matrix} \right)\\
		&= \left(\begin{matrix} u_1 \\ u_1 \\ u_3\end{matrix} \right) \times
		\left(\begin{matrix} 0 \\ u_1 \\ u_3 + u_1^2\end{matrix} \right)\\
		&= \left(\begin{matrix} u_1 \\ 0 \\ 2u_3 + 3u_1^2\end{matrix} \right)\\
		&= \left(\begin{matrix} u_1 \\ 0 \\ u_1^2\end{matrix} \right).
	\end{align*}
	Therefore $u_1 = 0$. Hence a typical 1-cocycle that is trivial on $\langle \sigma \rangle$ satisfies
	\begin{align*}
		\alpha_u(\tau) = \left(\begin{matrix} 0 \\ 0 \\ u \end{matrix} \right),\qquad (u\in k).
	\end{align*}
	This is a necessary condition. To show it is sufficient one can apply \cite[Proposition 2]{martin2004nonab} which involves looking at the presentation of $\Gamma$.
	Now we calculate the class $[\alpha_u] \in H^1(\Gamma, Q)$. Suppose $[\alpha_v] = [\alpha_u]$. Then
	there is a $q\in Q$ fixed under the action of $\sigma$, that is of the form
	\begin{align*}
		q = \left(\begin{matrix} 0 \\ 0 \\ \lambda\end{matrix}\right),
	\end{align*}
	such that $\alpha_v(\gamma) = q\times\alpha_u(\gamma)\times\gamma\cdot q^{-1}$. In particular, for $\gamma = \tau$
	\begin{align*}
		\left(\begin{matrix} 0 \\ 0 \\ v\end{matrix}\right) &=
		\left(\begin{matrix} 0 \\ 0 \\ \lambda\end{matrix}\right) \times
		\left(\begin{matrix} 0 \\ 0 \\ u\end{matrix}\right) \times
		\tau\cdot\left(\begin{matrix} 0 \\ 0 \\ \lambda\end{matrix}\right)\\
		&=
		\left(\begin{matrix} 0 \\ 0 \\ \lambda\end{matrix}\right) \times
		\left(\begin{matrix} 0 \\ 0 \\ u\end{matrix}\right) \times
		\left(\begin{matrix} 0 \\ 0 \\ \lambda\end{matrix}\right)\\
		&=
		\left(\begin{matrix} 0 \\ 0 \\ \lambda\end{matrix}\right) \times
		\left(\begin{matrix} 0 \\ 0 \\ u + \lambda\end{matrix}\right) \\
		&=
		\left(\begin{matrix} 0 \\ 0 \\ u\end{matrix}\right).
	\end{align*}
	Hence only if $u=v$ are two 1-cocycles of the particular form in the same class, and therefore $H^1(\Gamma, Q)$ is infinite.
\end{proof}

In view of Theorem \ref{thm:k2_h1} one could show that the map $H^1(\Gamma, Q) \rightarrow H^1(\Gamma_p, Q)$ is not injective. Furthermore, it is natural to ask whether Example \ref{eg:non_red} leads to a reductive counterexample to K\"ulshammer's second question, although we can quickly verify that the answer is ``not immediately''. For suppose there was a reductive group with unipotent radical \emph{containing} the multiplication law:
\begin{align*}
	&\ldots \epsilon_\alpha(u_\alpha) \ldots \epsilon_\beta(u_\beta) \ldots \epsilon_\gamma(u_\gamma) \times
	\ldots \epsilon_\alpha(v_\alpha) \ldots \epsilon_\beta(v_\beta) \ldots \epsilon_\gamma(v_\gamma)\\
	&=
	\ldots \epsilon_\alpha(u_\alpha + v_\alpha) \ldots \epsilon_\beta(u_\beta + v_\beta) \ldots \epsilon_\gamma(u_\gamma + v_\gamma + u_\alpha v_\alpha + u_\beta v_\beta + u_\alpha v_\beta).
\end{align*}
Then setting $u_\delta = v_\delta = 0$ whenever $\delta \neq \alpha$ gives
\begin{align*}
	\epsilon_\alpha(u_\alpha) \times \epsilon_\alpha(v_\alpha) &= \epsilon_\alpha(u_\alpha + v_\alpha) \epsilon_\gamma(u_\alpha v_\alpha),
\end{align*}
which is absurd.


