%!TEX root = /Users/dan/Documents/Thesis/Thesis.tex
% Chapter 4

\chapter{K\"ulshammer's Second Problem}
\label{Chapter4}
\lhead{Chapter 4. \emph{K\"ulshammer's Second Problem}}

% \begin{itemize}
% \item K\"ulshammer's First Problem
% \item K\"ulshammer's Second Problem
% \item Counter example for non-reductive $G$
% \item Overture to Ch. 3-5
% \end{itemize}

\section{K\"ulshammer's Second Problem}

Two questions raised by B. K\"ulshammer concerning representations of a finite group $\Gamma$ into a linear algebraic group $G$ over an algebraically closed field $k$. The first has a positive answer  and is essentially contained a paper by A. Weil \cite{weil1964remarks}:
\begin{itemize}
	\item[(K. I)] Let $\mathrm{char}(k)$ be prime to the order of $\Gamma$. Are there only finitely many representations $\rho:\Gamma\rightarrow G$ up to conjugation by $G$?
	\item[(K. II)] Let $p = \mathrm{char}(k)$ and $\Gamma_p \subset  \Gamma$ be a Sylow $p$-subgroup. Fix a conjugacy class of representations from $\Gamma_p\rightarrow G$. Are there, up to conjugation by $G$, only finitely many representations $\rho:\Gamma\rightarrow G$ whose restrictions to $\Gamma_p$ belong to the given class?
\end{itemize}

(K. II) has positive answer so long as $G$ is reductive and the characteristic of $k$ is good for $G$ \cite{slodowy1997two}. The same paper shows that the answer is ``no'' in general by way of a counterexample involving a non-reductive $G$.

We will explore the possibility of a reductive counterexample to (K. II).

\section{The Approach}

We are interested in knowing whether there can be infinitely many $G$-conjugacy classes of representations $\Gamma\rightarrow G$ that when restricted to $\Gamma_p$ hit some fixed $G$-conjugacy class of representations $\Gamma_p\rightarrow G$. 

\begin{theorem} \label{finiteGCR} There are only finitely many $G$-conjugacy classes of $G$-completely reducible representations $\Gamma\rightarrow G$.
\end{theorem}
\begin{proof}[Reference something]
\end{proof}

Although $G$ has infinitely many parabolic subgroups there are only finitely many $G$-conjugacy classes of parabolic subgroups, so we can choose a finite set $\{Q_i\}$ of representatives. We choose a set of Levi subgroups $\{M_i\}$, $M_i$ being a Levi subgroup of $Q_i$. By Theorem \ref{finiteGCR} there are only finitely many $M_i$-conjugacy classes of $M_i$-completely reducible representations $\sigma^{(i)}_0:\Gamma\rightarrow M_i$, so we fix a set of representatives $\{\sigma^{(i)}_{0,j}\}$.

We will show that for each representation $\rho:\Gamma\rightarrow G$ there exists a representation $\sigma$ that is $G$-conjugate to $\rho$ and that fits one of only finitely many commutative diagrams
\begin{displaymath}
	\xymatrix{
	\Gamma \ar[r]^{\sigma} \ar[rd]_{\sigma^{(i)}_{0,j}} & Q_i \ar[d]\\
	& M_i}
\end{displaymath}

Let $\rho$ be a representation from $\Gamma\rightarrow G$ and let $P$ be a minimal parabolic subgroup of $G$ containing $\rho(\Gamma)$. Then there is a $g$ in $G$ such that $P = g\cdot Q$, where $Q\in\{Q_i\}$. Let $\rho' = g\cdot \rho$. 

Define $\rho_0:\Gamma\rightarrow M$ by composing $\rho'$ with the projection $Q\rightarrow M$, $M\in\{M_i\}$ the chosen Levi subgroup for $Q$:
\begin{displaymath}
	\xymatrix{
	\Gamma \ar[r]^{\rho'} \ar[rd]_{\rho_0} & Q \ar[d]\\
	& M}
\end{displaymath}

Since $Q$ is a minimal parabolic containing $\rho'(\Gamma)$, $\rho_0$ is $M$-irreducible [reference] and therefore $M$-completely reducible. Hence we can choose an $m\in M$ such that $\sigma_0 = m\cdot \rho_0$ where $\sigma_0\in\{\sigma^{(i)}_{0,j}\}$. Let $\sigma = m\cdot \rho' = mg\cdot \rho$. This verifies what we set out to show.

[Say something about $\rho \rightsquigarrow (i,j) \rightsquigarrow H^1(\Gamma,V_i)$ so that I can state the next lemma]

For a given parabolic $P<G$, Levi $L<P$ and representation $\rho:\Gamma\rightarrow P$ we have defined the map $\rho_0:\Gamma\rightarrow L$ by the projection $P\rightarrow L$. Now define $\alpha_\rho:\Gamma\rightarrow R_u(P)$ by projecting on to $R_u(P)$, so that $\rho = \alpha_\rho\rho_0$. If $\rho_0$ is a homomorphism then
\begin{eqnarray*}
	\alpha_\rho(\gamma_1\gamma_2)\rho_0(\gamma_1\gamma_2) \,=\, \rho(\gamma_1\gamma_2) 
		&=& \rho(\gamma_1)\rho(\gamma_2) \\
		&=& \alpha_\rho(\gamma_1)\rho_0(\gamma_1)\alpha_\rho(\gamma_2)\rho_0(\gamma_2) \\
		&=& \alpha_\rho(\gamma_1)\rho_0(\gamma_1)\alpha_\rho(\gamma_2)\rho_0(\gamma_1)^{-1}\rho_0(\gamma_1)\rho_0(\gamma_2)\\
		&=&\alpha_\rho(\gamma_1)\rho_0(\gamma_1)\alpha_\rho(\gamma_2)\rho_0(\gamma_1)^{-1}\rho_0(\gamma_1\gamma_2),
\end{eqnarray*}
so that
\begin{eqnarray*}
	\alpha_\rho(\gamma_1\gamma_2) =
	\alpha_\rho(\gamma_1)\rho_0(\gamma_1)\cdot\alpha_\rho(\gamma_2),
\end{eqnarray*}
which is the 1-cocycle condition in (\ref{theOneCocycleCondition}).

Hence for the given $\rho$ we have a corresponding 1-cocycle $\alpha_\rho:\Gamma\rightarrow R_u(P)$ where $\Gamma$ acts on $R_u(P)$ via $\rho_0$.

Suppose we conjugate $\rho$ by an element $g\in G$. Then $\rho'=g\cdot\rho$ has corresponding 1-cocycle $\alpha_{\rho'}:\Gamma\rightarrow gR_u(P)g^{-1}$.

So the $G$-action on $\rho:\Gamma\rightarrow P$ almost corresponds to maps of 1-cocycles of the form:
\begin{displaymath}
	Z^1(\Gamma, R_u(P))\rightarrow Z^1(\Gamma, gR_u(P)g^{-1}),
\end{displaymath}
the catch being that conjugating $\rho$ by an arbitrary element of $g$ changes the action of $\Gamma$ on $R_u(P)$. If we choose $g\in Z(L)^\circ$ and consider the $Z(L)^\circ$-action:
\begin{displaymath}
	\xymatrix{
		\Gamma \ar[r]^{\rho} \ar[rd]_{\rho_0} & P \ar[d]\\
		& L} \qquad\mapsto\qquad
	\xymatrix{
		\Gamma \ar[r]^{g\cdot\rho} \ar[rd]_{\rho_0} & gPg^{-1} \ar[d]\\
		& L}
\end{displaymath}
then we really do get a map of 1-cocycles. 


\begin{lemma}\label{kToHOne} Let $\{\rho_\mu\, |\, \mu \in I\}$ be a collection of representations $\Gamma\rightarrow P$ for a fixed parabolic subgroup $P<G$. The following statements are equivalent:
	\begin{itemize}
		\item[(i)] There are only finitely many $P$-conjugacy classes of $\{\rho_\mu\}$.
		\item[(ii)] There are only finitely many $G$-conjugacy classes of $\{\rho_\mu\}$.
		\item[(iii)] For each $i$, the number of elements of $H^1(\Gamma, V_i)$ modulo $Z(M_i^\circ)$ that come from the $\rho_\mu$'s is finite.
	\end{itemize}
\end{lemma}
\begin{proof}
	(i) $\Rightarrow$ (ii) is obvious. 
\end{proof}

Let $K$ be a collection of representations $\Gamma \rightarrow G$ whose restrictions to $\Gamma_p$ belong to some fixed class. Let $\sigma\in K$, then $\sigma|_{\Gamma_p}$ and $\sigma_{0,j}^{(i)}|_{\Gamma_p}$ are conjugate. We have a 1-cocycle $\alpha$ corresponding to $\sigma$, while $\sigma_{0,j}^{(i)}$ corresponds to the trivial 1-cocycle. Hence $\alpha|_{\Gamma_p}$ is conjugate to the trivial 1-cocycle, that is, $\alpha|_{\Gamma_p}$ is a 1-coboundary. 

% Suppose we are given a representation $\rho:\Gamma\rightarrow G$. Let $P<G$ be the minimal parabolic subgroup of $G$ containing $\rho(\Gamma)$ and let $g$ be the element of $G$ such that $P=g\cdot Q$ where $Q$ is one of the fixed parabolics $\{Q_i\}$. Let $\rho'=g\cdot\rho$.
% 
% Choose a Levi subgroup $L<Q$, so that $Q = R_u(Q)\rtimes L$. 
% 
% Since $Q$ is the minimal parabolic containing $\rho'(\Gamma)$, $\rho_0$ must be $L$-irreducible. For if $\rho_0(\Gamma)$ is contained in a proper parabolic subgroup of $L$ \ldots
% 
% Hence $\rho_0$ is also $L$-completely reducible. By [Theorem] there are only finitely many $L$-conjugacy classes of the given $\rho_0$.
% 
% So for each representation $\Gamma\rightarrow G$ we can choose a $Q$ from a finite set of parabolic subgroups of $G$, with corresponding Levi subgroup $L$, and $\rho_0':\Gamma\rightarrow G$ such that the $G$-conjugacy class containing the given representation has representative $\rho'$ \ldots

[Therefore the problem is controlled by the 1-cohomology]

\section{An algebraic group version}

In an attempt to gain further insight into (K. II) we adjust the original question by letting $\Gamma$ be an infinite group $H$. The advantage being that a negative answer in the algebraic group version may provide a negative answer to (K. II) by choosing an appropriate finite subgroup $\Gamma$ of $H$. In many of the examples to follow we set $H = SL_2(K)$ with Sylow $p$-subgroup $H_p = U_2(K)$ consisting of upper unitriangular matrices. 

Let $P \subset G$ be a parabolic subgroup and $L \subset P$ the corresponding Levi subgroup. Fix a representation $\rho_0:H\rightarrow L$. We can assume $\rho_0(H)$ is $L$-irreducible, that is, not contained in a proper parabolic of $L$. 

Now define $\rho_\alpha:H \rightarrow P$ by $\rho_\alpha(h) = \alpha(h)\rho_0(h)$ where $\alpha:H\rightarrow R_u(P)$, $R_u(P)$ the unipotent radical of $P$. 

For $\rho_\alpha$ to be a homomorphism
\begin{eqnarray*}
	\alpha(h_1h_2)\rho_0(h_1h_2) &=& \alpha(h_1)\rho_0(h_1)\alpha(h_2)\rho_0(h_2) \\
	 &=& \alpha(h_1)\rho_0(h_1)\alpha(h_2)\rho_0(h_1)^{-1}\rho_0(h_1)\rho_0(h_2) \\
	 &=& \alpha(h_1)\rho_0(h_1)\alpha(h_2)\rho_0(h_1)^{-1}\rho_0(h_1h_2).
\end{eqnarray*}
That is $\alpha(h_1h_2) = \alpha(h_1) h_1\cdot\alpha(h_2)$, where the action $H \times R_u(P) \rightarrow R_u(P)$ is conjugation via $\rho_0$. This is a 1-cocycle condtion; $\alpha\in Z^1(H, R_u(P))$. $R_u(P)$ will not be abelian in general.

Now suppose $\rho_\alpha$ is $R_u(P)$-conjugate to some $\rho_\beta$, $\alpha, \beta \in Z^1(H, R_u(P))$. That is, there exists a $v \in R_u(P)$ such that for all $h\in H$
\begin{eqnarray*}
	\alpha(h)\rho_0(h) &=& v\beta(h)\rho_0(h)v^{-1} \\
	&=& v\beta(h)\rho_0(h)v^{-1}\rho_0(h)^{-1}\rho_0(h).
\end{eqnarray*}
That is $\alpha(h) = v\beta(h)h\cdot v^{-1}$. In particular if $\rho_\alpha$ is $R_u(P)$-conjugate to $\rho_0$, that is $\beta$ is trivial, then $\alpha$ takes the form of a 1-coboundary. Generally speaking $\alpha$ and $\beta$ project to the same 1-cohomology class. In the abelian case this reads ``$\alpha$ and $\beta$ differ by a 1-coboundary'':
\begin{eqnarray*}
	\alpha(h) = v\beta(h)h\cdot v^{-1} \quad\leadsto \quad\alpha(h) &=& v + \beta(h) - h\cdot v \\
	&=& \beta(h) + v - h\cdot v \\
	&=& \beta(h) + \chi_v(h).
\end{eqnarray*}

% 3) Look at the nonreductive counterexample in Slodowy's paper on Kulshammer's problem.  What is special about the 3-dimensional U that makes this counterexample work?  Can you find similar structure in the unipotent radical of a reductive group?
