%!TEX root = /Users/dan/Documents/Thesis/Thesis.tex
% Chapter 4

\chapter{The 1-Cohomology}
\label{Chapter4}
\lhead{Chapter 4. \emph{The 1-Cohomology}}

\section{Abelian 1-Cohomology}
	
\subsection{Definitions}
Let $H$ be a group and $V$ an abelian group (vector space) on which $H$ acts homomorphically (linearly). We call a map $\sigma$ from $H\rightarrow V$ a  \emph{1-cocycle} if it satisfies
\begin{eqnarray}\label{ch4::theOneCocycleCondition}
	\sigma(h_1h_2) = \sigma(h_1) + h_1\cdot\sigma(h_2),
\end{eqnarray}
for all $h_1, h_2$ in $H$. Denote by $Z^1\left( H, V \right)$ the collection of all 1-cocycles from $H\rightarrow V$.

We call the (\ref{ch4::theOneCocycleCondition}) the \emph{1-cocycle condition}.

For any $\sigma_1, \sigma_2$ in $Z^1\left(H, V\right)$
\begin{eqnarray*}
	\left(\sigma_1 + \sigma_2\right)(h_1h_2) &=& \sigma_1(h_1h_2) +  \sigma_2(h_1h_2) \\
	&=& \sigma_1(h_1) + h_1\cdot\sigma_1(h_2) +  \sigma_2(h_1) + h_1\cdot\sigma_2(h_2)\\
	&=& \left( \sigma_1(h_1) + \sigma_2(h_1) \right) + h_1\cdot\left(\sigma_1(h_2) + \sigma_2(h_2)\right) \\
	&=& \left(\sigma_1+\sigma_2\right)(h_1) + h_1\cdot\left(\sigma_1 + \sigma_2\right)(h_2),
\end{eqnarray*}
so $Z^1(H, V)$ is closed under pointwise addition.

The trivial map from $H \rightarrow V$ that sends every $h$ in $H$ to the identity 0 in $V$ is a 1-cocycle. Furthermore for any $\sigma$ in $Z^1(H, V)$ we have
\begin{eqnarray*}
	\sigma(1)\, =\, \sigma(1\cdot 1) &=& \sigma(1) + 1\cdot \sigma(1) \\
	&=& \sigma(1) + \sigma(1) \\
	&=& 2\,\sigma(1),
\end{eqnarray*}
which implies that
\begin{eqnarray*}
\sigma(1) = 0.
\end{eqnarray*}
From this we deduce that
\begin{eqnarray*}
	\sigma(hh^{-1})\, =\, \sigma(1) &=& 0 \\
	&=& \sigma(h) + h\cdot \sigma(h^{-1}),
\end{eqnarray*}
and so each $\sigma$ has an inverse defined by
\begin{eqnarray*}
	-\sigma(h) = h\cdot\sigma(h^{-1}).
\end{eqnarray*}
Therefore $Z^1\left(H, V\right)$ is a $\mathbb{Z}-$module under pointwise addition.

Given a $v$ in $V$ we define a \emph{1-coboundary} $\chi^H_v:H\rightarrow V$ to be
\begin{eqnarray*}
	\chi^H_v (h) = v - h\cdot v,
\end{eqnarray*}
and denote by $B^1\left(H, V\right)$ the collection of all 1-coboundaries. 

For any $v$ in $V$ and any $h_1, h_2$ in $H$
\begin{eqnarray*}
	\chi^H_v(h_1h_2) &=& v - (h_1h_2)\cdot v \\
	&=& v - h_1 \cdot \left(h_2\cdot v \right)\\
	&=& v - h_1 \cdot \left(v -v + h_2\cdot v \right)\\
	&=& v - h_1\cdot v + h_1\cdot \left( v - h_2\cdot v\right)\\
	&=& \chi^H_v(h_1) + h_1\cdot \chi^H_v(h_2),
\end{eqnarray*}
so every 1-coboundary is also a 1-cocycle. 

For any $u,v$ in $V$ and all $h$ in $H$
\begin{eqnarray*}
	(\chi^H_u + \chi^H_v)(h) &=& \chi^H_u(h) + \chi^H_v(h)\\
	&=& u - h\cdot u + v - h\cdot v \\
	&=& (u + v) - h\cdot (u + v) \\
	&=& \chi^H_{u + v} (h)
\end{eqnarray*}
is a 1-coboundary, and hence $B^1\left(H, V\right)$ is also closed under pointwise addition.

We see that $B^1(H, V)$ is a subgroup of $Z^1(H, V)$ via the two-step subgroup test. In fact it is easy to show that $B^1(H, V)$ is a $\mathbb{Z}-$submodule of $Z^1(H, V)$, so we may form the quotient module
\begin{eqnarray*}
	H^1\left(H, V\right) = Z^1\left(H, V\right) / B^1\left(H, V\right),
\end{eqnarray*}
called the \emph{1-cohomology}.
\begin{lemma} Suppose $H$ is linearly reductive. Then $H^1(H, V)$ is trivial \cite{hochschild1965structure}.
\end{lemma}

\subsection{Maps between 1-cohomologies}
Let $\phi$ be a homomorphism from $\tilde{H}\rightarrow H$, $\tilde{H}$ being another group that acts on $V$. Suppose that for every $h$ in $H$, $\phi$ satisfies
\begin{eqnarray*}
	\phi(h)\cdot v = h\cdot v,
\end{eqnarray*}
for all $v$ in $V$. If $\sigma$ is a 1-cocycle from $H\rightarrow V$ then we will show that the map denoted $Z^1(\phi)(\sigma)$ defined by
\begin{eqnarray*}
	Z^1(\phi)(\sigma) = \sigma \circ \phi,
\end{eqnarray*}
is a 1-cocycle from $\tilde{H}\rightarrow V$.

Take $h_1, h_2$ in $H$. We have
\begin{eqnarray*}
	Z^1(\phi)(\sigma)(h_1h_2) &=& (\sigma \circ \phi)(h_1h_2) \\
		&=& \sigma(\phi(h_1h_2)) \\
		&=& \sigma(\phi(h_1)\phi(h_2)) \\
		&=& \sigma(\phi(h_1)) + \phi(h_1)\cdot\sigma(\phi(h_2) \\
		&=& \sigma(\phi(h_1)) + h_1\cdot\sigma(\phi(h_2)) \\
		&=& (\sigma \circ \phi)(h_1) + (\sigma \circ \phi)(h_2) \\
		&=& Z^1(\phi)(\sigma)(h_1) + h_1\cdot Z^1(\phi)(\sigma)(h_2).
\end{eqnarray*}

Moreover, it can be shown that $Z^1(\phi)$ maps $B^1(H, V)$ into $B^1(\tilde{H}, V)$. This leads us to define a map of 1-cohomologies,
\begin{eqnarray*}
	H^1(\phi):H^1(H, V) \rightarrow H^1(\tilde{H}, V),
\end{eqnarray*}
defined by
\begin{displaymath}
	\xymatrix{
	Z^1(H, V) \ar[d]_{\pi} \ar[r]^{Z^1(\phi)} & Z^1(H, V) \ar[d]^{\pi}\\
	H^1(H, V) \ar[r]^{H^1(\phi)} & H^1(\tilde{H}, V)
	}
% \begin{CD}
	% Z^1(H, V) @>Z^1(\phi)>> Z^1(\tilde{H}, V)\\
	% @V{\pi}VV                                  @VV{\tilde{\pi}}V\\
	% H^1(H, V) @>>H^1(\phi)> H^1(\tilde{H}, V)
% \end{CD}
\end{displaymath}
where $\pi$ and $\tilde\pi$ are the respective canonical projections of $Z^1(H, V)$ onto $H^1(H, V)$ and $Z^1(\tilde{H}, V)$ onto $H^1(\tilde{H}, V)$. To show that the map $H^1(\phi)$ is well-defined it is sufficient to notice that $Z^1(\phi)$ is a homomorphism.

\begin{example}
Let $\tilde{H}$ be a subgroup of $H$ and $i:\tilde{H}\rightarrow H$ the inclusion map. Then $i$ gives rise to a well defined map
\begin{eqnarray*}
H^1(i):H^1(H, V)\rightarrow H^1(\tilde{H}, V).
\end{eqnarray*}
\end{example}

\begin{lemma}\label{ch4::mapFromSylow}
Let $H$ be a finite group and $\tilde{H} = H_p$ a \emph{Sylow $p$-subgroup} of $H$. If $V$ is a vector space then the map 
\begin{eqnarray*}
H^1(i):H^1(H, V)\rightarrow H^1(H_p, V)
\end{eqnarray*}
is injective.
\end{lemma}
\begin{proof}
Let $x$ be an element of $H^1(H, V)$ such that $H^1(i)(x) = 0$. Now choose a 1-cocycle $\sigma$ in $Z^1(H, V)$ such that $\pi(\sigma) = x$. Hence $Z^1(i)(\sigma)$ is a 1-coboundary as its image under $\tilde\pi$ is 0. That is to say $\sigma$ restricted to $H_p$ is equal to a 1-coboundary, say $\chi_v^{H_p}$. But since $\chi_v^{H_p}$ can be trivially extended to a 1-coboundary $\chi_v^H$ from $H\rightarrow V$, and
\begin{eqnarray*}
	\pi(\sigma - \chi_v^H) = x,
\end{eqnarray*}
we could well have chosen the 1-cocycle $(\sigma - \chi_v^H)$ as a representative for $x$. Hence there is no harm in assuming that $\sigma$ is 0 when restricted to $H_p$.
Now choose a set of representatives $h_1, \ldots, h_l$ in $H$ for the coset space $H/H_p$ and set
\begin{eqnarray*}
	v^* = \sum_{i =1}^l \sigma(h_i).
\end{eqnarray*}
Consider the 1-coboundary $\chi_{v^*}^H$ defined by $v^*$
\begin{eqnarray*}
	\chi_{v^*}^H(h) &=& v^* - h\cdot v^* \\
	&=& \sum_{i = 1}^l\sigma(h_i) - h\cdot \sum_{i = 1}^l\sigma(h_i) \\
	&=& \sum_{i = 1}^l\sigma(h_i) - \sum_{i = 1}^l h\cdot \sigma(h_i).
\end{eqnarray*}
By the 1-cocycle condition we have
\begin{eqnarray*}
	\sigma(h h_i) = \sigma(h) + h\cdot\sigma(h_i),
\end{eqnarray*}
from which we obtain
\begin{eqnarray*}
	 \sum_{i = 1}^l\sigma(h_i) - \sum_{i = 1}^l h\cdot \sigma(h_i) &=& \sum_{i = 1}^l\sigma(h_i) - \sum_{i = 1}^l \left(\sigma(hh_i) - \sigma(h) \right)\\
	 &=& \sum_{i = 1}^l\sigma(h_i) - \sum_{i = 1}^l \sigma(hh_i) +\sum_{i = 1}^l \sigma(h).
\end{eqnarray*}
Now as the value of $\sigma$ at a fixed $h$ depends only on the value of $\sigma$ at the representative $h_j$ of the coset containing $h$ we can collapse the middle term to yield
\begin{eqnarray*}
	\chi_{v^*}^H(h) &=& \sum_{i = 1}^l\sigma(h_i) - \sum_{i = 1}^l \sigma(hh_i) +\sum_{i = 1}^l \sigma(h)\\
	&=& \sum_{i = 1}^l\sigma(h_i) - \sum_{i = 1}^l \sigma(h_i) +\sum_{i = 1}^l \sigma(h) \\
	&=& l\, \sigma(h).
\end{eqnarray*}
Since $\gcd([H:H_p], p) = \gcd(l,p) = 1$, $l$ is invertible and so
\begin{eqnarray*}
	l^{-1}\chi_{v^*}^H(h) = \sigma(h).
\end{eqnarray*}
Therefore $\sigma$ is a 1-coboundary and so the kernel of $H(i)$ is trivial.
\end{proof}

\begin{example}
	Let
	\begin{displaymath}
		k = \bar{\mathbb{F}_p} = \bigcup_r \mathbb{F}_{p^r},
	\end{displaymath}
	$V$ a vector space on which $SL_2(k)$ acts, and $U(k)$ the subgroup of $SL_2(k)$ consisting of upper unitriangular matrices. Then $U(\mathbb{F}_{p^r})$ is a Sylow $p$-subgroup of $SL_2(\mathbb{F}_{p^r})$ for each $r$, and the map
	\begin{displaymath}
		H^1(SL_2(k), V) \rightarrow H^1(U(k), V)
	\end{displaymath}
	is injective.
\end{example}
\begin{proof}
	The group $GL_2(\mathbb{F}_{p^r})$ has order $(p^{2r} - 1)(p^{2r} - p^r)$ since there are $p^{2r} - 1$ choices of vectors for the first column (all choices excluding the zero vector), and $p^{2r} - p^r$ choices of vectors for the second column (all choices excluding multiples of the first vector). The determinant is a homomorphism of groups
	\begin{displaymath}
		\mathrm{det}:GL_2(\mathbb{F}_{p^r}) \rightarrow \mathbb{F}^*_{p^r},
	\end{displaymath}
	with kernel $SL_2(\mathbb{F}_{p^r})$. Therefore, by the First homomorphism theorem for groups
	\begin{displaymath}
		GL_2(\mathbb{F}_{p^r})\,/\,SL_2(\mathbb{F}_{p^r}) \sim \mathrm{det}(GL_2(\mathbb{F}_{p^r})) = \mathbb{F}^*_{p^r},
	\end{displaymath}
	and so
	\begin{eqnarray*}
		|SL_2(\mathbb{F}_{p^r})|
		&=& |GL_2(\mathbb{F}_{p^r})|\,/\,|\mathbb{F}^*_{p^r}|\\
		&=& (p^{2r} - 1)(p^{2r} - p^r)\,/\,(p^r - 1)\\
		&=& p^r(p^{2r} - 1).
	\end{eqnarray*}
	Since $|U(\mathbb{F}_{p^r})| = p^r$, $U(\mathbb{F}_{p^r})$ is a Sylow $p$-subgroup of $SL_2(\mathbb{F}_{p^r})$.
	
	Fix a non-trivial $y\in H^1(SL_2(k), V)$ and choose a representative $\tau\in Z^1(SL_2(k), V)$ for $y$. For each $g\in SL_2(\mathbb{F}_{p^r})$ define the morphism $f^{(r)}_g:V\rightarrow V$ by
	\begin{displaymath}
		f^{(r)}_g(v) = \tau(g) - \chi_v(g) = \tau(g) - v + g\cdot v.
	\end{displaymath}
	Consider the sequence of subsets of $V$ defined by
	\begin{displaymath}
		C_r = \{v \in V | f^{(r)}_g(v) = 0\}.
	\end{displaymath}
	Each subset $C_r$ is closed and the inclusion $\mathbb{F}_{p^r} \subset \mathbb{F}_{p^{r+1}}$ induces the reverse inclusion $C_r \supset C_{r+1}$. The Noetherian property for $V$ requires that the sequence becomes constant. However, $y\neq 0$ so $\tau$ is not a 1-coboundary on $SL_2(k)$, which means the $C_r$'s are eventually empty. That is, there exists an integer $s$ such that for any $v$ in $V$
	\begin{displaymath}
		(\tau - \chi_v)|_{SL_2(\mathbb{F}_{p^s})} \neq 0.
	\end{displaymath}
	Equivalently, if $y|_{SL_2(\mathbb{F}_{p^r})} = 0$ for all $r$ then $y = 0$.
	
	Take $x$ in the kernel of the map $H^1(SL_2(k), V) \rightarrow H^1(U(k), V)$. Then for each $r$, $x|_{U(\mathbb{F}_{p^r})} = 0$ so by (\ref{ch4::mapFromSylow}) $x|_{SL_2(\mathbb{F}_{p^r})} = 0$. Therefore $x=0$ and so $H^1(SL_2(k), V) \rightarrow H^1(U(k), V)$ is injective.
\end{proof}

We could also consider appropriate maps $f:V\rightarrow\tilde{V}$ and following a similar chain of arguments as before we can define
\begin{eqnarray*}
	H^1(f):H^1(H, V)\rightarrow H^1(H, \tilde{V}),
\end{eqnarray*}
or even	
\begin{eqnarray*}
	H^1(\phi, f):H^1(H, V)\rightarrow H^1(\tilde{H}, \tilde{V}).
\end{eqnarray*}

\section{Non-abelian 1-Cohomology}
	
\subsection{The non-abelian setting}

We will be interested in $H$, $V$ algebraic groups, where we require that 1-cocyles be morphisms of varieties. 

% Let us see why this is an interesting pursuit.
% 
% Let $H,G$ be algebraic groups, $P$ a parabolic subgroup of $G$, and $L$ a Levi subgroup of $P$. Let $\rho: H \rightarrow L$ be a morphism.  
% 
% Suppose $\rho_\alpha : H \rightarrow P$ is of the form $\rho_\alpha(h) = \alpha(h)\rho(h)$, where $\alpha:H \rightarrow R_u(P)$.
% 
% \begin{example}
% What properties must $\alpha$ satisfy for $\rho_\alpha$ to be a homomorphism? 
% \end{example}
% Since
% \begin{eqnarray*}
% \alpha(gh)\rho(gh) = \rho_\alpha(gh) &=& \rho_\alpha(g)\rho_\alpha(h) \\
% 	&=& \alpha(g)\rho(g)\alpha(h)\rho(h) \\
% 	&=& \alpha(g)\rho(g)\alpha(h)\rho(g)^{-1}\rho(g)\rho(h)\\
% 	&=&\alpha(g)\rho(g)\alpha(h)\rho(g)^{-1}\rho(gh),
% \end{eqnarray*}
% we have
% \begin{eqnarray*}
% \alpha(gh) = \alpha(g)\rho(g)\alpha(h)\rho(g)^{-1}.,
% \end{eqnarray*}
% or simply
% \begin{eqnarray}\label{nonab1c}
% \alpha(gh)=\alpha(g)*g\cdot\alpha(h),
% \end{eqnarray}
% where the action of $H$ on $R_u(P)$ is defined by $\rho$, and $*:R_u(P)\times R_u(P)\rightarrow R_u(P)$.
% 
% We call (\ref{nonab1c}) the 1-cocycle condition in our non-abelian setting and refer to morphisms $H\rightarrow V$ between algebraic groups that satisfies the 1-cocycle condition as 1-cocycles. We denote by $Z^1(H,V)$ the set of all 1-cocycles from $H$ into $V$.
% 
% \begin{example}
% When is $\rho$ $R_u(P)$-conjugate to $\rho_\alpha$?
% \end{example}
% Suppose there exists a $v\in R_u(P)$ such that $\rho_\alpha(h) = v\rho(h)v^{-1}$ for all $h\in H$. Then
% \begin{eqnarray*}
% 	\alpha(h)\rho(h) = \rho_\alpha(h) &=& v\rho(h)v^{-1}\\
% 	&=& v\rho(h)v^{-1}\rho(h)^{-1}\rho(h).
% \end{eqnarray*}
% Therefore, $\alpha$ is of the form
% \begin{eqnarray*}
% 	\alpha(h) = v*h\cdot v^{-1}.
% \end{eqnarray*}
% 
% So for a fixed $v\in V$, we define a 1-coboundary in our non-abelian setting to be a morphism $\chi_v: H \rightarrow V$ of the form
% \begin{eqnarray}
% \chi_v(h) = v*h\cdot v^{-1}, 
% \end{eqnarray}
% and denote the collection of all 1-coboundaries from $H$ into $V$ by $B^1(H,V)$. Indeed,
% \begin{eqnarray*}
% 	\chi_v(gh) &=& v\rho(gh)v^{-1}\rho(gh)^{-1} \\
% 	&=& v\rho(g)\rho(h)v^{-1}\rho(h)^{-1}\rho(g)^{-1} \\
% 	&=&  v\rho(g)\left[v^{-1}\rho(g)^{-1}\rho(g)v\right]\rho(h)v^{-1}\rho(h)^{-1}\rho(g)^{-1} \\
% 	&=& \left[v\rho(g)v^{-1}\rho(g)^{-1}\right]\left[\rho(g)v\rho(h)v^{-1}\rho(h)^{-1}\rho(g)^{-1}\right] \\
% 	&=& \left[v*g\cdot v^{-1}\right]*g\cdot \left[v*h\cdot v^{-1}\right] \\
% 	&=& \chi_v(g) * g\cdot \chi_v(h),
% \end{eqnarray*}
% so that $B^1(H,V)\subset Z^1(H,V)$.
% 
% \begin{example}
% When is $\rho_\alpha$ $R_u(P)$-conjugate to $\rho_\beta$?
% \end{example}
% Let $\alpha,\beta\in Z^1(H,R_u(P))$ and suppose there exists a $v\in R_u(P)$ such that $\rho_\beta(h) = v\rho_\alpha(h)v^{-1}$. Then
% \begin{eqnarray*}\label{nonabeq}
% 	\beta(h)\rho(h) &=& v\alpha(h)\rho(h)v^{-1} \\
% 	&=&v\alpha(h)\rho(h)v^{-1}\rho(h)^{-1}\rho(h),
% \end{eqnarray*}
% that is
% \begin{eqnarray}
% 	\beta(h) = v\alpha(h)*h\cdot v^{-1}.
% \end{eqnarray}
% The relation in (\ref{nonabeq}) gives rise to an equivalence relation on $Z^1(H,V)$. Now we define the (non-abelian) 1-cohomology, denoted by $H^1(H,V)$, to be the set of equivalence classes of $Z^1(H,V)$.

\subsection{Definitions}
Let $H, V$ be algebraic groups, $H$ acting on $V$. We call a map $\sigma$ from $H\rightarrow V$ a  \emph{1-cocycle} if it satisfies
\begin{eqnarray}\label{ch4::theNonabOneCocycleCondition}
	\sigma(h_1h_2) = \sigma(h_1) * h_1\cdot\sigma(h_2),
\end{eqnarray}
for all $h_1, h_2$ in $H$. Denote by $Z^1\left( H, V \right)$ the collection of all 1-cocycles from $H\rightarrow V$.

We call the (\ref{ch4::theNonabOneCocycleCondition}) the \emph{1-cocycle condition}.

% For any $\sigma_1, \sigma_2$ in $Z^1\left(H, V\right)$
% \begin{eqnarray*}
% 	\left(\sigma_1 + \sigma_2\right)(h_1h_2) &=& \sigma_1(h_1h_2) +  \sigma_2(h_1h_2) \\
% 	&=& \sigma_1(h_1) + h_1\cdot\sigma_1(h_2) +  \sigma_2(h_1) + h_1\cdot\sigma_2(h_2)\\
% 	&=& \left( \sigma_1(h_1) + \sigma_2(h_1) \right) + h_1\cdot\left(\sigma_1(h_2) + \sigma_2(h_2)\right) \\
% 	&=& \left(\sigma_1+\sigma_2\right)(h_1) + h_1\cdot\left(\sigma_1 + \sigma_2\right)(h_2),
% \end{eqnarray*}
% so $Z^1(H, V)$ is closed under pointwise addition.

% The trivial map from $H \rightarrow V$ that sends every $h$ in $H$ to the identity 1 in $V$ is a 1-cocycle. Furthermore for any $\sigma$ in $Z^1(H, V)$ we have
% \begin{eqnarray*}
% 	\sigma(1)\, =\, \sigma(1\cdot 1) &=& \sigma(1) + 1\cdot \sigma(1) \\
% 	&=& \sigma(1) + \sigma(1) \\
% 	&=& 2\,\sigma(1),
% \end{eqnarray*}
% which implies that
% \begin{eqnarray*}
% \sigma(1) = 0.
% \end{eqnarray*}
% From this we deduce that
% \begin{eqnarray*}
% 	\sigma(hh^{-1})\, =\, \sigma(1) &=& 0 \\
% 	&=& \sigma(h) + h\cdot \sigma(h^{-1}),
% \end{eqnarray*}
% and so each $\sigma$ has an inverse defined by
% \begin{eqnarray*}
% 	-\sigma(h) = h\cdot\sigma(h^{-1}).
% \end{eqnarray*}
% Therefore $Z^1\left(H, V\right)$ is a $\mathbb{Z}-$module under pointwise addition.

Given a $v$ in $V$ we define a \emph{1-coboundary} $\chi^H_v:H\rightarrow V$ to be
\begin{eqnarray*}
	\chi^H_v (h) = v * h\cdot v^{-1},
\end{eqnarray*}
and denote by $B^1\left(H, V\right)$ the collection of all 1-coboundaries. 

For any $v$ in $V$ and any $h_1, h_2$ in $H$
\begin{eqnarray*}
	\chi^H_v(h_1h_2) &=& v * (h_1h_2)\cdot v^{-1} \\
	&=& v * h_1 \cdot \left(h_2\cdot v^{-1} \right)\\
	&=& v * h_1 \cdot \left(v v^{-1} h_2\cdot v \right)\\
	&=& v * h_1\cdot v * h_1\cdot \left( v * h_2\cdot v^{-1}\right)\\
	&=& \chi^H_v(h_1) * h_1\cdot \chi^H_v(h_2),
\end{eqnarray*}
so every 1-coboundary is also a 1-cocycle. 

% For any $u,v$ in $V$ and all $h$ in $H$
% \begin{eqnarray*}
% 	(\chi^H_u + \chi^H_v)(h) &=& \chi^H_u(h) + \chi^H_v(h)\\
% 	&=& u - h\cdot u + v - h\cdot v \\
% 	&=& (u + v) - h\cdot (u + v) \\
% 	&=& \chi^H_{u + v} (h)
% \end{eqnarray*}
% is a 1-coboundary, and hence $B^1\left(H, V\right)$ is also closed under pointwise addition.

% Setting $v = -u$ in the above calculation provides the definition of an inverse of a 1-coboundary and hence shows that $B^1(H, V)$ is a subgroup of $Z^1(H, V)$ via the two-step subgroup test. In fact it is easy to show that $B^1(H, V)$ is a $\mathbb{Z}-$submodule of $Z^1(H, V)$, so we may form the quotient module
% \begin{eqnarray*}
% 	H^1\left(H, V\right) = Z^1\left(H, V\right) / B^1\left(H, V\right),
% \end{eqnarray*}
% called the \emph{1-cohomology}.
% \begin{lemma} Suppose $H$ is linearly reductive. Then $H^1(H, V) = 0$. [Hochschild]
% \end{lemma}

We say $\sigma_1, \sigma_2$ in $Z^1(H, V)$ are \emph{equivalent} if there exists a $v$ in $V$ such that
\begin{eqnarray}\label{ch4::equivalentOneCocycles}
	\sigma_1(h) = v * \sigma_2(h) * h\cdot v^{-1},
\end{eqnarray}
for all $h$ in $H$. We call the set of equivalence classes of $Z^1(H, V)$ under the equivalence relation defined by (\ref{ch4::equivalentOneCocycles}) the \emph{1-cohomology}, denoted $H^1(H, V)$.

\subsection{Maps between 1-cohomologies}
% Let $\phi$ be a homomorphism from $\tilde{H}\rightarrow H$, $\tilde{H}$ being another group that acts on $V$. Suppose that for every $h$ in $H$, $\phi$ satisfies
% \begin{eqnarray*}
% 	\phi(h)\cdot v = h\cdot v,
% \end{eqnarray*}
% for all $v$ in $V$. If $\sigma$ is a 1-cocycle from $H\rightarrow V$ then we will show that the map denoted $Z^1(\phi)(\sigma)$ defined by
% \begin{eqnarray*}
% 	Z^1(\phi)(\sigma) = \sigma \circ \phi,
% \end{eqnarray*}
% is a 1-cocycle from $\tilde{H}\rightarrow V$.
% 
% Take $h_1, h_2$ in $H$. We have
% \begin{eqnarray*}
% 	Z^1(\phi)(\sigma)(h_1h_2) &=& (\sigma \circ \phi)(h_1h_2) \\
% 		&=& \sigma(\phi(h_1h_2)) \\
% 		&=& \sigma(\phi(h_1)\phi(h_2)) \\
% 		&=& \sigma(\phi(h_1)) + \phi(h_1)\cdot\sigma(\phi(h_2) \\
% 		&=& \sigma(\phi(h_1)) + h_1\cdot\sigma(\phi(h_2)) \\
% 		&=& (\sigma \circ \phi)(h_1) + (\sigma \circ \phi)(h_2) \\
% 		&=& Z^1(\phi)(\sigma)(h_1) + h_1\cdot Z^1(\phi)(\sigma)(h_2).
% \end{eqnarray*}
% 
% Moreover, it can be shown that $Z^1(\phi)$ maps $B^1(H, V)$ into $B^1(\tilde{H}, V)$. This leads us to define a map of 1-cohomologies,
% \begin{eqnarray*}
% 	H^1(\phi):H^1(H, V) \rightarrow H^1(\tilde{H}, V),
% \end{eqnarray*}
% defined by
% \begin{displaymath}
% \begin{CD}
% 	Z^1(H, V) @>Z^1(\phi)>> Z^1(\tilde{H}, V)\\
% 	@V{\pi}VV                                  @VV{\tilde{\pi}}V\\
% 	H^1(H, V) @>>H^1(\phi)> H^1(\tilde{H}, V)
% \end{CD}
% \end{displaymath}
% where $\pi$ and $\tilde\pi$ are the respective canonical projections of $Z^1(H, V)$ onto $H^1(H, V)$ and $Z^1(\tilde{H}, V)$ onto $H^1(\tilde{H}, V)$. To show that the map $H^1(\phi)$ is well-defined it is sufficient to notice that $Z^1(\phi)$ is a homomorphism.

\begin{lemma} Let $B$ be a Borel subgroup of $SL_2$ acting on an algebraic group $V$. Then $H^1(i):H^1(SL_2, V)\rightarrow H^1(B, V)$ is injective.
\end{lemma}
\begin{proof}
Let $x$ be in the kernel of $H^1(i)$ and $\sigma$ and element of $Z^1(SL_2, V)$ that projects onto the class $x$. Since $Z^1(i)(\sigma)$ projects to the trivial 1-cohomology class we may as well assume that $\sigma|_B = 1$. For there exists some $v$ in $V$ such that for all $b$ in $B$
\begin{displaymath}
	Z^1(i)(\sigma)(b) = v*b\cdot v^{-1}.
\end{displaymath}
Consider the 1-cocycle $\hat{\sigma}:SL_2\rightarrow V$ defined by
\begin{displaymath}
	\hat{\sigma}(h) = v^{-1}*\sigma(h)*h\cdot v.
\end{displaymath}
Then by construction $\hat{\sigma}$ also projects to the class $x$, and for all $b$ in $B$
\begin{eqnarray*}
	\hat{\sigma}(b) &=& v^{-1}*\sigma(b)*b\cdot v\\
	&=& v^{-1}*(v*b\cdot v^{-1})*b\cdot v\\
	&=& v^{-1}*v*b\cdot (v^{-1} * v)\\
	&=& 1,
\end{eqnarray*}
so we may as well have chosen $\hat{\sigma}$ instead as a representative for $x$. 

Now consider the \emph{homogeneous space} $SL_2/B$ \cite{humphreys1975linear} and take the map 
\begin{displaymath}
	\tilde{\sigma}:SL_2/B \rightarrow V,
\end{displaymath}
defined in the usual way under the canonical projection $\pi:SL_2 \rightarrow SL_2/B$:
\begin{displaymath}
	\xymatrix{
	SL_2 \ar[r]^{\sigma} \ar[d]_{\pi} & V\\
	SL_2/B \ar[ru]_{\tilde{\sigma}}
	}
\end{displaymath}
This map is well defined and is a morphism \cite{borel1991linear}. Now since $SL_2/B$ is an irreducible projective variety \cite{humphreys1975linear}, $\tilde{\sigma}$ must be constant  \cite{borel1991linear}. Hence, as $\sigma$ takes the value 1 for any $b$ in $B$, $\tilde{\sigma}(hB) = 1$ for all cosets $hB$. Therefore, for all $h$ in $SL_2$
\begin{displaymath}
	\sigma(h) = \tilde{\sigma}(hB) = 1.
\end{displaymath}
We have shown that $\sigma$ is the 1-coboundary $\chi_1$ which means that the kernel of $H^1(i)$ is trivial.
\end{proof} 

\begin{lemma} Let $B$ be a Borel subgroup of $SL_2$ and $U$ be the unipotent radical of $B$. Then $H^1(B, V)\rightarrow H^1(U, V)$ is injective. Moreover
\begin{displaymath}
	H^1(SL_2, V)\rightarrow H^1(U, V)
\end{displaymath}
is injective.
\end{lemma}
\begin{proof}
As in the previous example, let $x$ be an element of the kernel of $H^1(i):H^1(B,V)\rightarrow H^1(U,V)$ and let $\sigma$ in $Z^1(B,V)$ be a representative for $x$ such that $\sigma|_U = 1$. Let $T$ be a maximal torus for $B$. For any $u$ in $U$ and $t$ in $T$ there is a $u'$ in $U$ such that
\begin{displaymath}
	ut = tu'.
\end{displaymath}
Hence $U$ acts trivially on $\sigma(T)$:
\begin{eqnarray*}
	\sigma(ut) &=& \sigma(tu')\\
	\sigma(u)*u\cdot\sigma(t) &=& \sigma(t)*t\cdot\sigma(u')\\
	u\cdot\sigma(t) &=& \sigma(t).
\end{eqnarray*}
Since $T$ is linearly reductive, $H^1(T, V)$ is trivial [prove or reference], so that there is a $v$ in $V$ such that for all $t$ in $T$
\begin{displaymath}
	\sigma(t) = \chi_v(t) = v*t\cdot v^{-1}.
\end{displaymath}
Consider the 1-cocycle $\tau$ in $Z^1(B, V)$ defined by
\begin{displaymath}
	\tau(b) = v^{-1}*\sigma(b)*b\cdot v.
\end{displaymath}
\begin{displaymath}
	v*t\cdot v^{-1} = w*b\cdot w^{-1}
\end{displaymath}
\begin{displaymath}
	v*t\cdot v^{-1} = u\cdot v*b\cdot v^{-1}
\end{displaymath}
\end{proof}
