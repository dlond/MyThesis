%!TEX root = /Users/dan/Documents/Thesis/Thesis.tex
% Chapter 4

\chapter{K\"ulshammer's Second Problem}
\label{Chapter4}
\lhead{Chapter 4. \emph{K\"ulshammer's Second Problem}}

\section{K\"ulshammer's Second Problem}

Two questions were raised by B. K\"ulshammer concerning representations of a finite group $\Gamma$ into a linear algebraic group $G$ over an algebraically closed field $k$. The first has a positive answer  and is essentially contained a paper by A. Weil \cite{weil1964remarks}:
\begin{itemize}
	\item[(K. I)] Let $\mathrm{char}(k)$ be prime to the order of $\Gamma$. Are there only finitely many representations $\rho:\Gamma\rightarrow G$ up to conjugation by $G$?
	\item[(K. II)] Let $p = \mathrm{char}(k)$ and $\Gamma_p \subset  \Gamma$ be a Sylow $p$-subgroup. Fix a conjugacy class of representations from $\Gamma_p\rightarrow G$. Are there, up to conjugation by $G$, only finitely many representations $\rho:\Gamma\rightarrow G$ whose restrictions to $\Gamma_p$ belong to the given class?
\end{itemize}

(K. II) has positive answer so long as $G$ is reductive and the characteristic of $k$ is good for $G$ \cite{slodowy1997two}. The same paper shows that the answer is ``no'' in general by way of a counterexample involving a non-reductive $G$.

We wish to determine whether there exists a reductive counterexample to (K. II).

\section{The Approach}

We are interested in knowing whether there can be infinitely many $G$-conjugacy classes of representations $\Gamma\rightarrow G$ that when restricted to $\Gamma_p$ hit some fixed $G$-conjugacy class of representations $\Gamma_p\rightarrow G$. A consequence of the following Theorem [reference] is that we will need to study representations into parabolic subgroups $P < G$.

\begin{theorem} \label{finiteGCR} There are only finitely many $G$-conjugacy classes of $G$-completely reducible representations $\Gamma\rightarrow G$.
\end{theorem}

So by Theorem \ref{finiteGCR}, if we have infinitely many $G$-conjugacy classes of representations $\Gamma\rightarrow G$ then infinitely many of those classes must be of non-$G$-completely reducible representations. The following Lemma states that the finiteness of $G$-conjugacy classes of a collection of representations $\Gamma\rightarrow G$ carries over to $P$-conjugacy classes for any parabolic subgroup $P<G$ containing the image of the representations.

\begin{lemma}\label{GIsPConj} Let $R=\{\rho_\lambda:\Gamma\rightarrow P\, |\, \lambda \in \Lambda\}$ be a collection of representations indexed by the set $\Lambda$, $P$ a fixed parabolic subgroup of $G$. Then $R$ is contained in a finite union of $G$-conjugacy classes if and only if it is contained in a finite union of $P$-conjugacy classes.
\end{lemma}
\begin{proof}
	Take two elements $\rho_\mu, \rho_\nu$ of $R$ in a particular $G$-conjugacy class. Then there exists an element $g\in G$ such that
	\begin{displaymath}
		g\cdot \rho_\mu = \rho_\nu.
	\end{displaymath}
	
	By definition $\rho_\nu$ maps into $P$, but on the other hand $\rho_\nu = g\cdot \rho_\mu$ maps into $Q = gPg^{-1}$. Therefore 
	\begin{displaymath}
		\rho_\nu: \Gamma\rightarrow P \cap Q.
	\end{displaymath}
	Let $T$ be a maximal torus of $G$ contained in $P\cap Q$ [ref], and let $\{n_1, \ldots, n_l\}$ be coset representatives for the Weyl group $W = N_G(T)/T$.
	
	Since $T$ and $gTg^{-1}$ are maximal tori of $Q$ they must be $Q$-conjugate, so there exists an element $q\in Q$ such that
	\begin{displaymath}
		qTq^{-1} = gTg^{-1}.
	\end{displaymath}
	By the definition of $Q$ there exists an element $p\in P$ such that $q = gpg^{-1}$, so in fact
	\begin{eqnarray*}
		gpg^{-1}Tgp^{-1}g^{-1} &=& gTg^{-1} \\
		\Rightarrow pg^{-1}Tgp^{-1} &=& T.
	\end{eqnarray*}
	We see that $gp^{-1}$ lies in $N_G(T)$. Let $n_i$ be the coset representative for the element of $W$ containing $gp^{-1}$ and let $t\in T$ be the element that satisfies
	\begin{displaymath}
		gp^{-1} = n_it.
	\end{displaymath}
	$T$ is a subgroup of $P$ so let $p^{-1}t^{-1} = p' \in P$ and we have
	\begin{eqnarray*}
		\rho_\mu &=& g^{-1}\cdot\rho_\nu\\
		&=& (p^{-1}t^{-1}n_i^{-1})\cdot\rho_\nu\\
		&=& p'\cdot(n_i^{-1}\cdot\rho_\nu). 
	\end{eqnarray*}
	Furthermore, as $\rho_\mu$ is an arbitrary element of $R\cap \left(G\cdot \rho_\nu\right)$ we have
	\begin{displaymath}
		R \cap \left(G\cdot \rho_\nu\right) \subset \bigcup_{i=1}^l P\cdot(n_i^{-1}\cdot\rho_\nu),
	\end{displaymath}
	where $l = |W|$.
	
	Therefore, a $G$-conjugacy class of $R$ is contained in a union of at most $l$ $P$-conjugacy classes. Thus it is clear that if $R$ is contained in a finite union of $G$-conjugacy classes then it is contained in a finite union of $P$-conjugacy classes.
	
	The converse is trivial.
\end{proof}

Although $G$ has infinitely many parabolic subgroups there are only finitely many $G$-conjugacy classes of parabolic subgroups, so we can choose a finite set $\{Q_i\}$ of representatives. We choose this particular set by fixing a maximal torus $T<G$ and a Borel subgroup $B<G$ containing $T$. Then a set of representative parabolic subgroups of $G$ can be sought via the root system of $G$ with respect to $T$. For each $Q_i$ there is a corresponding Levi subgroup $M_i<Q_i$ containing $B$. By Theorem \ref{finiteGCR} there are only finitely many $M_i$-conjugacy classes of $M_i$-irreducible representations $\Gamma \rightarrow M_i$ so we can fix a finite set of representatives $\{\sigma^j_{M_i}\}$. 

Let $V_i = R_u(Q_i)$ be the unipotent radical of $Q_i$, so that $Q_i = V_i \rtimes M_i$. We define the projection $\pi_i:Q_i \rightarrow M_i$ by $\pi_i(q) = m$, where $q = vm \in Q_i$, $v\in V_i$, $m\in M_i$.

We will show that for each representation $\rho:\Gamma\rightarrow G$ there exists an element $g\in G$ such that the representation $\sigma = g\cdot\rho$ fits one of only finitely many commutative diagrams of the following form, determined by the indices $i,j$:
\begin{displaymath}
	\xymatrix{
	\Gamma \ar[r]^{\sigma} \ar[rd]_{\sigma^j_{M_i}} & Q_i \ar[d]^{\pi_i}\\
	& M_i}
\end{displaymath}
We call this construction a \emph{standard commutative diagram} for $\rho$.

Let $\rho:\Gamma\rightarrow G$ be a representation and let $P$ be a minimal parabolic subgroup of $G$ containing $\rho(\Gamma)$. Then there is an element $h\in G$ such that $hPh^{-1} = Q_i$ for some $i$. Let $\rho' = h\cdot \rho$. 

Since $Q_i$ is a minimal parabolic subgroup containing $(\pi_i \circ \rho)(\Gamma)$, $\rho'_0$ is $M_i$-irreducible [reference]. Hence there exists an $m\in M_i$ such that
\begin{displaymath}
	m\cdot \rho'_0 = \sigma^j_{M_i}, 
\end{displaymath}
for some $j$. Let $g=mh\in G$ and define $\sigma = g\cdot \rho$. This verifies what we set out to show.

It is worth pointing out that the element $g\in G$ and the minimal parabolic $P<G$ used in the construction are not necessarily unique, hence the qualifier ``\emph{a} standard commutative diagram''. As an extreme example, if $\rho$ is the trivial representation then $\rho$ has minimal parabolic $P=B$, and any $g\in G$ could be used to conjugate $\rho(\Gamma)$ into $Q_i=B$. [Example of more than one minimal parabolic?]

For a given parabolic subgroup $P$ of $G$ with Levi subgroup $L$ and unipotent radical $V$, and a given representation $\rho:\Gamma\rightarrow P$ we have a map $\rho_L:\Gamma\rightarrow L$ defined by $\rho_L = \pi \circ \rho$. Now define $\alpha_\rho:\Gamma\rightarrow V$ by $\alpha_\rho(\gamma) = \rho(\gamma)\rho_L(\gamma)^{-1}$ for all $\gamma\in\Gamma$, so that $\rho = \alpha_\rho\rho_L$.

If $\rho$ is a homomorphism then
\begin{eqnarray*}
	\alpha_\rho(\gamma_1\gamma_2)\rho_L(\gamma_1\gamma_2) \,=\, \rho(\gamma_1\gamma_2) 
		&=& \rho(\gamma_1)\rho(\gamma_2) \\
		&=& \alpha_\rho(\gamma_1)\rho_L(\gamma_1)\alpha_\rho(\gamma_2)\rho_L(\gamma_2) \\
		&=& \alpha_\rho(\gamma_1)\rho_L(\gamma_1)\alpha_\rho(\gamma_2)\rho_L(\gamma_1)^{-1}\rho_L(\gamma_1)\rho_L(\gamma_2)\\
		&=&\alpha_\rho(\gamma_1)\rho_L(\gamma_1)\alpha_\rho(\gamma_2)\rho_L(\gamma_1)^{-1}\rho_L(\gamma_1\gamma_2),
\end{eqnarray*}
so that
\begin{eqnarray*}
	\alpha_\rho(\gamma_1\gamma_2) &=&
	\alpha_\rho(\gamma_1)\rho_L(\gamma_1)\alpha_\rho(\gamma_2)\rho_L(\gamma_1)^{-1}\\
	&=& \alpha_\rho(\gamma_1)\,\left(\gamma_1\cdot\alpha_\rho(\gamma_2)\right),
\end{eqnarray*}
where $\Gamma$ acts on $V$ by conjugation via $\rho_L$. Therefore $\alpha_\rho$ satisfies the (multiplicative) 1-cocycle condition in (\ref{theNonabOneCocycleCondition}) and so $\alpha_\rho\in Z^1(\Gamma, \rho_L, V)$. 

Conversely given a 1-cocycle $\alpha\in Z^1(\Gamma, \rho_L, V)$ we can construct a representation $\rho:\Gamma\rightarrow P$ by $\rho(\gamma) = \alpha(\gamma)\rho_L(\gamma)$ for all $\gamma\in \Gamma$.

Given a representation $\rho:\Gamma\rightarrow P$, define $Hom(\Gamma, P)_{\rho_L}$ to be the set of representations $\sigma:\Gamma\rightarrow P$ such that $\sigma_L = \rho_L$. We formalise the above findings in the following Lemma:

\begin{lemma}
  The map $h:Hom(\Gamma, P)_{\rho_L} \rightarrow Z^1(\Gamma, \rho_L, V)$ defined by
  \begin{displaymath}
    (h(\sigma))(\gamma) = \sigma(\gamma)\rho_L(\gamma)^{-1},
  \end{displaymath}
  is bijective.
  \label{lem:hom_z1}
\end{lemma}

For ease of notation we will often write $h(\sigma)$ as $\alpha_\sigma$. 

Let $v \in V$ and $\sigma \in Hom(\Gamma, P)_{\rho_L}$. Since $L$ normalizes $V$, $\pi \circ (v \cdot \sigma) = \sigma_L = \rho_L$ and so $v \cdot \sigma \in Hom(\Gamma, P)_{\rho_L}$. Thus $V$ acts on $Hom(\Gamma, P)_{\rho_L}$. We show that $h$ gives rise to a bijective map $\bar{h}: Hom(\Gamma,P)_{\rho_{0}}/V\rightarrow H^{1}(\Gamma, \rho_{0}, V)$.

\begin{lemma}
  The following diagram is commutative:
  \begin{displaymath}
    \xymatrix{
    Hom(\Gamma, P)_{\rho_{0}} \ar[r]^h \ar[d] & Z^{1}(\Gamma, \rho_L, V) \ar[d] \\
    Hom(\Gamma, P)_{\rho_{0}}/V \ar[r]^{\bar{h}} & H^{1}(\Gamma, \rho_L, V).
    }
  \end{displaymath}
  Furthermore, $\bar{h}$ is bijective.
  \label{lem:v_h1}
\end{lemma}
\begin{proof}  
  Take $\sigma,\tau\in Hom(\Gamma, P)_{\rho_L}$ such that $\sigma = v\cdot\tau$ for some fixed $v\in V$. Then for all $\gamma\in \Gamma$
  \begin{eqnarray*}
    \alpha_\sigma(\gamma) &=& \sigma(\gamma)\rho_L(\gamma)^{-1}\\
    &=& v\tau(\gamma)v^{-1}\rho_L(\gamma)^{-1}\\
    &=& v\tau(\gamma)\rho_L(\gamma)^{-1}\rho_L(\gamma)v^{-1}\rho_L(\gamma)^{-1}\\
    &=& v\tau(\gamma)\rho_L(\gamma)^{-1}(\gamma\cdot v^{-1})\\
    &=& v\alpha_\tau(\gamma)(\gamma\cdot v^{-1}).
  \end{eqnarray*}
  Hence $\alpha_\sigma$ and $\alpha_\tau$ belong to the same 1-cohomology class. This proves $\overline{h}$ is well-defined. 
  
  Conversely, let $\alpha_\sigma, \alpha_\tau \in Z^1(\Gamma, \rho_L, V)$ belong to the same 1-cohomology class, so there exists a $v \in V$ such that
  \begin{displaymath}
    \alpha_\sigma(\gamma) = v \alpha_\tau(\gamma)(\gamma \cdot v^{-1}).
  \end{displaymath}
  Then the corresponding representations $\sigma, \tau:\Gamma \rightarrow P$ are $V$-conjugate:
  \begin{eqnarray*}
    \sigma(\gamma) &=&  \alpha_\sigma(\gamma)\rho_L(\gamma) \\
    &=& v \alpha_\tau(\gamma)(\gamma \cdot v^{-1}) \rho_L(\gamma) \\
    &=& v \alpha_\tau(\gamma)\rho_L(\gamma) v^{-1} \rho_L(\gamma)^{-1} \rho_L(\gamma) \\
    &=& v \cdot \tau(\gamma).
  \end{eqnarray*}
  Therefore $\bar{h}^{-1}$ is well-defined, so $\bar{h}$ is a bijection.
\end{proof}

In general, take an element $g\in G$. We can conjugate $\sigma\in Hom(\Gamma, P)_{\rho_L}$ by $g$ to get an element $g\cdot\sigma\in Hom(\Gamma, gPg^{-1})_{g\cdot\rho_L}$ and $h(g\cdot\sigma) = \alpha_{g\cdot\sigma}\in Z^1(\Gamma, g\cdot\rho_L, gVg^{-1})$.

If $g\in P$ then $g=vl$ for some $v\in V$ and some $l\in L$, and since $gPg^{-1} = P$ and $gVg^{-1} = V$, conjugating gives rise to the maps $Hom(\Gamma, P)_{\rho_L}\rightarrow Hom(\Gamma, P)_{l\cdot\rho_L}$ and $Z^1(\Gamma, \rho_L, V)\rightarrow Z^1(\Gamma, l\cdot\rho_L, V)$.

If $l\in Z(L)$ then $l\cdot\rho_L = \rho_L$. Indeed $VZ(L)$ acts on $Hom(\Gamma, P)_{\rho_L}$ and also on $Z^1(\Gamma, \rho_L, V)$ via $h$. Furthermore, we see from Lemma \ref{lem:v_h1} that $V$ acts trivially on $H^1(\Gamma, \rho_L, V)$.

\begin{lemma}
  The $Z(L)$-action on $Hom(\Gamma, P)_{\rho_L}$ and $Z^1(\Gamma, \rho_L, V)$ descends to a $Z(L)$-action on $Hom(\Gamma, P)_{\rho_L}/V$ and $H^1(\Gamma, \rho_L, V)$.
  \label{lem:vzl_h1zl}
\end{lemma}
\begin{proof}
  We show that $\bar{h} : Hom(\Gamma, P)_{\rho_L}/V \rightarrow H^1(\Gamma, \rho_L, V)$ is $Z(L)$-equivariant, which means the map $\tilde{h} : Hom(\Gamma, P)_{\rho_L}/VZ(L) \rightarrow H^1(\Gamma, \rho_L, V)/Z(L)$ is well-defined. This follows quickly from the fact that the $Z(L)$-action on $Z^1(\Gamma, \rho_L, V)$ is defined by $h$, so that $h$ is automatically $Z(L)$-equivariant. To this end, let $z \in Z(L)$ $\sigma V \in Hom(\Gamma, P)_{\rho_L}/V$. Then
  \begin{displaymath}
    \bar{h}(z \cdot \sigma V) = (h(z \cdot \sigma))V = (z \cdot h(\sigma)) V = z \cdot \bar{h}(\sigma V).
  \end{displaymath}
\end{proof}

\begin{lemma}
  Let $R = \{\rho_\lambda:\Gamma\rightarrow P\,|\,\lambda\in\Lambda\}$ be a collection of representations indexed by the set $\Lambda$. Given an irreducible representation $\sigma_L:\Gamma\rightarrow L$ define
  \begin{displaymath}
    R_{\sigma_L} = \{\rho \in R\,|\,\rho_L = \sigma_L\}. 
  \end{displaymath}
  
  The following statements are equivalent:
  \begin{itemize}
    \item[(i)] $R$ is contained in a finite union of $P$-conjugacy classes.
    \item[(ii)] For each irreducible representation $\sigma_L:\Gamma\rightarrow L$, $R_{\sigma_L}$ is contained in a finite union of $VZ(L)^\circ$-conjugacy classes.
    \item[(iii)] For each irreducible representation $\sigma_L:\Gamma\rightarrow L$, if $R_{\sigma_L}$ is nonempty then 
      \begin{displaymath}
	H^{1}(\Gamma,\sigma_L,V)/Z(L)^\circ
      \end{displaymath}
      is finite.
  \end{itemize}
  \label{lem:p_h1}
\end{lemma}
\begin{proof}\quad

  $(i) \Rightarrow (ii)$ Assume $R$ is contained in a finite union of $P$-conjugacy classes and fix an irreducible representation $\sigma_L : \Gamma \rightarrow L$. Then $R_{\sigma_L}$ is contained a finite union of $P$-conjugacy classes. Take $\rho \in R_{\sigma_L}$ and suppose that $p \cdot \rho \in R_{\sigma_L}$ for some $p \in P$. Writing $p = vl$ for some $v \in V$ and some $l \in L$, $(vl) \cdot \rho \in R_{\sigma_L}$ implies that in fact $l \in C_L(\sigma_L(\Gamma))$. Furthermore, since $\sigma_L$ is irreducible it follows that $C_L(\sigma_L(\Gamma))/Z(L)^\circ$ is finite [reference], so we can choose a finite set $\{c_1, \ldots, c_m\}$ of coset representatives for $C_L(\sigma_L(\Gamma))/Z(L)^\circ$. Therefore
  \begin{displaymath}
    R_{\sigma_L} \cap (P \cdot \rho) \subset \bigcup_{i = 1}^{m} VZ(L)^\circ \cdot \left( c_i \cdot \rho \right).
  \end{displaymath}

  $(ii) \Rightarrow (iii)$ 
  
  $(iii) \Rightarrow (i)$ 
\end{proof}

\begin{theorem}
  Let $R=\{\rho_\lambda:\Gamma\rightarrow G\,|\,\lambda \in \Lambda\}$ be a collection of representations indexed by the set $\Lambda$ and define $M_i < Q_i < G$ and $\sigma_{0,i}^{j}$ as in [reference]. Then $R^{G}$ is a finite union of $G$-conjugacy classes if and only if for each $i,j$ the subset of $H^{1}(\Gamma, \sigma_{0,i}^{j}, V_{i})/Z(L)^\circ$ arising from $R$ is finite.
  \label{thm:g_h1}
\end{theorem}

\begin{theorem}
  Let $\Gamma$ be a finite (or algebraic) group and $G$ be an algebraic group over an algebraically closed field $k$ of characteristic $p$. Define $M_i < Q_i < G$ and $\sigma_{0,i}^{j}:\Gamma \rightarrow M_{i}$ as in [reference]. The answer to (the algebraic version of) K\"ulshammer's second question is positive if and only if each map
  \begin{displaymath}
    H^{1}(\Gamma, \sigma_{0,i}^{j}, V_{i}) \rightarrow H^{1}(\Gamma_{p}, \sigma_{0,i}^{j}, V_{i})
  \end{displaymath}
  is injective.
  \label{k2_h1}
\end{theorem}

% \section{An algebraic group version}
% 
% In an attempt to gain further insight into (K. II) we adjust the original question by letting $\Gamma$ be an infinite group $H$. The advantage being that a negative answer in the algebraic group version may provide a negative answer to (K. II) by choosing an appropriate finite subgroup $\Gamma$ of $H$. In many of the examples to follow we set $H = SL_2(K)$ with Sylow $p$-subgroup $H_p = U_2(K)$ consisting of upper unitriangular matrices. 
% 
% Let $P \subset G$ be a parabolic subgroup and $L \subset P$ the corresponding Levi subgroup. Fix a representation $\rho_0:H\rightarrow L$. We can assume $\rho_0(H)$ is $L$-irreducible, that is, not contained in a proper parabolic of $L$. 
% 
% Now define $\rho_\alpha:H \rightarrow P$ by $\rho_\alpha(h) = \alpha(h)\rho_0(h)$ where $\alpha:H\rightarrow R_u(P)$, $R_u(P)$ the unipotent radical of $P$. 
% 
% For $\rho_\alpha$ to be a homomorphism
% \begin{eqnarray*}
% 	\alpha(h_1h_2)\rho_0(h_1h_2) &=& \alpha(h_1)\rho_0(h_1)\alpha(h_2)\rho_0(h_2) \\
% 	 &=& \alpha(h_1)\rho_0(h_1)\alpha(h_2)\rho_0(h_1)^{-1}\rho_0(h_1)\rho_0(h_2) \\
% 	 &=& \alpha(h_1)\rho_0(h_1)\alpha(h_2)\rho_0(h_1)^{-1}\rho_0(h_1h_2).
% \end{eqnarray*}
% That is $\alpha(h_1h_2) = \alpha(h_1) h_1\cdot\alpha(h_2)$, where the action $H \times R_u(P) \rightarrow R_u(P)$ is conjugation via $\rho_0$. This is a 1-cocycle condtion; $\alpha\in Z^1(H, R_u(P))$. $R_u(P)$ will not be abelian in general.
% 
% Now suppose $\rho_\alpha$ is $R_u(P)$-conjugate to some $\rho_\beta$, $\alpha, \beta \in Z^1(H, R_u(P))$. That is, there exists a $v \in R_u(P)$ such that for all $h\in H$
% \begin{eqnarray*}
% 	\alpha(h)\rho_0(h) &=& v\beta(h)\rho_0(h)v^{-1} \\
% 	&=& v\beta(h)\rho_0(h)v^{-1}\rho_0(h)^{-1}\rho_0(h).
% \end{eqnarray*}
% That is $\alpha(h) = v\beta(h)h\cdot v^{-1}$. In particular if $\rho_\alpha$ is $R_u(P)$-conjugate to $\rho_0$, that is $\beta$ is trivial, then $\alpha$ takes the form of a 1-coboundary. Generally speaking $\alpha$ and $\beta$ project to the same 1-cohomology class. In the abelian case this reads ``$\alpha$ and $\beta$ differ by a 1-coboundary'':
% \begin{eqnarray*}
% 	\alpha(h) = v\beta(h)h\cdot v^{-1} \quad\leadsto \quad\alpha(h) &=& v + \beta(h) - h\cdot v \\
% 	&=& \beta(h) + v - h\cdot v \\
% 	&=& \beta(h) + \chi_v(h).
% \end{eqnarray*}

% 3) Look at the nonreductive counterexample in Slodowy's paper on Kulshammer's problem.  What is special about the 3-dimensional U that makes this counterexample work?  Can you find similar structure in the unipotent radical of a reductive group?
