%!TEX root = /Users/dan/Documents/Thesis/Thesis.tex
% Chapter 4

\chapter{K\"ulshammer's Second Problem}
\label{Chapter4}
\lhead{Chapter 4. \emph{K\"ulshammer's Second Problem}}

% \begin{itemize}
% \item K\"ulshammer's First Problem
% \item K\"ulshammer's Second Problem
% \item Counter example for non-reductive $G$
% \item Overture to Ch. 3-5
% \end{itemize}

\section{K\"ulshammer's Second Problem}

Two questions were raised by B. K\"ulshammer concerning representations of a finite group $\Gamma$ into a linear algebraic group $G$ over an algebraically closed field $k$. The first has a positive answer  and is essentially contained a paper by A. Weil \cite{weil1964remarks}:
\begin{itemize}
	\item[(K. I)] Let $\mathrm{char}(k)$ be prime to the order of $\Gamma$. Are there only finitely many representations $\rho:\Gamma\rightarrow G$ up to conjugation by $G$?
	\item[(K. II)] Let $p = \mathrm{char}(k)$ and $\Gamma_p \subset  \Gamma$ be a Sylow $p$-subgroup. Fix a conjugacy class of representations from $\Gamma_p\rightarrow G$. Are there, up to conjugation by $G$, only finitely many representations $\rho:\Gamma\rightarrow G$ whose restrictions to $\Gamma_p$ belong to the given class?
\end{itemize}

(K. II) has positive answer so long as $G$ is reductive and the characteristic of $k$ is good for $G$ \cite{slodowy1997two}. The same paper shows that the answer is ``no'' in general by way of a counterexample involving a non-reductive $G$.

We will explore the possibility of a reductive counterexample to (K. II).

\section{The Approach}

We are interested in knowing whether there can be infinitely many $G$-conjugacy classes of representations $\Gamma\rightarrow G$ that when restricted to $\Gamma_p$ hit some fixed $G$-conjugacy class of representations $\Gamma_p\rightarrow G$. The following Lemma states that the finiteness of $G$-conjugacy classes of representations $\Gamma\rightarrow G$ carries over to $P$-conjugacy classes for any parabolic subgroup $P<G$ containing the image of the representations.

\begin{lemma}\label{GIsPConj} Let $R=\{\rho_\lambda:\Gamma\rightarrow P\, |\, \lambda \in \Lambda\}$ be a collection of representations, $P$ a fixed parabolic subgroup $P<G$. Then $R$ is a finite union of $G$-conjugacy classes if and only if it is a finite union of $P$-conjugacy classes.
\end{lemma}
\begin{proof}
	Suppose there are only finitely many $G$-conjugacy classes of $R$ and take $\rho_\mu, \rho_\nu \in R$ such that
	\begin{displaymath}
		g\cdot \rho_\mu = \rho_\nu
	\end{displaymath}
	for some $g \in G$. By definition $\rho_\nu$ maps into $P$ but on the other hand $\rho_\nu = g\cdot \rho_\mu$ maps into $Q = gPg^{-1}$, so 
	\begin{displaymath}
		\rho_\nu: \Gamma\rightarrow P \cap Q
	\end{displaymath}
%	If $P = Q$ then $g\in N_G(P) = P$, hence $\rho_\mu, \rho_\nu$ are $P$-conjugate. 
	
%	Assume $P \neq Q$ and l
	Let $T < P \cap Q$ be a maximal torus of $G$ [ref]. Since $T,gTg^{-1}$ are maximal tori of $Q$ they must be $Q$-conjugate, say
	\begin{displaymath}
		qTq^{-1} = gTg^{-1},
	\end{displaymath}
	for some $q \in Q$. Then 
	\begin{displaymath}
		q^{-1}g \in N_G(T),
	\end{displaymath}
	and so
	\begin{displaymath}
		q^{-1}g = (gp^{-1}g^{-1})g = gp^{-1} \in N_G(T),
	\end{displaymath}
	for some $p \in P$. Hence, we can write
	\begin{eqnarray*}
		g &=& (gp^{-1})p\\
		&=& n(tp),
	\end{eqnarray*}
	where $n\in G$ is one of the finitely many representatives of the Weyl group $W=N_G(T)/T$ such that $\pi(gp^{-1}) = nT\in W$, and $t\in T$ such that $nt = gp^{-1}$. Since $T<P$, $tp\in P$.
	
	Therefore, a $G$-conjugacy class of $R$ splits up into at most $|N_G(T)/T|$ $P$-conjugacy classes which proves the forward direction. The converse is trivial.
\end{proof}


\begin{theorem} \label{finiteGCR} There are only finitely many $G$-conjugacy classes of $G$-completely reducible representations $\Gamma\rightarrow G$.\qedhere
\end{theorem}
\begin{proof}[See (a reference)]
\end{proof}

Although $G$ has infinitely many parabolic subgroups there are only finitely many $G$-conjugacy classes of parabolic subgroups, so we can choose a finite set $\{Q_i\}$ of representatives. We choose this particular set by fixing a maximal torus $T<G$ and a Borel subgroup $B<G$ containing $T$. Then a set of representative parabolic subgroups of $G$ can be sought via the root system of $G$ with respect to $T$ [reference or explain]. We choose a set of Levi subgroups $\{M_i\}$, $M_i$ being a Levi subgroup of $Q_i$. By Theorem \ref{finiteGCR} there are only finitely many $M_i$-conjugacy classes of $M_i$-completely reducible representations $\sigma_{0, i}:\Gamma\rightarrow M_i$, and we fix a finite set of representatives $\{\sigma^{j_i}_{0,i}\}$.

We will show that for each representation $\rho:\Gamma\rightarrow G$ there exists a representation $\sigma = g\cdot\rho$, $g\in G$, that fits precisely one of only finitely many commutative diagrams determined by the indices $i,j$:
\begin{displaymath}
	\xymatrix{
	\Gamma \ar[r]^{\sigma} \ar[rd]_{\sigma^{j_i}_{0,i}} & Q_i \ar[d]\\
	& M_i}
\end{displaymath}
where $Q_i$ is the minimal parabolic subgroup of $G$ containing $\rho(\Gamma)$ from our fixed set. We call this the \emph{standard commutative diagram} for $\rho$.

Let $\rho$ be a representation from $\Gamma\rightarrow G$ and let $P$ be a minimal parabolic subgroup of $G$ containing $\rho(\Gamma)$. Then there is a $g$ in $G$ such that $P = gQg^{-1}$ for some $Q\in\{Q_i\}$. Let $\rho' = g\cdot \rho$. 

Define $\rho_0:\Gamma\rightarrow M$ by composing $\rho'$ with the projection $Q\rightarrow M$, $M\in\{M_i\}$ the chosen Levi subgroup for $Q$:
\begin{displaymath}
	\xymatrix{
	\Gamma \ar[r]^{\rho'} \ar[rd]_{\rho_0} & Q \ar[d]\\
	& M}
\end{displaymath}

Since $Q$ is a minimal parabolic subgroup containing $\rho'(\Gamma)$, $\rho_0$ is $M$-irreducible [reference] and therefore $M$-completely reducible. Hence there exists an $m\in M$ such that $\sigma_0 = m\cdot \rho_0$ for some $\sigma_0\in\{\sigma^{j_i}_{0,i}\}$. Let $\sigma = m\cdot \rho' = mg\cdot \rho$. This verifies what we set out to show.

[Say something about $\rho \rightsquigarrow (i,j) \rightsquigarrow H^1(\Gamma,V_i)$ so that I can state the next lemma]

For a given parabolic subgroup $P<G$ with Levi subgroup $L<P$, and a given representation $\rho:\Gamma\rightarrow P$ we have defined the map $\rho_0:\Gamma\rightarrow L$ by composing $\rho$ with the projection $P\rightarrow L$. Now let $V = R_u(P)$ and define $\alpha_\rho:\Gamma\rightarrow V$ by composing $\rho$ with the projection $P\rightarrow V$, so that $\rho = \alpha_\rho\rho_0$. If $\rho_0$ is a homomorphism then
\begin{eqnarray*}
	\alpha_\rho(\gamma_1\gamma_2)\rho_0(\gamma_1\gamma_2) \,=\, \rho(\gamma_1\gamma_2) 
		&=& \rho(\gamma_1)\rho(\gamma_2) \\
		&=& \alpha_\rho(\gamma_1)\rho_0(\gamma_1)\alpha_\rho(\gamma_2)\rho_0(\gamma_2) \\
		&=& \alpha_\rho(\gamma_1)\rho_0(\gamma_1)\alpha_\rho(\gamma_2)\rho_0(\gamma_1)^{-1}\rho_0(\gamma_1)\rho_0(\gamma_2)\\
		&=&\alpha_\rho(\gamma_1)\rho_0(\gamma_1)\alpha_\rho(\gamma_2)\rho_0(\gamma_1)^{-1}\rho_0(\gamma_1\gamma_2),
\end{eqnarray*}
so that
\begin{eqnarray*}
	\alpha_\rho(\gamma_1\gamma_2) &=&
	\alpha_\rho(\gamma_1)\rho_0(\gamma_1)\alpha_\rho(\gamma_2)\rho_0(\gamma_1)^{-1}\\
	&=& \alpha_\rho(\gamma_1)\,\gamma_1\cdot\alpha_\rho(\gamma_2),
\end{eqnarray*}
where $\Gamma$ acts on $V$ by conjugation via $\rho_0$. Thus $\alpha_\rho$ satisfies the (multiplicative) 1-cocycle condition in (\ref{theNonabOneCocycleCondition}).

Hence for the given $\rho$ we have a corresponding 1-cocycle $\alpha_\rho:\Gamma\rightarrow V$. We use this to define the map
\begin{eqnarray}\label{repToH1}
	\mathcal{H}^1|_L:\mathrm{Hom}(\Gamma, P>L) \rightarrow H^1(\Gamma, \rho_0, V),
\end{eqnarray}
by projecting the corresponding 1-cocyles to their 1-cohomology classes.

Suppose we conjugate $\rho$ by an element $g\in G$. Then $\rho'=g\cdot\rho$ has corresponding 1-cocycle $\alpha_{\rho'}:\Gamma\rightarrow gVg^{-1}$.

So the $G$-action corresponds to maps of the form:
\begin{displaymath}
	H^1(\Gamma, \rho_0, V)\rightarrow H^1(\Gamma, g\cdot\rho_0, gVg^{-1}),
\end{displaymath}
Note that conjugating $\rho$ by an arbitrary element of $g$ changes $\rho_0$ and hence the action of $\Gamma$ on $V$. If we choose $g\in Z(L)^\circ$ and consider the $Z(L)^\circ$-action then the induced action of $\Gamma$ on $V$ does not change:
\begin{displaymath}
	\xymatrix{
		\Gamma \ar[r]^{\rho} \ar[rd]_{\rho_0} & P \ar[d]\\
		& L} \qquad\mapsto\qquad
	\xymatrix{
		\Gamma \ar[r]^{g\cdot\rho} \ar[rd]_{\rho_0} & gPg^{-1} \ar[d]\\
		& L}
\end{displaymath} 


\begin{lemma}\label{PConjIsHOne} Let $R=\{\rho_\lambda:\Gamma\rightarrow P\, |\, \lambda \in \Lambda\}$ be a collection of representations, $P$ a fixed parabolic subgroup $P<G$. Then $R_G = \{g\rho g^{-1}\,|\,\forall g\in G\, ,\forall\rho\in R\}$ is a finite union of $G$-conjugacy classes if and only if each $H^1(\Gamma, V_i)/Z(M_i)^\circ$ arising from the standard commutative diagrams for each element of $R$ has only finitely many elements.
\end{lemma}
\begin{proof}
	Suppose $R_G$ is a finite union of $G$-conjugacy classes. By Lemma \ref{GIsPConj} $R_G$ is a finite union of $P$-conjugacy classes, so there is a finite subset $S\subset R_G$ such that
	\begin{displaymath}
		R_G = \bigcup_{\rho \in S}\{p\rho p^{-1}\,|\,\forall p\in P\} = \bigcup_{\rho \in S}P\rho.
	\end{displaymath}
	Take two elements $\sigma$ and $\tau$ from one of the $P$-conjugacy classes above such that $\sigma$ and $\tau$ have the same standard commutative diagram:
	\begin{displaymath}
		\sigma',\tau':\xymatrix{
			\Gamma \ar[r] \ar[rd]_{\rho_0} & Q \ar[d]\\
		& M},
	\end{displaymath}
	where $\sigma' = g\cdot\sigma$ and $\tau' = h\cdot\tau$ for some $g, h\in G$. Suppose $P_1,P_2<P$ are minimal parabolic subgroups containing $\sigma(\Gamma), \tau(\Gamma)$ respectively. Then
	\begin{eqnarray*}
		gph^{-1}Qhp^{-1}g^{-1} &=& gpP_2p^{-1}g^{-1} \\
		&=& gP_1g^{-1} \\
		&=& Q,
	\end{eqnarray*}
	so that $gph^{-1}\in N_G(Q) = Q$, and
	\begin{eqnarray*}
		gph^{-1}\cdot\tau' &=& gp\cdot \tau \\
		&=& g\cdot \sigma \\
		&=& \sigma'.
	\end{eqnarray*}
	Hence $\sigma'$ and $\tau'$ are $Q$-conjugate so we can find elements $v\in V = R_u(Q)$ and $m\in M$ such that
	\begin{displaymath}
		\sigma' = (vm)\cdot\tau'.
	\end{displaymath}
	The standard commutative diagrams are the same so $\rho_0 = m\cdot\rho_0$, which implies $m\in C_M(\rho_0(\Gamma))$. Furthermore, since $\rho_0(\Gamma)$ is $M$-irreducible, $C_M(\rho_0(\Gamma))$ is a finite extension of $Z(M)^\circ$.
	
	Therefore $m = z_ic$ where $z_i$ is one of the finitely many coset representatives for $C_M(\rho_0(\Gamma))/Z(M)^\circ$ and $c\in Z(M)^\circ$.
	
	If $z_i$ is the identity element then $\sigma' = (vc)\cdot\tau'$, that is
	\begin{eqnarray*}
		\alpha_\sigma(\gamma)\rho_0(\gamma) &=& vc\alpha_\tau(\gamma)\rho_0(\gamma)(vc)^{-1} \\
		\Rightarrow \alpha_\sigma(\gamma) &=& vc\alpha_\tau(\gamma)\rho_0(\gamma)(vc)^{-1}\rho_0(\gamma)^{-1} \\
		&=& vc\alpha_\tau(\gamma)c^{-1}\rho_0(\gamma)v^{-1}\rho_0(\gamma)^{-1} \\
		&=& (vc)\alpha_\tau(\gamma)\, \gamma\cdot(vc)^{-1} \\
		&=& v\alpha_\tau(\gamma)\, \gamma\cdot v^{-1}
	\end{eqnarray*}
		
\end{proof}



Let $R$ be a collection of representations $\Gamma \rightarrow G$ whose restrictions to $\Gamma_p$ belong to some fixed class. Let $\sigma\in R$, then $\sigma|_{\Gamma_p}$ and $\sigma^{j_i}_{0,i}|_{\Gamma_p}$ are conjugate. We have a 1-cocycle $\alpha$ corresponding to $\sigma$, while $\sigma_{0,j}^{(i)}$ corresponds to the trivial 1-cocycle. Hence $\alpha|_{\Gamma_p}$ is conjugate to the trivial 1-cocycle, that is, $\alpha|_{\Gamma_p}$ is a 1-coboundary. 

% Suppose we are given a representation $\rho:\Gamma\rightarrow G$. Let $P<G$ be the minimal parabolic subgroup of $G$ containing $\rho(\Gamma)$ and let $g$ be the element of $G$ such that $P=g\cdot Q$ where $Q$ is one of the fixed parabolics $\{Q_i\}$. Let $\rho'=g\cdot\rho$.
% 
% Choose a Levi subgroup $L<Q$, so that $Q = R_u(Q)\rtimes L$. 
% 
% Since $Q$ is the minimal parabolic containing $\rho'(\Gamma)$, $\rho_0$ must be $L$-irreducible. For if $\rho_0(\Gamma)$ is contained in a proper parabolic subgroup of $L$ \ldots
% 
% Hence $\rho_0$ is also $L$-completely reducible. By [Theorem] there are only finitely many $L$-conjugacy classes of the given $\rho_0$.
% 
% So for each representation $\Gamma\rightarrow G$ we can choose a $Q$ from a finite set of parabolic subgroups of $G$, with corresponding Levi subgroup $L$, and $\rho_0':\Gamma\rightarrow G$ such that the $G$-conjugacy class containing the given representation has representative $\rho'$ \ldots



\begin{lemma}\label{kToHOne} Let $\Gamma$ be a finite (or algebraic) group and $G$ be an algebraic group over an algebraically closed field $k$ of characteristic $p$. Let $V_i = R_u(P_i)$ for each representative parabolic subgroup $P_i<G$. The answer to (the algebraic version of) K\"ulshammer's second question is positive if and only if the map
	\begin{displaymath}
		H^1(\Gamma, V_i) \rightarrow H^1(\Gamma_p, V_i)
	\end{displaymath}
	is injective.
\end{lemma}
\begin{proof}
	The proof.
\end{proof}

% \section{An algebraic group version}
% 
% In an attempt to gain further insight into (K. II) we adjust the original question by letting $\Gamma$ be an infinite group $H$. The advantage being that a negative answer in the algebraic group version may provide a negative answer to (K. II) by choosing an appropriate finite subgroup $\Gamma$ of $H$. In many of the examples to follow we set $H = SL_2(K)$ with Sylow $p$-subgroup $H_p = U_2(K)$ consisting of upper unitriangular matrices. 
% 
% Let $P \subset G$ be a parabolic subgroup and $L \subset P$ the corresponding Levi subgroup. Fix a representation $\rho_0:H\rightarrow L$. We can assume $\rho_0(H)$ is $L$-irreducible, that is, not contained in a proper parabolic of $L$. 
% 
% Now define $\rho_\alpha:H \rightarrow P$ by $\rho_\alpha(h) = \alpha(h)\rho_0(h)$ where $\alpha:H\rightarrow R_u(P)$, $R_u(P)$ the unipotent radical of $P$. 
% 
% For $\rho_\alpha$ to be a homomorphism
% \begin{eqnarray*}
% 	\alpha(h_1h_2)\rho_0(h_1h_2) &=& \alpha(h_1)\rho_0(h_1)\alpha(h_2)\rho_0(h_2) \\
% 	 &=& \alpha(h_1)\rho_0(h_1)\alpha(h_2)\rho_0(h_1)^{-1}\rho_0(h_1)\rho_0(h_2) \\
% 	 &=& \alpha(h_1)\rho_0(h_1)\alpha(h_2)\rho_0(h_1)^{-1}\rho_0(h_1h_2).
% \end{eqnarray*}
% That is $\alpha(h_1h_2) = \alpha(h_1) h_1\cdot\alpha(h_2)$, where the action $H \times R_u(P) \rightarrow R_u(P)$ is conjugation via $\rho_0$. This is a 1-cocycle condtion; $\alpha\in Z^1(H, R_u(P))$. $R_u(P)$ will not be abelian in general.
% 
% Now suppose $\rho_\alpha$ is $R_u(P)$-conjugate to some $\rho_\beta$, $\alpha, \beta \in Z^1(H, R_u(P))$. That is, there exists a $v \in R_u(P)$ such that for all $h\in H$
% \begin{eqnarray*}
% 	\alpha(h)\rho_0(h) &=& v\beta(h)\rho_0(h)v^{-1} \\
% 	&=& v\beta(h)\rho_0(h)v^{-1}\rho_0(h)^{-1}\rho_0(h).
% \end{eqnarray*}
% That is $\alpha(h) = v\beta(h)h\cdot v^{-1}$. In particular if $\rho_\alpha$ is $R_u(P)$-conjugate to $\rho_0$, that is $\beta$ is trivial, then $\alpha$ takes the form of a 1-coboundary. Generally speaking $\alpha$ and $\beta$ project to the same 1-cohomology class. In the abelian case this reads ``$\alpha$ and $\beta$ differ by a 1-coboundary'':
% \begin{eqnarray*}
% 	\alpha(h) = v\beta(h)h\cdot v^{-1} \quad\leadsto \quad\alpha(h) &=& v + \beta(h) - h\cdot v \\
% 	&=& \beta(h) + v - h\cdot v \\
% 	&=& \beta(h) + \chi_v(h).
% \end{eqnarray*}

% 3) Look at the nonreductive counterexample in Slodowy's paper on Kulshammer's problem.  What is special about the 3-dimensional U that makes this counterexample work?  Can you find similar structure in the unipotent radical of a reductive group?
