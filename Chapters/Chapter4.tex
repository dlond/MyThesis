%!TEX root = ../Thesis.tex
% Chapter 4

\chapter{K\"ulshammer's Second Problem}
\label{Chapter4}
\lhead{Chapter 4. \emph{K\"ulshammer's Second Problem}}

\section{K\"ulshammer's Second Problem}

Two questions were raised by B. K\"ulshammer concerning representations of a finite group $\Gamma$ into a linear algebraic group $G$ over an algebraically closed field $k$. 
\begin{itemize}
  \item[1.] \begin{quote} Let $\mathrm{char}(k)$ be prime to the order of $\Gamma$. Are there only finitely many representations $\rho:\Gamma\rightarrow G$ up to conjugation by $G$?
    \end{quote}
  \item[2.] \begin{quote} Let $p = \mathrm{char}(k)$ and $\Gamma_p < \Gamma$ be a Sylow $p$-subgroup. Fix a conjugacy class of representations from $\Gamma_p\rightarrow G$. Are there, up to conjugation by $G$, only finitely many representations $\rho:\Gamma\rightarrow G$ whose restrictions to $\Gamma_p$ belong to the given class?
    \end{quote}
\end{itemize}

As pointed out in the Introduction, the first has a positive answer and the second has positive answer so long as $G$ is reductive and the characteristic of $k$ is good for $G$. We wish to determine whether there exists a reductive counterexample to K\"ulshammer's second question. We also investigate the algebraic version of K\"ulshammer's second question and conclude this chapter with a necessary condition for a positive answer to both versions of K\"ulshammer's second question.

Throughout this Chapter we let $H,G$ be reductive linear algebraic groups, and $\Gamma$ a finite group.

\section{The Approach}

We are interested in knowing whether there can be infinitely many $G$-conjugacy classes of representations $\Gamma\rightarrow G$ that when restricted to $\Gamma_p$ hit some fixed $G$-conjugacy class of representations $\Gamma_p\rightarrow G$. A consequence of the following Theorem \cite[Theorem 1.2]{martin2003reductive} is that we will need to study representations into proper parabolic subgroups $P < G$.

\begin{theorem} \label{thm:finiteGCR} Let $\Gamma$ be a finite group There are only finitely many $G$-conjugacy classes of $G$-completely reducible representations $\Gamma\rightarrow G$.
\end{theorem}

So by Theorem \ref{thm:finiteGCR}, if we have infinitely many $G$-conjugacy classes of representations $\Gamma\rightarrow G$ then infinitely many of those classes must be of non-$G$-completely reducible representations. The following Lemma states that the finiteness of $G$-conjugacy classes of a collection of representations $\Gamma\rightarrow G$ carries over to $P$-conjugacy classes for any parabolic subgroup $P<G$ containing the image of the representations.

\begin{lemma} Let $R=\{\rho_\lambda:\Gamma\rightarrow P\, |\, \lambda \in \Lambda\}$ be a collection of representations indexed by the set $\Lambda$, $P$ a fixed parabolic subgroup of $G$. Then $R$ is contained in a finite union of $G$-conjugacy classes if and only if it is contained in a finite union of $P$-conjugacy classes.
  \label{lem:GPconj}
\end{lemma}
\begin{proof}
	Take two elements $\rho_\mu, \rho_\nu$ of $R$ in a particular $G$-conjugacy class. Then there exists an element $g\in G$ such that
	\begin{displaymath}
		g\cdot \rho_\mu = \rho_\nu.
	\end{displaymath}
	
	By definition $\rho_\nu(\Gamma) \subset P$, but on the other hand $\rho_\nu = g\cdot \rho_\mu$, therefore $\rho_\nu(\Gamma) \subset P \cap Q$.
	
	Let $T$ be a maximal torus of $G$ contained in $P\cap Q$ and let $\{n_1, \ldots, n_l\}$ be coset representatives for the Weyl group $W = N_G(T)/T$.
	
	Since $T$ and $gTg^{-1}$ are maximal tori of $Q$ they must be $Q$-conjugate, so there exists an element $q\in Q$ such that
	\begin{displaymath}
		qTq^{-1} = gTg^{-1}.
	\end{displaymath}
	By the definition of $Q$ there exists an element $p\in P$ such that $q = gpg^{-1}$, so in fact
	\begin{eqnarray*}
		gpg^{-1}Tgp^{-1}g^{-1} &=& gTg^{-1} \\
		\Rightarrow pg^{-1}Tgp^{-1} &=& T.
	\end{eqnarray*}
	We see that $gp^{-1}$ lies in $N_G(T)$. Let $n_i$ be the coset representative for the element of $W$ containing $gp^{-1}$ and let $t\in T$ be the element that satisfies
	\begin{displaymath}
		gp^{-1} = n_it.
	\end{displaymath}
	$T$ is a subgroup of $P$ so let $p^{-1}t^{-1} = p' \in P$ and we have
	\begin{eqnarray*}
		\rho_\mu &=& g^{-1}\cdot\rho_\nu\\
		&=& (p^{-1}t^{-1}n_i^{-1})\cdot\rho_\nu\\
		&=& p'\cdot(n_i^{-1}\cdot\rho_\nu). 
	\end{eqnarray*}
	Furthermore, as $\rho_\mu$ is an arbitrary element of $R\cap \left(G\cdot \rho_\nu\right)$ we have
	\begin{displaymath}
		R \cap \left(G\cdot \rho_\nu\right) \subset \bigcup_{i=1}^l P\cdot(n_i^{-1}\cdot\rho_\nu),
	\end{displaymath}
	where $l = |W|$.
	
	Therefore, a $G$-conjugacy class of $R$ is contained in a union of at most $l$ $P$-conjugacy classes. Thus it is clear that if $R$ is contained in a finite union of $G$-conjugacy classes then it is contained in a finite union of $P$-conjugacy classes.
	
	The converse is trivial.
\end{proof}

We direct the reader's attention to the following notation: For a given parabolic subgroup $P$ of $G$ with Levi subgroup $L$ and unipotent radical $V$, and a given representation $\rho:\Gamma\rightarrow P$ we have a map $\rho_L:\Gamma\rightarrow L$ defined by
\begin{eqnarray}
  \rho_L = \pi \circ \rho,
  \label{eqn:proj_l}
\end{eqnarray}
where $\pi$ is the projection $Q \rightarrow L$.

Now define $\alpha_\rho:\Gamma\rightarrow V$ by 
\begin{eqnarray*}
  \alpha_\rho(\gamma) = \rho(\gamma)\rho_L(\gamma)^{-1},
\end{eqnarray*}
for all $\gamma\in\Gamma$. Hence $\rho = \alpha_\rho\rho_L$.

$\Gamma$ acts on $V$ by conjugation via $\rho_L$.
\begin{eqnarray*}
	\alpha_\rho(\gamma_1\gamma_2)\rho_L(\gamma_1\gamma_2) \,=\, \rho(\gamma_1\gamma_2) 
		&=& \rho(\gamma_1)\rho(\gamma_2) \\
		&=& \alpha_\rho(\gamma_1)\rho_L(\gamma_1)\alpha_\rho(\gamma_2)\rho_L(\gamma_2) \\
		&=& \alpha_\rho(\gamma_1)\rho_L(\gamma_1)\alpha_\rho(\gamma_2)\rho_L(\gamma_1)^{-1}\rho_L(\gamma_1)\rho_L(\gamma_2)\\
		&=&\alpha_\rho(\gamma_1)\rho_L(\gamma_1)\alpha_\rho(\gamma_2)\rho_L(\gamma_1)^{-1}\rho_L(\gamma_1\gamma_2),
\end{eqnarray*}
so that
\begin{eqnarray*}
	\alpha_\rho(\gamma_1\gamma_2) &=&
	\alpha_\rho(\gamma_1)\rho_L(\gamma_1)\alpha_\rho(\gamma_2)\rho_L(\gamma_1)^{-1}\\
	&=& \alpha_\rho(\gamma_1)\,\left(\gamma_1\cdot\alpha_\rho(\gamma_2)\right),
\end{eqnarray*}
where $\Gamma$ acts on $V$ by conjugation via $\rho_L$. Therefore $\alpha_\rho$ satisfies the 1-cocycle condition (Equation \ref{eqn:na_z}) and so $\alpha_\rho$ is a 1-cocycle from $\Gamma \rightarrow V$. At this point we make a change to our notation in previous Chapters to make it explicit that the action of $\Gamma$ on $V$ depends on $\rho_L$ and write $\alpha_\rho \in Z^1(\Gamma, \rho_L, V)$.

Conversely given a 1-cocycle $\alpha\in Z^1(\Gamma, \rho_L, V)$ we can construct a map $\rho:\Gamma\rightarrow P$ by $\rho(\gamma) = \alpha(\gamma)\rho_L(\gamma)$ for all $\gamma\in \Gamma$. The construction is a homomorphism from $\Gamma \rightarrow P$, for take $\gamma_1, \gamma_2 \in \Gamma$:
\begin{eqnarray*}
  \rho(\gamma_1 \gamma_2) &=& \alpha(\gamma_1 \gamma_2) \rho_L(\gamma_1 \gamma_2) \\
  &=& \alpha(\gamma_1)(\gamma_1 \cdot \alpha(\gamma_2)) \rho_L(\gamma_1) \rho_L(\gamma_2) \\
  &=& \alpha(\gamma_1) \rho_L(\gamma_1) \alpha(\gamma_2) \rho_L(\gamma_1)^{-1} \rho_L(\gamma_1) \rho_L(\gamma_2) \\
  &=& \alpha(\gamma_1) \rho_L(\gamma_1) \alpha(\gamma_2) \rho_L(\gamma_2) \\
  &=& \rho(\gamma_1) \rho(\gamma_2).
\end{eqnarray*}

Given a representation $\rho:\Gamma\rightarrow P$, define $Hom(\Gamma, P)_{\rho_L}$ to be the set of representations $\sigma:\Gamma\rightarrow P$ such that $\sigma_L = \rho_L$. We formalise the above findings in the following Lemma:

\begin{lemma}
  The map $h:Hom(\Gamma, P)_{\rho_L} \rightarrow Z^1(\Gamma, \rho_L, V)$ defined by
  \begin{displaymath}
    (h(\sigma))(\gamma) = \sigma(\gamma)\rho_L(\gamma)^{-1},
  \end{displaymath}
  is bijective.
  \label{lem:hom_z1}
\end{lemma}

For ease of notation we will often write $h(\sigma)$ as $\alpha_\sigma$. Also, since $h$ is bijective we do no harm to use the otherwise suggestive notation $\alpha_\sigma$ when picking elements from $Z^1(\Gamma, \rho_L, V)$.

Let $v \in V$ and $\sigma \in Hom(\Gamma, P)_{\rho_L}$. Since $\pi$ kills $V$, $\pi \circ (v \cdot \sigma) = \sigma_L = \rho_L$ and so $v \cdot \sigma \in Hom(\Gamma, P)_{\rho_L}$. Thus $V$ acts on $Hom(\Gamma, P)_{\rho_L}$.

Denote by $\overline{\sigma}$ an element of $Hom(\Gamma, P)_{\rho_L}/V$ with a representative element $\sigma \in Hom(\Gamma, P)_{\rho_L}$ and $[\alpha_\sigma]$ an element of $H^1(\Gamma, \rho_L, V)$ with a representative element $\alpha_\sigma \in Z^1(\Gamma, \rho_L, V)$. We show that $h$ gives rise to a bijection $\overline{h}: Hom(\Gamma,P)_{\rho_L}/V\rightarrow H^{1}(\Gamma, \rho_L, V)$ defined by
\begin{displaymath}
  \overline{h}(\overline{\sigma}) = [h(\sigma)] = [\alpha_\sigma],
\end{displaymath}
for all $\overline{\sigma} \in Hom(\Gamma, P)_{\rho_L}/V$.

\begin{lemma}
  The following diagram is commutative:
  \begin{displaymath}
    \xymatrix{
    Hom(\Gamma, P)_{\rho_L} \ar[r]^h \ar[d] & Z^{1}(\Gamma, \rho_L, V) \ar[d] \\
    Hom(\Gamma, P)_{\rho_L}/V \ar[r]^{\overline{h}} & H^{1}(\Gamma, \rho_L, V).
    }
  \end{displaymath}
  Furthermore, $\overline{h}$ is bijective.
  \label{lem:v_h1}
\end{lemma}
\begin{proof}  
  Suppose $\overline{\sigma} = \overline{\tau}$ for some $\sigma,\tau\in Hom(\Gamma, P)_{\rho_L}$, that is to say $\sigma = v\cdot\tau$ for some $v\in V$. Then for all $\gamma\in \Gamma$
  \begin{eqnarray*}
    \alpha_\sigma(\gamma) &=& \sigma(\gamma)\rho_L(\gamma)^{-1}\\
    &=& v\tau(\gamma)v^{-1}\rho_L(\gamma)^{-1}\\
    &=& v\tau(\gamma)\rho_L(\gamma)^{-1}\rho_L(\gamma)v^{-1}\rho_L(\gamma)^{-1}\\
    &=& v\tau(\gamma)\rho_L(\gamma)^{-1}(\gamma\cdot v^{-1})\\
    &=& v\alpha_\tau(\gamma)(\gamma\cdot v^{-1}).
  \end{eqnarray*}
  Hence $[h(\sigma)] = [h(\tau)]$ and so $\overline{h}$ is well-defined. 
  
  Since $h$ is onto and $Z^1(\Gamma, \rho_L, V) \rightarrow H^1(\Gamma, \rho_L, V)$ is onto, so is $\overline{h}$.
  
  Now suppose $[\alpha_\sigma] = [\alpha_\tau]$ for some $\alpha_\sigma, \alpha_\tau \in Z^1(\Gamma, \rho_L, V)$. Then there exists $v \in V$ such that
  \begin{displaymath}
    \alpha_\sigma(\gamma) = v \alpha_\tau(\gamma)(\gamma \cdot v^{-1}),
  \end{displaymath}
  for all $\gamma \in \Gamma$.

  Therefore corresponding representations $\sigma, \tau \in Hom(\Gamma, P)_{\rho_L}$ are $V$-conjugate:
  \begin{eqnarray*}
    \sigma(\gamma) &=&  \alpha_\sigma(\gamma)\rho_L(\gamma) \\
    &=& v \alpha_\tau(\gamma)(\gamma \cdot v^{-1}) \rho_L(\gamma) \\
    &=& v \alpha_\tau(\gamma)\rho_L(\gamma) v^{-1} \rho_L(\gamma)^{-1} \rho_L(\gamma) \\
    &=& v \tau(\gamma) v^{-1} \\
    &=& (v \cdot \tau)(\gamma).
  \end{eqnarray*}
  That is to say $\overline{\sigma} = \overline{\tau}$ and so $\overline{h}$ is bijective.
\end{proof}

More generally, we can conjugate $\sigma\in Hom(\Gamma, P)_{\rho_L}$ by $g \in G$ to get an element 
\begin{displaymath}
  g \cdot \sigma \in Hom(\Gamma, gPg^{-1})_{g\cdot\rho_L},
\end{displaymath}
and
\begin{displaymath}
  \alpha_{g \cdot \sigma} \in Z^1(\Gamma, g\cdot\rho_L, gVg^{-1}),
\end{displaymath}
under $h$.

If $g\in P$ then $g=vl$ for some $v\in V$ and some $l\in L$, and since $gPg^{-1} = P$ and $gVg^{-1} = V$, conjugating gives rise to the maps
\begin{displaymath}
  Hom(\Gamma, P)_{\rho_L} \rightarrow Hom(\Gamma, P)_{l\cdot\rho_L},
\end{displaymath}
and
\begin{displaymath}
  Z^1(\Gamma, \rho_L, V)\rightarrow Z^1(\Gamma, l\cdot\rho_L, V),
\end{displaymath}
again, via $h$.

Furthermore, if $l\in Z(L)^\circ$ then $l\cdot\rho_L = \rho_L$. Indeed $Z(L)^\circ$ acts on $Hom(\Gamma, P)_{\rho_L}$ and on $Z^1(\Gamma, \rho_L, V)$ in the following way
\begin{eqnarray*}
  (z \cdot \sigma) (\gamma) &=& z \sigma(\gamma) z^{-1} \\
  (z \cdot \alpha_\sigma) (\gamma) &=&  z \alpha_\sigma(\gamma) z^{-1},
\end{eqnarray*}
and $h$ is $Z(L)^\circ$-equivariant:
\begin{eqnarray*}
  h(z \cdot \sigma)(\gamma) &=&  z \sigma(\gamma) z^{-1} \rho_L(\gamma)^{-1} \\
  &=& z \sigma(\gamma) \rho_L(\gamma)^{-1} z^{-1} \\
  &=& (z \cdot h(\sigma))(\gamma),
\end{eqnarray*}
that is
\begin{eqnarray}
  \alpha_{z \cdot \sigma} &=& z \cdot \alpha_\sigma,
  \label{eqn:h_z_equivar}
\end{eqnarray}
for all $\alpha_\sigma \in Z^1(\Gamma, \rho_L, V)$ and all $z \in Z(L)^\circ$.

We show that the $Z(L)^\circ$-action on $Hom(\Gamma, P)_{\rho_L}$ and $Z^1(\Gamma, \rho_L, V)$ descends to give a $Z(L)^\circ$-action on $Hom(\Gamma, P)_{\rho_L}/V$ and $H^1(\Gamma, \rho_L, V)$, respectively. The actions will be well-defined as a consequence of the fact that $L$ normalizes $V$.

Let $z \in Z(L)^\circ$ and $\overline{\sigma} \in Hom(\Gamma, P)_{\rho_L}/V$. We define the $Z(L)^\circ$-action on $Hom(\Gamma, P)_{\rho_L}/V$ so that the projection $Hom(\Gamma, P)_{\rho_L} \rightarrow Hom(\Gamma, P)_{\rho_L}/V$ is $Z(L)^\circ$-equivariant:
\begin{eqnarray}
  z \cdot \overline{\sigma} = \overline{z \cdot \sigma}.
  \label{eqn:z_act_rv}
\end{eqnarray}

Suppose $\overline{\sigma} = \overline{\tau}$. Then there is $v \in V$ such that $\sigma = v \cdot \tau \in Hom(\Gamma, P)_{\rho_L}$, and $v' \in V$ such that $zv = v' z$. Therefore
\begin{eqnarray*}
  \overline{z \cdot \sigma} &=& \overline{zv \cdot \tau} \\
  &=& \overline{v' z \cdot \tau} \\
  &=& \overline{z \cdot \tau}.
\end{eqnarray*}

Hence the $Z(L)^\circ$-action on $Hom(\Gamma, P)_{\rho_L}/V$ in Equation \ref{eqn:z_act_rv} is well-defined.

Similarly, the $Z(L)^\circ$-action on $H^1(\Gamma, \rho_L, V)$ is defined so that the projection to the 1-cohomology $Z^1(\Gamma, \rho_L, V) \rightarrow H^1(\Gamma, \rho_L, V)$ is $Z(L)^\circ$-equivariant:
\begin{eqnarray}
  z \cdot [\alpha_\sigma] = [z \cdot \alpha_\sigma].
  \label{eqn:z_act_h}
\end{eqnarray}

Suppose $[\alpha_\sigma] = [\alpha_\tau]$. Then there is $v \in V$ such that for all $\gamma \in \Gamma$,
\begin{displaymath}
  \alpha_\sigma(\gamma) = v \alpha_\tau(\gamma) (\gamma \cdot v^{-1}),
\end{displaymath}
and there is $v' \in V$ such that $zv = v' z$. Therefore, for all $\gamma \in \Gamma$
\begin{eqnarray*}
	(z \cdot \alpha_\sigma)(\gamma) 
  &=& z v \alpha_\tau(\gamma) (\gamma \cdot v^{-1}) z^{-1} \\
  &=& z v \alpha_\tau(\gamma) \rho_L(\gamma) v^{-1} \rho_L(\gamma)^{-1} z^{-1} \\
  &=& v' z \alpha_\tau(\gamma) z^{-1} \rho_L(\gamma) v'^{-1} \rho_L(\gamma)^{-1} \\
  &=& v' z \alpha_\tau(\gamma) z^{-1} (\gamma \cdot v'^{-1}) \\
  &=& v' ((z \cdot \alpha_\tau)(\gamma)) (\gamma \cdot v'^{-1}).
\end{eqnarray*}

Therefore $[z \cdot \alpha_\sigma] = [z \cdot \alpha_\tau]$ and the $Z(L)^\circ$-action on $H^1(\Gamma, \rho_L, V)$ in Equation \ref{eqn:z_act_h} is well-defined.

Since $h$ is $Z(L)^\circ$-equivariant, it follows that $\overline{h}$ is also:
\begin{eqnarray*}
  \overline{h}(z \cdot \overline{\sigma}) &=& \overline{h}(\overline{z \cdot \sigma}) \\
  &=& [\alpha_{z \cdot \sigma}] \\
  &=& [z \cdot \alpha_\sigma] \qquad (\textrm{Equation }\ref{eqn:h_z_equivar}) \\
  &=& z \cdot [\alpha_\sigma] \\
  &=& z \cdot \overline{h}(\overline{\sigma}).
\end{eqnarray*}

In summary we have a well-defined $Z(L)^\circ$-action on $Hom(\Gamma, P)_{\rho_L}/V$ and $H^1(\Gamma, \rho_L, V)$, and a $Z(L)^\circ$-equivariant bijection 
\begin{displaymath}
  \overline{h}:Hom(\Gamma, P)_{\rho_L}/V \rightarrow H^1(\Gamma, \rho_L, V).
\end{displaymath}

Hence the following Lemma:

\begin{lemma}
  The bijection $h: Hom(\Gamma, P)_{\rho_L} \rightarrow Z^1(\Gamma, \rho_L, V)$ gives rise to a bijection
  \begin{displaymath}
    \tilde{h}: Hom(\Gamma, P)_{\rho_L}/VZ(L)^\circ \rightarrow H^1(\Gamma, \rho_L, V)/Z(L)^\circ.
  \end{displaymath}
  \label{lem:vzl_h1zl}
\end{lemma}
\begin{proof}
  It remains to show that $\left(Hom(\Gamma, P)_{\rho_L}/V\right)/Z(L)^\circ \simeq Hom(\Gamma, P)_{\rho_L}/VZ(L)^\circ$.

  Denote by $\pi_Z$ the canonical projection from $Hom(\Gamma, P)_{\rho_L}/V \rightarrow \left( Hom(\Gamma, P)_{\rho_L}/V \right)/Z(L)^\circ$ and $\tilde{\sigma}$ the element of $Hom(\Gamma, P)_{\rho_L}/VZ(L)^\circ$ with representative $\sigma \in Hom(\Gamma, P)_{\rho_L}$. We show that the map
  \begin{displaymath}
    \varphi: (Hom(\Gamma, P)_{\rho_L}/V)/Z(L)^\circ \rightarrow Hom(\Gamma, P)_{\rho_L}/VZ(L)^\circ,
  \end{displaymath}
  defined by
  \begin{displaymath}
    \varphi(\pi_Z(\overline{\sigma})) = \tilde{\sigma},
  \end{displaymath}
  is well defined and a bijection. To this end consider the following statements:
  \begin{eqnarray*}
    && \pi_Z(\overline{\sigma}) = \pi_Z(\overline{\tau}) \\
    &\Longleftrightarrow& \textrm{there exists } z \in Z(L)^\circ \textrm{ such that } \overline{\sigma} = z \cdot \overline{\tau} \\
    &\Longleftrightarrow& \textrm{there exists } z \in Z(L)^\circ \textrm{ such that } \overline{\sigma} = \overline{z \cdot \tau} \qquad (\textrm{Equation } \ref{eqn:z_act_h}) \\
    &\Longleftrightarrow& \textrm{there exists } v \in V, z \in Z(L)^\circ \textrm{ such that } \sigma = v \cdot (z \cdot \tau) \\
    &\Longleftrightarrow& \textrm{there exists } v \in V, z \in Z(L)^\circ \textrm{ such that } \sigma = (vz) \cdot \tau \\
    &\Longleftrightarrow& \tilde{\sigma} = \tilde{\tau}.
  \end{eqnarray*}
  Hence the forward direction implies that $\varphi$ is well-defined and the reverse direction implies that $\varphi$ is injective. Surjectivity follows from the fact that the canonical projections:
  \begin{displaymath}
    Hom(\Gamma, P)_{\rho_L} \rightarrow Hom(\Gamma, P)_{\rho_L}/V \rightarrow (Hom(\Gamma, P)_{\rho_L}/V)/Z(L)^\circ, 
  \end{displaymath}
  and
  \begin{displaymath}
    Hom(\Gamma, P)_{\rho_L} \rightarrow Hom(\Gamma, P)_{\rho_L}/VZ(L)^\circ, 
  \end{displaymath}
  are surjective.
\end{proof}

Let $R = \{ \rho_\lambda : \Gamma \rightarrow G \,|\, \lambda \in \Lambda \}$ be a collection of representations indexed by the set $\Lambda$. For $\rho \in R$, we say that a parabolic subgroup $P$ of $G$ is $\rho$-minimal if $P$ is minimal among the parabolic subgroups of $G$ that contain $\rho(\Gamma)$. For a parabolic subgroup $P < G$ define
\begin{displaymath}
  R_P = \{ \rho \in R \,|\, P \textrm{ is } \rho\textrm{-minimal} \}.
\end{displaymath}

Since there are only finitely many $G$-conjugacy classes of parabolic subgroups of $G$ (a standard result, e.g. \cite[Theorem 30.1(a)]{humphreys1975linear}), we can choose a finite set of representative parabolic subgroups $\{Q_i\}_{i = 0}^n$ of $G$ such that every parabolic subgroup $P < G$ is $G$-conjugate to precisely one $Q_i$. Every $\rho \in R$ has a minimal parabolic subgroup and so there exists an element of each $G$-conjugacy class in $R$ with minimal parabolic $Q_i$ for some $i$, hence
\begin{eqnarray}
  G \cdot R = \bigcup_i G \cdot R_{Q_i}.
  \label{eqn:gr_gqi}
\end{eqnarray}

Furthermore, fix a particular $Q_i$ with Levi subgroup $M_i$. Since $Q_i$ is minimal for each $\rho \in R_{Q_i}$, $\rho_{M_i}$ (Equation \ref{eqn:proj_l}) is $M_i$-irreducible \cite[Lemma 6.2(ii)]{bate2005geometric}.

For an $M_i$-irreducible representation $\sigma : \Gamma \rightarrow M_i$ define
\begin{displaymath}
  R_{\sigma} = \{ \rho \in R_{Q_i} \,|\, \rho_{M_i} = \sigma \}.
\end{displaymath}

Since there are only finitely many $M_i$-conjugacy classes of $M_i$-irreducible representations $\Gamma \rightarrow M_i$ (Theorem \ref{thm:finiteGCR}), we can choose a finite set of representative $M_i$-irreducible representations $\{\sigma_i^j : \Gamma \rightarrow M_i \}_{j=0}^{n_i}$, such that every $M_i$-irreducible representation $\Gamma \rightarrow M_i$ is $M_i$-conjugate to precisely one $\sigma_i^j$. Every $M_i$-conjugacy class in $R_{Q_i}$ has an element $\rho$ such that $\rho_{M_i} = \sigma_i^j$ for some $j$, hence
\begin{displaymath}
  M_i \cdot R_{Q_i} = \bigcup_j M_i \cdot R_{\sigma_i^j},
\end{displaymath}
and therefore
\begin{eqnarray}
  G \cdot R = \bigcup_i G \cdot R_{Q_i} = \bigcup_i \bigcup_j G \cdot R_{\sigma_i^j}.
  \label{eqn:gr_grsigma}
\end{eqnarray}

Fix an $M_i$-irreducible representation $\sigma: \Gamma \rightarrow M_i$. We have a map from $Hom(\Gamma, P_i)_\sigma \rightarrow Hom(\Gamma, P_i)_\sigma / V_i Z(M_i)^\circ$ given by the canonical projection and a map $\tilde{h}: Hom(\Gamma, P_i)_\sigma / V_i Z(M_i)^\circ \rightarrow H^1(\Gamma, \sigma, V_i)/Z(M_i)^\circ$. We define the map
\begin{displaymath}
  \mathcal{H}: Hom(\Gamma, P_i)_\sigma \rightarrow H^1(\Gamma, \sigma, V_i) / Z(M_i)^\circ
\end{displaymath}
to be the composition of the above canonical projection with $\tilde{h}$. That is, $\mathcal{H}(\rho) = \tilde{h}(\tilde{\rho})$ for all $\rho \in Hom(\Gamma, P_i)_\sigma$, where $\tilde{\rho}$ is the projection of $\rho$ to $Hom(\Gamma, P_i)_\sigma / V_i Z(M_i)^\circ$.

We note that each subset $R_{\sigma_i^j} \subset R_{Q_i}$ is a $VZ(M_i)^\circ$-stable subset of $Hom(\Gamma, Q_i)_{\sigma_i^j}$, so it makes sense to calculate
\begin{displaymath}
  \mathcal{H}(R_{\sigma_i^j}) \subset H^1(\Gamma, \sigma_i^j, V_i) / Z(M_i)^\circ.
\end{displaymath}

\begin{lemma}
  Let $R_P = \{\rho_\lambda:\Gamma\rightarrow P\,|\,\lambda\in\Lambda\}$ be a collection of representations such that $P$ $\rho_\lambda$-minimal for each $\rho_\lambda$.
  
  The following statements are equivalent:
  \begin{itemize}
    \item[(i)] $R_P$ is contained in a finite union of $P$-conjugacy classes.
    \item[(ii)] For each irreducible representation $\sigma:\Gamma\rightarrow L$, $R_{\sigma}$ is contained in a finite union of $VZ(L)^\circ$-conjugacy classes.
    \item[(iii)] For each irreducible representation $\sigma:\Gamma\rightarrow L$,
      \begin{displaymath}
	\mathcal{H}(R_{\sigma}) \subset H^{1}(\Gamma,\sigma,V)/Z(L)^\circ
      \end{displaymath}
      is finite.
  \end{itemize}
  \label{lem:p_h1}
\end{lemma}
\begin{proof}\quad

  $(i) \Rightarrow (ii)$ Assume $R_P$ is contained in a finite union of $P$-conjugacy classes and fix an irreducible representation $\sigma : \Gamma \rightarrow L$. Then $R_{\sigma}$ is contained a finite union of $P$-conjugacy classes. Take $\rho \in R_{\sigma}$ and suppose that $p \cdot \rho \in R_{\sigma}$ for some $p \in P$. Writing $p = vl$ for some $v \in V$ and some $l \in L$, $(vl) \cdot \rho \in R_{\sigma}$ implies that in fact $l \in C_L(\sigma(\Gamma))$. Furthermore, since $\sigma$ is irreducible it follows that $C_L(\sigma(\Gamma))/Z(L)^\circ$ is finite \cite[Lemma 6.2]{martin2003reductive}, so we can choose a finite set $\{c_1, \ldots, c_m\}$ of coset representatives for $Z(L)^\circ\backslash C_L(\sigma(\Gamma))$. Therefore
  \begin{displaymath}
    R_{\sigma} \cap (P \cdot \rho) \subset \bigcup_{i = 1}^{m} VZ(L)^\circ \cdot \left( c_i \cdot \rho \right).
  \end{displaymath}
  Since $R_{\sigma}$ is contained in a finite number of $P$-conjugacy classes, we are done.

  $(ii) \Rightarrow (i)$ Assume that for each irreducible representation $\sigma : \Gamma \rightarrow L$, $R_{\sigma}$ is contained in a finite union of $VZ(L)^\circ$-conjugacy classes, so for each $\sigma$ there is a finite set $\Phi^\sigma \subset R_P$ such that
  \begin{displaymath}
    R_\sigma \subset \bigcup_{\phi \in \Phi^\sigma} VZ(L)^\circ \cdot \phi.
  \end{displaymath}
  We do no harm to assume that $R_P = P \cdot R_P$. Denote by $Hom(\Gamma, L)_{irr}$ the collection of all irreducible representations from $\Gamma \rightarrow L$. By Theorem \ref{thm:finiteGCR} we can choose a finite set $\Sigma \subset Hom(\Gamma, L)_{irr}$ such that
  \begin{displaymath}
    Hom(\Gamma, L)_{irr} = \bigcup_{\sigma \in \Sigma} L \cdot \sigma.
  \end{displaymath}
  For each $\sigma \in \Sigma$ define 
  \begin{displaymath}
    R_{L \cdot \sigma} = \bigcup_{l \in L} R_{l \cdot \sigma}.
  \end{displaymath}
  Since $P$ is minimal for each $\rho \in R_P$, $\rho_L$ is $L$-irreducible. Hence
  \begin{displaymath}
    R_P = \bigcup_{\sigma \in \Sigma} R_{L \cdot \sigma}.
  \end{displaymath}

  Suppose $\rho \in R_{L \cdot \sigma}$. Then there exists $l \in L$ such that $\rho_L = l \cdot \sigma$, so that $l^{-1} \cdot \rho_L = \sigma$. Since we assumed $R_P = P \cdot R_P$, $l^{-1} \cdot \rho \in R_P$ and therefore $l^{-1} \cdot \rho \in R_\sigma$. Conversely if $\rho \in L \cdot R_\sigma$ then $l \cdot \rho \in R_{L \cdot \sigma}$ for some $l \in L$. Hence
  \begin{displaymath}
    R_{L \cdot \sigma} = L \cdot R_\sigma.
  \end{displaymath}

  Therefore
  \begin{displaymath}
    R_P = \bigcup_{\sigma \in \Sigma} L \cdot R_\sigma \subset \bigcup_{\sigma \in \Sigma}\bigcup_{\phi \in \Phi^\sigma} LVZ(L)^\circ \cdot \phi = \bigcup_{\sigma \in \Sigma}\bigcup_{\phi \in \Phi^\sigma} P \cdot \phi.
  \end{displaymath}

  $(ii) \Leftrightarrow (iii)$ This follows directly from the fact that $\tilde{h}$ is a bijection (Lemma \ref{lem:vzl_h1zl}).

\end{proof}

We are now ready to state precisely the connection with $G$-conjugacy classes of representations and the 1-cohomology.

\begin{theorem}
  Let $R=\{\rho_\lambda:\Gamma\rightarrow G\,|\,\lambda \in \Lambda\}$ be a collection of representations indexed by the set $\Lambda$. 
  
  Suppose $R = G \cdot R$. Then $R$ is a finite union of $G$-conjugacy classes if and only if for each $i, j$ the subset $\mathcal{H}(R_{\sigma_i^j}) \subset H^1(\Gamma, \sigma_i^j, V_i) / Z(M_i)^\circ$ is finite.
  \label{thm:g_h1}
\end{theorem}
\begin{proof}
  Assume $R$ is a finite union of $G$-conjugacy classes. Then for each $Q_i$, $R_{Q_i}$ is contained in a finite union of $G$-conjugacy classes. By Lemma \ref{lem:GPconj} $R_{Q_i}$ is contained in a finite union of $Q_i$-conjugacy classes, and by Lemma \ref{lem:p_h1} $\tilde{h}(R_{\sigma_i^j}/VZ(M_i)^\circ)$ is finite for each $j$.

  Conversely, suppose that for each $i$, for each $j$, $\tilde{h}(R_{\sigma_i^j}/VZ(M_i)^\circ)$ is finite. Then by Lemma \ref{lem:p_h1}, each $R_{Q_i}$ is contained in a finite union of $Q_i$-conjugacy classes. Since
  \begin{displaymath}
    G \cdot R = \bigcup_i G \cdot R_{Q_i} \qquad (\textrm{Equation }\ref{eqn:gr_gqi})
  \end{displaymath}
  we are done.
\end{proof}

\begin{theorem}
  Let $G$ be an algebraic group over an algebraically closed field $k$ of characteristic $p$, $\Gamma$ a finite group and $\Gamma_p < \Gamma$ a Sylow $p$-subgroup. Define $M_i < Q_i < G$ and $\sigma_i^{j}:\Gamma \rightarrow_{irr} M_{i}$ as above. Let $\iota$ be the inclusion map $\iota : \Gamma_p \rightarrow \Gamma$ and for each $i,j$ let $Z^1(\iota), H^1(\iota)$ be the corresponding 1-cocycle and 1-cohomology restriction maps
  \begin{displaymath}
    Z^1(\iota) : Z^1(\Gamma, \sigma_i^j, V_i) \rightarrow Z^1(\Gamma_p, \sigma_i^j, V_i),
  \end{displaymath}
  and
  \begin{displaymath}
    H^1(\iota) : H^1(\Gamma, \sigma_i^j, V_i) \rightarrow H^1(\Gamma_p, \sigma_i^j, V_i).
  \end{displaymath}
  
  The answer to K\"ulshammer's second question for $\Gamma, G$ is positive only if $H^1(\iota)$ is injective for each $i,j$.
  \label{thm:k2_h1}
\end{theorem}
\begin{proof}
  Fix a representation $\rho_0: \Gamma_p \rightarrow G$ and let $X = G \cdot \rho_0$. For a map $\varphi$ from $\Gamma$ denote by $\varphi^p$ its restriction to $\Gamma_p$. Let
  \begin{eqnarray*}
    R &=&  \{ \rho \in \textrm{Hom}(\Gamma, G) \,|\, \rho^p \in X\}, \\
    R_{Q_i} &=&  \{ \rho \in R \,|\, Q_i \textrm{ is } \rho \textrm{-minimal} \}, \\
    R_{\sigma_i^j} &=&  \{ \rho \in R_{Q_i} \,|\, \rho_{M_i} = \sigma_i^j \},
  \end{eqnarray*}
  as done previously.
  
  Fix $g \in G$ and consider the set $R_j^p$ defined
  \begin{displaymath}
    R_j^p = \{ \rho \in R_{\sigma_i^j} \,|\, \rho^p = g \cdot \rho_0 \}.
  \end{displaymath}

  Then
  \begin{eqnarray}
    R_{\sigma_i^j} \subset G \cdot R_j^p.
    \label{eqn:rs_grp}
  \end{eqnarray}

  Assume $H^1(\iota)$ is injective. Since $Z^1(\iota)(h(\rho))$ is equal to a fixed 1-cocycle for all $\rho \in R_j^p$ the image of $R_j^p$ in $H^1(\Gamma_p, \sigma_i^j, V_i)$ is finite, hence the image of $R_j^p$ in $H^1(\Gamma, \sigma_i^j, V_i)$ is finite. 
  
  Therefore, the image of $R_j^p$ in $H^1(\Gamma, \sigma_i^j, V_i)/Z(M_i)^\circ$ is finite. By Lemma \ref{lem:p_h1}, $R_j^p$ is contained in a finite union of $Q_i$-conjugacy classes, hence by Equation \ref{eqn:rs_grp}, $R_{\sigma_i^j}$ is contained in a finite union of $G$-conjugacy classes.

  So by Equation \ref{eqn:gr_grsigma}, $G \cdot R = R$ is contained in a finite union of $G$-conjugacy classes. Therefore the answer to K\"ulshammer's second question for $\Gamma, G$ is positive.
\end{proof}

