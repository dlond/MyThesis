%!TEX root = ../Thesis.tex
% Chapter 7

\chapter{Future work}
\label{Chapter7}
\lhead{Chapter 7. \emph{Future work}}

Two major open questions remain: the first is an extension of Lemma \ref{lem:sl2_b_inj}
\begin{quote}
  Is it true that $H^1(SL_2(k), V) \rightarrow H^1(U(k), V)$ is injective, where $U$ is the unipotent radical of $SL_2(k)$ and $V$ is an algebraic group on which $SL_2(k)$ acts?
\end{quote}

In view of Lemma \ref{lem:sl2_b_inj} we need only consider whether $H^1(B, V) \rightarrow H^1(U, V)$ is injective, $B$ a Borel subgroup of $SL_2(k)$. There is evidence for this conjecture in our calculations in Chapter \ref{Chapter6}.

If the first were true, we would further investigate the results in Chapter \ref{Chapter5} which culminate in verifying that under certain conditions 1-cocycles from $Z^1(SL_2(k), V)$ which are trivial on a maximal torus $T$ have image lying in a product of commuting root groups of $V$. Although $V$ is rarely abelian, could we use the fact that their image is abelian to show that $H^1(SL_2(k), V) \rightarrow H^1(U, V)$ is injective?

If the answer to these two open questions turns out to be true, then constructing an argument similar to Theorem \ref{thm:k2_h1} for the algebraic analogue of K\"ulshammer's second question would show that the answer is positive for $SL_2(k)$ and reductive $G$.
