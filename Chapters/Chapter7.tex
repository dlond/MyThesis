%!TEX root = ../Thesis.tex
% Chapter 7

\chapter{Future work}
\label{Chapter7}
\lhead{Chapter 7. \emph{Future work}}

One direction to pursue would be the further investigation of the $B_4$ calculation in Section \ref{b4}. The discovery of infinitely many embeddings of $SL_2(k)$ in $B_4$ is encouraging as a potential counterexample to the algebraic version of K\"ulshammer's second question.

It would also be interesting to prove or disprove Conjecture \ref{bigclaim}, perhaps for arbitrary parabolics $P$. This would provide a formula for candidate 1-cocycles from which we can use to calculate the nonabelian 1-cohomology by direct computation or computer program, following the examples of Chapter \ref{Chapter6}.

We would also like to fully explore the consequences of Lemma \ref{lem:first} with regards to the restriction of 1-cohomologies
\begin{align*}
H^1(SL_2(k), V_\alpha)_{\omega_r} \rightarrow H^1(U_2(k), V_\alpha)_{\omega_r},
\end{align*}
where we may be able to use the fact that $\sigma$ lies in an abelian subgroup of $V_\alpha$ to show that the map is injective (cf. Example \ref{eg:sl2ab}).

A further course of interest is the thorough examination of Cram's non-reductive counterexample to K\"ulshammer's second question (\cite[Appendix]{slodowy1997two}). We may be able to exploit certain properties of Cram's counterexample to construct a reductive counterexample, or alternatively, provide some insight towards a proof that there is no reductive counterexample.
