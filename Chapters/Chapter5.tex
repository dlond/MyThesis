%!TEX root = ../Thesis.tex
% Chapter 5

\chapter{1-Cohomology Calculations: Theoretical Results}
\label{Chapter5}
\lhead{Chapter 5. \emph{1-Cohomology Calculation: Theoretical Results}}

In this Chapter we present some theoretical results for 1-cohomology calculations of the form $H^1(SL_2(k), V_\alpha)_{\rho_r}$ where $V_\alpha$ is the unipotent radical of a minimal parabolic subgroup of reductive $G$.

Let $G$ be a reductive group over an algebraically closed field $k$ of characteristic $p>0$, and let $B$ be a Borel subgroup of $G$ containing a maximal torus $T$ of $G$. This amounts to fixing a base $\Delta$ of $\Phi$, where $\Phi$ is the root system for $G$. Then $\Phi$ yields a set of representative parabolic subgroups $P_I$, $I \subset \Delta$ as in \cite[\S 30]{humphreys1975linear}. Denote by $\Phi^+$ the positive roots of $G$.

\section{Minimal Parabolic Subgroups}

Let $P_\alpha<G$ be the parabolic subgroup of $G$ corresponding to the simple root $\alpha\in\Delta$, with Levi subgroup $L_\alpha$ and unipotent radical $V_\alpha$.
%\begin{align*}
%V_\alpha=R_u(P_\alpha) &= \langle U_\delta \,|\, \delta \in \Phi^+ - \{\alpha\} \rangle,\\
%P_\alpha &= V_\alpha \rtimes L_\alpha.
%\end{align*}
We fix an ordering on the roots so that products such as
\begin{align*}
V_\alpha = \prod_{\delta\in\Phi^+ -\{\alpha\}} U_\delta,
\end{align*}
are unambiguous.

There exists a homomorphism $\rho_0$ from $ SL_2(k)$ into $L_\alpha$ such that
\begin{align*}
\rho_0 \left(\begin{matrix} 1 &  u \\ 0 & 1 \end{matrix} \right) &= \epsilon_\alpha(u), \\
\rho_0 \left(\begin{matrix} 1 & 0 \\ u & 1 \end{matrix} \right) &= \epsilon_{-\alpha}(u),
\end{align*}
where $\epsilon_\alpha : k \rightarrow U_\alpha$ is an isomorphism of algebraic groups \cite[Theorem 26.3(c)]{humphreys1975linear}. Any such isomorphism satisfies
\begin{align*}
	t \epsilon_\alpha(x)t^{-1} = \epsilon_\alpha(\alpha(t)x),
\end{align*}
for all $t\in T, x\in k$. 

We fix an integer $r > 0$ and define $\rho_r:SL_2(k) \rightarrow L_\alpha$ to be the composition of $\rho_0$ and the Frobenius map, defined by
\begin{align*}
F_r&:SL_2(k)\rightarrow SL_2(k) \\
& (A_{ij}) \mapsto (A_{ij})^{p^r}.
\end{align*}
That is $\rho_r = \rho_0 \circ F_r$ and satisfies
\begin{align*}
\rho_r \left(\begin{matrix} 1 &  u \\ 0 & 1 \end{matrix} \right) &= \epsilon_\alpha(u^{p^r}) \\
\rho_r \left(\begin{matrix} 1 & 0 \\ u & 1 \end{matrix} \right) &= \epsilon_{-\alpha}(u^{p^r}).
\end{align*}

Now we have an action of $SL_2(k)$ on $V_\alpha$ defined by
\begin{align*}
	y \cdot v = \rho_r(y)v\rho_r(y^{-1}),
\end{align*}
for $y \in SL_2(k), v \in V_\alpha$. By \cite[Theorem 26.3(c)]{humphreys1975linear}
\begin{align}
	\label{eqn:t_act}
	\left(\begin{matrix} t & 0 \\ 0 & t^{-1}\end{matrix}\right) \cdot \prod_{\delta\in\Phi^+-\{\alpha\}} \epsilon_\delta (\lambda_\delta) &=
	\prod_{\delta\in\Phi^+-\{\alpha\}} \epsilon_\delta\left(t^{\langle \delta, \alpha \rangle p^r} \lambda_\delta\right), \\ 
	\label{eqn:n_act}
	\left(\begin{matrix} 0 & 1 \\ -1 & 0 \end{matrix}\right) \cdot \prod_{\delta\in\Phi^+-\{\alpha\}}\epsilon_\delta (\lambda_\delta) &=
	\prod_{\delta\in\Phi^+-\{\alpha\}} n_\alpha \epsilon_\delta \left( \lambda_\delta\right)\, n_\alpha^{-1},
\end{align}
for all $t \in k^*, \lambda_\delta \in k$, where $n_\alpha = \epsilon_\alpha(1)\epsilon_{-\alpha}(-1)\epsilon_\alpha(1)$.

We set out to find the general form of a 1-cocycle $\tau \in Z^1(SL_2(k), V_\alpha)_{\rho_r}$. Then we can apply the canonical projection $\psi:Z^1(SL_2(k), V_\alpha)_{\rho_r} \rightarrow H^1(SL_2(k), V_\alpha)_{\rho_r}$ to calculate the 1-cohomology.
By Lemma \ref{trivial_on_t}, there exists $\sigma \in Z^1(SL_2(k), V_\alpha)_{\rho_r}$ such that $\psi(\tau) = \psi(\sigma)$ and $\sigma\left(y\right) = 1$ for all $y\in T_2(k)$. Hence we may restrict ourselves to 1-cocycles of this form. As a starting point, we have the following Proposition.

\begin{proposition} \label{claim1}
\begin{align*}
\sigma\left(\begin{matrix} 1 & u \\ 0 & 1 \end{matrix}\right) = \prod_{\delta\in\mathcal{D}} \epsilon_\delta\left(x_\delta\left(u\right)\right),
\end{align*}
where $\mathcal{D} = \{\delta \in \Phi^+-\{\alpha\}\,|\, \langle \delta, \alpha \rangle > 0\}$, and the $x_\delta\in k[X]$ are polynomials in one variable which satisfy
\begin{align}\label{eqn:x}
x_\delta\left(t^2u\right) &= t^{\langle \delta, \alpha \rangle p^r} x_\delta\left(u\right),
\end{align}
for all $t \in k^*, u\in k$.
\end{proposition}
\begin{proof}
Let $\delta \in \Phi^+-\{\alpha\}$. We have the chain of morphisms
\begin{align*}
k\simeq U_2(k) 
\stackrel{\iota}\longrightarrow SL_2(k) 
\stackrel{\sigma}\longrightarrow V_\alpha 
\stackrel{\pi_\delta}\longrightarrow k,
\end{align*}
where $\iota$ is the inclusion map and $\pi_\delta$ is the projection onto the root subgroup $U_\delta$. Hence  $x_\delta = \pi_\delta\, \circ\, \sigma\, \circ\, \iota$ is a morphism from $k \rightarrow k$.

We apply $\sigma$ to the identity
\begin{align*}
\left(\begin{matrix}
1 & t^2u \\ 0 & 1
\end{matrix}\right) =
\left(\begin{matrix}
t & 0 \\ 0 & t^{-1}
\end{matrix}\right)
\left(\begin{matrix}
1 & u \\ 0 & 1
\end{matrix}\right)
\left(\begin{matrix}
t^{-1} & 0 \\ 0 & t
\end{matrix}\right),
\end{align*}
using the 1-cocycle condition on the right to obtain
\begin{align}
\sigma
\left(\begin{matrix}
1 & t^2u \\ 0 & 1
\end{matrix}\right)
&=\sigma\left(
\left(\begin{matrix}
t & 0 \\ 0 & t^{-1}
\end{matrix}\right)
\left(\begin{matrix}
1 & u \\ 0 & 1
\end{matrix}\right)
\left(\begin{matrix}
t^{-1} & 0 \\ 0 & t
\end{matrix}\right)
\right) \nonumber \\
&=
\left(\sigma
\left(\begin{matrix}
t & 0 \\ 0 & t^{-1}
\end{matrix}\right)
\right)\left[
\left(\begin{matrix}
t & 0 \\ 0 & t^{-1}
\end{matrix}\right) \cdot
\sigma\left(
\left(\begin{matrix}
1 & u \\ 0 & 1
\end{matrix}\right)
\left(\begin{matrix}
t^{-1} & 0 \\ 0 & t
\end{matrix}\right)
\right)\right] \nonumber \\
&=
\left(\begin{matrix}
t & 0 \\ 0 & t^{-1}
\end{matrix}\right) \cdot
\left[\left(\sigma
\left(\begin{matrix}
1 & u \\ 0 & 1
\end{matrix}\right)
\right)
\left[\left(\begin{matrix}
1 & u \\ 0 & 1
\end{matrix}\right) \cdot
\sigma
\left(\begin{matrix}
t^{-1} & 0 \\ 0 & t
\end{matrix}\right)\right]\right]
\nonumber
\\
&=
\left(\begin{matrix}
t & 0 \\ 0 & t^{-1}
\end{matrix}\right) \cdot
\sigma
\left(\begin{matrix}
1 & u \\ 0 & 1
\end{matrix}\right)\nonumber .%label{tfixesalpha}.
\end{align}
So by Equation \ref{eqn:t_act}, $x_\delta\left(t^2u\right) = t^{\langle \delta, \alpha \rangle p^r} x_\delta\left(u\right)$, as claimed in Equation \ref{eqn:x}.

Since $x_\delta$ is a polynomial function there can only be non-negative powers of $t$ on the left-hand side of Equation \ref{eqn:x}, which forces $\langle \delta, \alpha \rangle \geq 0$. However, if $\langle \delta, \alpha \rangle = 0$ then $x_\delta$ is constant and hence zero, as $\sigma\left(y\right) = 1$ for $y\in T_2(k)$ implies that $x_\delta(0)=0$. Therefore the non-zero $x_\delta$ occur only when $\langle \delta, \alpha \rangle > 0$.
\end{proof}
\begin{remark} \label{claim1'}
Using a similar argument we see that
\begin{align*}
	\sigma\left(\begin{matrix}1 & 0\\u & 1\end{matrix}\right) = \prod_{\delta \in \mathcal{D}^-} \epsilon_\delta\left(x_\delta(u)\right),
\end{align*}
where $\mathcal{D}^- = \{\delta \in \Phi^+ - \{\alpha\}\,|\,\langle\delta,\alpha\rangle < 0\}$, and the $x_\delta \in k[X]$ are polynomials in one variable which satisfy
\begin{align*}
	x_\delta(t^{-2}u) = t^{\langle\delta,\alpha\rangle p^r}x_\delta(u).
\end{align*}
\end{remark}

\begin{proposition}\label{imb:imu}
The images of $U_2(k)$ and $B_2(k)$ under $\sigma$ are equal.
\end{proposition}
\begin{proof}
	Let $a\in k^*, b\in k$. Since
\begin{align*}
	\left(\begin{matrix}a & b\\0 & a^{-1}\end{matrix}\right) = 
	\left(\begin{matrix}a & 0\\0 & a^{-1}\end{matrix}\right)
	\left(\begin{matrix}1 & a^{-1}b\\0 & 1\end{matrix}\right),
\end{align*}
we get
\begin{align*}
	\sigma\left(\begin{matrix}a & b\\0 & a^{-1}\end{matrix}\right) &=
	\left(\sigma\left(\begin{matrix}a & 0\\0 & a^{-1}\end{matrix}\right)\right)
	\left(
	\left(\begin{matrix}a & 0\\0 & a^{-1}\end{matrix}\right) \cdot
	\sigma\left(\begin{matrix}1 & a^{-1}b\\0 & 1\end{matrix}\right)
	\right)\\
	&= \left(\begin{matrix}a & 0\\0 & a^{-1}\end{matrix}\right) \cdot
	\sigma\left(\begin{matrix}1 & a^{-1}b\\0 & 1\end{matrix}\right) \\
	&= \left(\begin{matrix}a & 0\\0 & a^{-1}\end{matrix}\right) \cdot
	\prod_{\delta\in\mathcal{D}} \epsilon_\delta\left(x_\delta(a^{-1}b)\right)\quad(\textrm{Proposition \ref{claim1}}) \\
	&= 
	\prod_{\delta\in\mathcal{D}} \epsilon_\delta\left(a^{\langle\delta, \alpha\rangle p^r}x_\delta(a^{-1}b)\right)\quad(\textrm{by Equation \ref{eqn:t_act}})\\
	&\subset \sigma\left(U_2(k)\right)
\end{align*}
The reverse inclusion is obvious.
\end{proof}

Next we prove some useful facts about root systems not containing $G_2$ or $C_3$.

\begin{proposition} \label{ufixes} Suppose $\Phi$ does not contain $G_2$ and let $\alpha,\beta\in\Phi$. If $\alpha + \beta \in \Phi$ then $\langle \alpha, \beta \rangle \leq 0$.
\end{proposition}
\begin{proof}We have
\begin{align*}
\langle \alpha, \beta \rangle > 0 \Leftrightarrow (\alpha, \beta) >0 \Leftrightarrow \cos(\theta) > 0,
\end{align*}
where $\theta$ is the angle between $\alpha$ and $\beta$. Hence acute angles correspond to positive pairs. Since $\alpha,\beta$ lie in a rank 2 root subsystem of $\Phi$ \cite[A.4]{humphreys1975linear}, we only need to refer to the root system diagrams for $A_1\times A_1, A_2, B_2$ (Appendix \ref{AppendixC}) to see that no two roots meeting at an acute angle add to give another root. Therefore, if $\langle \alpha, \beta \rangle > 0$ then $\alpha + \beta \notin \Phi$.
\end{proof}

\begin{remark}\label{g2counter}
If $\Phi = G_2$, $\Delta = \{\alpha, \beta\}$ ($\alpha$ short) then $\alpha, 2\alpha + \beta, 3\alpha + \beta \in \Phi$ and $\langle \alpha, 2\alpha + \beta \rangle = 1$. This shows that Proposition \ref{ufixes} breaks down when $\Phi = G_2$.
\end{remark}
\begin{remark}\label{g2rem}
	The only irreducible root system containing $G_2$ is $G_2$ itself.
\end{remark}

\begin{proposition} \label{uabelian}
Suppose $\Phi$ does not contain $G_2$ or $C_3$. Let $\delta_1, \delta_2 \in \Phi$ and $\gamma \in \Delta$ be distinct roots such that $\langle \delta_i, \gamma \rangle > 0$ $(i = 1, 2)$. If $\delta_1 + \delta_2$ is a root, then $\delta_1$ and $\delta_2$ are of opposite sign.
\end{proposition}
\begin{proof}
Suppose $\delta_1 + \delta_2 \in \Phi$. Let $\theta_i$ be the absolute value of the angle between $\delta_i$ and $\gamma$ $(i = 1,2)$, and let $\theta_3$ be the absolute value of the angle between $\delta_1$ and $\delta_2$. Then
\begin{align*}
&\langle \delta_i, \gamma\rangle > 0 \\
\Rightarrow &(\delta_i, \gamma) > 0 \\
\Rightarrow &\cos(\theta_i) > 0 \\
\Rightarrow &\theta_i < \pi/2,\quad (i = 1,2).
\end{align*}
By Proposition \ref{ufixes} $\langle \delta_1, \delta_2 \rangle \leq 0$, so $\theta_3 \geq \pi/2$.

Without loss of generality, this leads to the following four cases to consider.
\begin{itemize}
\item[(i)]$\theta_1 = \pi/3,\quad\theta_2 = \pi/3,\quad\theta_3 = 2\pi/3$,
\item[(ii)]$\theta_1 = \pi/4,\quad\theta_2 = \pi/4,\quad\theta_3 = \pi/2$,
\item[(iii)]$\theta_1 = \pi/4,\quad\theta_2 = \pi/3,\quad\theta_3 = \pi/2$,
\item[(iv)]$\theta_1 = \pi/3,\quad\theta_2 = \pi/3,\quad\theta_3 = \pi/2$,
\end{itemize}

For cases (i) and (ii) the three roots must lie in a plane, hence in a rank 2 subsystem of $\Phi$. Since we ruled out $\Phi=G_2$, this leaves $A_1 \times A_1, A_2,$ or $B_2$. In fact, they lie in $A_2$ and $B_2$ subsystems, respectively. Consulting the root system diagrams for each case (Appendix \ref{AppendixC}) we see that if $\gamma = \delta_1 + \delta_2$ then $\delta_1$ and $\delta_2$ are of opposite sign, as claimed.

Like case (ii), for case (iii) $\delta_1, \delta_2$ must lie in a $B_2$ subsystem, and they must have the same length. (Note that $A_1\times A_1$ is ruled out because $\delta_1+\delta_2$ is a root). However, $\theta_1 = \pi/4$ implies that $\delta_1, \gamma$ are roots of different length in a $B_2$ subsystem, while $\theta_2 = \pi/3$ implies that $\delta_2, \gamma$ are roots of the same length in an $A_2$ subsystem. Hence $\delta_1$ and $\delta_2$ have different lengths, a contradiction. Therefore we rule out case (iii).

For case (iv) $\delta_1, \delta_2$ again lie in a $B_2$ subsystem, so $\delta_1$ and $\delta_2$ are short roots because $\delta_1+\delta_2$ is a root. Furthermore $\delta_i, \gamma$ lie in an $A_2$ subsystem $(i = 1, 2)$, so the roots $\delta_1, \delta_2, \gamma$ are the same length. Hence $\gamma$ is a short root. Therefore, if $\delta_1, \delta_2, \gamma$  lie in a rank 3 subsystem $X$, $X$ contains two short root $A_2$ subsystems and a $B_2$ subsystem. This rules out $B_3$, so $X=C_3$. Since we excluded $C_3$, we can rule out case (iv).
\end{proof}

\begin{remark}\label{c3counter}
Let $\Phi = C_3$, $\Delta = \{\alpha, \beta, \gamma\}$ where $\gamma$ is the long root and $\beta$ is the short root connected to $\gamma$. Then
\begin{align*}
	\langle \alpha + \beta, \alpha \rangle &= 1, \\
	\langle \alpha + \beta + \gamma, \alpha \rangle &= 1, \\
	(\alpha + \beta) + (\alpha + \beta + \gamma) &= 2\alpha + 2\beta + \gamma \in \Phi, \textrm{ and} \\
	\alpha + \beta, \alpha + \beta + \gamma &\in \Phi^+.
\end{align*}
This shows that Proposition \ref{uabelian} breaks down when $\Phi$ contains $C_3$ as a root subsystem. 
\end{remark}
\begin{remark}\label{c3rem}
Unlike the case for $G_2$ (Remark \ref{g2rem}), it is not obvious whether or not an irreducible root sytem contains $C_3$.
\end{remark}

\begin{lemma}[First Main Lemma]\label{lem:first} Suppose $\Phi$ does not contain $G_2$ or $C_3$. Then
\begin{itemize}
\item[(i)] $y_1 \cdot \sigma(y_2) = \sigma(y_2)$, for all $y_1, y_2\in U_2(k)$, and
\item[(ii)] $\sigma\left(B_2(k)\right)$ lies in a product of commuting root groups of $V_\alpha$.
\end{itemize}
\end{lemma}
\begin{proof}
Let $u_1,u_2 \in k$. By Proposition \ref{claim1}
\begin{align*}
\left(\begin{matrix}1 & u_1 \\ 0 & 1 \end{matrix}\right)
\cdot
\sigma\left(\begin{matrix} 1 & u_2 \\ 0 & 1\end{matrix}\right)
&=
\epsilon_\alpha(u_1^{p^r}) \left[\,\prod_{\delta\in\mathcal{D}} \epsilon_\delta\left(x_\delta\left(u_2\right)\right)\right] \epsilon_\alpha(-u_1^{p^r}),
\end{align*}
where $\mathcal{D}=\{\delta \in \Phi^+ -\{\alpha\} \,|\,\langle \delta, \alpha \rangle > 0\}$.

Let $\delta \in \mathcal{D}$. Since $\langle \delta, \alpha \rangle > 0$, $\alpha + \delta$ is not a root (Proposition \ref{ufixes}), so $U_\alpha$ commutes with $U_\delta$. This proves (i).

Let $\delta_1, \delta_2 \in \mathcal{D}$. Since $\langle \delta_i, \alpha \rangle > 0$ $(i=1,2)$, $\delta_1 + \delta_2$ is not a root (Proposition \ref{uabelian}), so $U_{\delta_1}$ and $U_{\delta_2}$ commute. Therefore $\sigma\left(U_2(k)\right)$ lies in a product of commuting root groups, so $\sigma\left(B_2(k)\right)$ lies in a product of commuting root groups (Proposition \ref{imb:imu}).
\end{proof}

\begin{lemma}[Second Main Lemma]\label{lem:second}
Let $\sigma\in Z^1(SL_2(k), V_\alpha)_{\rho_r}$ such that $\sigma(y) = 1$ for all $y\in T_2(k)$ and suppose
\begin{itemize}
	\item[(i)] $y_1 \cdot \sigma\left(y_2\right) = \sigma\left(y_2\right)$, for all $y_1,y_2 \in U_2(k)$, and
	\item[(ii)] $\sigma\left(U_2(k)\right)$ lies in a product of commuting root groups of $V_\alpha$.
\end{itemize}
Then for all $\delta \in \mathcal{D}$, there exist $\mu_\delta\in k$ such that for all $a\in k^*$ and all $b\in k$
\begin{align*}
\sigma\left(\begin{matrix}a & b\\0 & a^{-1}\end{matrix}\right) = \prod_{\delta\in\mathcal{D}} \epsilon_\delta\left(a^{(\langle\delta,\alpha\rangle p^r - n(\delta))}b^{n(\delta)}\mu_\delta\right),
\end{align*}
where $n(\delta) = p^{r-2+\langle\delta,\alpha\rangle}$, and $\mathcal{D}$ is defined by
\begin{align*}
	\mathcal{D} = \left\{ \begin{array}{ll}
		\{\delta \in \Phi^+ - \{\alpha\}\,|\, \langle \delta, \alpha \rangle = 1\textrm{ or }2\}, & \textrm{if } p = 2 \\ \\
		\{\delta \in \Phi^+ - \{\alpha\}\,|\, \langle \delta, \alpha \rangle = 2\}, & \textrm{otherwise}.
	\end{array}\right.
\end{align*}
\end{lemma}
\begin{proof}
We apply $\sigma$ to both sides of the equation
\begin{align*}
	\left(\begin{matrix} 1 & u_1 + u_2 \\ 0 & 1 \end{matrix}\right) = 
	\left(\begin{matrix} 1 & u_1 \\ 0 & 1 \end{matrix}\right)
	\left(\begin{matrix} 1 & u_2 \\ 0 & 1 \end{matrix}\right),
\end{align*}
to get 
\begin{align}
\sigma\left(\begin{matrix} 1 & u_1 + u_2 \\ 0 & 1 \end{matrix}\right) &=
	\left(\sigma\left(\begin{matrix} 1 & u_1 \\ 0 & 1 \end{matrix}\right)\right)
	\left(\left(\begin{matrix} 1 & u_1 \\ 0 & 1 \end{matrix}\right)\cdot\sigma\left(\begin{matrix} 1 & u_2 \\ 0 & 1 \end{matrix}\right)\right)\nonumber \\ &=
	\sigma\left(\begin{matrix} 1 & u_1 \\ 0 & 1 \end{matrix}\right)
	\sigma\left(\begin{matrix} 1 & u_2 \\ 0 & 1 \end{matrix}\right)\quad(\textrm{by (i)}) \label{xdelta_homs}.
\end{align}
By Proposition \ref{claim1} there exist polynomials $x_\delta$ such that
\begin{align*}
	\sigma\left(\begin{matrix}1 & u\\0 & 1\end{matrix}\right) = \prod_{\delta\in\mathcal{D}} \epsilon_\delta\left(x_\delta(u)\right).
\end{align*}
Applying Equation \ref{xdelta_homs} yields
\begin{align*}
	\prod_{\delta\in\mathcal{D}} \epsilon_\delta(x_\delta(u_1 + u_2)) &= 
	\prod_{\delta\in\mathcal{D}} \epsilon_\delta(x_\delta(u_1))
	\prod_{\delta\in\mathcal{D}} \epsilon_\delta(x_\delta(u_2)) \\
	&= \prod_{\delta\in\mathcal{D}} \epsilon_\delta(x_\delta(u_1))\epsilon_\delta(x_\delta(u_2))\quad(\textrm{by (ii)}) \\
	&= \prod_{\delta\in\mathcal{D}} \epsilon_\delta(x_\delta(u_1) + x_\delta(u_2)).
\end{align*}
Hence each $x_\delta$ is an additive polynomial, so it is of the form 
\begin{align*}
	x_\delta(\lambda) = \sum_{i=0}^n \mu_i\lambda^{p^i}\quad(\textrm{\cite[\S 20.3, Lemma A]{humphreys1975linear}}),
\end{align*}
for some $n\in\mathbb{N}$ and some $\mu_i\in k$ depending on $\delta$.
Then by Equation \ref{eqn:x} of Proposition \ref{claim1}
\begin{align*}
	\sum_{i=0}^n \mu_i(t^2u)^{p^i} = t^{\langle \delta, \alpha \rangle p^r}\sum_{i = 0}^n\mu_i u^{p^i},
\end{align*}
for all $t\in k^*, u \in k$.

Suppose $x_\delta \neq 0$, so there exists $j \geq 0$ such that $\mu_j \neq 0$. Then
\begin{align}\label{eqn:xfinal}
	\mu_j t^{2p^j} u^{p^j} &= t^{\langle \delta, \alpha \rangle p^r} \mu_j u^{p^j}\nonumber\\
	\Rightarrow t^{2p^j} &= t^{\langle \delta, \alpha \rangle p^r}\nonumber\\
	\Rightarrow 2p^j &= \langle \delta, \alpha \rangle p^r.
\end{align}

In \cite[\S 3.4]{carter1989simple} it is shown that the possible pairings of any two roots are bounded by $\pm 3$. Hence by Proposition \ref{claim1} $\langle \delta, \alpha \rangle = 1, 2$ or 3. Therefore the only solutions to Equation \ref{eqn:xfinal} are
\begin{align*}
&p=2, \quad \langle \delta, \alpha \rangle = 1, \quad j = r - 1, \\
&p\geq 2, \quad \langle \delta, \alpha \rangle = 2, \quad j = r.
\end{align*}
This shows that it is only possible for $x_\delta$ to have one nonzero term, either the $(r-1)^{th}$ or the $r^{th}$ depending on whether $\langle \delta, \alpha \rangle = 1$ or $2$, respectively. Furthermore, the $(r-1)^{th}$ term is zero if $p \neq 2$.

Let $\mu_\delta = \mu_j$ and let $n(\delta) = p^{r-2+\langle\delta,\alpha\rangle}$, so that
\begin{align*}
	\sigma\left(\begin{matrix}1 & u\\0 & 1\end{matrix}\right) = \prod_{\delta\in\mathcal{D}'} \epsilon_\delta\left(u^{n(\delta)}\mu_\delta\right),
\end{align*}
where
\begin{align*}
		\mathcal{D}' = \left\{\begin{array}{ll}
			\{\delta \in \Phi^+ \,|\, \langle \delta, \alpha\rangle = 1\textrm{ or }2\},&\textrm{if }p = 2, \\ \\
			\{\delta \in \Phi^+ \,|\, \langle \delta, \alpha\rangle = 2\},&\textrm{otherwise}.
		\end{array}\right.
\end{align*}

Furthermore
\begin{align*}
\sigma\left(\begin{matrix}a & b\\0 & a^{-1}\end{matrix}\right) &=
\sigma\left(\begin{matrix}a & 0\\0 & a^{-1}\end{matrix}\right)\left[
\left(\begin{matrix}a & 0\\0 & a^{-1}\end{matrix}\right)\cdot
\sigma\left(\begin{matrix}1 & a^{-1}b\\0 & 1\end{matrix}\right)\right] \\ &=
\left(\begin{matrix}a & 0\\0 & a^{-1}\end{matrix}\right)\cdot
\sigma\left(\begin{matrix}1 & a^{-1}b\\0 & 1\end{matrix}\right) \\
&=
\left(\begin{matrix}a & 0\\0 & a^{-1}\end{matrix}\right)\cdot
\prod_{\delta\in\mathcal{D}'} \epsilon_\delta\left((a^{-1}b)^{n(\delta)}\mu_\delta\right)\\
&=\prod_{\delta\in\mathcal{D}'} \epsilon_\delta\left(a^{\langle\delta,\alpha\rangle p^r} (a^{-1}b)^{n(\delta)}\mu_\delta\right)\quad(\textrm{Equation \ref{eqn:t_act}})\\
&= \prod_{\delta\in\mathcal{D}'} \epsilon_\delta\left(a^{(\langle\delta,\alpha\rangle p^r - n(\delta))} b^{n(\delta)} \mu_\delta\right).
\end{align*}
This completes the proof.
\end{proof}
\begin{remark} \label{rem:second}
	Suppose that
\begin{itemize}
\item[(i$'$)] $y_1 \cdot\sigma(y_2) = \sigma(y_2)$, for all $y_1, y_2 \in U^-_2(k)$, and
\item[(ii$'$)] $\sigma(U^-_2(k))$ lies in a product of commuting root groups of $V_\alpha$.
\end{itemize}
Then for all $\delta \in \mathcal{D}^-$, there exist $\nu_\delta \in k$ such that for all $a \in k^*$ and all $c \in k$
\begin{align*}
	\sigma\left(\begin{matrix}a & 0\\c & a^{-1}\end{matrix}\right) = \prod_{\delta\in\mathcal{D}^-}\epsilon_\delta\left(a^{(\langle\delta,\alpha\rangle p^r - n(\delta))}b^{n(\delta)}\nu_\delta\right),
\end{align*}
where $n(\delta) = p^{r-2+\langle\delta,\alpha\rangle}$, and $\mathcal{D}^-$ is defined by
\begin{align*}
\mathcal{D}^-\left\{\begin{array}{ll}
	\{\delta \in \Phi^+-\{\alpha\} \,|\, \langle\delta,\alpha\rangle = -1\textrm{ or }-2\},&\textrm{if }p = 2, \\ \\
	\{\delta \in \Phi^+-\{\alpha\} \,|\, \langle\delta,\alpha\rangle = -2\},&\textrm{otherwise}.
\end{array}\right.
\end{align*}
The proof follows the proof of Lemma \ref{lem:second}, replacing $\alpha$ with $-\alpha$.
\end{remark}

\begin{corollary}\label{sigmaneg}
Suppose $\sigma$ satisfies the hypotheses of Lemma \ref{lem:second}. Then
\begin{align*}
	\sigma\left(\begin{matrix}-a & -b\\0 & -a^{-1}\end{matrix}\right) = \sigma\left(\begin{matrix}a & b\\0 & a^{-1}\end{matrix}\right).
\end{align*}
\end{corollary}

\begin{proof} It is obvious for $p=2$. Suppose $p>2$, then
\begin{align*}
	\sigma\left(\begin{matrix}-a & -b\\0 & -a^{-1}\end{matrix}\right) &= \prod_{\delta\in\mathcal{D}} \epsilon_\delta\left((-a)^{2p^r - p^r}(-b)^{p^r}\mu_\delta \right) \\
	&= \prod_{\delta\in\mathcal{D}} \epsilon_\delta\left((-a)^{p^r}(-b)^{p^r}\mu_\delta \right) \\
	&= \prod_{\delta\in\mathcal{D}} \epsilon_\delta\left((ab)^{p^r}\mu_\delta \right) \\
	&=\sigma\left(\begin{matrix}a & b\\0 & a^{-1}\end{matrix}\right).
\end{align*}
\end{proof}

\section{Discussion} \label{ch5_discussion}

In the next Chapter we investigate the counterexamples to Propositions \ref{ufixes} and \ref{uabelian} in the respective Remarks \ref{g2counter} and \ref{c3counter} by deriving the form of $\sigma\in Z^1(SL_2(k), V_\alpha)_{\rho_r}$ for the appropriate simple roots $\alpha\in\Delta$. This is carried out in Sections \ref{g2:alpha} and \ref{c3:alpha}, respectively. We show that $\sigma$ still satisfies the conclusion of Lemma \ref{lem:second}. This gives some evidence for the following Conjecture.
\begin{conjecture}\label{bigclaim} Let $G$ be a reductive group, $P_\alpha$ a minimal parabolic subgroup of $G$ with unipotent radical $V_\alpha$. Let $x \in H^1(SL_2(k), V_\alpha)_{\rho_r}$. Then there exists $\sigma \in Z^1(SL_2(k), V_\alpha)_{\rho_r}$ such that $\psi(\sigma) = x$. Furthermore, there exist $\mu_\delta\in k$ such that for all $a\in k^*$, $b\in k$
\begin{align*}
	\sigma\left(\begin{matrix}a & b \\0 & a^{-1}\end{matrix}\right) = \prod_{\delta\in\mathcal{D}} \epsilon_\delta\left(a^{(\langle\delta,\alpha\rangle p^r -n(\delta))}b^{n(\delta)}\mu_\delta\right),
\end{align*}
where $n(\delta)=p^{r-2+\langle\delta,\alpha\rangle}$, and $\mathcal{D}$ is defined by
\begin{align*}
\mathcal{D} = 
		\left\{\begin{array}{ll}
			\{\delta \in \Phi^+ \,|\, \langle \delta, \alpha \rangle = 1\textrm{ or }2\}, & \textrm{if }p = 2,\\ \\
			\{\delta \in \Phi^+ \,|\, \langle \delta, \alpha \rangle = 2\}, & \textrm{otherwise}.
		\end{array}\right.
\end{align*}
\end{conjecture}

If the above ask Conjecture is true, it would be worthwhile to see whether it can be extended to all parabolic subgroups $P_I$, $I\subset \Delta$ of $G$. This would be useful for further 1-cohomology calculations following the method we employ in the next Chapter.

Furthermore, given that $H^1(SL_2(k), V_\alpha)_{\rho_r}\rightarrow H^1(B_2(k), V_\alpha)_{\rho_r}$ is already injective by Lemma \ref{sl2_b_inj} and $\sigma\left(B_2(k)\right)$ lies in an abelian subgroup of $V_\alpha$, it may be possible to show that the restriction map
\begin{align*}
	H^1(SL_2(k), V_\alpha)_{\rho_r} \rightarrow H^1(U_2(k), V_\alpha)_{\rho_r},
\end{align*}
is injective (cf. Example \ref{ab_example}). Thus there is evidence that the answer algebraic version of K\"ulshammer's second question for $SL_2(k)$ and connected reductive $G$ is ``yes'' (cf. \ref{main_thm}).
