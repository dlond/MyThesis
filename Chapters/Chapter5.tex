%!TEX root = ../Thesis.tex
% Chapter 5

\chapter{1-Cohomology Calculations: Theoretical Results}
\label{Chapter5}
\lhead{Chapter 5. \emph{1-Cohomology Calculation: Theoretical Results}}

In this Chapter we present some theoretical results of 1-cohomology calculations of the form $H^1(SL_2(k), V)$. 

Following the approach to the algebraic version of K\"ulshammer's second question in the previous Chapter we let $G$ be a reductive group and fix representative parabolic subgroups. In particular we fix a Borel subgroup $B<G$ containing a maximal torus $T$, hence fix a base $\Delta$ of $\Phi$ the root system for $G$. Then we can choose representative parabolic subgroups $P_I$, $I \subset \Delta$ as in \cite[\S 30]{humphreys1975linear}.

\section{Minimal Parabolic Subgroups}

Let $G$ be a reductive group over an algebraically closed field $k$ of characteristic $p$. Let $\Phi$ be the roots for $G$ with $\Delta \subset \Phi^+ \subset \Phi$ the simple and positive roots, respectively, associated to a fixed maximal torus $T < G$ contained by a fixed Borel subgroup $B < G$.

Let $P_\alpha<G$ be the parabolic subgroup of $G$ corresponding to the simple root $\alpha\in\Delta$, with Levi subgroup $L_\alpha$ and unipotent radical $V_\alpha$:
\begin{align*}
V_\alpha=R_u(P_\alpha) &= \langle U_\delta \,|\, \delta \in \Phi^+, \delta \neq \alpha \rangle,\\
P_\alpha &= V_\alpha L_\alpha 
\end{align*}

There exists a homomorphism $\rho_0$ from $ SL_2(k)$ into $L_\alpha$ under which
\begin{align*}
\rho_0 \left(\begin{matrix} 1 &  u \\ 0 & 1 \end{matrix} \right) &= \epsilon_\alpha(u), \\
\rho_0 \left(\begin{matrix} 1 & 0 \\ u & 1 \end{matrix} \right) &= \epsilon_{-\alpha}(u),
\end{align*}
where $\epsilon_\alpha : k \rightarrow U_\alpha$ is an isomorphism \cite[Theorem 26.3(c)]{humphreys1975linear}.

We fix an integer $r > 0$ and define $\rho_r:SL_2(k) \rightarrow L_\alpha$ composed of $\rho_0$ and the Frobenius map,
\begin{align*}
F_r&:SL_2(k)\rightarrow SL_2(k) \\
& (A_{ij}) \mapsto (A_{ij})^{p^r}.
\end{align*}
That is
\begin{align*}
\rho_r &= \rho_0 \circ F_r,
\end{align*}
and satisfies
\begin{align*}
\rho_r \left(\begin{matrix} 1 &  u \\ 0 & 1 \end{matrix} \right) &= \epsilon_\alpha(u^{p^r}) \\
\rho_r \left(\begin{matrix} 1 & 0 \\ u & 1 \end{matrix} \right) &= \epsilon_{-\alpha}(u^{p^r}).
\end{align*}

We let $SL_2(k)$ act on $V_\alpha$ via $\rho_r$ and we consider 1-cocycles $\sigma \in Z^1(SL_2(k), \rho_r, V_\alpha)$. As we are interested in 1-cohomology classes, we may as well only consider those 1-cocycles that are zero on a maximal torus of $SL_2(k)$ (Lemma \ref{lem:nonab_lin_red}), so let $\sigma \in Z^1(SL_2(k), \rho_r, V_\alpha)$ such that
\begin{align*}
\sigma\left(\left(\begin{matrix} t & 0 \\ 0 & t^{-1}\end{matrix}\right)\right) = 0,
\end{align*}
for all $t\in k^*$.
We fix an ordering of the roots so that expressions such as $\prod_\delta U_\delta$ are unambiguous.

We can say a few things about these particular 1-cocycles which help us calculate the 1-cohomology.  By \cite[Theorem 26.3(c)]{humphreys1975linear}
\begin{align}
\left(\begin{matrix} t & 0 \\ 0 & t^{-1}\end{matrix}\right) \cdot \prod_\delta \epsilon_\delta (\lambda_\delta) &=
\prod_\delta \epsilon_\delta\left( (t^{p^r})^{\langle \delta, \alpha\rangle}\lambda_\delta\right),
\label{eqn:t_act}
\end{align}
\begin{align}
\left(\begin{matrix} 0 & -1 \\ 1 & 0 \end{matrix}\right) \cdot \prod_\delta \epsilon_\delta (\lambda_\delta) &=
\prod_\delta n_\alpha \epsilon_\delta\left( \lambda_\delta\right)\, n_\alpha^{-1},
\label{eqn:n_act}
\end{align}
where $n_\alpha = \epsilon_\alpha(1)\epsilon_{-\alpha}(-1)\epsilon_\alpha(1)$ and $\lambda_\delta \in k$.

\begin{lemma} \label{claim1}
\begin{align*}
\sigma\left(\left(\begin{matrix} 1 & u \\ 0 & 1 \end{matrix}\right)\right) = \prod_\delta\, \epsilon_\delta\left(x_\delta\left(u\right)\right),
\end{align*}
where $\delta$ ranges $\Phi^+ - \{\alpha\}$ such that $\langle \delta, \alpha \rangle > 0$, and $x_\delta\in k[X]$ are polynomials in one variable.
\end{lemma}
\begin{proof}
We have the chain of morphisms
\begin{align*}
k\simeq \left(\begin{matrix}1 & * \\ 0 & 1\end{matrix}\right) 
\stackrel{\iota}\longrightarrow SL_2(k) 
\stackrel{\sigma}\longrightarrow V_\alpha 
\stackrel{\pi_\delta}\longrightarrow k
\end{align*}
where $\iota$ is the inclusion map and $\pi_\delta$ the projection onto the root subgroup $V_\delta$. Hence, by the definition
\begin{align*}
x_\delta = \pi_\delta\, \circ\, \sigma\, \circ\, \iota
\end{align*}
is a morphism from $k \rightarrow k$.

Now since
\begin{align}
\left(\begin{matrix}
t & 0 \\ 0 & t^{-1}
\end{matrix}\right)
\left(\begin{matrix}
1 & u \\ 0 & 1
\end{matrix}\right)
\left(\begin{matrix}
t^{-1} & 0 \\ 0 & t
\end{matrix}\right)
=
\left(\begin{matrix}
1 & t^2u \\ 0 & 1
\end{matrix}\right),
\label{eqn:tut}
\end{align}
we use the 1-cocycle condition (Equation \ref{eqn:na_z}) to obtain
\begin{align*}
\sigma\left(
\left(\begin{matrix}
1 & t^2u \\ 0 & 1
\end{matrix}\right)
\right)
&=\sigma\left(
\left(\begin{matrix}
t & 0 \\ 0 & t^{-1}
\end{matrix}\right)
\left(\begin{matrix}
1 & u \\ 0 & 1
\end{matrix}\right)
\left(\begin{matrix}
t^{-1} & 0 \\ 0 & t
\end{matrix}\right)
\right)\\
&=
\sigma\left(
\left(\begin{matrix}
t & 0 \\ 0 & t^{-1}
\end{matrix}\right)
\right)
\left(\begin{matrix}
t & 0 \\ 0 & t^{-1}
\end{matrix}\right) \cdot
\sigma\left(
\left(\begin{matrix}
1 & u \\ 0 & 1
\end{matrix}\right)
\left(\begin{matrix}
t^{-1} & 0 \\ 0 & t
\end{matrix}\right)
\right)\\
&=
\sigma\left(
\left(\begin{matrix}
t & 0 \\ 0 & t^{-1}
\end{matrix}\right)
\right)
\left(\begin{matrix}
t & 0 \\ 0 & t^{-1}
\end{matrix}\right) \cdot
\sigma\left(
\left(\begin{matrix}
1 & u \\ 0 & 1
\end{matrix}\right)
\right)
\left(\begin{matrix}
1 & u \\ 0 & 1
\end{matrix}\right) \cdot
\sigma\left(
\left(\begin{matrix}
t^{-1} & 0 \\ 0 & t
\end{matrix}\right)
\right)\\
&=
\left(\begin{matrix}
t & 0 \\ 0 & t^{-1}
\end{matrix}\right) \cdot
\sigma\left(
\left(\begin{matrix}
1 & u \\ 0 & 1
\end{matrix}\right)
\right).
\end{align*}
So by Equation \ref{eqn:t_act},
\begin{align*}
x_\delta\left(t^2u\right) &= (t^{p^r})^{\langle \delta, \alpha\rangle}x_\delta\left(u\right).
\end{align*}
Since $x_\delta$ is a polynomial function there can only be non-negative powers of $t$ on the left-hand side of the equality which forces $\langle \delta, \alpha \rangle \geq 0$. However, if $\langle \delta, \alpha \rangle = 0$ then $x_\delta$ is constant and hence zero, as $\sigma$ is zero on $\left(\begin{matrix} * & 0 \\ 0 & *\end{matrix}\right)$. Therefore the non-zero $x_\delta$ occur precisely when $\langle \delta, \alpha \rangle > 0$.
\end{proof}

Next we prove a couple of useful facts about root systems not containing $G_2$ or $C_3$.

\begin{lemma} \label{ufixes} Suppose $\Phi$ does not contain $G_2$ and let $\alpha,\beta\in\Phi$. If $\alpha + \beta \in \Phi$ then $\langle \alpha, \beta \rangle \leq 0$.
\end{lemma}
\begin{proof} $\alpha, \beta$ lie in a rank-2 subsystem of $\Phi$. We have
\begin{align*}
\langle \alpha, \beta \rangle > 0 \Longleftrightarrow (\alpha, \beta) >0 \Longleftrightarrow \cos(\theta) > 0,
\end{align*}
where $\theta$ is the angle between $\alpha$ and $\beta$. Hence acute angles correspond to positive pairs. Referring to the $A_2$ and $B_2$ root system diagrams (Appendix \ref{AppendixC}) we find that no two roots meeting at an acute angle add to give another root. Therefore if $\langle \alpha, \beta \rangle > 0$ then $\alpha + \beta \notin \Phi$.
\end{proof}

We must exclude the case $\Phi = G_2$ here since $\alpha, 2\alpha + \beta$ and $3\alpha + \beta$ are all roots ($\alpha$ short) but $\langle \alpha, 2\alpha + \beta \rangle = 1$.

\begin{lemma} \label{uabelian}
Suppose $\Phi$ does not contain $G_2$ or $C_3$. Let $\delta_1, \delta_2 \in \Phi$ and $\gamma \in \Delta$ be roots such that $\langle \delta_i, \gamma \rangle > 0$ $(i = 1, 2)$. If $\delta_1 + \delta_2$ is a root, then $\delta_1$ and $\delta_2$ are of opposite sign.
\end{lemma}
\begin{proof}
Suppose $\delta_1 + \delta_2 \in \Phi$. Let $\theta_i$ be the absolute value of the angle between $\delta_i$ and $\gamma$, $(i = 1,2)$ and let $\theta_3$ be the absolute value of the angle between $\delta_1$ and $\delta_2$. Then
\begin{align*}
&\langle \delta_i, \gamma\rangle > 0 \qquad (i=1,2) \\
\Longrightarrow &(\delta_i, \gamma) > 0 \\
\Longrightarrow &\cos(\theta_i) > 0 \\
\Longrightarrow &\theta_i < \pi/2,
\end{align*}
and similarly, using Lemma \ref{ufixes}
\begin{align*}
&\langle \delta_1, \delta_2 \rangle \leq 0 \\
&\Longrightarrow \theta_3 \geq \pi/2.
\end{align*}
So, without loss of generality, this leads to consider four cases:
\begin{align*}
\textbf{1:}&\theta_1 = \pi/3,\quad\theta_2 = \pi/3,\quad\theta_3 = 2\pi/3; \\
\textbf{2:}&\theta_1 = \pi/3,\quad\theta_2 = \pi/3,\quad\theta_3 = \pi/2; \\
\textbf{3:}&\theta_1 = \pi/4,\quad\theta_2 = \pi/3,\quad\theta_3 = \pi/2; \\
\textbf{4:}&\theta_1 = \pi/4,\quad\theta_2 = \pi/4,\quad\theta_3 = \pi/2.
\end{align*}

For the cases in which $\theta_3 = \pi/2$ we can reason from the root system diagrams (Appendix \ref{AppendixC}) that $\delta_1$ and $\delta_2$ lie in a $B_2$ subsystem of $\Phi$, and they have the same length. Since $\delta_1+\delta_2$ is a root it must be that $\delta_1$ and $\delta_2$ are short roots and their sum is a long root. However we can rule out the third case. For if $\theta_1 = \pi/4$ then $\delta_1$ and $\gamma$ are roots of different length in a $B_2$ subsystem, but $\theta_2 = \pi/3$ implies that $\delta_2$ and $\gamma$ are roots of the same length in an $A_2$ subsystem, which is absurd.

The three roots must lie in a plane for cases one and four. That is, they lie in some rank 2 subsystem; $A_2$ and $B_2$ respectively. Consulting the root system diagrams, recall that $\gamma \in \Delta$, yields $\gamma = \delta_1 + \delta_2$ and the result holds.

In the second case we see that $\delta_1, \delta_2$ and $\gamma$ do not lie together in a rank 2 subsystem, and that these roots are the same length which implies that $\gamma$ is a short root. In fact, since a pair of short roots lie in subsystems of type $A_2$ it must be that the rank 3 subsystem in which the four roots, $\delta_1, \delta_2, \delta_1 + \delta_2, \gamma$, lie is of type $C_3$, but this is impossible by assumption. 
\end{proof}

We excluded $\Phi$ containing $C_3$ for berevity. The particular roots of $C_3$ which result in case two of Lemma \ref{uabelian} arises with $\gamma$ being the short simple root that is not connected to the long simple root (see Example \ref{eg:c3}).

We return to the 1-cohomology calculation but assume that the root system for $G$ does not contain $G_2$ or $C_3$.

\begin{corollary}\label{uact} For any $u_1, u_2 \in k$
\begin{align*}
\left(\begin{matrix}1 & u_1 \\ 0 & 1 \end{matrix}\right)
\cdot
\sigma\left(\left(\begin{matrix} 1 & u_2 \\ 0 & 1\end{matrix}\right)\right)
=
\sigma\left(\left(\begin{matrix} 1 & u_2 \\ 0 & 1\end{matrix}\right)\right).
\end{align*}
Furthermore, the $x_\delta$ are homomorphisms.
\end{corollary}

\begin{proof}
We have
\begin{align*}
\left(\begin{matrix}1 & u_1 \\ 0 & 1 \end{matrix}\right)
\cdot
\sigma\left(\left(\begin{matrix} 1 & u_2 \\ 0 & 1\end{matrix}\right)\right)
&=
\epsilon_\alpha(u_1^{p^r}) \prod_\delta \epsilon_\delta\left(x_\delta\left(u_2\right)\right) \epsilon_\alpha(-u_1^{p^r}),
\end{align*}
with $\langle \delta, \alpha \rangle > 0$. By Lemma \ref{ufixes}, $\alpha + \delta \notin \Phi$ so each $\epsilon_\delta$ commutes with the $\epsilon_\alpha$. Hence
\begin{align*}
  \left( \begin{matrix} 1 & u \\ 0 & 1 \end{matrix}\right) \cdot \sigma\left(\left(\begin{matrix} 1 & u_2 \\ 0 & 1 \end{matrix} \right)\right)
    &= 
    \prod_\delta \epsilon_\delta\left(x_\delta\left( u_2 \right)\right) \\
    &= 
    \sigma\left(\left(\begin{matrix} 1 & u_2 \\ 0 & 1 \end{matrix} \right) \right).
    \end{align*}
\end{proof}

\begin{corollary} The image of the group of upper triangular matrices of $SL_2(k)$ under $\sigma$ lies in a product of commuting root groups of $V_\alpha$.
  \label{cor:im_ab}
\end{corollary}
\begin{proof}
First consider
\begin{align*}
\sigma\left(\left( \begin{matrix} 1 & b \\ 0 & 1 \end{matrix}\right)\right) &= \prod_\delta \epsilon_\delta\left(x_\delta(b)\right).
\end{align*}
Suppose the roots $\delta_1$ and $\delta_2$ appear on the right hand side. By Lemma \ref{claim1} $\delta_i \in \Phi^+ - \{\alpha\}$ and $\langle \delta_i, \alpha \rangle > 0$ $(i=1,2)$, so Lemma \ref{uabelian} asserts that $\delta_1 + \delta_2$ is not a root, hence, $\epsilon_{\delta_1}$ and $\epsilon_{\delta_2}$ commute. 

For any $a, b\in k$ with $a\neq 0$
\begin{align*}
\sigma\left(\left(\begin{matrix} a & ab \\ 0 & a^{-1}\end{matrix}\right)\right) 
&= \left(\begin{matrix} a & 0 \\ 0 & a^{-1}\end{matrix} \right) \cdot
\sigma\left(\left(\begin{matrix} 1 & b \\ 0 & 1\end{matrix}\right)\right) \\
&= \prod_\delta \epsilon_\delta\left(a^{\langle \delta, \alpha \rangle p^r}x_\delta\left(b\right)\right).
\end{align*}
\end
{proof}

Since the $x_\delta$ are homomorphisms from $k\rightarrow k$ they must take the form
\begin{align*}
k\mapsto\sum_i \mu_i k^{p^i},
\end{align*}
for some $\mu_i$ in $k$. Furthermore, combining Equation \ref{eqn:tut} with the result in Corollary \ref{uact} we get that
\begin{align}
\prod_\delta \epsilon_\delta\left(x_\delta\left(a^2b\right)\right) = \prod_\delta \epsilon_\delta\left(a^{\langle \delta, \alpha \rangle p^r}x_\delta\left(b\right)\right),
\label{eqn:blah}
\end{align}
severely restricting the possible polynomials $x_\delta$. In fact, they are confined to be polynomials involving just one term, and the degree has already been decided upon fixing the integer $r$ in the definition of $\rho_r$. For suppose $x_\delta$ and hence some $\mu_j$ is non-zero. Then equating the coefficients of $b$ in Equation \ref{eqn:blah} yields
\begin{align*}
\mu_ja^{2p^j} &= \mu_j a^{\langle \delta, \alpha \rangle p^r}\\
\Longrightarrow2p^j &= \langle \delta, \alpha \rangle p^r.
\end{align*}

In \cite[\S 3.4]{carter1989simple} it is shown that the possible pairings of any two roots are bounded by $\pm 3$. Hence by Lemma \ref{claim1} $\langle \delta, \alpha \rangle = 1, 2$ or 3. It is now clear that if $\langle \delta, \alpha \rangle = 3$ then $x_\delta = 0$.

If $\langle \delta, \alpha \rangle = 1$ the characteristic of $k$ must be 2 and $j = r-1$. Otherwise $\langle \delta, \alpha \rangle = 2$ and $j = r$, but the characteristic of $k$ is so far unrestricted.

In order to capture the excluded cases where the root system for $G$ contains $G_2$ or $C_3$ we provide the following examples.

The results leading up to Corollary \ref{cor:im_ab} provide a general set of tools for calculating the 1-cohomology of a rank 2 algebraic group $G$ which we employ in Chapter \ref{Chapter6}, but they allude to something much grander which remains unproven: we conjecture
\begin{quote}
	Let $G$ be a reductive group over a closed field of positive characteristic $p$ and let $\Gamma = SL_2(k)$. Let $I \subset \Delta$ and $\sigma \in Z^1(SL_2(k), V_I)$ such that $\sigma(T)  = 0$. Then the image of $\sigma$ lies in a product of commuting root groups.
	\label{conj:im_ab}
\end{quote}

There is some work involved in extending the results in this Chapter to parabolics of rank $>$ 1. Furthermore we need a way to extend the results to $G$ with root system containing $G_2$ or $C_3$, as Examples \ref{eg:g2}, \ref{eg:c3} only verify the conjecture for $G$ with root system equal to $G_2, C_3$, respectively.

If the conjecture was true then we could attempt to apply Lemma \ref{lem:a_h_restriction} together with an algebraic version of Theorem \ref{thm:k2_h1} to show that the answer to the algebraic version of K\"ulshammer's second question is positive for $SL_2(k)$ and any reductive $G$, with no restrictions on the characteristic of $k$.


