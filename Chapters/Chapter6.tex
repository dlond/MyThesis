%!TEX root = ../Thesis.tex
% Chapter 6

\chapter{Example 1-Cohomology Calculations}
\label{Chapter6}
\lhead{Chapter 6. \emph{Example 1-Cohomology Calculations}}
In this Chapter we calculate some examples of 1-cohomology.

We calculate the 1-cohomology of $B_2$, aided by the results in the previous Chapter. We also investigate the counterexamples to Propositions \ref{ufixes} and \ref{uabelian} as pointed out in the respective Remarks \ref{g2counter} and \ref{c3counter} of the previous Chapter.

In our last example calculation we use the 1-cohomology to find a family of maps from $SL_2(k)$ to $B_4$ which gives infinitely many conjugacy classes. We would like to say whether or not this leads to a counterexample to K\"ulshammer's second question however the calculation is not developed enough for us to apply Theorem \ref{main_thm}.

\section{General Method}
The general method we use of calculating the 1-cohomology is outlined below.

For each $\alpha \in \Delta$ (the simple roots)
	\begin{itemize}
	\item[1.] Fix $x\in H^1(SL_2(k), V_\alpha)_{\rho_r}$. Then there exists $\sigma\in Z^1(SL_2(k), V_\alpha)_{\rho_r}$ such that $\psi(\sigma) = x$ and $\sigma\left(y\right) = 1 \in V_\alpha$ for all $y\in T_2(k)$ (Lemma \ref{trivial_on_t}).
	\item[2.] Use Lemma \ref{lem:second} to determine $\sigma\left(\begin{matrix}a & b\\0 & a^{-1}\end{matrix}\right)$. 
	\item[3.] If $\sigma\left(y\right) = 1$ for all $y\in B_2(k)$ then $H^1(SL_2(k), V_\alpha)_{\rho_r}$ is trivial by Lemma \ref{sl2_b_inj}.
	\item[4.] Otherwise, determine $\sigma\left(\begin{matrix}a & b\\c & d\end{matrix}\right)$ for $c \neq 0$ by
	\begin{align*}
	\sigma\left(\begin{matrix}a & b\\c & d\end{matrix}\right) &= \sigma\left(
			\left(\begin{matrix}1 & ac^{-1}\\0 & 1\end{matrix}\right)
			\left(\begin{matrix}0 & 1\\-1 & 0\end{matrix}\right)
			\left(\begin{matrix}-c & -d\\0 & -c^{-1}\end{matrix}\right)
			\right),
	\end{align*}
	and show that $\sigma$ is a well-defined 1-cocycle on $SL_2(k)$.
	\item[5.] Calculate $H^1(SL_2(k), V_\alpha)_{\rho_r}$.
	\item[6.] Calculate $H^1(SL_2(k), V_\alpha)_{\rho_r}/C_L(\rho_r)$ and use Theorem \ref{main_thm} to get information about conjugacy classes in $\mathrm{Hom}(SL_2(k), P_\alpha)$.
	\end{itemize}
We elaborate on step 4. Given $\sigma\left(\begin{matrix}a & b \\ 0 & a^{-1}\end{matrix}\right)$ and $\sigma\left(\begin{matrix}0 & 1\\-1 & 0\end{matrix}\right)$, the obvious way to piece together a 1-cocycle on $SL_2(k)$ is, for $c\neq 0$
\begin{align}\label{fullsigma}
	\sigma\left(\begin{matrix}a & b\\c & d\end{matrix}\right)
	&=
	\sigma\left(
		\left(\begin{matrix}1 & ac^{-1}\\0 & 1\end{matrix}\right)
		\left(\begin{matrix}0 & 1\\-1 & 0\end{matrix}\right)
		\left(\begin{matrix}-c & -d\\0 & -c^{-1}\end{matrix}\right)
	\right) \nonumber \\
	&=
	\sigma\left(\begin{matrix}1 & ac^{-1}\\0 & 1\end{matrix}\right)\left[
		\left(\begin{matrix}1 & ac^{-1}\\0 & 1\end{matrix}\right)\cdot
	\sigma\left(
		\left(\begin{matrix}0 & 1\\-1 & 0\end{matrix}\right)
		\left(\begin{matrix}-c & -d\\0 & -c^{-1}\end{matrix}\right)
	\right)
	\right]\nonumber \\
	&=
	\sigma\left(\begin{matrix}1 & ac^{-1}\\0 & 1\end{matrix}\right)\left[
		\left(\begin{matrix}1 & ac^{-1}\\0 & 1\end{matrix}\right)\cdot
	\left(\sigma
		\left(\begin{matrix}0 & 1\\-1 & 0\end{matrix}\right)
		\left[
		\left(\begin{matrix}0 & 1\\-1 & 0\end{matrix}\right)\cdot
		\sigma\left(\begin{matrix}-c & -d\\0 & -c^{-1}\end{matrix}\right)
		\right]
	\right)
	\right]\nonumber \\
	&=
	\sigma\left(\begin{matrix}1 & ac^{-1}\\0 & 1\end{matrix}\right)\left[
		\left(\begin{matrix}1 & ac^{-1}\\0 & 1\end{matrix}\right)\cdot
	\left(\sigma
		\left(\begin{matrix}0 & 1\\-1 & 0\end{matrix}\right)
		\left[
		\left(\begin{matrix}0 & 1\\-1 & 0\end{matrix}\right)\cdot
		\sigma\left(\begin{matrix}c & d\\0 & c^{-1}\end{matrix}\right)
		\right]
	\right)
	\right],
\end{align}
where we apply Corollary \ref{sigmaneg} at the last step.

This gives a candidate for a 1-cocycle on $SL_2(k)$. In \cite[Proposition 2]{martin2004nonab} Martin establishes a criterion for a function to belong to the set of 1-cocycles in terms of group presentations. We can use this criterion to prove that our candidate $\sigma$ is a well-defined 1-cocycle on $SL_2(k)$ based on Steinberg's presentation for $SL_2$ \cite[\S 12.1]{carter1989simple}. This amounts to verifying the following equations hold.
\begin{align}
	&\sigma(x_r(t_1))\left[x_r(t_1)\cdot \sigma(x_r(t_2))\right] = \sigma(x_r(t_1+t_2)), \label{pita1}\\
	&\sigma(h_r(t_1))\left[h_r(t_1) \cdot \sigma(h_r(t_2)))\right] = \sigma(h_r(t_1t_2)), \label{pita2}\\
	&\sigma(n_r(t))\left[n_r(t) \cdot \left[\sigma(x_r(u)) \left[x_r(u) \cdot \sigma(n_r(t)^{-1})\right]\right]\right] = \sigma(x_{-r}(-t^{-2}u)), \label{pita3}
\end{align}
for $r \in \{\pm \alpha\}$, where
\begin{align*}
	x_\alpha(t) &= \left(\begin{matrix}1 & t\\0 & 1\end{matrix}\right), \\
	x_{-\alpha}(t) &= \left(\begin{matrix}1 & 0\\t & 1\end{matrix}\right), \\
	n_r(t)&=x_r(t)x_{-r}(-t^{-1})x_r(t),\textrm{ and} \\
	h_r(t)&=n_r(t)n_r(-1).
\end{align*}

The checks are routine and lengthy, so we omit the details in our examples that follow. Note that not all of the equations need to be checked, for example Equation \ref{pita1} is automatic for $r=\alpha$, since $\sigma$ is already a well-defined 1-cocycle on $B_2(k)$ at this stage.

We will need to determine the action of $SL_2(k)$ on $V_\alpha$ and the group law on $V_\alpha$ which often involves lengthy computations using the commutator relations. For clarity, we will use column vector notation.
\begin{align*}
\left(\begin{matrix}
	v_{1}\\ v_{2}\\ \vdots\\ v_{n}
\end{matrix}\right) = \epsilon_{\delta_1}(v_1)\epsilon_{\delta_2}(v_2)\cdots\epsilon_{\delta_n}(v_n) \in V_\alpha
\end{align*}

Determining the values of $\sigma$ in Equation \ref{fullsigma} are straightforward. Lemma \ref{lem:second} gives $\sigma\left(\begin{matrix}a & b \\ 0 & a^{-1}\end{matrix}\right)$ up to some constants $\mu_\delta\in k$, while $\sigma\left(\begin{matrix}0 & 1\\-1 & 0\end{matrix}\right)$ is computed as follows.
\begin{proposition} \label{prop:A}
Let $\sigma$ satisfy the hypotheses of Lemma \ref{lem:second}. Then there exist constants $\lambda_\delta\in k$ such that
\begin{align}
	\sigma\left(\begin{matrix}0 & 1 \\ -1 & 0\end{matrix}\right) &=
	\sigma
		\left(\begin{matrix} 1 & -1\\ 0 & 1\end{matrix}\right)
	\left[
		\left(\begin{matrix} 1 & -1\\ 0 & 1\end{matrix}\right)\cdot
		\left(
		\sigma
		\left(\begin{matrix} 1 & 0\\ 1 & 1\end{matrix}\right)
		\left[
		\left(\begin{matrix} 1 & 0\\ 1 & 1\end{matrix}\right)\cdot
		\sigma\left(\begin{matrix} 1 & -1\\ 0 & 1\end{matrix}\right)\right]
		\right)
	\right] \label{prop:A:eqn1}\\
	&= \prod_{\delta\in\mathcal{D}} \epsilon_\delta(\lambda_\delta),\label{prop:A:eqn2}
\end{align}
where $\mathcal{D} = \{\delta\in \Phi^+-\{\alpha\}\,|\,\langle\delta,\alpha\rangle = 0\}$.
\end{proposition}
\begin{proof}
Since
\begin{align*}
		\left(\begin{matrix}t & 0\\0 & t^{-1}\end{matrix}\right)
		\left(\begin{matrix}0 & 1\\-1 & 0\end{matrix}\right) =
		\left(\begin{matrix}0 & 1\\-1 & 0\end{matrix}\right)
		\left(\begin{matrix}t^{-1} & 0\\0 & t\end{matrix}\right),
\end{align*}
we have
\begin{align*}
	\sigma\left(
		\left(\begin{matrix}t & 0\\0 & t^{-1}\end{matrix}\right)
		\left(\begin{matrix}0 & 1\\-1 & 0\end{matrix}\right)
	\right) &= 
	\sigma\left(
		\left(\begin{matrix}0 & 1\\-1 & 0\end{matrix}\right)
		\left(\begin{matrix}t^{-1} & 0\\0 & t\end{matrix}\right)
	\right) \\
	\Rightarrow
		\left(\begin{matrix}t & 0\\0 & t^{-1}\end{matrix}\right)\cdot
		\sigma\left(\begin{matrix}0 & 1\\-1 & 0\end{matrix}\right)
	 &= 
	\sigma
		\left(\begin{matrix}0 & 1\\-1 & 0\end{matrix}\right).
\end{align*}
Therefore $\sigma\left(\begin{matrix}0 & 1\\ -1 & 0\end{matrix}\right)$ is fixed under the action of $T_2(k)$ and so is of the form
\begin{align*}
\sigma\left(\begin{matrix}0 & 1\\-1 & 0\end{matrix}\right) = \prod_{\delta\in\mathcal{D}} \epsilon_\delta(\lambda_\delta),
\end{align*}
for some $\lambda_\delta\in k$, where
\begin{align*}
	\mathcal{D} = \{\delta\in \Phi^+-\{\alpha\}\,|\,\langle\delta,\alpha\rangle = 0\}.
\end{align*}
Furthermore, since
\begin{align*}
\left(\begin{matrix} 0 & 1\\ -1 & 0\end{matrix}\right)
&=
\left(\begin{matrix} 1 & -1\\ 0 & 1\end{matrix}\right)
\left(\begin{matrix} 1 & 0\\ 1 & 1\end{matrix}\right)
\left(\begin{matrix} 1 & -1\\ 0 & 1\end{matrix}\right),
\end{align*}
we have
\begin{align*}
	\sigma
		\left(\begin{matrix} 0 & 1\\ -1 & 0\end{matrix}\right)
	&=
	\sigma\left(
		\left(\begin{matrix} 1 & -1\\ 0 & 1\end{matrix}\right)
		\left(\begin{matrix} 1 & 0\\ 1 & 1\end{matrix}\right)
		\left(\begin{matrix} 1 & -1\\ 0 & 1\end{matrix}\right)
	\right) \\
	&=
	\sigma
		\left(\begin{matrix} 1 & -1\\ 0 & 1\end{matrix}\right)
	\left[
		\left(\begin{matrix} 1 & -1\\ 0 & 1\end{matrix}\right)\cdot
		\sigma
		\left(
		\left(\begin{matrix} 1 & 0\\ 1 & 1\end{matrix}\right)
		\left(\begin{matrix} 1 & -1\\ 0 & 1\end{matrix}\right)
		\right)
	\right] \\
	&=
	\sigma
		\left(\begin{matrix} 1 & -1\\ 0 & 1\end{matrix}\right)
	\left[
		\left(\begin{matrix} 1 & -1\\ 0 & 1\end{matrix}\right)\cdot
		\left(
		\sigma
		\left(\begin{matrix} 1 & 0\\ 1 & 1\end{matrix}\right)
		\left[
		\left(\begin{matrix} 1 & 0\\ 1 & 1\end{matrix}\right)\cdot
		\sigma\left(\begin{matrix} 1 & -1\\ 0 & 1\end{matrix}\right)\right]
		\right)
	\right].
\end{align*}
\end{proof}


%	Throughout we use the notation of Humphreys \cite[Chapter XI]{humphreys1975linear}.

\section{$G = B_2$}
\label{b2}

Let $T$ be a maximal torus of $B_2$ over an algebraically closed field $k$ of characteristic $p$. We label the positive roots for $B_2$ as $\alpha, \beta, \alpha + \beta, 2\alpha + \beta$. We have from \cite[\S 33.4]{humphreys1975linear}:
\begin{align*}
\epsilon_\beta (y) \epsilon_\alpha (x) &= \epsilon_\alpha (x) \epsilon_\beta (y) \epsilon_{\alpha + \beta} (xy) \epsilon_{2\alpha+\beta} (x^2y) \\
\epsilon_{\alpha + \beta} (y) \epsilon_\alpha (x) &= \epsilon_\alpha (x) \epsilon_{\alpha + \beta} (y) \epsilon_{2\alpha + \beta} (2xy),
\end{align*}
and 
\begin{align*}
n_\alpha \epsilon_\beta(x) n_\alpha^{-1} &= \epsilon_{2\alpha+\beta}(x)\\
n_\alpha \epsilon_{\alpha+\beta}(x) n_\alpha^{-1} &= \epsilon_{\alpha+\beta}(-x)\\
n_\alpha \epsilon_{2\alpha+\beta}(x) n_\alpha^{-1} &= \epsilon_{\beta}(x)\\
n_\beta \epsilon_\alpha(x) n_\beta^{-1} &= \epsilon_{\alpha+\beta}(x)\\
n_\beta \epsilon_{\alpha+\beta}(x) n_\beta^{-1} &= \epsilon_{\alpha}(-x)\\
n_\beta \epsilon_{2\alpha+\beta}(x) n_\beta^{-1} &= \epsilon_{2\alpha+\beta}(x)
\end{align*}
A proper parabolic subgroup of $B_2$ is conjugate to one of
\begin{align*}
P_\alpha &= \langle B, U_{-\alpha} \rangle\\
P_\beta &= \langle B, U_{-\beta} \rangle,
\end{align*}
where $B$ is the Borel subgroup of $B_2$ containing $T$
\begin{align*}
B=\langle T, U_\alpha, U_\beta, U_{\alpha + \beta}, U_{2\alpha+\beta}\rangle.
\end{align*}
The two parabolic subgroups have the Levi decompositions
\begin{align*}
P_\alpha &= R_u(P_\alpha) \rtimes L_\alpha \\
&= \langle U_\beta, U_{\alpha + \beta}, U_{2\alpha + \beta} \rangle \rtimes \langle T, U_\alpha, U_{-\alpha} \rangle \\ 
P_\beta &= R_u(P_\beta) \rtimes L_\beta \\
&= \langle U_\alpha, U_{\alpha+\beta}, U_{2\alpha + \beta} \rangle \rtimes \langle T, U_\beta, U_{-\beta} \rangle \\
\end{align*}

\subsection{$V = R_u(P_\alpha)$}
\label{b2:alpha}

Let $V_\alpha=R_u(P_\alpha)=\langle U_\beta, U_{\alpha + \beta}, U_{2\alpha + \beta} \rangle$. Note that $V_\alpha$ is abelian.
We will write $\mathbf{v}\in V_\alpha$ as a column vector for convenience,
\begin{align*}
	\left(\begin{matrix}v_1\\v_2\\v_3\end{matrix}\right) = \epsilon_\beta(v_1)\epsilon_{\alpha+\beta}(v_2)\epsilon_{2\alpha+\beta}(v_3),
\end{align*}
and we compute the following.
\begin{align*}
\left(\begin{matrix} 1 & u \\ 0 & 1\end{matrix}\right) \cdot \mathbf{v} 
&= 
\rho_r\left(\begin{matrix} 1 & u \\ 0 & 1\end{matrix}\right) \mathbf{v}\left( \rho_r\left(\begin{matrix} 1 & u \\ 0 & 1\end{matrix}\right)\right)^{-1} \\
&=
\epsilon_ \alpha (u^{p^r}) \epsilon_ \beta (v_1)\epsilon_{\alpha+\beta}(v_2) \epsilon_{2\alpha+\beta}(v_3) \epsilon_ \alpha (-u^{p^r}) \\
&=
\epsilon_ \alpha (u^{p^r}) \epsilon_ \beta (v_1) \epsilon_{\alpha+\beta}(v_2) \epsilon_ \alpha (-u^{p^r}) \epsilon_{2\alpha+\beta}(v_3) \\
&=
\epsilon_ \alpha (u^{p^r}) \epsilon_ \beta (v_1)  \epsilon_ \alpha (-u^{p^r}) \epsilon_{\alpha+\beta}(v_2) \epsilon_{2\alpha+\beta}(-2u^{p^r}v_2)\epsilon_{2\alpha+\beta}(v_3)\\
&=
\epsilon_ \alpha (u^{p^r}) \epsilon_ \alpha (-u^{p^r})  \epsilon_ \beta (v_1) \epsilon_{\alpha+\beta}(-u^{p^r}v_1) \epsilon_{2\alpha+\beta}(u^{2p^r}v_1) \epsilon_{\alpha+\beta}(v_2) \epsilon_{2\alpha+\beta}(v_3-2u^{p^r}v_2)\\
&=
\epsilon_ \beta (v_1)  \epsilon_{\alpha+\beta}(v_2-u^{p^r}v_1) \epsilon_{2\alpha+\beta}(v_3-2u^{p^r}v_2 + u^{2p^r}v_1)\\
&= \left(\begin{matrix} v_1 \\ v_2-u^{p^r}v_1 \\ v_3-2u^{p^r}v_2 + u^{2p^r}v_1 \end{matrix}\right),
\end{align*}
\begin{align*}
\left(\begin{matrix} t & 0 \\ 0 & t^{-1}\end{matrix}\right) \cdot \mathbf{v}
&=
\rho_r\left(\begin{matrix}t & 0 \\ 0 & t^{-1}\end{matrix}\right) \mathbf{v} \left(\rho_r\left(\begin{matrix}t & 0 \\ 0 & t^{-1}\end{matrix}\right)\right)^{-1} \\
&=
\alpha^\vee(t^{p^r})\epsilon_\beta(v_1)\epsilon_{\alpha+\beta}(v_2)\epsilon_{2\alpha+\beta}(v_3)\left(\alpha^\vee(t^{p^r})\right)^{-1} \\
&=
\epsilon_\beta\left(\beta(\alpha^\vee(t^{p^r}))v_1\right)\epsilon_{\alpha+\beta}\left((\alpha+\beta)(\alpha^\vee(t^{p^r}))v_2\right)\epsilon_{2\alpha+\beta}\left((2\alpha+\beta)(\alpha^\vee(t^{p^r}))v_3\right) \\
&=
\epsilon_\beta\left(t^{\langle\beta,\alpha\rangle p^r}v_1\right)\epsilon_{\alpha+\beta}\left(t^{\langle\alpha+\beta,\alpha\rangle p^r}v_2\right)\epsilon_{2\alpha+\beta}\left(t^{\langle2\alpha+\beta,\alpha\rangle p^r}v_3\right) \\
&=
\left(\begin{matrix} t^{-2p^r}v_1 \\ v_2 \\ t^{2p^r}v_3 \end{matrix}\right) \\
\left(\begin{matrix} 0 & 1 \\ -1 & 0 \end{matrix}\right) \cdot \mathbf{v} 
&=
\rho_r\left(\begin{matrix} 0 & 1 \\ -1 & 0\end{matrix}\right) \mathbf{v}\left( \rho_r\left(\begin{matrix} 0 & 1 \\ -1 & 0\end{matrix}\right)\right)^{-1} \\
&= 
n_ \alpha  \epsilon_ \beta (v_1)\epsilon_{\alpha+\beta}(v_2) \epsilon_{2\alpha+\beta}(v_3) n_ \alpha^{-1}\\
&= 
n_ \alpha  \epsilon_\beta (v_1) n_ \alpha^{-1}n_ \alpha \epsilon_{\alpha+\beta}(v_2) n_ \alpha^{-1} n_ \alpha \epsilon_{2\alpha+\beta}(v_3) n_ \alpha^{-1}\\
&= 
\epsilon_{2\alpha+\beta} (v_1) \epsilon_{\alpha+\beta}(-v_2)  \epsilon_{\beta}(v_3) \\
&= 
\epsilon_{\beta}(v_3) \epsilon_{\alpha+\beta}(-v_2) \epsilon_{2\alpha+\beta} (v_1)\\
&= \left(\begin{matrix} v_3 \\ -v_2 \\ v_1 \end{matrix}\right).
\end{align*}
This is enough to determine the action of $SL_2(k)$ on $V_\alpha$. We leave the details to the reader.
\begin{align*}
\left(\begin{matrix}a & b \\ c & d\end{matrix}\right) \cdot \mathbf{v} 
&= \left(\begin{matrix}
	d^{2p^r}v_1 - 2(cd)^{p^r}v_2 + c^{2p^r}v_3 \\
	(ad + bc)^{p^r}v_2 - (bd)^{p^r}v_1 - (ac)^{p^r}v_3 \\
	b^{2p^r}v_1 - 2(ab)^{p^r}v_2 + a^{2p^r}v_3
\end{matrix}\right)
\end{align*}

Let $x\in H^1(SL_2(k), V_\alpha)_{\rho_r}$. By Proposition \ref{trivial_on_t} there exists $\sigma\in Z^1(SL_2(k), V_\alpha)_{\rho_r}$ such that $\sigma(y) = 1$ for all $y\in T_2(k)$ and $\psi(\sigma) = x$. By Lemma \ref{lem:first}, we can apply Lemma \ref{lem:second}, which gives
\begin{align*}
	\sigma\left(\begin{matrix}a & b\\0 & a^{-1}\end{matrix}\right) &= \epsilon_{2\alpha+\beta}\left(\mu_3(ab)^{p^r}\right),
\end{align*}
for all $a\in k^*, b\in k$, for some fixed $\mu_3\in k$.

Suppose $p\neq 2$, and let $\mathbf{w} = \left(\begin{matrix}0\\2^{-1}\mu_3\\0\end{matrix}\right)\in V_\alpha$. Now consider $\chi^{SL_2(k)}_\mathbf{w} \in B^1(SL_2(k), V_\alpha)_{\rho_r}$.

\begin{align*}
	\chi^{SL_2(k)}_\mathbf{w}\left(\begin{matrix}a & b\\0 & a^{-1}\end{matrix}\right) &= \mathbf{w} - \left(\begin{matrix}a & b\\0 & a^{-1}\end{matrix}\right) \cdot \mathbf{w} \\
	%&=
	%\left(\begin{matrix}0\\2^{-1}\mu_3\\0\end{matrix}\right)
	%-\left(
	%\left(\begin{matrix}a & 0\\0 & a^{-1}\end{matrix}\right)
	%\left(\begin{matrix}1 & a^{-1}b\\0 & 1\end{matrix}\right)\cdot
	%\left(\begin{matrix}0\\2^{-1}\mu_3\\0\end{matrix}\right)
	%\right)\\
	&=
	\left(\begin{matrix}0\\2^{-1}\mu_3\\0\end{matrix}\right)
	-
	\left(\begin{matrix}a & b\\0 & a^{-1}\end{matrix}\right)\cdot
	\left(\begin{matrix}0\\2^{-1}\mu_3\\0\end{matrix}\right) \\
	%&=
	%\left(\begin{matrix}0\\2^{-1}\mu_3\\0\end{matrix}\right)
	%-
	%\left(\begin{matrix}a & 0\\0 & a^{-1}\end{matrix}\right)\cdot
	%\left(\begin{matrix}0\\2^{-1}\mu_3\\-(a^{-1}b)^{p^r}\mu_3\end{matrix}\right) \\
	&=
	\left(\begin{matrix}0\\2^{-1}\mu_3\\0\end{matrix}\right)
	-
	\left(\begin{matrix}0\\2^{-1}\mu_3\\-2(ab)^{p^r} (2^{-1}\mu_3)\end{matrix}\right) \\
	&=\left(\begin{matrix}0\\2^{-1}\mu_3\\0\end{matrix}\right)
	-\left(\begin{matrix}0\\2^{-1}\mu_3\\-(ab)^{p^r}\mu_3\end{matrix}\right) \\
	&=\left(\begin{matrix}0\\0\\(ab)^{p^r}\mu_3\end{matrix}\right) \\
	&= \sigma\left(\begin{matrix}a & b\\0 & a^{-1}\end{matrix}\right).
\end{align*}
This shows that if $p\neq 2$ then $H^1(B_2(k), V_\alpha)_{\rho_r}$ is trivial, and hence $H^1(SL_2(k), V_\alpha)_{\rho_r}$ is trivial by Lemma \ref{sl2_b_inj}.

We proceed with $p=2$.
By Lemma \ref{lem:second} and Remark \ref{rem:second} there exist constants $\mu_1,\mu_3\in k$ such that
	\begin{align*}
	\sigma
			\left(\begin{matrix}a & b \\ 0 & a^{-1}\end{matrix}\right)
	&=
	\left(\begin{matrix}0 \\ 0 \\ \mu_3(ab)^{2^r}\end{matrix}\right),
	\end{align*}
and 
\begin{align*}
	\sigma\left(\begin{matrix}d^{-1} & 0 \\ c & d\end{matrix}\right)
	&=
	\left(\begin{matrix}\mu_1(cd)^{2^r}\\0\\0\end{matrix}\right).
	\end{align*}
	
Furthermore, by Equation \ref{prop:A:eqn2} there exists $\lambda_2\in k$ such that
\begin{align*}
	\sigma\left(\begin{matrix}0 & 1\\1 & 0\end{matrix}\right) = \left(\begin{matrix}0 \\ \lambda_2 \\ 0 \end{matrix}\right),
\end{align*}
and by Equation \ref{prop:A:eqn1}
\begin{align*}
\sigma\left(\begin{matrix}0 & 1 \\ 1 & 0\end{matrix}\right)
&=
	\sigma
		\left(\begin{matrix} 1 & 1\\ 0 & 1\end{matrix}\right) +
		\left(\begin{matrix} 1 & 1\\ 0 & 1\end{matrix}\right)\cdot
		\left(
		\sigma
		\left(\begin{matrix} 1 & 0\\ 1 & 1\end{matrix}\right) +
		\left(\begin{matrix} 1 & 0\\ 1 & 1\end{matrix}\right)\cdot
		\sigma\left(\begin{matrix} 1 & 1\\ 0 & 1\end{matrix}\right)
		\right)
	\\
&=
		\left(\begin{matrix} 0 \\ 0 \\ \mu_3 \end{matrix}\right) +
		\left(\begin{matrix} 1 & 1\\ 0 & 1\end{matrix}\right)\cdot
		\left(
		\left(\begin{matrix} \mu_1 \\ 0 \\ 0\end{matrix}\right) +
		\left(\begin{matrix} 1 & 0\\ 1 & 1\end{matrix}\right)\cdot
		\left(\begin{matrix} 0 \\ 0 \\ \mu_3\end{matrix}\right)
		\right)
	\\
&=
		\left(\begin{matrix} 0 \\ 0 \\ \mu_3 \end{matrix}\right) +
		\left(\begin{matrix} 1 & 1\\ 0 & 1\end{matrix}\right)\cdot
		\left(
		\left(\begin{matrix} \mu_1 \\ 0 \\ 0\end{matrix}\right) +
		\left(\begin{matrix} \mu_3 \\ \mu_3 \\ \mu_3\end{matrix}\right)
		\right)
	\\
&=
		\left(\begin{matrix} 0 \\ 0 \\ \mu_3 \end{matrix}\right) +
		\left(\begin{matrix} 1 & 1\\ 0 & 1\end{matrix}\right)\cdot
		\left(\begin{matrix} \mu_1 + \mu_3 \\ \mu_3 \\ \mu_3\end{matrix}\right)
	\\
&=
		\left(\begin{matrix} 0 \\ 0 \\ \mu_3 \end{matrix}\right) +
		\left(\begin{matrix} \mu_1 + \mu_3 \\ \mu_3 + \mu_1 + \mu_3 \\ \mu_1 + \mu_3 + \mu_3\end{matrix}\right)
	\\
&=
		\left(\begin{matrix} \mu_1 + \mu_3 \\ \mu_1 \\ \mu_1 + \mu_3\end{matrix}\right).
\end{align*}
Therefore $\lambda_2 = \mu_1 = \mu_3$. Writing $\mu = \mu_1 (= \mu_3)$, we have
\begin{align*}
\sigma\left(\begin{matrix}a & b\\0 & a^{-1}\end{matrix}\right) = \left(\begin{matrix}0 \\ 0 \\ (ab)^{2^r}\mu\end{matrix}\right),\quad
\sigma\left(\begin{matrix}d^{-1} & 0\\c & d\end{matrix}\right) = \left(\begin{matrix}(cd)^{2^r}\mu \\ 0 \\ 0\end{matrix}\right),\quad
\sigma\left(\begin{matrix}0 & 1\\1 & 0\end{matrix}\right) = \left(\begin{matrix}0 \\ \mu \\ 0\end{matrix}\right).
\end{align*}

	Suppose $c\neq 0$. Then
	\begin{align}
	\sigma
			\left(\begin{matrix}a & b \\ c & d\end{matrix}\right)
			 &=
	\sigma\left(
			\left(\begin{matrix} 1 & ac^{-1} \\ 0 & 1 \end{matrix}\right)
			\left(\begin{matrix} 0 & 1 \\ 1 & 0 \end{matrix}\right)
			\left(\begin{matrix} c & d \\ 0 & c^{-1} \end{matrix}\right)
			\right) \nonumber \\
		&=
	\sigma
			\left(\begin{matrix} 1 & ac^{-1} \\ 0 & 1 \end{matrix}\right)
			 + 
	\left(\begin{matrix} 1 & ac^{-1} \\ 0 & 1 \end{matrix}\right)\cdot
	\left(
			\sigma
				\left(\begin{matrix} 0 & 1 \\ 1 & 0 \end{matrix}\right)
				 +
			\left(\begin{matrix} 0 & 1 \\ 1 & 0 \end{matrix}\right)\cdot
			\sigma
				\left(\begin{matrix} c & d \\ 0 & c^{-1} \end{matrix}\right)
			\right)\nonumber \\
		&=
			\left(\begin{matrix} 0 \\ 0 \\ (ac^{-1})^{2^r}\mu\end{matrix}\right)
			 + 
	\left(\begin{matrix} 1 & ac^{-1} \\ 0 & 1 \end{matrix}\right)\cdot
	\left(
				\left(\begin{matrix} 0 \\ \mu \\ 0 \end{matrix}\right)
				 +
			\left(\begin{matrix} 0 & 1 \\ 1 & 0 \end{matrix}\right)\cdot
				\left(\begin{matrix} 0 \\ 0 \\ (cd)^{2^r}\mu \end{matrix}\right)
			\right)\nonumber \\
		&=
			\left(\begin{matrix} 0 \\ 0 \\ (ac^{-1})^{2^r}\mu\end{matrix}\right)
			 + 
	\left(\begin{matrix} 1 & ac^{-1} \\ 0 & 1 \end{matrix}\right)\cdot
	\left(
				\left(\begin{matrix} 0 \\ \mu \\ 0 \end{matrix}\right)
				 +
				\left(\begin{matrix} (cd)^{2^r}\mu \\ 0 \\ 0 \end{matrix}\right)
			\right)\nonumber \\
		&=
			\left(\begin{matrix} 0 \\ 0 \\ (ac^{-1})^{2^r}\mu\end{matrix}\right)
			 + 
	\left(\begin{matrix} 1 & ac^{-1} \\ 0 & 1 \end{matrix}\right)\cdot
				\left(\begin{matrix} (cd)^{2^r}\mu \\ \mu \\ 0 \end{matrix}\right)
 \nonumber \\
		&=
			\left(\begin{matrix} 0 \\ 0 \\ (ac^{-1})^{2^r}\mu\end{matrix}\right)
			 + 
			\left(\begin{matrix} (cd)^{2^r}\mu \\ \mu + (ac^{-1})^{2^r}(cd)^{2^r}\mu\\ (ac^{-1})^{2^{r+1}}(cd)^{2^r}\mu \end{matrix}\right)
 \nonumber \\
		&=
			\left(\begin{matrix} (cd)^{2^r}\mu \\ \mu + (ac^{-1})^{2^r}(cd)^{2^r}\mu\\(ac^{-1})^{2^r}\mu + (ac^{-1})^{2^{r+1}}(cd)^{2^r}\mu \end{matrix}\right)
 \nonumber \\
		&=
			\left(\begin{matrix} (cd)^{2^r}\mu \\ (1 + ad)^{2^r}\mu \\ (1+ad)^{2^r}(ac^{-1})^{2^r}\mu \end{matrix}\right)
 \nonumber \\
		&=
			\left(\begin{matrix} (cd)^{2^r}\mu \\ (bc)^{2^r}\mu \\ (ab)^{2^r}\mu \end{matrix}\right)\label{eqn:b2a}.
	\end{align}
Note that substituting $c=0$ gives
\begin{align*}
\sigma\left(\begin{matrix}a & b \\ 0 & a^{-1}\end{matrix}\right) = \left(\begin{matrix} 0 \\ 0 \\ (ab)^{2^r}\mu\end{matrix}\right),
\end{align*}
substituting $b=0$ gives
\begin{align*}
\sigma\left(\begin{matrix}d^{-1} & 0 \\ c & d\end{matrix}\right) = \left(\begin{matrix} (cd)^{2^r}\mu \\ 0 \\ 0\end{matrix}\right),
\end{align*}
and substituting $a = d = 0, b = c = 1$ gives
\begin{align*}
\sigma\left(\begin{matrix}0 & 1 \\ 1 & 0\end{matrix}\right) = \left(\begin{matrix} 0 \\ \mu \\ 0\end{matrix}\right),
\end{align*}
so we take Equation \ref{eqn:b2a} to be the general form of a 1-cocycle $\sigma:SL_2(k)\rightarrow V_\alpha$ such that $\sigma(y) = 1$ for all $y\in T_2(k)$.

We can show that $\sigma$ as defined in Equation \ref{eqn:b2a} is a well-defined 1-cocycle by verifying that Equations \ref{pita1}--\ref{pita3} hold for $r=\alpha, -\alpha$ but we leave the details to the reader.

We can now calculate the 1-cohomology by finding $\tau\in Z^1(SL_2(k), V_\alpha)_{\rho_r}$ such that $\tau = \sigma + \chi^{SL_2(k)}_\mathbf{v}$, for some $\mathbf{v}\in V_\alpha$. We may assume $\tau(y) = 1$ for all $y \in T_2(k)$, hence $\mathbf{v}$ is fixed by the action of $T_2(k)$. 

To this end, let $\mathbf{v} = \left(\begin{matrix}0\\v\\0\end{matrix}\right)$. Since
	$\left(\begin{matrix}a & b\\c & d\end{matrix}\right) \cdot \left(\begin{matrix}0\\v\\0\end{matrix}\right) =
	\left(\begin{matrix}0\\v\\0\end{matrix}\right),
\chi^{SL_2(k)}_v$ is trivial.
This shows that for each $\mu\in k$, $\psi(\sigma_\mu)$ is a distinct element in the 1-cohomology, where
\begin{align*}
	\sigma_\mu\left(\begin{matrix}a & b\\c & d\end{matrix}\right) = \left(\begin{matrix}(cd)^{2^r}\mu\\(bc)^{2^r}\mu\\(ad)^{2^r}\mu\end{matrix}\right).
\end{align*}

We now consider the action of $Z(L_\alpha)^\circ$, the connected centre of the Levi subgroup $L_\alpha$. We have $Z(L_\alpha)^\circ = \langle \gamma^\vee(x)\,|\,x \in k \rangle$ where $\gamma$ is a root in $\Phi$ such that $\langle \alpha, \gamma \rangle = 0$.

Since $\langle \alpha, \alpha + \beta\rangle = 0$, we may choose $\gamma = \alpha + \beta$.
Therefore $Z(L_ \alpha)^\circ = \langle (\alpha + \beta)^\vee(x)\,|\,x \in k \rangle$. Taking an element $\mathbf{s} = (\alpha + \beta)^\vee(s)$ of $Z(L_\alpha)^\circ$ we compute the action of $\mathbf{s}$ on the 1-cocycle $\sigma_\mu$ as follows.
\begin{align*}
\left(\mathbf{s}\cdot \sigma_\mu\right)
\left(\begin{matrix} a & b \\ c & d\end{matrix} \right) 
&=
(\alpha + \beta)^\vee(s) \epsilon_\beta \left(\mu (cd)^{2^r} \right)\epsilon_{\alpha+\beta} \left(\mu(bc)^{2^r} \right)\epsilon_{2\alpha + \beta} \left(\mu(ab)^{2^r} \right)(\alpha + \beta)^\vee(s)^{-1}\\
&=  \epsilon_\beta\left(s^{\langle\beta , \alpha+\beta\rangle}\mu (cd)^{2^r} \right)\epsilon_{\alpha+\beta} \left(s^{\langle \alpha+\beta, \alpha+\beta \rangle} \mu(bc)^{2^r} \right)\epsilon_{2\alpha + \beta} \left(s^{\langle 2\alpha+\beta, \alpha+\beta\rangle}\mu(ab)^{2^r}\right) \\
&=
\left(\begin{matrix}
(s^2\mu)(cd)^{2^r} \\
(s^2\mu)(bc)^{2^r} \\
(s^2\mu)(ab)^{2^r}
\end{matrix}\right).
\end{align*}

So we see that the infinitely many equivalence classes of 1-cocycles collapse to just two classes when we consider the action of $Z(L_\alpha)^\circ$, represented by the 1-cocycles $\sigma_0$ and $\sigma_1$.

\subsection{$V = R_u(P_\beta)$}
\label{b2:beta}

Let $V_\beta = R_u(P_\beta) = \langle U_\alpha, U_{\alpha + \beta}, U_{2\alpha + \beta} \rangle$.

Note that $V$ is not abelian in general. The Group Law for $V$ can be computed as follows. Writing $\mathbf{v}\in V_\beta$ as
\begin{align*}
	\left(\begin{matrix} v_1\\v_2\\v_3\end{matrix}\right) = \epsilon_\alpha(v_1)\epsilon_{\alpha+\beta}(v_2)\epsilon_{2\alpha+\beta}(v_3),
\end{align*}
for $\mathbf{v}, \mathbf{w}\in V_\beta$ we have
\begin{align*}
\mathbf{v}\mathbf{w}
&= 
\epsilon_\alpha(v_1)\epsilon_{\alpha+\beta}(v_2)\epsilon_{2\alpha+\beta}(v_3) \epsilon_\alpha(w_1)\epsilon_{\alpha+\beta}(w_2)\epsilon_{2\alpha+\beta}(w_3)\\
&= 
\epsilon_\alpha(v_1)\epsilon_{\alpha+\beta}(v_2) \epsilon_\alpha(w_1)\epsilon_{\alpha+\beta}(w_2)\epsilon_{2\alpha+\beta}(v_3)\epsilon_{2\alpha+\beta}(w_3)\\
&= 
\epsilon_\alpha(v_1) \epsilon_\alpha(w_1) \epsilon_{\alpha + \beta}(v_2)\epsilon_{2\alpha+\beta}(2v_2w_1)\epsilon_{\alpha+\beta}(w_2)\epsilon_{2\alpha+\beta}(v_3)\epsilon_{2\alpha+\beta}(w_3)\\
&= 
\epsilon_\alpha(v_1 + w_1) \epsilon_{\alpha + \beta}(v_2 + w_2)\epsilon_{2\alpha+\beta}(v_3 + w_3 + 2v_2w_1)\\
&=
\left(\begin{matrix}
v_1 + w_1 \\
v_2 + w_2 \\
v_3 + w_3 + 2v_2w_1
\end{matrix}\right).
\end{align*}

We compute the following.
\begin{align*}
\left(\begin{matrix} 1 & u \\ 0 & 1\end{matrix}\right) \cdot \mathbf{v} &= \rho_r\left(\begin{matrix} 1 & u \\ 0 & 1\end{matrix}\right) \mathbf{v}\left( \rho_r\left(\begin{matrix} 1 & u \\ 0 & 1\end{matrix}\right)\right)^{-1} \\
&=\epsilon_\beta (u^{p^r}) \epsilon_\alpha (v_1)\epsilon_{\alpha+\beta}(v_2) \epsilon_{2\alpha+\beta}(v_3) \epsilon_\beta (-u^{p^r}) \\
&=\epsilon_\alpha (v_1) \epsilon_\beta (u^{p^r}) \epsilon_{\alpha+\beta}(u^{p^r}v_1) \epsilon_{2\alpha+\beta}(u^{p^r}v_1^2) \epsilon_{\alpha+\beta}(v_2) \epsilon_{2\alpha+\beta}(v_3) \epsilon_\beta (-u^{p^r})  \\
&=\epsilon_\alpha (v_1) \epsilon_\beta (u^{p^r}) \epsilon_{\alpha+\beta}(v_2 + u^{p^r}v_1) \epsilon_{2\alpha+\beta}(v_3 + u^{p^r}v_1^2)  \epsilon_\beta (-u^{p^r})  \\
&=\epsilon_\alpha (v_1) \epsilon_{\alpha+\beta}(u^{p^r}v_1) \epsilon_{2\alpha+\beta}(u^{p^r}v_1^2) \epsilon_{\alpha+\beta}(v_2) \epsilon_{2\alpha+\beta}(v_3)\epsilon_\beta (u^{p^r})  \epsilon_\beta (-u^{p^r})  \\
&=\epsilon_\alpha (v_1)  \epsilon_{\alpha+\beta}(v_2 + u^{p^r}v_1) \epsilon_{2\alpha+\beta}(v_3 + u^{p^r}v_1^2) \\
&= \left(\begin{matrix} v_1 \\ v_2 + u^{p^r}v_1\\ v_3 + u^{p^r}v_1^2 \end{matrix}\right)\\
\left(\begin{matrix} t & 0 \\ 0 & t^{-1}\end{matrix}\right) \cdot \mathbf{v} &=
\rho_r\left(\begin{matrix} t & 0 \\ 0 & t^{-1}\end{matrix}\right) \mathbf{v}\left( \rho_r\left(\begin{matrix} t & 0 \\ 0 & t^{-1}\end{matrix}\right)\right)^{-1} \\
&= \beta^\vee(t^{p^r}) \epsilon_\alpha (v_1)
\epsilon_{\alpha+\beta}(v_2)
\epsilon_{2\alpha+\beta}(v_3) (\beta^\vee(t^{p^r}))^{-1} \\
&= \epsilon_\alpha \left(\alpha(\beta^\vee(t^{p^r}))v_1\right)
\epsilon_{\alpha+\beta} \left((\alpha+\beta)(\beta^\vee(t^{p^r}))v_2 \right)
\epsilon_{2\alpha+\beta} \left((2\alpha+\beta)(\beta^\vee(t^{p^r}))v_3 \right)\\
&= \epsilon_\alpha \left((t^{p^r})^{\langle \alpha, \beta \rangle}v_1 \right)
\epsilon_{\alpha+\beta} \left((t^{p^r})^{\langle \alpha+\beta, \beta \rangle}v_2 \right)
\epsilon_{2\alpha+\beta} \left((t^{p^r})^{\langle 2\alpha+\beta, \beta \rangle}v_3 \right)\\
&= \left(\begin{matrix} t^{-p^r}v_1 \\ t^{p^r}v_2\\ v_3 \end{matrix}\right) \\
\left(\begin{matrix} 0 & 1 \\ -1 & 0 \end{matrix}\right) \cdot \mathbf{v} &=
\rho_r\left(\begin{matrix} 0 & 1 \\ -1 & 0\end{matrix}\right) \mathbf{v}\left( \rho_r\left(\begin{matrix} 0 & 1 \\ -1 & 0\end{matrix}\right)\right)^{-1} \\
&= n_\beta  \epsilon_\alpha (v_1)\epsilon_{\alpha+\beta}(v_2) \epsilon_{2\alpha+\beta}(v_3) n_\beta^{-1}\\
&= n_\beta  \epsilon_\alpha (v_1) n_\beta^{-1}n_\beta \epsilon_{\alpha+\beta}(v_2) n_\beta^{-1} n_\beta \epsilon_{2\alpha+\beta}(v_3) n_\beta^{-1}\\
&= \epsilon_{\alpha+\beta} (v_1) \epsilon_{\alpha}(-v_2)  \epsilon_{2\alpha+\beta}(v_3) \\
&=\epsilon_{\alpha}(-v_2)  \epsilon_{\alpha+\beta} (v_1)  \epsilon_{2\alpha+\beta}(v_3 - 2v_1v_2) \\
&= \left(\begin{matrix} -v_2 \\ v_1 \\ v_3 - 2v_1v_2 \end{matrix}\right).
\end{align*}

This is enough to determine the action of $SL_2(k)$ on $V_\beta$ (we omit the details).
\begin{align*}
\left(\begin{matrix}
a & b \\ c & d
\end{matrix} \right) \cdot \mathbf{v} &=
\left(\begin{matrix}
c^{p^r}v_2 + d^{p^r}v_1 \\
a^{p^r}v_2 + b^{p^r}v_1 \\
v_3 + (ac)^{p^r}v_2^2 + (bd)^{p^r}v_1^2 + 2(bc)^{p^r}v_1v_2
\end{matrix}\right).
\end{align*}

Let $x\in H^1(SL_2(k), V_\beta)_{\rho_r}$. Then by Proposition \ref{trivial_on_t} there exists $\sigma \in Z^1(SL_2(k), V_\beta)_{\rho_r}$ such that $\sigma(y) = 1$ for all  $y\in T_2(k)$ and $\phi(\sigma) = x$. By Lemma \ref{lem:first} we can apply Lemma \ref{lem:second} to yield
\begin{align*}
\sigma\left(\begin{matrix}a & b\\0 & a^{-1}\end{matrix}\right) &= \epsilon_{\alpha+\beta}(\mu_2(ab)^{2^{r-1}}),\\
\sigma\left(\begin{matrix}d^{-1} & 0\\c & d\end{matrix}\right) &= \epsilon_{\alpha}(\mu_1(cd)^{2^{r-1}}) \quad\textrm{(Remark \ref{rem:second})},
\end{align*}
for some $\mu_1,\mu_2 \in k$ if $p = 2$, and $\sigma(B_2(k)) = 1$ if $p>2$. Hence by Lemma \ref{} $H^1(SL_2(k), V_\beta)_{\rho_r}$ is trivial if $p > 2$.

Proceeding with $p=2$ we see that $V_\beta$ is abelian so we revert to additive notation.

By Equation \ref{prop:A:eqn2} there exists $\lambda_3 \in k$ such that
\begin{align*}
\sigma\left(\begin{matrix}0 & 1\\1 & 0\end{matrix}\right) = \left(\begin{matrix}0\\0\\ \lambda_3\end{matrix}\right),
\end{align*}
and by Equation \ref{prop:A:eqn1}
\begin{align*}
\sigma\left(
	\begin{matrix} 0 & 1 \\ 1 & 0 \end{matrix}
\right)
&=
\sigma
	\left(\begin{matrix} 1 & 1 \\ 0 & 1 \end{matrix}\right)
 +
\left(\begin{matrix} 1 & 1 \\ 0 & 1 \end{matrix}\right) \cdot
\left(
	\sigma
		\left(\begin{matrix} 1 & 0 \\ 1 & 1 \end{matrix}
	\right) +
	\left(\begin{matrix} 1 & 0 \\ 1 & 1 \end{matrix}\right)\cdot
	\sigma
		\left(\begin{matrix} 1 & 1 \\ 0 & 1 \end{matrix}\right)
\right)\\ 
&=
\left(\begin{matrix} 0 \\ \mu_2 \\ 0 \end{matrix}\right)
+
\left(\begin{matrix} 1 & 1 \\ 0 & 1 \end{matrix}\right) \cdot
\left(
	\left(\begin{matrix} \mu_1 \\ 0 \\ 0 \end{matrix}\right)
	+
	\left(\begin{matrix} 1 & 0 \\ 1 & 1 \end{matrix}\right)\cdot
	\left(\begin{matrix} 0 \\ \mu_2 \\ 0 \end{matrix}\right)
\right)\\ 
&=
\left(\begin{matrix} 0 \\ \mu_2 \\ 0 \end{matrix}\right)
+
\left(\begin{matrix} 1 & 1 \\ 0 & 1 \end{matrix}\right) \cdot
\left(
	\left(\begin{matrix} \mu_1 \\ 0 \\ 0 \end{matrix}\right)
	+
	\left(\begin{matrix} \mu_2 \\ \mu_2 \\ \mu_2^2 \end{matrix}\right)
\right)\\ 
&=
\left(\begin{matrix} 0 \\ \mu_2 \\ 0 \end{matrix}\right)
+
\left(\begin{matrix} 1 & 1 \\ 0 & 1 \end{matrix}\right) \cdot
\left(\begin{matrix} \mu_1 + \mu_2 \\ \mu_2 \\ \mu_2^2 \end{matrix}\right)\\
&=
\left(\begin{matrix} 0 \\ \mu_2 \\ 0 \end{matrix}\right)
+
\left(\begin{matrix} \mu_1 + \mu_2 \\ \mu_1 \\ \mu_1^2 \end{matrix}\right)\\
&=
\left(\begin{matrix} \mu_1 + \mu_2 \\ \mu_1 + \mu_2 \\ \mu_1^2 \end{matrix}\right).
\end{align*}

Therefore $\mu_1 = \mu_2$ and $\lambda_3 = \mu_1^2 (=\mu_2^2)$. Writing $\mu = \mu_1 (=\mu_2)$, we have
\begin{align*}
\sigma\left(\begin{matrix}a & b\\0 & a^{-1}\end{matrix}\right) = \left(\begin{matrix}0 \\ \mu(ab)^{2^{r-1}}\\ 0\end{matrix}\right), \quad
\sigma\left(\begin{matrix}d^{-1} & 0\\c & d\end{matrix}\right) = \left(\begin{matrix}\mu(cd)^{2^{r-1}}\\ 0\\0\end{matrix}\right), \quad
\sigma\left(\begin{matrix}0 & 1\\1 & 0\end{matrix}\right) = \left(\begin{matrix}0\\ 0\\ \mu^2\end{matrix}\right).
\end{align*}

Suppose $c\neq0$. Then
\begin{align}
\sigma\left(\begin{matrix} a & b \\ c & d \end{matrix}\right) 
&=
\sigma
	\left(\begin{matrix} 1 & ac^{-1} \\ 0 & 1 \end{matrix}\right)
	\left(\begin{matrix} 0 & 1 \\ 1 & 0 \end{matrix}\right)
	\left(\begin{matrix} c & d \\ 0 & c^{-1} \end{matrix}\right)
 \nonumber\\
&=
\sigma
	\left(\begin{matrix} 1 & ac^{-1} \\ 0 & 1 \end{matrix}\right)
 +
\left(\begin{matrix} 1 & ac^{-1} \\ 0 & 1 \end{matrix}\right) \cdot
\left(
	\sigma
		\left(\begin{matrix} 0 & 1 \\ 1 & 0 \end{matrix}\right)
	 +
	\left(\begin{matrix} 0 & 1 \\ 1 & 0 \end{matrix}\right) \cdot
	\sigma
		\left(\begin{matrix} c & d \\ 0 & c^{-1} \end{matrix}\right)
\right)\nonumber \\
&=
\left(\begin{matrix} 0 \\ \mu(ac^{-1})^{2^{r-1}} \\ 0 \end{matrix}\right)
+
\left(\begin{matrix} 1 & ac^{-1} \\ 0 & 1 \end{matrix}\right) \cdot
\left(
	\left(\begin{matrix} 0 \\ 0 \\ \mu^2 \end{matrix}\right)
	+
	\left(\begin{matrix} 0 & 1 \\ 1 & 0 \end{matrix}\right) \cdot
	\left(\begin{matrix} 0 \\ \mu(cd)^{2^{r-1}} \\ 0 \end{matrix}\right)
\right) \nonumber\\
&=
\left(\begin{matrix} 0 \\ \mu(ac^{-1})^{2^{r-1}} \\ 0 \end{matrix}\right)
+
\left(\begin{matrix} 1 & ac^{-1} \\ 0 & 1 \end{matrix}\right) \cdot
\left(
	\left(\begin{matrix} 0 \\ 0 \\ \mu^2 \end{matrix}\right)
	+
	\left(\begin{matrix} \mu(cd)^{2^{r-1}} \\ 0 \\ 0 \end{matrix}\right)
\right) \nonumber\\
&=
\left(\begin{matrix} 0 \\ \mu(ac^{-1})^{2^{r-1}} \\ 0 \end{matrix}\right)
+
\left(\begin{matrix} 1 & ac^{-1} \\ 0 & 1 \end{matrix}\right) \cdot
\left(\begin{matrix} \mu(cd)^{2^{r-1}}\\ 0 \\ \mu^2 \end{matrix}\right) \nonumber\\
&=
\left(\begin{matrix} 0 \\ \mu(ac^{-1})^{2^{r-1}} \\ 0 \end{matrix}\right)
+
\left(\begin{matrix} \mu(cd)^{2^{r-1}} \\ (ac^{-1})^{2^r} \mu(cd)^{2^{r-1}}  \\ \mu^2 +  (ac^{-1})^{2^r} \left(\mu(cd)^{2^{r-1}}\right)^2  \end{matrix}\right)
 \nonumber\\
&=
\left(\begin{matrix}  \mu(cd)^{2^{r-1}}  \\ \mu\left(ac^{-1} + a^2c^{-1}d \right)^{2^{r-1}} \\ \mu^2\left( 1 + ad\right)^{2^r} \end{matrix}\right) \nonumber \\
&=
\left(\begin{matrix}  \mu(cd)^{2^{r-1}}  \\ \mu\left(ab \right)^{2^{r-1}} \\ \mu^2\left( bc \right)^{2^r} \end{matrix}\right). \label{eqn:b2b}
\end{align}

Note that substituting $c=0$ gives
\begin{align*}
\sigma\left(\begin{matrix}a & b \\ 0 & a^{-1}\end{matrix}\right) = \left(\begin{matrix}0 \\ \mu(ab)^{2^{r-1}}\\ 0\end{matrix}\right),
\end{align*}
substituting $b=0$ gives
\begin{align*}
\sigma\left(\begin{matrix}d^{-1} & 0 \\ c & d\end{matrix}\right) = \left(\begin{matrix}\mu(cd)^{2^{r-1}} \\ 0 \\0 \end{matrix}\right),
\end{align*}
and substituting $a = d = 0, b = c=1$ gives
\begin{align*}
\sigma\left(\begin{matrix}0 & 1 \\ 1 & 0\end{matrix}\right) = \left(\begin{matrix}0  \\ 0\\ \mu^2\end{matrix}\right),
\end{align*}
so we take Equation \ref{eqn:b2b} to be the general form of a 1-cocycle $\sigma:SL_2(k)\rightarrow V_\beta$ such that $\sigma(y) = 1$ for all  $y\in T_2(k)$. We can now show that Equation \ref{eqn:b2b} is a well-defined by verifying Equations \ref{pita1}--\ref{pita3}, but as in the previous calculation we omit the details.

We calculate the 1-cohomology by calculating $\tau = \sigma + \chi^{SL_2(k)}_\mathbf{v}$ such that $\mathbf{v} \in V_\beta$ is fixed under the action of $T_2(k)$. To this end let $\mathbf{v} = \left(\begin{matrix}0 \\ 0 \\ v_3\end{matrix}\right)$.
\begin{align*}
\tau\left(\begin{matrix} a & b \\ c & d \end{matrix}\right) &=
\mathbf{v} + \sigma\left(\begin{matrix} a & b \\ c & d \end{matrix}\right) + 
\left(\begin{matrix} a & b \\ c & d \end{matrix}\right) \cdot \mathbf{v} \\
 &=
\left(\begin{matrix} 0 \\ 0 \\ v_3 \end{matrix}\right) + 
\left(\begin{matrix}  \mu(cd)^{2^{r-1}}  \\ \mu\left(ab \right)^{2^{r-1}} \\ \mu^2\left( bc \right)^{2^r} \end{matrix}\right)+ 
\left(\begin{matrix} a & b \\ c & d \end{matrix}\right) \cdot 
\left(\begin{matrix} 0 \\ 0 \\ v_3 \end{matrix}\right) \\
 &=
\left(\begin{matrix} 0 \\ 0 \\ v_3 \end{matrix}\right) + 
\left(\begin{matrix}  \mu(cd)^{2^{r-1}}  \\ \mu\left(ab \right)^{2^{r-1}} \\ \mu^2\left( bc \right)^{2^r} \end{matrix}\right)+ 
\left(\begin{matrix} 0 \\ 0 \\ v_3 \end{matrix}\right) \\
&= \left(\begin{matrix}  \mu(cd)^{2^{r-1}}  \\ \mu\left(ab \right)^{2^{r-1}} \\ \mu^2\left( bc \right)^{2^r} \end{matrix}\right).
\end{align*}
Therefore, for each $\mu$ in $k$ we get a distinct element of the 1-cohomology $\psi(\sigma_\mu)$, where
\begin{align*}
\sigma_\mu\left(\begin{matrix} a & b \\ c & d \end{matrix}\right) &=
\left(\begin{matrix}  \mu(cd)^{2^{r-1}}  \\ \mu\left(ab \right)^{2^{r-1}} \\ \mu^2\left( bc \right)^{2^r} \end{matrix}\right).
\end{align*}

But as before if we consider the action of $Z(L_\beta)$ on our 1-cocycles
\begin{align*}
(\mathbf{s}\cdot \sigma_\mu)\left(\begin{matrix} a & b \\ c & d \end{matrix}\right) &=
(2\alpha + \beta)^\vee(s) \cdot \sigma_\mu\left(\left(\begin{matrix} a & b \\ c & d \end{matrix}\right)\right)\\
&=
\left(\begin{matrix}  (s\mu)(cd)^{2^{r-1}}  \\ (s\mu)\left(ab \right)^{2^{r-1}} \\ (s\mu)^2\left( bc \right)^{2^r} \end{matrix}\right).
\end{align*}
our infinitely equivalence classes collapse to just two classes, $\psi(\sigma_0), \psi(\sigma_1)$.

	\section{$G = G_2$}
	\label{g2}

	Let $G=G_2$. Fix a maximal torus, labeling the positive simple roots $\Delta=\{\alpha, \beta\}$ with $\beta$ being the long root. 

	\subsection{$V = R_u(P_\alpha)$}
	\label{g2:alpha}
	Let $V_\alpha = R_u(P_\alpha) = \langle U_\beta, U_{\alpha+\beta}, U_{2\alpha+\beta}, U_{3\alpha+\beta}, U_{3\alpha+2\beta}\rangle$.
	We write $v \in V_\alpha$ as a column vector,
	\begin{align*}
	\left(\begin{matrix}
			v_1\\
			v_2\\
			v_3\\
			v_4\\
			v_5
			\end{matrix}\right) &=
\epsilon_{\beta}(v_1)
\epsilon_{\alpha+\beta}(v_2)
\epsilon_{2\alpha+\beta}(v_3)
\epsilon_{3\alpha+\beta}(v_4)
	\epsilon_{3\alpha+2\beta}(v_5),
	\end{align*}
	so that we can clearly present the group law for $V_\alpha$, derived from the commutation relations in \cite[\S 33.5]{humphreys1975linear}.
	\begin{align*}
	\left(\begin{matrix}
			u_1\\
			u_2\\
			u_3\\
			u_4\\
			u_5
			\end{matrix}\right)\times
	\left(\begin{matrix}
			v_1\\
			v_2\\
			v_3\\
			v_4\\
			v_5
			\end{matrix}\right)
	&= \left(\begin{matrix}
			u_1 + v_1\\
			u_2 + v_2\\
			u_3 + v_3\\
			u_4 + v_4\\
			u_5 + v_5 + 3u_3v_2 - u_4v_1\\
			\end{matrix}\right).
	\end{align*}
	Let $\sigma$ be in $Z^1(SL_2, V_\alpha)_{\rho_r}$ such that $\sigma\left(\begin{matrix}t & 0\\0 & t^{-1}\end{matrix}\right) = 1$.
	By Proposition \ref{claim1}
	\begin{align*}
	\sigma\left(\begin{matrix} 1 & u \\ 0 & 1 \end{matrix}\right) =
\epsilon_{2\alpha+\beta}(x_3(u))
\epsilon_{3\alpha+\beta}(x_4(u))
	\end{align*}
	where the $x_3, x_4$ satisfy
	\begin{align}
	x_3(t^2u) &= t^{p^r}x_3(u)\label{eqn:g2x3}\\
							 x_4(t^2b) &= t^{3p^r}x_4(u)\label{eqn:g2x4},
	\end{align}
	for all $t\in k^*, u\in k$.

	Furthermore, since
	\begin{align*}
\sigma\left(\begin{matrix} 1 & u_1 + u_2 \\ 0 & 1\end{matrix}\right)
&= \left(\sigma\left(\begin{matrix} 1 & u_1 \\ 0 & 1\end{matrix}\right)\right)
	\left(\begin{matrix} 1 & u_1 \\ 0 & 1\end{matrix}\right) \cdot
	\sigma
	\left(\begin{matrix} 1 & u_2 \\ 0 & 1\end{matrix}\right),
	\end{align*}
	we get
	\begin{align*}
&\epsilon_{2\alpha+\beta}\left(x_3(u_1+u_2)\right)\epsilon_{3\alpha+\beta}\left(x_4(u_1+u_2)\right)
\\ &\quad	= \epsilon_{2\alpha+\beta}\left(x_3(u_1)\right)\epsilon_{3\alpha+\beta}\left(x_4(u_1)\right)
\epsilon_\alpha(u_1^{p^r})
\epsilon_{2\alpha+\beta}\left(x_3(u_2)\right)\epsilon_{3\alpha+\beta}\left(x_4(u_2)\right)
\epsilon_\alpha(-u_1^{p^r})
\\ &\quad	= \epsilon_{2\alpha+\beta}\left(x_3(u_1)\right)\epsilon_{3\alpha+\beta}\left(x_4(u_1)\right)
\epsilon_\alpha(u_1^{p^r})
\epsilon_{2\alpha+\beta}\left(x_3(u_2)\right)
\epsilon_\alpha(-u_1^{p^r})
\epsilon_{3\alpha+\beta}\left(x_4(u_2)\right)
\\ &\quad	= \epsilon_{2\alpha+\beta}\left(x_3(u_1)\right)\epsilon_{3\alpha+\beta}\left(x_4(u_1)\right)
\epsilon_{2\alpha+\beta}\left(x_3(u_2)\right)
\epsilon_{3\alpha+\beta}\left(x_4(u_2) -3u_1^{p^r}x_3(u_2)\right)
	\\ &\quad	= 
\epsilon_{2\alpha+\beta}\left(x_3(u_1) + x_3(u_2)\right)
\epsilon_{3\alpha+\beta}\left(x_4(u_1) + x_4(u_2) - 3u_1^{p^r}x_3(u_2)\right)
	\end{align*}
	We see that $x_3$ is an additive polynomial, so it is of the form
	\begin{align*}
	x_3(\lambda)&= \sum_{i=0}^m \mu_i \lambda^{p^i}\quad(\textrm{\cite[\S 20.3, Lemma A]{humphreys1975linear}}),
	\end{align*}
	for some $\mu_i\in k, m\in \mathbb{N}$, and $x_4$ satisfies
	\begin{align}\label{eqn:g2x42}
	x_4(\lambda_1 + \lambda_2) = x_4(\lambda_1) + x_4(\lambda_2) -3\lambda_2^{p^r}x_3(\lambda_2).
	\end{align}
	Suppose $x_3\neq 0$, so there exists $j\geq 0$ such that $\mu_j\neq 0$. Then by Equation \ref{eqn:g2x3}
	\begin{align*}
	&\mu_j (t^2u)^{p^j} = t^{p^r}\mu_j u^{p^j}\\
		&\Rightarrow t^{2p^j} = t^{p^r}\\
		&\Rightarrow p = 2, j = r-1.
		\end{align*}
		But then $x_3 = 0$, for
		\begin{align*}
		x_4(0) &= x_4(\lambda + \lambda) \\
							&= x_4(\lambda)+x_4(\lambda)-3\lambda^{2^r}x_3(\lambda)\quad(\textrm{Equation \ref{eqn:g2x42}}) \\
							&= 3\lambda^{2^r}x_3(\lambda),
		\end{align*}
		which implies that $x_3$ is constant, hence zero, as $\sigma(T_2(k)) = 1$.

		Therefore $x_3 = 0$ in any case, and so by Equation \ref{eqn:g2x42}, $x_4$ is an additive polynomial. Then it is of the form
		\begin{align*}
		x_4(\lambda) = \sum_{i=0}^n \nu_i \lambda^{p^r},
		\end{align*}
		for some $\nu_i \in k, n\in\mathbb{N}$.
		If $x_4\neq 0$ then some $\nu_j\neq 0$, and we get
		\begin{align*}
		&\nu_j(t^2u)^{p^j} = t^{3p^r}\nu_j u^{p^j}\quad(\textrm{Equation \ref{eqn:g2x4}})\\
			&\Rightarrow t^{2p^j} = t^{3p^r}\\
			&\Rightarrow 2p^j = 3p^r,
		\end{align*}
		which implies that 2 divides $p$ and 3 divides $p$, a contradiction. Hence $x_4=0$ and
		\begin{align*}
		\sigma\left(\begin{matrix}* & *\\0 & *\end{matrix}\right) &= \sigma\left(\begin{matrix}1 & *\\0 & 1\end{matrix}\right) \quad(\textrm{Proposition \ref{imb:imu}}) \\
			&= 1.
			\end{align*}
			Therefore $H^1(SL_2(k), V_\alpha)_{\rho_r}$ is trivial by Lemma \ref{sl2_b_inj}.

The purpose of this calculation was to show that although the hypotheses of Lemma \ref{lem:first} fail (Remark \ref{g2counter}) we still (trivially) obtain the same result as the conclusion of Lemma \ref{lem:second}; that is, the equation for $\sigma(B_2(k)$.

			%\subsection{$V = R_u(P_\beta)$}
			%\label{g2:beta}
			%TODO Calculation goes here if I have time.

			\section{$G = C_3$}
			\label{c3}
			Let $G=C_3$. Fix a maximal torus, labeling the positive simple roots $\Delta=\{\alpha, \beta, \gamma\}$ with $\gamma$ being the long root and connected to $\beta$. 

			\subsection{$V = R_u(P_\alpha)$}
			\label{c3:alpha}

			Let $V_\alpha = R_u(P_\alpha) = \langle U_\beta, U_\gamma, U_{\alpha+\beta}, U_{\beta+\gamma}, U_{\alpha+\beta+\gamma}, U_{2\beta+\gamma}, U_{\alpha+2\beta+\gamma}, U_{2\alpha+2\beta+\gamma}\rangle$.
			We will write $v$ in $V_\alpha$ as a column vector,
			\begin{align*}
			\left(\begin{matrix}
					v_1\\
					v_2\\
					v_3\\
					v_4\\
					v_5\\
					v_6\\
					v_7\\
					v_8
					\end{matrix}\right)
=\epsilon_{\beta}(v_1)
\epsilon_{\gamma}(v_2)
\epsilon_{\alpha+\beta}(v_3)
\epsilon_{\beta+\gamma}(v_4)
\epsilon_{\alpha+\beta+\gamma}(v_5)
\epsilon_{2\beta+\gamma}(v_6)
\epsilon_{\alpha+2\beta+\gamma}(v_7)
	\epsilon_{2\alpha+2\beta+\gamma}(v_8),
	\end{align*}
	so that we can write the group law for $V_\alpha$ (\cite[\S 33.3, \S 33.4]{humphreys1975linear}) as
	\begin{align*}
	\left(\begin{matrix}
			u_1\\
			u_2\\
			u_3\\
			u_4\\
			u_5\\
			u_6\\
			u_7\\
			u_8
			\end{matrix}\right)
	\times
	\left(\begin{matrix}
			v_1\\
			v_2\\
			v_3\\
			v_4\\
			v_5\\
			v_6\\
			v_7\\
			v_8
			\end{matrix}\right)
	&=
	\left(\begin{matrix}
			u_1 + v_1\\
			u_2 + v_2\\
			u_3 + v_3\\
			u_4 + v_4 + u_2 + v_1\\
			u_5 + v_5 - u_3v_2\\
			u_6 + v_6 + u_2v_1^2 + 2u_4v_1\\
			u_7 + v_7 + u_2u_3v_1 + u_2v_1v_3 + u_5v_1 + u_4v_3\\
			u_8 + v_8 - u_3^2v_2 - 2u_3v_2v_3 + 2u_5v_3
			\end{matrix}\right).
	\end{align*}
	Let $\sigma$ be in $Z^1(SL_2, V_\alpha)$ such that $\sigma\left(\begin{matrix}* & 0\\0 & *\end{matrix}\right)$. By Proposition \ref{claim1}
	\begin{align*}
	\sigma\left(\begin{matrix}1 & u\\0 & 1\end{matrix}\right) = 
\epsilon_{\alpha+\beta}\left(x_3(u)\right)
\epsilon_{\alpha+\beta+\gamma}\left(x_5(u)\right)
\epsilon_{2\alpha+2\beta+\gamma}\left(x_8(u)\right)
	\end{align*}
	where $x_3, x_5, x_8$ satisfy
	\begin{align}
	x_3(t^2u) &= t^{p^r}x_3(u)\label{c3x3} \\
	x_5(t^2u) &= t^{p^r}x_5(u)\label{c3x5} \\
	x_8(t^2u) &= t^{2p^r}x_8(u),\label{c3x8}
	\end{align}
	Since $\alpha + \delta \notin \Phi$ for $\delta \in \{\alpha + \beta, \alpha + \beta + \gamma, 2\alpha + 2\beta + \gamma\}$, $\sigma\left(\begin{matrix}1 & u\\0 & 1\end{matrix}\right)$ is unchanged under the action of $\left(\begin{matrix}1 & *\\0 & 1\end{matrix}\right)$ for any $u\in k$. Then
\begin{align*}
	\sigma\left(\begin{matrix} 1 & u_1 + u_2 \\0 & 1\end{matrix}\right) &= 
	\left(\sigma\left(\begin{matrix} 1 & u_1\\0 & 1\end{matrix}\right)\right)
	\left(\left(\begin{matrix} 1 & u_1\\0 & 1\end{matrix}\right)\cdot
	\sigma\left(\begin{matrix} 1 & u_2\\0 & 1\end{matrix}\right)\right) \\ &=
	\sigma\left(\begin{matrix} 1 & u_1\\0 & 1\end{matrix}\right)
	\sigma\left(\begin{matrix} 1 & u_2 \\0 & 1\end{matrix}\right).
\end{align*}
Therefore
\begin{align*}
&\epsilon_{\alpha+\beta}\left(x_3(u_1 + u_2)\right)
\epsilon_{\alpha+\beta+\gamma}\left(x_5(u_1 + u_2)\right)
\epsilon_{2\alpha+2\beta+\gamma}\left(x_8(u_1 + u_2)\right) \\
&=
\epsilon_{\alpha+\beta}\left(x_3(u_1)\right)
\epsilon_{\alpha+\beta+\gamma}\left(x_5(u_1)\right)
\epsilon_{2\alpha+2\beta+\gamma}\left(x_8(u_1)\right)
\epsilon_{\alpha+\beta}\left(x_3(u_2)\right) \\
&\qquad\epsilon_{\alpha+\beta+\gamma}\left(x_5(u_2)\right) 
\epsilon_{2\alpha+2\beta+\gamma}\left(x_8(u_2)\right) \\
&=
\epsilon_{\alpha+\beta}\left(x_3(u_1)\right)
\epsilon_{\alpha+\beta+\gamma}\left(x_5(u_1)\right)
\epsilon_{\alpha+\beta}\left(x_3(u_2)\right)
\epsilon_{2\alpha+2\beta+\gamma}\left(x_8(u_1)\right) \\
&\qquad\epsilon_{\alpha+\beta+\gamma}\left(x_5(u_2)\right) 
\epsilon_{2\alpha+2\beta+\gamma}\left(x_8(u_2)\right) \\
&=
\epsilon_{\alpha+\beta}\left(x_3(u_1)\right)
\epsilon_{\alpha+\beta}\left(x_3(u_2)\right)
\epsilon_{\alpha+\beta+\gamma}\left(x_5(u_1)\right)
\epsilon_{2\alpha+2\beta+\gamma}\left(x_8(u_1) + 2x_3(u_2)x_5(u_1)\right) \\
&\qquad\epsilon_{\alpha+\beta+\gamma}\left(x_5(u_2)\right) 
\epsilon_{2\alpha+2\beta+\gamma}\left(x_8(u_2)\right) \\
&=
\epsilon_{\alpha+\beta}\left(x_3(u_1) + x_3(u_2)\right)
\epsilon_{\alpha+\beta+\gamma}\left(x_5(u_1) + x_5(u_2)\right) \\
&\qquad\epsilon_{2\alpha+2\beta+\gamma}\left(x_8(u_1) + x_8(u_2) + 2x_3(u_2)x_5(u_1)\right)
\end{align*}

We see that $x_3, x_5$ are additive polynomials, so by \cite[\S 20.3, Lemma A]{humphreys1975linear} they are of the form
\begin{align*}
	x_3(\lambda) &= \sum_{i=0}^m \mu_i\lambda^{p^r} \\
	x_5(\lambda) &= \sum_{i=0}^n \nu_i\lambda^{p^r},
\end{align*}
for some $\mu_i, \nu_i \in k, m,n \in \mathbb{N}$, while $x_8$ is of the form
\begin{align}\label{c3x8}
	x_8(\lambda_1 + \lambda_2) = x_8(\lambda_1) + x_8(\lambda_2) + 2x_3(\lambda_2)x_5(\lambda_1).
\end{align}

Suppose $x_3 \neq 0$. Then there exists $j \geq 0$ such that $\mu_j \neq 0$, so by Equation \ref{c3x3}
\begin{align*}
	&\mu_j(t^2u)^{p^j} = t^{p^r}\mu_ju^{p^j} \\
	&\Rightarrow t^{2p^j} = t^{p^r} \\
	&\Rightarrow 2p^j = p^r.
\end{align*}

Therefore $p = 2$, $j=r-1$ and $x_3(\lambda) = \mu \lambda^{2^{r-1}}$ for all $\lambda\in k$, for some fixed $\mu\in k$. The same argument shows that if $x_5 \neq 0$ then $p=2$ and $x_5(\lambda) = \nu \lambda^{2^{r-1}}$ for all $\lambda \in k$, for some fixed $\nu\in k$.

Consider Equation \ref{c3x8}. If $x_3$ or $x_4$ is nonzero then $p=2$, so $2x_3(\lambda_2)x_5(\lambda_1) = 0$. Conversely $2x_3(\lambda_2)x_5(\lambda_1) = 0$ if either of $x_3$ or $x_5$ is zero. Hence in either case $x_8$ is an additive polynomial, so by \cite[\S 20.3, Lemma A]{humphreys1975linear} $x_8$ is of the form
\begin{align*}
	x_8(\lambda_1 + \lambda_2) = x_8(\lambda_1) + x_8(\lambda_2).
\end{align*}
Moreover, in either case $\sigma\left(U_2(k)\right)$ lies in an abelian subgroup of $V_\alpha$. Hence Lemma \ref{lem:second} applies. The two cases are
\begin{align*}
p=2: \qquad&x_3(\lambda) = \mu \lambda^{2^{r-1}} \\
	&x_5(\lambda) = \nu \lambda^{2^{r-1}} \\
	&x_8(\lambda) = \omega \lambda^{2^r}, \\
p>2: \qquad&x_3=x_5=0 \\
	&x_8(\lambda) = \omega \lambda^{p^r}.
\end{align*}

The point of this example calculation was to show that we could eventually apply Lemma \ref{lem:second} even though Proposition \ref{uabelian} does not apply, so we could not use Lemma \ref{lem:first}. This provides some evidence for Conjecture \ref{bigclaim}.

%\subsection{$V = R_u(P_\beta)$}
%\label{c3:beta}
%Calculation goes here

%\subsection{$V = R_u(P_\gamma)$}
%\label{c3:gamma}
%Calculation goes here

\section{$G = B_4$}
\label{b4}

Let $G = B_4$. Let $\mathrm{char}(k)=2$ and set $V:=\langle U_\phi\, |\, \phi \in \Phi^+, \phi \neq \gamma + \delta,\phi \neq \gamma+2\delta\} \rangle$. We will write
\begin{align*}
\mathbf{v} &= \epsilon_\alpha(v_1) \epsilon_\beta(v_2) \epsilon_{\alpha+\beta}(v_3) \epsilon_{\beta+\gamma}(v_4) \epsilon_{\alpha+\beta+\gamma}(v_5) \epsilon_{\beta+\gamma+\delta}(v_6) \epsilon_{\alpha+\beta+\gamma+\delta}(v_7) \epsilon_{\beta+\gamma+2\delta}(v_8)\\ 
&\quad\epsilon_{\alpha+\beta+\gamma+2\delta}(v_9)\epsilon_{\beta+2\gamma+2\delta}(v_{10}) \epsilon_{\alpha+\beta+2\gamma+2\delta}(v_{11}) \epsilon_{\alpha+2\beta+2\gamma+2\delta}(v_{12}) \in V
\end{align*}
as a column vector:
\begin{align*}
\mathbf{v} = \left( \begin{matrix}
	         v_1 \\
	         v_2 \\
	         v_3 \\
	         v_4 \\
	         v_5 \\
	         v_6 \\
	         v_7 \\
	         v_8 \\
	         v_9 \\
	         v_{10} \\
	         v_{11} \\
	         v_{12} 
	      \end{matrix}\right). 
\end{align*}
The Group Law on $V$ is
\begin{align*}
	     \mathbf{u}
	     \mathbf{v}&=
	     \mathbf{u} + \mathbf{v} +
	     \left( \begin{matrix}
	         0 \\
	         0 \\
	         u_2v_1\\
	         0 \\
	         u_4v_1 \\
	         0 \\
	         u_6v_1\\
	         0 \\
	         u_8v_1\\
	         0 \\
	         u_{10}v_1\\
	         u_{10}v_1v_2 + u_8v_1v_4 + u_6^2v_1 + u_{11}v_2 + u_{10}v_3 + u_9v_4 + u_8v_5
	         	\end{matrix}\right).
\end{align*}

For integers $r,s\geq 0$ we have a homomorphism $\rho_{r,s}:SL_2\rightarrow \widetilde{A}_1\widetilde{A}_1 < L_{\{\gamma,\delta\}}$ defined by
\begin{align*}
\rho_{r,s}   \left(\begin{matrix} % or pmatrix or bmatrix or Bmatrix or ...
      1 & u \\
      0 & 1 \\
   \end{matrix}\right) &= \epsilon_\delta(u^{2^r})\cdot\epsilon_{\gamma+\delta}(u^{2^s}) \\
\rho_{r,s}   \left(\begin{matrix} % or pmatrix or bmatrix or Bmatrix or ...
      t & 0 \\
      0 & t^{-1} \\
   \end{matrix}\right) &= \delta^\vee(t^{2^r})\cdot(\gamma+\delta)^\vee(t^{2^s}) \\
\rho_{r,s}   \left(\begin{matrix} % or pmatrix or bmatrix or Bmatrix or ...
      0 & 1 \\
      1 & 0 \\
   \end{matrix}\right) &= n_\delta\cdot n_{\gamma+\delta} 
\end{align*}
from which we obtain an action of $SL_2$ on $V$:
\begin{align*}
\left( \begin{matrix}
	      a & b \\
	      c & d \\
	   \end{matrix}\right) \cdot \mathbf{v} &=
	   \left( \begin{matrix}
	   v_1 \\
	   c^{2^{s+1}} v_{10} + d^{2^{s+1}}v_2 \\
	   c^{2^{s+1}} v_{11} + d^{2^{s+1}}v_3 \\
	   c^{2^{r+1}} v_{8} + d^{2^{r+1}}v_4 \\
	   c^{2^{r+1}} v_{9} + d^{2^{r+1}}v_5 \\
	   v_6 + (bd)^{2^r}v_4 + (bd)^{2^s}v_2 + (ac)^{2^r}v_8 + (ac)^{2^s}v_{10} \\
	   v_7 + (bd)^{2^r}v_5 + (bd)^{2^s}v_3 + (ac)^{2^r}v_9 + (ac)^{2^s}v_{11} \\
	   a^{2^{r+1}}v_8 + b^{2^{r+1}}v_4 \\
	   a^{2^{r+1}}v_9 + b^{2^{r+1}}v_5 \\
	   a^{2^{s+1}}v_{10} + b^{2^{s+1}}v_2 \\
	   a^{2^{s+1}}v_{11} + b^{2^{s+1}}v_3 \\
	   v_{12} + (bd)^{2^{r+1}}v_4v_5 + (bd)^{2^{s+1}}v_2v_3 + (bc)^{2^{r+1}}(v_4v_9 + v_5v_8)\\ +\, (bc)^{2^{s+1}}(v_2v_{11} + v_3v_{10}) + (ac)^{2^{r+1}}(v_8v_9) + (ac)^{2^{s+1}}(v_{10}v_{11})
	   \end{matrix} \right)
\end{align*}

Now let $\sigma$ be a 1-cocycle from $SL_2$ to $V$ such that for all $t$ in $k^*$
\begin{align*}
\sigma\left(\begin{matrix} % or pmatrix or bmatrix or Bmatrix or ...
      t & 0 \\
      0 & t^{-1} \\
   \end{matrix}\right) = \left( \begin{matrix} 0 \\ \vdots \\ 0 \end{matrix}\right).
\end{align*}
Since $\sigma$ is a morphism of varieties, each component of $\sigma\left(\begin{matrix} 1 & u \\ 0 & 1\end{matrix}\right)$ should be a polynomial function of $u$, so we let
\begin{align*}
\sigma \left( \begin{matrix} % or pmatrix or bmatrix or Bmatrix or ...
      1 & u \\
      0 & 1 \\
   \end{matrix}\right) = \left( \begin{matrix} p_1(u) \\ \vdots \\ p_{12}(u) \end{matrix} \right),
\end{align*}
where each $p_i$ ($1\leq i \leq 12$) is as required. Applying $\sigma$ to the identity
\begin{align*}
  \left( \begin{matrix}
      t & 0 \\
      0 & t^{-1} \\
   \end{matrix}\right)
   \left(\begin{matrix}
      1 & u \\
      0 & 1 \\
   \end{matrix}\right)
   \left(\begin{matrix}
      t^{-1} & 0 \\
      0 & t \\
   \end{matrix}\right) &=
\left(   \begin{matrix}
      1 & t^2u \\
      0 & 1 \\
   \end{matrix}\right),
 \end{align*}
 gives rise to the following equations
 \begin{align}
 \label{tAct}
 p_i(t^2u) &= \left\{    \begin{array}{ll}
       p_i(u), & i = 1,6,7,12 \\
       t^{-2^{r+1}}p_i(u), & i = 4,5 \\
       t^{-2^{s+1}}p_i(u), & i = 2,3 \\
       t^{2^{r+1}}p_i(u), & i = 8,9 \\
       t^{2^{s+1}}p_i(u), & i = 10,11 \\
    \end{array}
 \right.
 \end{align}
It is clear that for $i = 1,6,7,12$ the polynomials $p_i$ must be constant-valued, say $\lambda_i$ for some fixed $\lambda_i$ in $k$ (resp). Furthermore, since $p_i(t^2u)$ involves only non-negative powers of $t$, $p_i$ must be the zero polynomial for $i=2,3,4,5$. Now consider the identity
\begin{align*}
  \left( \begin{matrix}
      1 & u_1 \\
      0 & 1 \\
   \end{matrix}\right)
   \left(\begin{matrix}
      1 & u_2 \\
      0 & 1 \\
   \end{matrix}\right) &=
    \left(\begin{matrix}
      1 & u_1 + u_2 \\
      0 & 1 \\
   \end{matrix}\right).
\end{align*}
Applying $\sigma$ to both sides yields
\begin{align*}
p_1(u_1 + u_2) &= p_1(u_1) + p_1(u_2) \\
p_6(u_1 + u_2) &= p_6(u_1) + p_6(u_2) \\
p_7(u_1 + u_2) &= p_7(u_1) + p_7(u_2) + p_6(u_1)p_1(u_2)\\
p_8(u_1 + u_2) &= p_8(u_1) + p_8(u_2) \\
p_9(u_1 + u_2) &= p_9(u_1) + p_9(u_2) + p_8(u_1)p_1(u_2)\\
p_{10}(u_1 + u_2) &= p_{10}(u_1) + p_{10}(u_2)\\
p_{11}(u_1 + u_2) &= p_{11}(u_1) + p_{11}(u_2) + p_{10}(u_1)p_1(u_2)\\
p_{12}(u_1 + u_2) &= p_{12}(u_1) + p_{12}(u_2) + \left(p_6(u_1)\right)^2p_1(u_2).
\end{align*}
Now we see that the constant polynomials $p_1,p_6,p_7,p_{12}$ must in fact be the zero polynomial and the remaining polynomials must be homomorphisms from $k\rightarrow k$. That is for some $w_j, x_j, y_j, z_j$ in $k$ and all $u$ in $k$
\begin{align*}
p_8(u) &= \sum_{j=0}^N w_j u^{2^j} \\
p_9(u) &= \sum_{j=0}^N x_j u^{2^j} \\
p_{10}(u) &= \sum_{j=0}^N y_j u^{2^j} \\
p_{11}(u) &= \sum_{j=0}^N z_j u^{2^j}, 
\end{align*}
If $\sigma$ is not the trivial 1-cocycle then one of the polynomials above is not the zero polynomial. Suppose for instance that $p_8$ is not the zero polynomial, so that $w_l\neq 0$ for some index $l\geq 0$. Then by Equation \ref{tAct}
\begin{align*}
\sum_{j=0}^N w_j (t^2u)^{2^j} &= t^{2^{r+1}}\sum_{j=0}^N w_j u^{2^j} \\
\Rightarrow\quad w_l (t^2u)^{2^l} &= t^{2^{r+1}} w_l u^{2^l} \\
\Rightarrow\quad l &= r.
\end{align*}
Similarly, for the polynomials $p_9,p_{10},,p_{11}$, we get that
\begin{align*}
p_8(u) = wu^{2^{r}}, \quad p_9(u) = xu^{2^{r}}, \\
p_{10}(u) = yu^{2^{s}}, \quad p_{11}(u) = zu^{2^{s}},
\end{align*}
for some $w,x,y,z$ in $k$.

So, we have
\begin{align*}
\sigma\left(\begin{matrix} a & b \\ 0 & a^{-1} \end{matrix}\right) &=
\sigma\left(\begin{matrix} a & 0 \\ 0 & a^{-1} \end{matrix}\right)\left[
\left(\begin{matrix} a & 0 \\ 0 & a^{-1} \end{matrix}\right)\cdot
\sigma\left(\begin{matrix} 1 & a^{-1}b \\ 0 & 1 \end{matrix}\right) \right]\\
&=
\left( \begin{matrix}
0 \\
0 \\
0 \\
0 \\
0 \\
0 \\
0 \\
w(ab)^{2^{r+1}} \\
x(ab)^{2^{r+1}} \\
y(ab)^{2^{s+1}} \\
z(ab)^{2^{s+1}} \\
0
\end{matrix} \right).
\end{align*}

We apply the same argument using the fact that each component of $\sigma\left(\begin{matrix} 1 & 0 \\ u & 1\end{matrix}\right)$ is a polynomial function, say $p'_i(u)$ for all $u$ in $k$, to get
\begin{align*}
\sigma\left(\begin{matrix} d^{-1} & 0 \\ c & d \end{matrix}\right) &=
\left( \begin{matrix}
0 \\
y'(cd)^{2^s} \\
z'(cd)^{2^s} \\
w'(cd)^{2^r} \\
x'(cd)^{2^r} \\
0 \\
0 \\
0 \\
0 \\
0 \\
0 \\
0
\end{matrix} \right),
\end{align*}
for some $w', x', y', z'$ in $k$.

From this we deduce that
\begin{align*}
\sigma\left(\begin{matrix} 0 & 1 \\ 1 & 0 \end{matrix}\right) &=
\sigma\left(
\left(\begin{matrix} 1 & 1 \\ 0 & 1 \end{matrix}\right)
\left(\begin{matrix} 1 & 0 \\ 1 & 1 \end{matrix}\right)
\left(\begin{matrix} 1 & 1 \\ 0 & 1 \end{matrix}\right)
\right) \\
&=
\sigma
\left(\begin{matrix} 1 & 1 \\ 0 & 1 \end{matrix}\right)
\left[
\left(\begin{matrix} 1 & 1 \\ 0 & 1 \end{matrix}\right)
\cdot
\sigma\left(
\left(\begin{matrix} 1 & 0 \\ 1 & 1 \end{matrix}\right)
\left(\begin{matrix} 1 & 1 \\ 0 & 1 \end{matrix}\right)
\right)\right] \\
&=
\sigma
\left(\begin{matrix} 1 & 1 \\ 0 & 1 \end{matrix}\right)
\left[
\left(\begin{matrix} 1 & 1 \\ 0 & 1 \end{matrix}\right)
\cdot
\left(
\sigma\left(\begin{matrix} 1 & 0 \\ 1 & 1 \end{matrix}\right)
\left[
\left(\begin{matrix} 1 & 0 \\ 1 & 1 \end{matrix}\right)
\cdot
\sigma\left(\begin{matrix} 1 & 1 \\ 0 & 1 \end{matrix}\right)
\right]\right)\right] \\
&=
\left(\begin{matrix}
0 \\
y + y' \\
z + z' \\
w + w' \\
x + x' \\
w' + y' \\
x' + z' \\
w + w' \\
x + x' \\
y + y' \\
z + z' \\
w'x' + y'z'
\end{matrix}\right).
\end{align*}

Furthermore, since $\sigma\left(\begin{matrix} 0 & 1 \\ 1 & 0\end{matrix}\right)$ is fixed under the action of $\left(\begin{matrix} t & 0 \\ 0 & t^{-1}\end{matrix}\right)$, we have
\begin{align*}
\sigma\left(\begin{matrix} 0 & 1 \\ 1 & 0\end{matrix}\right) &=
\left(\begin{matrix}
n_1 \\
0 \\
0 \\
0 \\
0 \\
n_6 \\
n_7 \\
0 \\
0 \\
0 \\
0 \\
n_{12}
\end{matrix}\right),
\end{align*} 
for some $n_1, n_6, n_7, n_{12}$ in $k$. So in fact
\begin{align*}
w' &= w \\
x' &= x \\
y' &= y \\
z' &= z \\
n_1 &= 0\\
n_6 &=w+y\\
n_7 &= x+z\\
n_{12} &= wx + yz.
\end{align*}

Consider
$\sigma\left(\begin{matrix} a & b \\ c & d \end{matrix}\right)$.
If $c=0$ then we already have
\begin{align*}
\sigma\left(\begin{matrix} a & b \\ 0 & a^{-1} \end{matrix}\right) &=
\sigma\left(\begin{matrix} a & 0 \\ 0 & a^{-1} \end{matrix}\right) \left[
\left(\begin{matrix} a & 0 \\ 0 & a^{-1} \end{matrix}\right)\cdot
\sigma\left(\begin{matrix} 1 & a^{-1}b \\ 0 & 1 \end{matrix}\right) \right]\\
&=
\left( \begin{matrix}
0 \\
0 \\
0 \\
0 \\
0 \\
0 \\
0 \\
w(ab)^{2^{r+1}} \\
x(ab)^{2^{r+1}} \\
y(ab)^{2^{s+1}} \\
z(ab)^{2^{s+1}} \\
0
\end{matrix} \right).
\end{align*}
Otherwise, $c\neq 0$ and we can write
\begin{align*}
\left(\begin{matrix} a & b \\ c & d \end{matrix}\right) &= 
\left(\begin{matrix} 1 & ac^{-1} \\ 0 & 1 \end{matrix}\right)
\left(\begin{matrix} 0 & 1 \\ 1 & 0 \end{matrix}\right)
\left(\begin{matrix} c & d \\ 0 & c^{-1} \end{matrix}\right),
\end{align*}
and so
\begin{align*}
\sigma\left(\begin{matrix} a & b \\ c & d \end{matrix}\right) &= 
\sigma\left(
\left(\begin{matrix} 1 & ac^{-1} \\ 0 & 1 \end{matrix}\right)
\left(\begin{matrix} 0 & 1 \\ 1 & 0 \end{matrix}\right)
\left(\begin{matrix} c & d \\ 0 & c^{-1} \end{matrix}\right)
\right) \\
&=
\sigma \left(\begin{matrix} 1 & ac^{-1} \\ 0 & 1 \end{matrix}\right) \left[
\left(\begin{matrix} 1 & ac^{-1} \\ 0 & 1 \end{matrix}\right) \cdot
\sigma \left( 
\left(\begin{matrix} 0 & 1 \\ 1 & 0 \end{matrix}\right)
\left(\begin{matrix} c & d \\ 0 & c^{-1} \end{matrix}\right)
\right) \right]\\
&=
\sigma \left(\begin{matrix} 1 & ac^{-1} \\ 0 & 1 \end{matrix}\right) \left[
\left(\begin{matrix} 1 & ac^{-1} \\ 0 & 1 \end{matrix}\right) \cdot
\left( 
\sigma \left(\begin{matrix} 0 & 1 \\ 1 & 0 \end{matrix}\right)\left[
\left(\begin{matrix} 0 & 1 \\ 1 & 0 \end{matrix}\right) \cdot
\sigma\left(\begin{matrix} c & d \\ 0 & c^{-1} \end{matrix}\right)
\right]\right) \right]\\
&=
\left(\begin{matrix}
0 \\
y(cd)^{2^s} \\
z(cd)^{2^s} \\
w(cd)^{2^r} \\
x(cd)^{2^r} \\
n_6 + w(ad)^{2^r} + y(ad)^{2^s} \\
n_7 + x(ad)^{2^r} + z(ad)^{2^s} \\
w(ab)^{2^r} \\
x(ab)^{2^r}  \\
y(ab)^{2^s} \\
z(ab)^{2^r} \\
n_{12} +wx(ad)^{2^{r+1}} + yz(ad)^{2^{s+1}}
\end{matrix}\right) \\
&=
\left(\begin{matrix}
0 \\
y(cd)^{2^s} \\
z(cd)^{2^s} \\
w(cd)^{2^r} \\
x(cd)^{2^r} \\
w(bc)^{2^r} + y(bc)^{2^s} \\
x(bc)^{2^r} + z(bc)^{2^s} \\
w(ab)^{2^r} \\
x(ab)^{2^r}  \\
y(ab)^{2^s} \\
z(ab)^{2^r} \\
wx(bc)^{2^{r+1}} + yz(bc)^{2^{s+1}}
\end{matrix}\right).
\end{align*}
We see that in any case
\begin{align*}
\sigma\left(\begin{matrix} a & b \\ c & d \end{matrix} \right) &= 
\left(\begin{matrix}
0 \\
y(cd)^{2^s} \\
z(cd)^{2^s} \\
w(cd)^{2^r} \\
x(cd)^{2^r} \\
w(bc)^{2^r} + y(bc)^{2^s} \\
x(bc)^{2^r} + z(bc)^{2^s} \\
w(ab)^{2^r} \\
x(ab)^{2^r}  \\
y(ab)^{2^s} \\
z(ab)^{2^r} \\
wx(bc)^{2^{r+1}} + yz(bc)^{2^{s+1}}
\end{matrix}\right).
\end{align*}

This is our candidate for a general 1-cocycle on $SL_2(k)$. It turns out that $\sigma$ is a well-defined 1-cocycle on $SL_2(k)$ by verifying Equations \ref{pita1}--\ref{pita3} but we omit the details here.

Given the general form of a 1-cocycle we can now calculate the 1-cohomology. 
We let $\tau \in \psi(\sigma)$, assuming that $\tau(T_2(k)) = 1$.
Then there exists $\mathbf{v}$ in $V$ that is fixed under the action of $\left(\begin{matrix}t & 0 \\ 0 & t^{-1}\end{matrix}\right)$, such that $\tau(x) = \mathbf{v}\sigma(x)(x\cdot\mathbf{v}^{-1})$, for all $x\in SL_2(k)$.

We have
\begin{align*}
\tau\left(\begin{matrix} a & b \\ c & d \end{matrix}\right) &=
\mathbf{v}\sigma\left(\begin{matrix} a & b \\ c & d \end{matrix}\right) \left[
\left(\begin{matrix} a & b \\ c & d \end{matrix}\right) \cdot \mathbf{v}^{-1}\right]\\
&=
\left(\begin{matrix} 
v_1  \\
0 \\
0 \\
0 \\
0 \\
v_6 \\
v_7 \\
0 \\
0 \\
0 \\
0 \\
v_{12}
\end{matrix}\right)
\sigma\left(\begin{matrix} a & b \\ c & d \end{matrix}\right) \left[
\left(\begin{matrix} a & b \\ c & d \end{matrix}\right) \cdot 
\left(\begin{matrix} 
v_1  \\
0 \\
0 \\
0 \\
0 \\
v_6 \\
v_7 + v_1v_6 \\
0 \\
0 \\
0 \\
0 \\
v_{12} + v_1v_6^2
\end{matrix}\right)\right]\\
&=
\left(\begin{matrix}
0 \\
y(cd)^{2^s} \\
(z+yv_1)(cd)^{2^s} \\
w(cd)^{2^r} \\
(x+wv_1)(cd)^{2^r} \\
w(bc)^{2^r} + y(bc)^{2^s} \\
(x+wv_1)(bc)^{2^r} + (z+yv_1)(bc)^{2^s} \\
w(ab)^{2^r} \\
(x+wv_1)(ab)^{2^r}  \\
y(ab)^{2^s} \\
(z+yv_1)(ab)^{2^r} \\
w(x+wv_1)(bc)^{2^{r+1}} + y(z+yv_1)(bc)^{2^{s+1}}
\end{matrix}\right).
\end{align*}

We can denote this relationship by
\begin{align*}
(w,x,y,z) &\sim (w, x+\lambda w, y, z + \lambda y),
\end{align*}
where the 4-tuple $(w,x,y,z)$ represents the 1-cocycle 
\begin{align*}
\left(\begin{matrix} a & b \\ c & d \end{matrix}\right) &\mapsto
\left(\begin{matrix}
0 \\
y(cd)^{2^s} \\
z(cd)^{2^s} \\
w(cd)^{2^r} \\
x(cd)^{2^r} \\
w(bc)^{2^r} + y(bc)^{2^s} \\
x(bc)^{2^r} + z(bc)^{2^s} \\
w(ab)^{2^r} \\
x(ab)^{2^r}  \\
y(ab)^{2^s} \\
z(ab)^{2^r} \\
wx(bc)^{2^{r+1}} + yz(bc)^{2^{s+1}}
\end{matrix}\right).
\end{align*}

That is, $\psi((w,x,y,z)) = \psi((w',x', y', z'))$ if and only if $w'=w, x'=x+\lambda w, y'=y, z'=z+\lambda y$, for some $\lambda\in k$.

We find infinitely many equivalence classes of 1-cocycles. For instance, for each $x, z\in k$, $\psi((0,x,0,z))$ is a distinct element of the 1-cohomology.

Now we consider the action of $Z(L_{\{\gamma,\delta\}})^\circ$ on the 1-cohomology. An element $\mathbf{s} = \alpha^\vee(t)(\beta + \gamma + \delta)^\vee(u)\in Z(L_{\{\gamma,\delta\}})^\circ$ acts on the 1-cocycle $\sigma$ by
\begin{align*}
(\mathbf{s}\cdot\sigma)\left(\begin{matrix} a & b \\ c & d\end{matrix}\right)
&=
\left(\begin{matrix}
0 \\
t^{-1}u^{2}y(cd)^{2^s} \\
tz(cd)^{2^s} \\
t^{-1}u^{2}w(cd)^{2^r} \\
tx(cd)^{2^r} \\
t^{-1}u^{2}(w(bc)^{2^r} + y(bc)^{2^s}) \\
t(x(bc)^{2^r} + z(bc)^{2^s}) \\
t^{-1}u^{2}w(ab)^{2^r} \\
tx(ab)^{2^r}  \\
t^{-1}u^{2}y(ab)^{2^s} \\
tz(ab)^{2^r} \\
u^2(wx(bc)^{2^{r+1}} + yz(bc)^{2^{s+1}})
\end{matrix}\right).
\end{align*}

