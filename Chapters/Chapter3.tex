%!TEX root = ../Thesis.tex
% Chapter 3

\chapter{The 1-Cohomology}
\label{Chapter3}
\lhead{Chapter 3. \emph{The 1-Cohomology}}

\section{Abelian 1-Cohomology}
The abelian 1-cohomology is standard (cf. \cite{brown1976cohomology}). We present this Section as motivation for the less-known non-abelian theory to appear in the next Section.

Let $K$ be an algebraic group and $V$ an abelian group on which $K$ acts by group automorphisms. We denote the action by $x \cdot v$, for $x \in K, v \in V$.

\begin{definition} We call a morphism $\alpha$ from $K\rightarrow V$ a \emph{1-cocycle} if it satisfies
	\begin{align}
		\alpha(xy) = \alpha(x) + x\cdot\alpha(y),
		\label{eqn:a_z}
	\end{align}
	for all $x, y$ in $K$. Denote by $Z^1\left( K, V \right)$ the collection of all 1-cocycles from $K\rightarrow V$.

	We call Equation \ref{eqn:a_z} the \emph{1-cocycle condition}.
\end{definition}

For any $\alpha, \beta$ in $Z^1\left(K, V\right)$,
\begin{align*}
	\left(\alpha + \beta\right)(xy) &=  \alpha(xy) +  \beta(xy) \\
	&=  \alpha(x) + x\cdot\alpha(y) +  \beta(x) + x\cdot\beta(y)\\
	&=  \left( \alpha(x) + \beta(x) \right) + x\cdot\left(\alpha(y) + \beta(y)\right) \\
	&=  \left(\alpha+\beta\right)(x) + x\cdot\left(\alpha + \beta\right)(y).
\end{align*}
It is easy to check that $\alpha + \beta$ is a morphism, so $Z^1(K, V)$ is closed under pointwise addition.

The trivial map from $K \rightarrow V$ that sends every $h \in K$ to the identity $0 \in V$ is a 1-cocycle. Furthermore for any $\alpha \in Z^1(K, V)$ we have
\begin{align*}
	\alpha(1)\, =\, \alpha(1\cdot 1) &=  \alpha(1) + 1\cdot \alpha(1) \\
	&=  \alpha(1) + \alpha(1) \\
	&=  2\,\alpha(1),
\end{align*}
so $\alpha(1) = 0$. Then, for all $x \in K$,
\begin{align*}
	0 = \alpha(1) = \alpha(xx^{-1}) = \alpha(x) + x \cdot \alpha(x^{-1})
\end{align*}
Hence each $\alpha$ has a negative defined by
\begin{align*}
	-\alpha(x) = x\cdot\alpha(x^{-1}).
\end{align*}
Therefore $Z^1\left(K, V\right)$ is an abelian $\mathbb{Z}-$module under pointwise addition.

\begin{definition} Let $v \in V$. The morphism $\chi^K_v:K\rightarrow V$ defined by
\begin{align*}
	\chi^K_v (x) = v - x\cdot v,
\end{align*}
is called a \emph{1-coboundary}. We denote by $B^1\left(K, V\right)$ the collection of all 1-coboundaries. 
\end{definition}

For any $v \in V$ and any $x, y \in K$
\begin{align*}
	\chi^K_v(xy) &=  v - (xy)\cdot v \\
	&=  v - x \cdot \left(y\cdot v \right)\\
	&=  v - x \cdot \left(v -v + y\cdot v \right)\\
	&=  v - x\cdot v + x\cdot \left( v - y\cdot v\right)\\
	&=  \chi^K_v(x) + x\cdot \chi^K_v(y),
\end{align*}
so $B^1(K, V) \subset Z^1(K, V)$. 

For any $u,v \in V$ and all $h \in K$,
\begin{align*}
	(\chi^K_u + \chi^K_v)(h) &=  \chi^K_u(h) + \chi^K_v(h)\\
	&=  u - h\cdot u + v - h\cdot v \\
	&=  (u + v) - h\cdot (u + v) \\
	&=  \chi^K_{u + v} (h).
\end{align*}
Hence $B^1\left(K, V\right)$ is also closed under pointwise addition.

Evidently $B^1(K, V)$ is a subgroup of $Z^1(K, V)$. In fact $B^1(K, V)$ is a $\mathbb{Z}-$submodule of $Z^1(K, V)$, so we may form the quotient module.
\begin{definition} The \emph{1-cohomology} is the quotient module defined by
\begin{align*}
	H^1\left(K, V\right) = Z^1\left(K, V\right) / B^1\left(K, V\right).
\end{align*}
We denote by $\psi$ the canonical projection from $Z^1(K, V)\rightarrow H^1(K, V)$.
\end{definition}

TODO: $V$ is a vector space, $K$ acts linearly $\Rightarrow$ $B^1(K, V) \subset Z^1(K, V)$ are vector spaces: formula for scalar multiplication, formula for $\lambda\chi^K_V, \lambda \in k$.

We conclude this Section with a useful Lemma \cite[Proposition 1]{kemper2000characterization}.
\begin{lemma} Suppose $K$ is linearly reductive. Then $H^1(K, V)$ is trivial.
  \label{lem:lin_red_h}
\end{lemma}

\section{Non-abelian 1-Cohomology}
	
We no longer require that $V$ is abelian; henceforth $V$ is an algebraic group on which $K$ acts by group automorphisms. Much of the preceding Section is a direct analogue, for example many formulas are just rewritten in multiplicative notation.
One main difference will see is that 1-coboundaries are less useful in the non-abelian setting.

\begin{definition} We call a morphism $\alpha$ from $K\rightarrow V$ a \emph{1-cocycle} if it satisfies
\begin{align}
  \alpha(xy) = \alpha(x) (x\cdot\alpha(y)),
  \label{eqn:na_z}
\end{align}
for all $x, y \in K$. Denote by $Z^1\left( K, V \right)$ the collection of all 1-cocycles from $K\rightarrow V$.

We call Equation \ref{eqn:na_z} the \emph{1-cocycle condition}.
\end{definition}

\begin{remark} Unlike the previous Section there is no natural addition operation on $Z^1(K, V)$.
\end{remark}

\begin{definition} Let $v \in V$. The morphism $\chi^K_v:K\rightarrow V$ defined by
\begin{align*}
	\chi^K_v (h) = v (h\cdot v^{-1}),
\end{align*}
is called a \emph{1-coboundary}. We denote by $B^1\left(K, V\right)$ the collection of all 1-coboundaries.
\end{definition}

For any $v \in V$ and any $x, y \in K$,
\begin{align*}
	\chi^K_v(xy) &=  v \left((xy) \cdot v^{-1}\right) \\
	&=  v (x \cdot v^{-1}) (x \cdot v) ((xy) \cdot v^{-1}) \\
	&=  v \left(x \cdot v^{-1}\right) (x \cdot v) \left(x \cdot \left(y \cdot v^{-1}\right)\right) \\
	&=  v \left(x \cdot v^{-1}\right) \left(x \cdot v \left(y \cdot v^{-1}\right)\right) \\
	&=  \chi^K_v(x) (x \cdot \chi^K_v(y)),
\end{align*}
so $B^1(K, V) \subset Z^1(K, V)$.

In the abelian case we use the fact that we can take the quotient $Z^1(K, V)/B^1(K, V)$, as $B^1(K, V)$ is a $\mathbb{Z}$-submodule of $Z^1(K, V)$, but in the non-abelian they are just sets. Moreover, in the abelian case
\begin{align*}
	\psi(\alpha) = \psi(\beta) \Leftrightarrow \exists v \in V, \alpha = \beta + \chi^K_v.
\end{align*}
Again, this makes no sense in the non-abelian setting. Instead we have the following Definition.

\begin{definition} Define the relation $\sim$ on $Z^1(K, V)$ by
	\begin{align}
		\alpha \sim \beta \Leftrightarrow \exists v \in V, \alpha = v\beta(x)(x \cdot v^{-1}), \forall x \in K.
		\label{eqn:h_equiv}
	\end{align}
Then $\sim$ is an equivalence relation (TODO: show in appendix?).

Denote by $H^1(K, V)$ the \emph{1-cohomology}, defined to be the set of equivalence classes of $Z^1(K, V)$ under the relation in Equation \ref{eqn:h_equiv}, and denote by $\psi$ the canonical projection from $Z^1(K, V) \rightarrow H^1(K, V)$.
\end{definition}

R. Richardson \cite[Lemma 6.2.6]{richardson1982orbits} provides a result analogous to Lemma \ref{lem:lin_red_h}:
\begin{lemma}
  Suppose $K$ is linearly reductive and $V$ is unipotent. Then $H^1(K, V)$ is trivial.
  \label{lem:nonab_lin_red}
\end{lemma}

\section{Maps Between 1-Cohomologies}

The next few results (Lemma \ref{zeta}--Lemma \ref{h1maps}) could be stated in terms of the abelian 1-cohomology. Indeed we provide Lemma \ref{lem:a_h_restriction}, where $V$ is necessarily a vector space. However, for reasons of efficiency we state these few \emph{definition-like} results leading up to Lemma \ref{h1maps} using multiplicative notation only.

As in the previous Section, $K, V$ are algebraic groups such that $K$ acts on $V$ by group automorphisms. We denote the action of $K$ on $V$ by $x \cdot v$, for $x \in K, v \in V$.

\begin{lemma} \label{zeta}
	Let $K'$ be an algebraic group that acts on $V$ by group automorphisms, so we can define $Z^1(K', V), H^1(K', V)$. Denote the action of $K'$ on $V$ by $x \ast v$, for $x \in K', v \in V$.

	Let $\zeta:K'\rightarrow K$ be a homomorphism and suppose that $x \ast v = \zeta(x) \cdot v$ for all $x \in K', v \in V$. Then for all $\alpha \in Z^1(K, V), \alpha \circ \zeta \in Z^1(K', V)$.
\end{lemma}
\begin{proof}
	Let $\alpha \in Z^1(K, V)$ and let $x, y \in K'$. Then
	\begin{align*}
		(\alpha \circ \zeta)(xy) &= \alpha(\zeta(x)\zeta(y) \\
			&= \alpha(\zeta(x))\left(\zeta(x) \cdot \alpha(\zeta(y))\right) \\
			&= \alpha(\zeta(x))\left(x \ast \alpha(\zeta(y))\right) \\
			&= (\alpha \circ \zeta)(x)\left(x \ast (\alpha \circ \zeta)(y)\right).
	\end{align*}
	Therefore, since $\alpha \circ \zeta$ is a morphism and satisfies the 1-cocycle condition, $\alpha \circ \zeta \in Z^1(K', V)$.
\end{proof}

\begin{lemma} \label{xi}
	Let $V'$ be an algebraic group on which $K$ acts by group automorphisms. Denote the action of $K$ on $V'$ by $x \wedge v$ for $x \in K, v \in V'$.

	Let $\xi: V \rightarrow V'$ be a $K$-equivariant homomorphism, that is, $x \wedge \xi(v) = \xi(x \cdot v)$ for all $x \in K, v \in V$. Then for all $\alpha \in Z^1(K, V), \xi \circ \alpha \in Z^1(K, V')$.
\end{lemma}
\begin{proof}
	Let $\alpha \in Z^1(K, V)$ and let $x, y \in K$. Then
	\begin{align*}
		(\xi \circ \alpha)(xy) &= \xi\left(\alpha(x)(x \cdot \alpha(y))\right) \\
			&= \xi(\alpha(x))\xi(x \cdot \alpha(y)) \\
			&= \xi(\alpha(x))\left(x \wedge \xi(\alpha(y))\right) \\
			&= (\xi \circ \alpha)(x)\left(x \wedge (\xi \circ \alpha)(y)\right).
	\end{align*}
	Therefore, since $\xi \circ \alpha$ is a morphism and satisfies the 1-cocycle condition, $\xi \circ \alpha \in Z^1(K, V')$.
\end{proof}

\begin{lemma} \label{h1maps} Let $K', V'$ be algebraic groups, such that
	\begin{itemize}
		\item[(a)] $K'$ acts on $V$ by group automorphisms, denoted by $x \ast v$, for $x \in K', v \in V$,
		\item[(b)] $K'$ acts on $V'$ by group automorphisms, denoted by $x \wedge v$, for $x \in K', v \in V'$.
	\end{itemize}
	Let $\zeta:K' \rightarrow K$ and $\xi: V \rightarrow V'$ be homomorphisms such that
	\begin{itemize}
		\item[(c)] $x \ast v = \zeta(x) \cdot v$ for all $x \in K', v \in V$,
		\item[(d)] $\xi(x \ast v) = x \wedge \xi(v)$ for all $x \in K', v \in V$.
	\end{itemize}
	Then the function defined by
	\begin{align*}
		Z^1(\zeta, \xi)(\alpha) = \xi \circ \alpha \circ \zeta,
	\end{align*}
	maps $Z^1(K, V) \rightarrow Z^1(K', V')$.

	Furthermore, $Z^1(\zeta, \xi)$ descends to give a well-defined map
	\begin{align*}
		H^1(\zeta, \xi):H^1(K, V) \rightarrow H^1(K', V'),
	\end{align*}
	defined by
	\begin{align*}
		H^1(\zeta, \xi)(\psi(\alpha)) = \left(\psi' \circ Z^1(\zeta, \xi)\right)(\alpha),
	\end{align*}
	for all $\alpha \in Z^1(K, V)$, where $\psi:Z^1(K, V)\rightarrow H^1(K, V)$ and $\psi':Z^1(K', V') \rightarrow H^1(K', V')$ are the canonical projections.

	Moreover, the following diagram commutes:
	\begin{align*}
		\xymatrix{
			Z^1(K, V) \ar[r]^{Z^1(\zeta, \xi)} \ar[d]^{\psi} & Z^1(K', V') \ar[d]^{\psi'} \\
			H^1(K, V) \ar[r]^{H^1(\zeta, \xi)}               & H^1(K', V').
		}
	\end{align*}
\end{lemma}
\begin{proof}
	Let $\alpha \in Z^1(K, V)$ and let $x,y \in K'$. Then
	\begin{align*}
		(\xi \circ \alpha \circ \zeta)(xy) &= \xi \left((\alpha \circ \zeta)(x)\right) \xi \left(x \ast (\alpha \circ \zeta)(y)\right) \quad\textrm{(Lemma \ref{zeta})} \\
			&= \xi \left((\alpha \circ \zeta)(x)\right) \left(x \wedge \xi(\alpha \circ \zeta)(y)\right) \quad\textrm{(by (d))} \\
			&= (\xi \circ \alpha \circ \zeta)(x) \left(x \wedge (\xi \circ \alpha \circ \zeta)(y)\right).
	\end{align*}
	Therefore $(\xi \circ \alpha \circ \zeta) \in Z^1(K', V')$.

	It remains to show $H^1(\zeta, \xi)$ is well-defined. Let $\alpha,\beta \in Z^1(K, V)$ such that $\psi(\alpha) = \psi(\beta)$. Then there exists $v \in V$ such that
	\begin{align*}
		\beta(x) = v\alpha(x)(x \cdot v^{-1}),
	\end{align*}
	for all $x \in K$.

	Then, for all $x \in K'$
	\begin{align*}
		\left(Z^1(\zeta, \xi)(\beta)\right)(x) &= \xi\left(\beta(\zeta(x))\right) \\
			&= \xi \left( v \alpha(\zeta(x))\left(\zeta(x) \ast v^{-1}\right) \right) \\
			&= \xi(v) \xi(\alpha(\zeta(x))) \xi\left( \zeta(x) \ast v^{-1}\right) \\
			&= \xi(v) \xi(\alpha(\zeta(x)))\left(x \wedge \xi(v^{-1})\right) \\
			&= \xi(v) \left(Z^1(\zeta, \xi)(\alpha)\right)(x) \left(x \wedge \xi(v^{-1})\right).
	\end{align*}
	This shows that $\psi'\left(Z^1(\zeta, \xi)(\alpha)\right) = \psi'\left(Z^1(\zeta, \xi)(\beta)\right)$, hence $H^1(\zeta, \xi)$ is well-defined. This completes the proof.
\end{proof} 

\begin{remark}
	The slightly unfortunate choice of notation ``$Z^1(K, V)$'' doesn't explicitly mention the action. The consequence is that given suitable homomorphisms
	\begin{align*}
		\zeta&:K \rightarrow K, \\
		\xi&: V \rightarrow V,
	\end{align*}
	the statement
	\begin{align*}
		H^1(\zeta, \xi):H^1(K, V) \rightarrow H^1(K, V)
	\end{align*}
	is misleading on its own: Is the 1-cohomology on the left of the arrow the same as the one on the right? If nothing is said about the action defining the 1-cocycles on the right, we take that to mean they are the same. More satisfyingly, and by pure coincidence, later on in Chapter \ref{Chapter4} we adopt a modified notation for the 1-cocycles and the 1-cohomology which carries with it the action defining the 1-cocycles.
\end{remark}

\begin{example}
	Let $K' < K$ and let $\zeta$ be the inclusion of $K'$ in $K$. Let $\xi$ be the identity map on $V$. Then by Lemma \ref{h1maps}, the map $Z^1(\zeta, \xi)$ defined by
\begin{align*}
	Z^1(\zeta, \xi)(\alpha) = \xi \circ \alpha \circ \zeta = \alpha \circ \zeta,
\end{align*}
maps $Z^1(K, V)$ into $Z^1(K', V)$. Furthermore, the map
\begin{align*}
	H^1(\zeta, \xi)(\psi(\alpha)) = \psi' \circ Z^1(\zeta, \xi),
\end{align*}
is a well-defined map of 1-cohomologies $H^1(K, V) \rightarrow H^1(K', V)$.
\end{example}

The next Lemma is standard \cite[Theorem 10.3]{brown1976cohomology} but we give our own proof here. The Lemma deals with the abelian 1-cohomology of a finite group, so we alter our notation appropriately.
\begin{lemma}
Let $V$ be a vector space over a field of characteristic $p$. Let $\Gamma$ be a finite group and $\Gamma_p < \Gamma$ a \emph{Sylow $p$-subgroup} of $\Gamma$. Let $\zeta$ be the inclusion of $\Gamma_p$ in $\Gamma$, and $\xi$ the identity map on $V$. The map 
\begin{align*}
H^1(\zeta, \xi):H^1(\Gamma, V)\rightarrow H^1(\Gamma_p, V)
\end{align*}
is injective.
\label{lem:a_h_restriction}
\end{lemma}
\begin{proof}
	We will show that $\mathrm{Ker}\left(H^1(\zeta, \xi)\right)$ is trivial. First we show that we can choose $\psi(\beta) \in \mathrm{Ker}\left(H^1(\zeta, \xi)\right)$ such that $Z^1(\zeta, \xi)(\beta) = 0$.
	
	Let $\alpha \in Z^1(\Gamma, V)$ such that $H^1(\zeta, \xi)(\psi(\alpha)) = \psi'(0)$. Hence $Z^1(\zeta, \xi)(\alpha)$ is a 1-coboundary, so there exists $v \in V$ such that
\begin{align*}
	Z^1(\zeta, \xi)(\alpha) = \chi^{\Gamma_p}_v.
\end{align*}
But
\begin{align*}
	\chi^{\Gamma_p}_v = Z^1(\zeta, \xi)(\chi^\Gamma_v),
\end{align*}
so let $\beta = \alpha - \chi^\Gamma_v \in Z^1(\Gamma, V)$. Then
\begin{align*}
	\psi(\beta) &= \psi(\alpha - \chi^\Gamma_v) \\
		&= \psi(\alpha) \in \mathrm{Ker}\left(H^1(\zeta, \xi)\right),
	\end{align*}
	and
	\begin{align*}
		Z^1(\zeta, \xi)(\beta) &= Z^1(\zeta, \xi)(\alpha - \chi^\Gamma_v) \\
			&= Z^1(\zeta, \xi)(\alpha) - Z^1(\zeta, \xi)(\chi^\Gamma_v) \\
			&= \chi^{\Gamma_p}_v - \chi^{\Gamma_p}_v = 0.
		\end{align*}

		Choose a set of representatives $\{\gamma_1, \ldots, \gamma_l\} \subset \Gamma$ for the right coset space of $\Gamma_p$ in $\Gamma$ and let
\begin{align*}
	w = \sum_{i =1}^l \beta(\gamma_i).
\end{align*}
Let $\gamma \in \Gamma$. Then
\begin{align*}
	\chi_{w}^\Gamma(\gamma) &=  w - \gamma\cdot w \\
	&=  \sum_{i = 1}^l\beta(\gamma_i) - \gamma\cdot \sum_{i = 1}^l\beta(\gamma_i) \\
	&=  \sum_{i = 1}^l\beta(\gamma_i) - \sum_{i = 1}^l \gamma\cdot \beta(\gamma_i) \\
	&=  \sum_{i = 1}^l\beta(\gamma_i) - \sum_{i = 1}^l \left(\beta(\gamma\gamma_i) - \beta(\gamma) \right) \quad(\textrm{Equation }\ref{eqn:a_z})\\
	&=  \sum_{i = 1}^l\beta(\gamma_i) - \sum_{i = 1}^l \beta(\gamma\gamma_i) +\sum_{i = 1}^l \beta(\gamma).
\end{align*}
Let $\gamma_j$ be the coset representative of $\gamma$, so that $\gamma \in \Gamma_p\gamma_j$. Then $\beta(\gamma\gamma_i) = \beta(\Gamma_p\gamma_j\gamma_i) = \beta(\gamma_j\gamma_i)$.



------
\begin{align*}
	\chi_{v^*}^\Gamma(\gamma) 
	&=  \sum_{i = 1}^l\beta(\gamma_i) - \sum_{i = 1}^l \beta(\gamma\gamma_i) +\sum_{i = 1}^l \beta(\gamma)\\
	&=  \sum_{i = 1}^l\beta(\gamma_i) - \sum_{i = 1}^l \beta(\gamma_i) +\sum_{i = 1}^l \beta(\gamma) \\
	&=  l\, \beta(\gamma).
\end{align*}
Since $\gcd([\Gamma:\Gamma_p], p) = \gcd(l,p) = 1$, $l$ has an inverse $l^{-1} = m$ and so
\begin{align*}
	m\chi_{v^*}^\Gamma(\gamma) = \beta(\gamma).
\end{align*}
Therefore $\beta$ is the 1-coboundary
\begin{align}
  m\chi_{v^*}^\Gamma = \chi_{mv^*}^\Gamma
\end{align}
and so $\mathrm{Ker}\left(H^1(\zeta, \xi)\right)$ is trivial.
\end{proof}

\begin{example}
	Let
	\begin{align}
		k = \overline{\mathbb{F}_p} = \bigcup_r \mathbb{F}_{p^r}.
	\end{align}
	Note that in general
	\begin{align}
	  \mathbb{F}_{p^r} \not\subset \mathbb{F}_{p^{r+1}},
	\end{align}
	but we do have
	\begin{align}
	  \mathbb{F}_{p^r} \subset \mathbb{F}_{p^{(r + 1)!}}.
	\end{align}
	Let $V$ a be vector space on which $SL_2(k)$ acts, and $U(k)$ the subgroup of $SL_2(k)$ consisting of upper unitriangular matrices. Then $U(\mathbb{F}_{p^r})$ is a Sylow $p$-subgroup of $SL_2(\mathbb{F}_{p^r})$ for each $r$, and the map
	\begin{align}
		H^1(\iota): H^1(SL_2(k), V) \rightarrow H^1(U(k), V)
	\end{align}
	is injective.
\label{eg:sl2ab}
\end{example}
\begin{proof}
	The group $GL_2(\mathbb{F}_{p^r})$ has order $(p^{2r} - 1)(p^{2r} - p^r)$ since there are $p^{2r} - 1$ choices of vectors for the first column (all choices excluding the zero vector), and $p^{2r} - p^r$ choices of vectors for the second column (all choices excluding multiples of the first vector). The determinant is a homomorphism of groups
	\begin{align}
		\mathrm{det}:GL_2(\mathbb{F}_{p^r}) \rightarrow \mathbb{F}^*_{p^r},
	\end{align}
	with kernel $SL_2(\mathbb{F}_{p^r})$. Therefore, by the First Komomorphism Theorem for groups
	\begin{align}
		GL_2(\mathbb{F}_{p^r})\,/\,SL_2(\mathbb{F}_{p^r}) \simeq \mathrm{det}(GL_2(\mathbb{F}_{p^r})) = \mathbb{F}^*_{p^r},
	\end{align}
	and so
	\begin{align*}
		|SL_2(\mathbb{F}_{p^r})|
		&=  |GL_2(\mathbb{F}_{p^r})|\,/\,|\mathbb{F}^*_{p^r}|\\
		&=  (p^{2r} - 1)(p^{2r} - p^r)\,/\,(p^r - 1)\\
		&=  p^r(p^{2r} - 1).
	\end{align*}
	Since $|U(\mathbb{F}_{p^r})| = p^r$, $U(\mathbb{F}_{p^r})$ is a Sylow $p$-subgroup of $SL_2(\mathbb{F}_{p^r})$.
	
	Fix a non-trivial $y\in H^1(SL_2(k), V)$ and choose a representative $\beta\in Z^1(SL_2(k), V)$ for $y$. For each $g\in SL_2(\mathbb{F}_{p^r})$ define the morphism $f_g:V\rightarrow V$ by
	\begin{align}
		f_g(v) = \beta(g) - \chi_v(g) = \beta(g) - v + g\cdot v.
	\end{align}
	Consider sequence of subsets of $V$ defined by
	\begin{align}
		C_r = \{v \in V \,|\, f_g(v) = 0\}.
	\end{align}
	Each subset $C_r$ is closed and the inclusion $\mathbb{F}_{p^{r!}} \subset \mathbb{F}_{p^{(r+1)!}}$ induces the reverse inclusion $C_{r!} \supset C_{(r+1)!}$. The Noetherian property for $V$ requires that the sequence of subsets of $V$ defined by
	\begin{align}
		\left\{C_{i!}\right\}_{i = 1}^\infty
	\end{align}
	becomes constant. Kowever, $y\neq 0$ so $\beta$ is not a 1-coboundary on $SL_2(k)$, which means the $C_r$'s are eventually empty. That is, there exists an integer $s$ such that for any $v$ in $V$
	\begin{align}
		(\beta - \chi_v)|_{SL_2(\mathbb{F}_{p^s})} \neq 0.
	\end{align}
	Equivalently, if $y|_{SL_2(\mathbb{F}_{p^r})} = 0$ for all $r$ then $y = 0$.
	
	Take $x$ in the kernel of the map 
	\begin{align}
	 H^1(\iota) : H^1(SL_2(k), V) \rightarrow H^1(U(k), V).
       \end{align}
	Then for each $r$, $x|_{U(\mathbb{F}_{p^r})} = 0$ so by Lemma \ref{lem:a_h_restriction}, $x|_{SL_2(\mathbb{F}_{p^r})} = 0$. Therefore $x=0$ and so $H^1(\iota)$ is injective.
\end{proof}

\begin{lemma} Let $B$ be a Borel subgroup of $SL_2$ acting on an algebraic group $V$ and let $\iota : B \rightarrow SL_2$ be the inclusion map. Then $H^1(\iota):H^1(SL_2, V)\rightarrow H^1(B, V)$ is injective.
  \label{lem:sl2_b_inj}
\end{lemma}
\begin{proof}
Let $x$ be in the kernel of $H^1(\iota)$ and $\alpha$ and element of $Z^1(SL_2, V)$ that projects onto the class $x$. Since $Z^1(\iota)(\alpha)$ projects to the trivial 1-cohomology class we may as well assume that $\alpha|_B = 1$. For there exists some $v \in V$ such that for all $b \in B$
\begin{align}
	\left(Z^1(\iota)(\alpha) \right)(b) = v (b \cdot v^{-1}).
\end{align}
Consider the 1-cocycle $\hat{\alpha}:SL_2\rightarrow V$ defined by
\begin{align}
	\hat{\alpha}(h) = v^{-1} \alpha(h) (h \cdot v).
\end{align}
Then by construction $\hat{\alpha}$ also projects to the class $x$, and for all $b \in B$
\begin{align*}
	\hat{\alpha}(b) &=  v^{-1} \alpha(b) (b \cdot v) \\
	&=  v^{-1} (v (b\cdot v^{-1})) (b \cdot v)\\
	&=  v^{-1} v (b\cdot v)^{-1} (b \cdot v)\\
	&=  1,
\end{align*}
so we may as well have chosen $\hat{\alpha}$ instead as a representative for $x$. 

Now consider the \emph{homogeneous space} $SL_2/B$ and take the map 
\begin{align}
	\tilde{\alpha}:SL_2/B \rightarrow V,
\end{align}
to be the unique morphism such that the following diagram commutes:
\begin{align}
	\xymatrix{
	SL_2 \ar[r]^{\alpha} \ar[d]_{\pi} & V\\
	SL_2/B \ar[ru]_{\tilde{\alpha}}
	},
\end{align}
$\pi$ the canonical projection $\pi:SL_2 \rightarrow SL_2/B$. That is, $\tilde{\alpha}(hB) = \alpha(h)$ for all $h \in SL_2$.

Now since $SL_2/B$ is an irreducible projective variety \cite[Theorem 21.3]{humphreys1975linear}, $\tilde{\alpha}$ must be constant  \cite{borel1991linear}. Kence, as $\alpha$ takes the value 1 for any $b \in B$, $\tilde{\alpha}(hB) = 1$ for all cosets $hB$. Therefore, for all $h \in SL_2$
\begin{align}
	\alpha(h) = \tilde{\alpha}(hB) = 1.
\end{align}
We have shown that $\alpha$ is the trivial 1-coboundary $\chi_1$ which means that the kernel of $H^1(\iota)$ is trivial.
\end{proof} 

TODO: Borel $B < SL_2$, $U = R_u(B)$, $SL_2$ acts on unipotent $V$ $\Rightarrow$ $H^1(B, V) \rightarrow H^1(U, V)$ is injective. Not sure if it's true, but would imply $H^1(SL_2, V) \rightarrow H^1(U, V)$ is injective. Then apply main theorem of CH4.
