%!TEX root = ../Thesis.tex
% Chapter 3

\chapter{The 1-Cohomology}
\label{Chapter3}
\lhead{Chapter 3. \emph{The 1-Cohomology}}

\section{Abelian 1-Cohomology}
The abelian 1-cohomology is standard (cf. \cite{brown1976cohomology}). We present this Section as motivation for the less-known non-abelian theory to appear in the next Section.

If $K$ is an algebraic group and $V$ is an algebraic variety then by an action of $K$ on $V$ we mean an action $K\times V\rightarrow V$ which is also a morphism of varieties.

If $\Gamma$ is a finite group then $\Gamma$ is a variety over $k$, and if $X$ is a variety then any function $f:\Gamma \rightarrow X$ is a morphism of varieties.

Let $K$ be an algebraic group and $V$ an abelian algebraic group with identity $0$ on which $K$ acts by group automorphisms. We denote the action by $x \cdot v$, for $x \in K, v \in V$.

\begin{definition} We call a morphism $\alpha: K\rightarrow V$ a \emph{1-cocycle} if it satisfies
	\begin{align}
		\alpha(xy) = \alpha(x) + x\cdot\alpha(y),
		\label{eqn:a_z}
	\end{align}
	for all $x, y$ in $K$. Denote by $Z^1\left( K, V \right)$ the collection of all 1-cocycles from $K$ to $V$.

	We call Equation \ref{eqn:a_z} the \emph{1-cocycle condition}.
\end{definition}

For any $\alpha, \beta$ in $Z^1\left(K, V\right)$,
\begin{align*}
	\left(\alpha + \beta\right)(xy) &=  \alpha(xy) +  \beta(xy) \\
	&=  \alpha(x) + x\cdot\alpha(y) +  \beta(x) + x\cdot\beta(y)\\
	&=  \left( \alpha(x) + \beta(x) \right) + x\cdot\left(\alpha(y) + \beta(y)\right) \\
	&=  \left(\alpha+\beta\right)(x) + x\cdot\left(\alpha + \beta\right)(y),
\end{align*}
for all $x,y \in K$. Moreover, $\alpha + \beta$ is a morphism, so $Z^1(K, V)$ is closed under pointwise addition.

The trivial map from $K$ to $V$ that sends every $h \in K$ to $0 \in V$ is a 1-cocycle. 
Define $-\alpha$ by
\begin{align*}
	(-\alpha)(x) = -\alpha(x),
\end{align*}
for all $x \in K$. Then
\begin{align*}
	(-\alpha)(xy) &= -\alpha(xy) \\
		&= -\left(\alpha(x) + x \cdot \alpha(y)\right) \\
		&= -\alpha(x) + x\cdot (-\alpha(y)) \\
		&= (-\alpha)(x) + x \cdot (-\alpha)(y),
\end{align*}
for all $x,y \in K$. Hence $-\alpha \in Z^1(K, V)$. Furthermore, $(\alpha + (-\alpha))(x) = 0$ for all $x \in K$. Therefore each $\alpha \in Z^1(K, V)$ has a negative $-\alpha\in Z^1(K, V)$.

Associativity of pointwise addition in $Z^1(K, V)$ holds because associativity holds in $V$. Therefore $Z^1\left(K, V\right)$ is a group. Moreover, since $V$ is abelian, $Z^1(K, V)$ is abelian.
\begin{lemma} Let $\alpha \in Z^1(K, V)$. Then $-\alpha(x) = x \cdot \alpha(x^{-1})$ for all $x \in K$.
\end{lemma}
\begin{proof}
Let $1\in K$ be the identity. Then
\begin{align*}
	\alpha(1)\, =\, \alpha(1\times 1) &=  \alpha(1) + 1\cdot \alpha(1) \\
	&=  \alpha(1) + \alpha(1) \\
	&=  2\,\alpha(1).
\end{align*}
This shows $\alpha(1) = 0$. Then, for all $x \in K$,
\begin{align*}
	0 = \alpha(1) = \alpha(xx^{-1}) = \alpha(x) + x \cdot \alpha(x^{-1})
\end{align*}
Hence
\begin{align*}
	-\alpha(x) = x\cdot\alpha(x^{-1}),
\end{align*}
for all $x \in K$. 
\end{proof}

\begin{definition} Let $v \in V$. The morphism $\chi^K_v:K\rightarrow V$ defined by
\begin{align*}
	\chi^K_v (x) = v - x\cdot v,
\end{align*}
for all $x \in K$, is called a \emph{1-coboundary}. We denote by $B^1\left(K, V\right)$ the collection of all 1-coboundaries from $K$ to $V$. 
\end{definition}

For any $v \in V$
\begin{align*}
	\chi^K_v(xy) &=  v - (xy)\cdot v \\
	&=  v - x \cdot \left(y\cdot v \right)\\
	&=  v - x \cdot \left(v -v + y\cdot v \right)\\
	&=  v - x\cdot v + x\cdot \left( v - y\cdot v\right)\\
	&=  \chi^K_v(x) + x\cdot \chi^K_v(y),
\end{align*}
for all $x, y \in K$. Therefore $B^1(K, V) \subset Z^1(K, V)$.
Furthermore, for any $v, w \in V$,
\begin{align*}
	(\chi^K_v - \chi^K_w)(x) &=  \chi^K_v(x) - \chi^K_w(x)\\
	&=  v - x \cdot v - w - x\cdot v \\
	&=  (v - w) - x\cdot (v - w) \\
	&=  \chi^K_{v - w} (x),
\end{align*}
for all $x \in K$. We see that $B^1\left(K, V\right)$ is an abelian subgroup of $Z^1(K, V)$, so we may form the quotient.
\begin{definition} The \emph{1-cohomology} is the quotient defined by
\begin{align*}
	H^1\left(K, V\right) = Z^1\left(K, V\right) / B^1\left(K, V\right).
\end{align*}
We denote by $\psi$ the canonical projection from $Z^1(K, V)$ to $H^1(K, V)$.
\end{definition}

\begin{remark}\label{identity}
	We defined the identity of $Z^1(K, V)$ to be the trivial map which sends all $x \in K$ to $0 \in V$. This map is precisely the 1-coboundary $\chi^K_0$. Appropriately, we say a 1-cohomology is \emph{trivial} if $Z^1(K, V) = B^1(K, V)$.
\end{remark}

\begin{lemma} \label{vspace} If $V$ is a vector space and $K$ acts linearly on $V$ then $Z^1(K, V)$ is a vector space and $B^1(K, V)\subset Z^1(K, V)$ is a vector subspace.
\end{lemma}
\begin{proof}
	Suppose $V$ is a vector space over $F$. We define scalar multiplication on $Z^1(K, V)$ as follows.

For $\lambda \in F, \alpha \in Z^1(K, V)$ define the map $\lambda\alpha$ by $(\lambda\alpha)(x) = \lambda\alpha(x)$ for all $x \in K$. Then $\lambda\alpha$ is a 1-cocycle, for
\begin{align*}
	\lambda \alpha(xy) &= \lambda\left( \alpha(x) + x \cdot \alpha(y)\right) \\
	&= \lambda\alpha(x) + \lambda(x\cdot \alpha(y)) \\
	&= \lambda\alpha(x) + x \cdot (\lambda\alpha(y)).
\end{align*}
So $Z^1(K, V)$ is a subspace of the vector space of functions from $K$ to $V$. Hence $Z^1(K, V)$ is a vector space.

Now let $\chi^K_v \in B^1(K, V)$. Then for all $x \in K$
\begin{align*}
	\lambda\chi^K_v(x) &= \lambda\left(v - x \cdot v\right) \\
		&= \lambda v - \lambda(x \cdot v) \\
		&= \lambda v - x \cdot (\lambda v) \\
		&= \chi^K_{\lambda v}(x),
\end{align*}
Hence $B^1(K, V)$ is a subspace of $Z^1(K, V)$.
\end{proof}

We conclude this Section with a useful Lemma \cite[Proposition 1]{kemper2000characterization}.
\begin{lemma} Suppose $K$ is linearly reductive. Then $H^1(K, V)$ is trivial.
  \label{lem:lin_red_h}
\end{lemma}

\section{Non-abelian 1-Cohomology}
	
Richardson introduces the non-abelian 1-cohomology in \cite{richardson1982orbits}. We reproduce the definitions and selected results below.

We no longer require that $V$ is abelian, henceforth $V$ is an algebraic group on which $K$ acts by group automorphisms. Accordingly, we denote the identity of $V$ by $1$. Much of the preceding Section is a direct analogue; for the most part, formulas are just rewritten in multiplicative notation.
One main difference will see is that 1-coboundaries are less useful in the non-abelian setting, especially when defining the 1-cohomology.

\begin{definition} We call a morphism $\alpha:K\rightarrow V$ a \emph{1-cocycle} if it satisfies
\begin{align}
  \alpha(xy) = \alpha(x) (x\cdot\alpha(y)),
  \label{eqn:na_z}
\end{align}
for all $x, y \in K$. Denote by $Z^1\left( K, V \right)$ the collection of all 1-cocycles from $K$ to $V$.

We call Equation \ref{eqn:na_z} the \emph{1-cocycle condition}.
\end{definition}

\begin{remark} Unlike the previous Section there is no natural addition operation on $Z^1(K, V)$.
\end{remark}

\begin{definition} Let $v \in V$. The morphism $\chi^K_v:K\rightarrow V$ defined by
\begin{align*}
	\chi^K_v (x) = v (x\cdot v^{-1}),
\end{align*}
for all $x \in K$ is called a \emph{1-coboundary}. We denote by $B^1\left(K, V\right)$ the collection of all 1-coboundaries from $K$ to $V$.
\end{definition}

For any $v \in V$ and any $x, y \in K$,
\begin{align*}
	\chi^K_v(xy) &=  v \left((xy) \cdot v^{-1}\right) \\
	&=  v (x \cdot v^{-1}) (x \cdot v) ((xy) \cdot v^{-1}) \\
	&=  v \left(x \cdot v^{-1}\right) (x \cdot v) \left(x \cdot \left(y \cdot v^{-1}\right)\right) \\
	&=  v \left(x \cdot v^{-1}\right) \left(x \cdot \left( v \left(y \cdot v^{-1}\right)\right)\right) \\
	&=  \chi^K_v(x) (x \cdot \chi^K_v(y)),
\end{align*}
so $B^1(K, V) \subset Z^1(K, V)$.

In the abelian case we use the fact that we can take the quotient $Z^1(K, V)/B^1(K, V)$, as $B^1(K, V)$ is an abelian subgroup of $Z^1(K, V)$, but in the non-abelian they are just sets. Moreover, although we did not explicitly mention it, in the abelian case the following holds
\begin{align*}
	\psi(\alpha) = \psi(\beta) \Leftrightarrow \exists v \in V, \alpha = \beta + \chi^K_v.
\end{align*}
In the non-abelian case, we have the following.

\begin{lemma} Define the relation $\sim$ on $Z^1(K, V)$ by
	\begin{align}
		\alpha \sim \beta \Leftrightarrow \exists v \in V,\forall x \in K, \alpha = v\beta(x)(x \cdot v^{-1}).
		\label{eqn:h_equiv}
	\end{align}
Then $\sim$ is an equivalence relation.
\end{lemma}
\begin{proof}
The relation is symmetric, since $\alpha(x) = 1\alpha(x)(1 \cdot x^{-1})$ for all $\alpha\in Z^1(K, V)$ and all $x \in K$. It is reflexive since
\begin{align*}
	\alpha(x) = v \beta(x)(x \cdot v^{-1}) \Rightarrow \beta(x) = v^{-1}\alpha(x)(x\cdot v).
\end{align*}
We show the relation is transitive. Suppose
\begin{align*}
	\alpha(x) &= v\beta(x)(x\cdot v^{-1}) \\
	\beta(x) &= w\gamma(x)(x\cdot w^{-1}).
\end{align*}
Then
\begin{align*}
	\alpha(x) &= v\left(w \gamma(x)(x\cdot w^{-1})\right)\left(x\cdot v^{-1}\right) \\
		&= vw \gamma(x)\left(x\cdot w^{-1}v^{-1}\right) \\
		&= (vw) \gamma(x)\left(x\cdot (vw)^{-1}\right).
\end{align*}
Therefore, Equation \ref{eqn:h_equiv} defines an equivalence relation on $Z^1(K, V)$.
\end{proof}

\begin{definition}
Denote by $H^1(K, V)$ the \emph{1-cohomology}, defined to be the set of equivalence classes of $Z^1(K, V)$ under the relation in Equation \ref{eqn:h_equiv}, and denote by $\psi$ the canonical projection from $Z^1(K, V)$ to $H^1(K, V)$.
\end{definition}

\begin{remark}\label{trivial}
Following Remark \ref{identity}, we identify the trivial map in $Z^1(K, V)$ by $\chi^K_1$, and we say a 1-cohomology is trivial if $H^1(K, V) = \psi(\chi^K_1)$, or equivalently if $Z^1(K, V) = B^1(K, V)$.
\end{remark}

R. Richardson \cite[Lemma 6.2.6]{richardson1982orbits} provides a result analogous to Lemma \ref{lem:lin_red_h}:
\begin{lemma}
  Suppose $K$ is linearly reductive and $V$ is unipotent. Then $H^1(K, V)$ is trivial.
  \label{lem:nonab_lin_red}
\end{lemma}

\section{Maps Between 1-Cohomologies}

We use multiplicative notation for the next few definition-like results (Lemma \ref{zeta}--Lemma \ref{h1maps}) but they also hold in the abelian setting as this is just a special case of the nonabelian case. We discuss a consequence of the main Lemma in the abelian setting in Corollary \ref{zlinear}, and both Lemma \ref{brown} and Example \ref{ab_example} require that $V$ is a vector space. In such cases we revert to additive notation to emphasise the fact that $V$ is abelian.

As in the previous Section, $K, V$ are algebraic groups such that $K$ acts on $V$ by group automorphisms. We denote the action of $K$ on $V$ by $x \cdot v$, for $x \in K, v \in V$.

\begin{lemma} \label{zeta}
	Let $K'$ be an algebraic group and let $\zeta:K'\rightarrow K$ be a homomorphism. We have an action of $K'$ on $V$ by group automorphisms, defined by $x \cdot v = \zeta(x) \cdot v$ for all $x \in K', v \in V$.

	Then for all $\alpha \in Z^1(K, V)$, $\alpha \circ \zeta \in Z^1(K', V)$.
\end{lemma}
\begin{proof}
	Let $\alpha \in Z^1(K, V)$. Evidently $\alpha \circ \zeta$ is a morphism from $K'$ to $V$. Let $x, y \in K'$, then
	\begin{align*}
		(\alpha \circ \zeta)(xy) &= \alpha\left(\zeta(xy)\right) \\
			&= \alpha\left(\zeta(x)\zeta(y)\right) \\
			&= \alpha(\zeta(x))\left(\zeta(x) \cdot \alpha(\zeta(y))\right) \\
			&= \alpha(\zeta(x))\left(x \cdot \alpha(\zeta(y))\right) \\
			&= (\alpha\circ\zeta)(x)\left(x \cdot (\alpha\circ\zeta)(y)\right).
	\end{align*}
	Therefore, since $\alpha \circ \zeta$ is a morphism and satisfies the 1-cocycle condition, $\alpha \circ \zeta \in Z^1(K', V)$.
\end{proof}

\begin{lemma} \label{xi}
	Let $V'$ be an algebraic group on which $K$ acts by group automorphisms.
	Let $\xi: V \rightarrow V'$ be a $K$-equivariant homomorphism, that is, $x \cdot \xi(v) = \xi(x \cdot v)$ for all $x \in K, v \in V$. Then for all $\alpha \in Z^1(K, V)$, $\xi \circ \alpha \in Z^1(K, V')$. 
\end{lemma}
\begin{proof}
	Let $\alpha \in Z^1(K, V)$. Evidently $\xi \circ \alpha$ is a morphism from $K$ to $V'$. Let $x, y \in K$, then
	\begin{align*}
		(\xi \circ \alpha)(xy) &= \xi\left(\alpha(xy)\right) \\
			&= \xi\left(\alpha(x)(x \cdot \alpha(y))\right) \\
			&= \xi(\alpha(x))\xi(x \cdot \alpha(y)) \\
			&= \xi(\alpha(x))\left(x \cdot \xi(\alpha(y))\right) \\
			&= (\xi \circ \alpha)(x)\left(x \cdot (\xi \circ \alpha)(y)\right).
	\end{align*}
	Therefore, since $\xi \circ \alpha$ is a morphism and satisfies the 1-cocycle condition, $\xi \circ \alpha \in Z^1(K, V')$.
\end{proof}

\begin{lemma}[Map of 1-Cohomologies] \label{h1maps} Let $K', V'$ be algebraic groups such that $K'$ acts on $V'$ by group automorphisms.

	Let $\zeta:K' \rightarrow K$ be a homomorphism and let $\xi: V \rightarrow V'$ be a $K'$-equivariant homomorphism; that is, suppose that $\xi(\zeta(x) \cdot v) = x \cdot \xi(v)$ for all $x \in K', v \in V$.

	Then the function $Z^1(\zeta, \xi)$ defined by
	\begin{align*}
		Z^1(\zeta, \xi)(\alpha) = \xi \circ \alpha \circ \zeta,
	\end{align*}
	maps $Z^1(K, V)$ to $Z^1(K', V')$.

	Furthermore, $Z^1(\zeta, \xi)$ descends to give a well-defined map
	\begin{align*}
		H^1(\zeta, \xi):H^1(K, V) \rightarrow H^1(K', V'),
	\end{align*}
	defined by
	\begin{align*}
		H^1(\zeta, \xi)(\psi(\alpha)) = \left(\psi' \circ Z^1(\zeta, \xi)\right)(\alpha),
	\end{align*}
	for all $\alpha \in Z^1(K, V)$, where $\psi'$ is the canonical projection from $Z^1(K', V')$ to $H^1(K', V')$.
	Moreover, the following diagram commutes:
	\begin{align*}
		\xymatrix{
			Z^1(K, V) \ar[r]^{Z^1(\zeta, \xi)} \ar[d]_{\psi} & Z^1(K', V') \ar[d]^{\psi'} \\
			H^1(K, V) \ar[r]^{H^1(\zeta, \xi)}               & H^1(K', V').
		}
	\end{align*}
\end{lemma}
\begin{proof}
	Let $\alpha \in Z^1(K, V)$. By Lemma \ref{zeta}, $\alpha \circ \zeta \in Z^1(K', V)$, where the action of $K'$ on $V$ is given by
	\begin{align*}
		x\cdot v = \zeta(x)\cdot v,
	\end{align*}
	for $x\in K', v\in V$.
	
	By Lemma \ref{xi}, $\xi \circ \left(\alpha \circ \zeta\right) = \xi \circ \alpha \circ \zeta \in Z^1(K', V')$. Therefore $Z^1(\zeta, \xi)$ maps $Z^1(K, V)$ to $Z^1(K', V')$.

	It remains to show $H^1(\zeta, \xi)$ is well-defined. Let $\alpha,\beta \in Z^1(K, V)$ such that $\psi(\alpha) = \psi(\beta)$. Then there exists $v \in V$ such that
	\begin{align*}
		\beta(x) = v\alpha(x)(x \cdot v^{-1}),
	\end{align*}
	for all $x \in K$.

	Then, for all $x \in K'$
	\begin{align*}
		\left(Z^1(\zeta, \xi)(\beta)\right)(x) &= \xi\left(\beta(\zeta(x))\right) \\
			&= \xi \left( v \alpha(\zeta(x))\left(\zeta(x) \cdot v^{-1}\right) \right) \\
			&= \xi(v) \xi(\alpha(\zeta(x))) \xi\left( \zeta(x) \cdot v^{-1}\right) \\
			&= \xi(v) \xi(\alpha(\zeta(x)))\left(x \cdot \xi(v^{-1})\right) \\
			&= \xi(v) \left(\left(Z^1(\zeta, \xi)(\alpha)\right)(x)\right) \left(x \cdot \xi(v^{-1})\right).
	\end{align*}
	This shows that $\psi'\left(Z^1(\zeta, \xi)(\alpha)\right) = \psi'\left(Z^1(\zeta, \xi)(\beta)\right)$, hence $H^1(\zeta, \xi)$ is well-defined. It is clear that the diagram commutes.
\end{proof} 
\begin{remark} \label{maps_functorial}
 The maps $Z^1(\zeta, \xi), H^1(\zeta, \xi)$ are \emph{functorial}. Suppose that
\begin{align*}
	&K'' \stackrel{\zeta'}\longrightarrow K' \stackrel{\zeta}\longrightarrow K, \textrm{ and} \\
	&V \stackrel{\xi}\longrightarrow V' \stackrel{\xi'}\longrightarrow V'',
\end{align*}
are homomorphisms of groups, and suppose that $K, K', K''$ act on $V, V', V''$ by group automorphisms such that the appropriate equivariance properties hold. Then
\begin{align*}
Z^1(\zeta \circ \zeta', \xi' \circ \xi) &= Z^1(\zeta, \xi) \circ Z^1(\zeta', \xi'), \textrm{ and} \\
H^1(\zeta \circ \zeta', \xi' \circ \xi) &= H^1(\zeta, \xi) \circ H^1(\zeta', \xi')\quad(\textrm{cf. Equation \ref{eqn:functorial}}).
\end{align*}
\end{remark}
\begin{remark}
	The slightly unfortunate choice of notation ``$Z^1(K, V)$'' does not make the action explicit. A consequence is that given suitable homomorphisms
	\begin{align*}
		\zeta&:K \rightarrow K, \\
		\xi&: V \rightarrow V,
	\end{align*}
	the statement
	\begin{align*}
		H^1(\zeta, \xi):H^1(K, V) \rightarrow H^1(K, V)
	\end{align*}
	is misleading on its own: Is the action of $K$ on $V$ the same on the left and the right? If nothing is said about the action defining the 1-cocycles on the right, we take that to mean the two actions are the same.

	Similarly, when $\zeta$ is the inclusion of some $K' < K$ in $K$ and $\xi$ is the identity map on $V$, it is implicit that the action of $K'$ on $V$ is defined by the action of $K$ on $V$, that is $x \cdot v = \zeta(x) \cdot v$, $x \in K', v \in V$, unless specified otherwise (cf. Example \ref{h1subgp}, Lemma \ref{brown}).

This issue arises in Chapter \ref{Chapter4}, where we will adopt a modified notation for these 1-cocycles, 1-cohomology, etc., that makes it clear what the action is.
\end{remark}

\begin{lemma} \label{zlinear} Let $K,K',V,V',\zeta,\xi$ satisfy the requirements of Lemma \ref{h1maps} and suppose $V,V'$ are abelian. Then $Z^1(\zeta, \xi)$ is a homomorphism and maps $B^1(K, V)$ into $B^1(K', V')$.

Moreover, if $V, V'$ are vector spaces and the actions are linear, then $Z^1(K, V)$ is a linear map.
\end{lemma}
\begin{proof}
	We prove the case where $V, V'$ are abelian. Let $\alpha, \beta \in Z^1(\zeta, \xi)$. Then
\begin{align*}
	\left(Z^1(\alpha + \beta)\right)(x) &= \xi\left((\alpha + \beta)(\zeta(x))\right) \\
		&= \xi\left(\alpha(\zeta(x)) + \beta(\zeta(x))\right) \\
		&= \xi\left(\alpha(\zeta(x))\right) + \xi\left(\beta(\zeta(x))\right) \\
		&= Z^1(\zeta, \xi)(\alpha) + Z^1(\zeta, \xi)(\beta).
\end{align*}
Clearly $Z^1(\zeta, \xi)\left(\chi^K_v\right) = \chi^{K'}_{\xi(v)}$ for any $v\in V$, so $Z^1(\zeta, \xi)$ maps $B^1(K, V)$ into $B^1(K', V')$.

The case where $V, V'$ are vector spaces is left as an exercise.
\end{proof}

\begin{example} \label{h1subgp}
	Let $K' < K$, let $\zeta$ be the inclusion of $K'$ in $K$, and let $\xi$ be the identity map on $V$. Then by Lemma \ref{h1maps}, the map $Z^1(\zeta, \xi)$ defined by
\begin{align*}
	Z^1(\zeta, \xi)(\alpha) = \xi \circ \alpha \circ \zeta = \alpha \circ \zeta,
\end{align*}
maps $Z^1(K, V)$ into $Z^1(K', V)$, and the map
\begin{align*}
	H^1(\zeta, \xi)(\psi(\alpha)) = \psi' \circ Z^1(\zeta, \xi),
\end{align*}
is a well-defined map of 1-cohomologies from $H^1(K, V)$ to $H^1(K', V)$. 
\end{example}

The situation in Example \ref{h1subgp} will be common, so we introduce further notation to suppress $\xi$ when it is the identity map on $V$.
\begin{definition} Let $K' < K$, let $\zeta$ be the inclusion of $K'$ in $K$, and let $\xi$ be the identity map on $V$. Define
\begin{align*}
	Z^1(\zeta) = Z^1(\zeta, \xi), \\
	H^1(\zeta) = H^1(\zeta, \xi).
\end{align*}
\end{definition}

\begin{definition} Denote by $\mathrm{Ker}\left(Z^1(\zeta, \xi)\right)$ the collection of all $\alpha\in Z^1(K, V)$ such that $Z^1(\zeta, \xi)(\alpha) = \chi^{K'}_1 \in B^1(K', V')$. Similarly, denote by $\mathrm{Ker}\left(H^1(\zeta, \xi)\right)$ the collection of all $x \in H^1(K, V)$ such that $H^1(\zeta, \xi)(x) = \psi'\left(\chi^{K'}_1\right)\in H^1(K', V')$.
\end{definition}

The following Lemma is useful when showing that $H^1(\zeta, \xi)$ is injective (cf. Lemma \ref{brown}, Lemma \ref{sl2_b_inj}).

\begin{lemma}\label{kerh1} Let $K, K', V, V', \zeta, \xi$ satisfy the requirements of Lemma \ref{h1maps} and suppose $\xi$ is surjective. 
Let $x\in \mathrm{Ker}\left(H^1(\zeta, \xi)\right)$. Then there exists $\beta \in Z^1(K, V)$ such that $x = \psi(\beta)$ and
\begin{align*}
	Z^1(\zeta, \xi)(\beta) = \chi^{K'}_1.
\end{align*}
\end{lemma}
\begin{proof}
	Let $\alpha \in Z^1(K, V)$ such that $\psi(\alpha) \in \mathrm{Ker}\left(H^1(\zeta, \xi)\right)$. Hence $Z^1(\zeta, \xi)(\alpha)$ is a 1-coboundary, so there exists $v \in V'$ such that
\begin{align*}
	Z^1(\zeta, \xi)(\alpha) = \chi^{K'}_v.
\end{align*}
Since $\xi$ is surjective there exists $w \in V$ such that $\xi(w) = v$.
Let $\beta \in Z^1(K, V)$ be defined by
\begin{align*}
	\beta(x) = w^{-1}\alpha(x)(x \cdot w),
\end{align*}
for all $x \in K$. Then $\psi(\beta) = \psi(\alpha)$, and for all $x \in K'$
\begin{align*}
	\left(Z^1(\zeta, \xi)(\beta)\right)(x) &= \xi\left(\beta(\zeta(x))\right) \\
		&= \xi\left(w^{-1}\alpha(\zeta(x))(\zeta(x)\cdot w )\right) \\
		&= \xi\left(w^{-1}\right)\xi\left(\alpha(\zeta(x))\right)\xi\left(\zeta(x)\cdot w )\right) \\
		&= v^{-1} \left(Z^1(\zeta, \xi)(\alpha)\right)(x) (\zeta(x) \cdot v ) \\
		&= v^{-1} v(\zeta(x) \cdot v^{-1}) (\zeta(x) \cdot v ) \\
		&= (v^{-1} v) \left((\zeta(x) \cdot v)^{-1} (\zeta(x) \cdot v )\right) \\
		&= 1.
\end{align*}
	Hence $Z^1(\zeta, \xi)(\beta) = \chi^{K'}_1$.
\end{proof}

The next Lemma is standard. It is stated in \cite[TODO]{brown1976cohomology} with the proof left to the reader, so we give our own proof here. The Lemma deals with the abelian 1-cohomology of a finite group, so we alter our notation appropriately.

\begin{lemma} \label{brown}
Let $V$ be a vector space over $k$, $\mathrm{char}(k) = p$. Let $\Gamma$ be a finite group that acts linearly on $V$, and let $\Gamma_p$ be a \emph{Sylow $p$-subgroup} of $\Gamma$. Let $\zeta$ be the inclusion of $\Gamma_p$ in $\Gamma$. Then the map 
\begin{align*}
H^1(\zeta):H^1(\Gamma, V)\rightarrow H^1(\Gamma_p, V)
\end{align*}
is injective.
\end{lemma}
\begin{proof}
	Let $x\in \mathrm{Ker}\left(H^1(\zeta)\right)$. By Lemma \ref{kerh1} there exists $\beta \in Z^1(\Gamma, V)$ such that $x = \psi(\beta)$ and $\beta(\gamma) = 0$ for all $\gamma \in \Gamma_p$.
	
	Choose a set of representatives $\{\gamma_1, \ldots, \gamma_l\} \subset \Gamma$ for the left coset space of $\Gamma_p$ in $\Gamma$.
	For any $\gamma \in \Gamma$ and $\gamma' \in \Gamma_p$,
\begin{align*}
	\beta(\gamma \gamma') = \beta(\gamma) + \gamma \cdot \beta(\gamma') = \beta(\gamma) +\gamma \cdot 0 = \beta(\gamma).
\end{align*} 
Hence $\beta$ is constant on the left $\Gamma_p$-cosets.

We have an action of $\Gamma$ on the left $\Gamma_p$-coset space, defined by $\gamma \cdot (\widetilde{\gamma}\Gamma_p) = (\gamma\widetilde{\gamma})\Gamma_p$. It is clear that this action is independent of the choice of representative $\widetilde{\gamma}$ of the coset $\widetilde{\gamma}\Gamma_p$, so the action is well-defined. Hence for a fixed $\gamma \in \Gamma$, left multiplication by $\gamma$ permutes the left $\Gamma_p$-cosets $\{\gamma_1\Gamma_p, \gamma_2\Gamma_p, \ldots, \gamma_l\Gamma_p\}$. It follow that the ordered set $\{\gamma\gamma_1, \gamma\gamma_2, \ldots, \gamma\gamma_l\}$ meets each left $\Gamma_p$-coset exactly once. Therefore
\begin{align}\label{betaconst}
	\sum_{i=1}^l \beta(\gamma\gamma_i) = \sum_{i=1}^l \beta(\gamma_i),
\end{align}
although the summands may be in a different order.

Let $w = \sum_{i=1}^l \beta(\gamma_i)$ and consider the 1-coboundary $\chi^\Gamma_w \in B^1(\Gamma, V)$.
\begin{align*}
	\chi_{w}^\Gamma(\gamma) &=  w - \gamma\cdot w \\
	&=  \sum_{i = 1}^l\beta(\gamma_i) - \gamma\cdot \sum_{i = 1}^l\beta(\gamma_i) \\
	&=  \sum_{i = 1}^l\beta(\gamma_i) - \sum_{i = 1}^l\gamma \cdot \beta(\gamma_i) \\
	&=  \sum_{i = 1}^l\beta(\gamma_i) - \sum_{i = 1}^l\left(\beta(\gamma\gamma_i) - \beta(\gamma)\right) \quad(\textrm{Equation }\ref{eqn:a_z})\\
	&=  \sum_{i = 1}^l\beta(\gamma_i) - \sum_{i = 1}^l \beta(\gamma\gamma_i) +\sum_{i = 1}^l \beta(\gamma) \\
	&=  \sum_{i = 1}^l\beta(\gamma_i) - \sum_{i = 1}^l \beta(\gamma_i) +\sum_{i = 1}^l \beta(\gamma) \quad(\textrm{Equation }\ref{betaconst})\\
	&=  \sum_{i = 1}^l \beta(\gamma) \\
	&= l\beta(\gamma),
\end{align*}
for all $\gamma \in \Gamma$.

Since $\mathrm{gcd}(l, p) = \mathrm{gcd}\left([\Gamma_p:\Gamma], p\right) = 1$, the positive integer $l$ is not zero in $k$. Therefore, by Lemma \ref{vspace}
\begin{align*}
	\beta = \chi^\Gamma_{l^{-1}w} \in B^1(\Gamma, V),
\end{align*}
which proves $\mathrm{Ker}\left(H^1(\zeta)\right)$ is trivial.
\end{proof}

\begin{example} \label{ab_example}
	Let $k = \overline{\mathbb{F}_p} = \bigcup_{r\in \mathbb{N}} \mathbb{F}_{p^r}$.
Let $V$ a be vector space over $k$ on which $SL_2(k)$ acts linearly, and let $U_2(k)$ be the subgroup of $SL_2(k)$ consisting of upper unitriangular matrices. Let $\zeta$ be the inclusion of $U_2(k)$ in $SL_2(k)$.

Then the map
	\begin{align}
		H^1(\zeta): H^1(SL_2(k), V) \rightarrow H^1(U_2(k), V)
	\end{align}
	is injective.
\label{eg:sl2ab}
\end{example}
\begin{proof}
	Let $r \in \mathbb{N}$ and denote the inclusion maps
\begin{align*}
	\zeta_r&:U_2(\mathbb{F}_{p^r}) \hookrightarrow SL_2(\mathbb{F}_{p^r}), \\
	\iota_r&:SL_2(\mathbb{F}_{p^r}) \hookrightarrow SL_2(k), \\
	\iota'_r&:U_2(\mathbb{F}_{p^r}) \hookrightarrow U_2(k).
\end{align*}
By Lemma \ref{h1maps} (Remark \ref{maps_functorial}) we get the following commutative diagram,
\begin{align}\label{eqn:cd}
	\xymatrix@C=40pt{
		H^1(SL_2(k), V) \ar[r]^{H^1(\zeta)} \ar[d]_{H^1(\iota_r)} & H^1(U_2(k), V) \ar[d]^{H^1(\iota'_r)} \\
		H^1(SL_2(\mathbb{F}_{p^r}), V)\, \ar[r]_{H^1(\zeta_r)} &\, H^1(U_2(\mathbb{F}_{p^r}), V).
	}
\end{align}

It is elementary to show that $U(\mathbb{F}_{p^r})$ is a Sylow $p$-subgroup of $SL_2(\mathbb{F}_{p^r})$ (Appendix \ref{u_sylow}, or \cite[TODO]{carter1989simple}), so by Lemma \ref{brown}, $H^1(\zeta_r)$ is injective for all $r \in \mathbb{N}$.

Let $\beta\in Z^1(SL_2(k), V)$ such that $\beta \notin B^1(SL_2(k), V)$, that is,
\begin{align}\label{betanob1}
	\beta \neq \chi^{SL_2(k)}_v,
\end{align}
for any $v \in V$. For each $x\in SL_2(\mathbb{F}_{p^r})$ define the morphism $f_x:V\rightarrow V$ by
	\begin{align*}
		f_x(v) = \beta(x) - \chi^{SL_2(k)}_v(x).
	\end{align*}
Since $\mathbb{F}_{p^{r!}} \subset \mathbb{F}_{p^{(r+1)!}}$ we have $SL_2(\mathbb{F}_{p^{r!}}) \subset SL_2(\mathbb{F}_{p^{(r+1)!}})$.
Consider the sequence $\left\{C_{r}\right\}_{r \in \mathbb{N}}$ defined by
	\begin{align*}
		C_{r} = \{v \in V \,|\,\forall x\in SL_2(\mathbb{F}_{p^{r}}),\, f_x(v) = 0\}.
	\end{align*}
	Then
\begin{align*}
	\bigcap_{r\in \mathbb{N}}C_{r!} &= \{v \in V \,|\, \forall x \in SL_2(k),\, f_x(v) = 0\} \\
		&= \emptyset \quad(\textrm{Equation \ref{betanob1}}).
\end{align*}
	Each $C_{r}$ is closed, and the inclusion $SL_2(\mathbb{F}_{p^{r!}}) \subset SL_2(\mathbb{F}_{p^{(r+1)!}})$ induces the reverse inclusion for the subsequence $C_{r!} \supset C_{(r+1)!}$.
Then the Noetherian property for $V$ requires that the subsequence $\left\{C_{r!}\right\}_{r \in \mathbb{N}}$ becomes constant, and since $\cap_{r\in\mathbb{N}}C_{r!} = \emptyset$, the subsequence $\left\{C_{r!}\right\}_{r \in \mathbb{N}}$ is eventually empty.
That is, there exists $s\in\mathbb{N}$ such that
	\begin{align*}
		Z^1(\iota_s)(\beta) \neq \chi_v^{SL_2(\mathbb{F}_{p^{s}})},
	\end{align*}
	for any $v \in V$. We have shown that if $\beta \in Z^1(SL_2(k), V)$ such that $Z^1(\iota_{r!})(\beta) \in B^1(SL_2(\mathbb{F}_{p^{r!}}), V)$ for all $r \in \mathbb{N}$, then $\beta \in B^1(SL_2(k), V)$.

So, let $\alpha \in Z^1(SL_2(k), V)$ such that $\psi(\alpha) \in \mathrm{Ker}\left(H^1(\zeta)\right)$.
Then, consulting the commutative diagram in Equation \ref{eqn:cd},
\begin{align*}
	&\psi(\alpha) \in \mathrm{Ker}\left(H^1(\iota'_r) \circ H^1(\zeta)\right), \forall r \in \mathbb{N} \\
	\Rightarrow &\,\psi(\alpha) \in \mathrm{Ker}\left(H^1(\zeta_r) \circ H^1(\iota_r)\right), \forall r \in \mathbb{N}  \\
	\Rightarrow &\,H^1(\iota_r)(\psi(\alpha)) \in \mathrm{Ker}\left(H^1(\zeta_r)\right), \forall r \in \mathbb{N}  \\
	\Rightarrow &\,H^1(\iota_r)(\psi(\alpha))\textrm{ is trivial }, \forall r \in \mathbb{N} \\% \quad(\textrm{since }H^1(\zeta_r)\textrm{ is injective}) \\
	\Rightarrow &\,Z^1(\iota_r)(\alpha) \in B^1(SL_2(\mathbb{F}_{p^r}), V), \forall r \in \mathbb{N}  \\
	\Rightarrow &\,\alpha \in B^1(SL_2(k), V) \\ %\quad(\textrm{by (ii)}).
	\Rightarrow &\,\psi(\alpha) \in H^1(SL_2(k), V) \textrm{ is trivial}.
\end{align*}
This shows $H^1(\zeta)$ is injective.
\end{proof}

\begin{remark}
TODO: Borel $B < SL_2$, $U = R_u(B)$, $SL_2$ acts on unipotent $V$ $\Rightarrow$ $H^1(B, V) \rightarrow H^1(U, V)$ is injective. Not sure if it's true, but would imply $H^1(SL_2, V) \rightarrow H^1(U, V)$ is injective. Then apply main theorem of CH4.
\end{remark}

\begin{lemma} Let $V$ be an algebraic group, and suppose $SL_2(k)$ acts on $V$ by group automorphisms. Let $B$ be a Borel subgroup of $SL_2(k)$ and let $\zeta: B \rightarrow SL_2(k)$ be the inclusion map. Then $H^1(\zeta): H^1(SL_2(k), V)\rightarrow H^1(B, V)$ is injective.
  \label{sl2_b_inj}
\end{lemma}
\begin{proof}
Let $x \in \mathrm{Ker}\left(H^1(\zeta)\right)$. Then by Lemma \ref{kerh1} there exists $\beta \in Z^1(SL_2(k), V)$ such that $x = \psi(\beta)$ and $\beta(b) = 1$ for all $b \in B$.

Let $y\in SL_2(k), b \in B$. Since $\beta(yb) = \beta(y)(y\cdot \beta(b)) = \beta(y)$, there exists a unique morphism $\widetilde{\beta}$ such that the following diagram commutes,
\begin{align*}
	\xymatrix{
	SL_2(k) \ar[r]^{\beta} \ar[d]_{\pi} & V\\
	SL_2/B \ar[ru]_{\widehat{\beta}}
	},
\end{align*}
where $\pi$ is the canonical projection $\pi:SL_2 \rightarrow SL_2/B$ (\cite[TODO]{humphreys1975linear}). Hence $\widehat{\beta}(xB) = \beta(x)$ for all $x \in SL_2(k)$.

Since $SL_2(k)/B$ is an irreducible projective variety \cite[Theorem 21.3]{humphreys1975linear}, $\widehat{\beta}$ must be constant. Therefore, since $\beta(x) = 1$ for all $x \in B$, $\beta(x) = \widehat{\beta}(xB) = 1$ for all $x \in SL_2(k)$. Hence $\beta = \chi^{SL_2(k)}_1 \in B^1(SL_2(k), V)$ and so $\mathrm{Ker}\left(H^1(\zeta)\right)$ is trivial.
\end{proof} 

\begin{lemma}\label{trivial_on_t} Suppose $V$ is unipotent and let $x \in H^1(SL_2(k), V)$. Then there exists $\beta \in Z^1(SL_2(k), V)$ such that $x = \psi(\beta)$ and $\beta\left(\begin{matrix}* & 0\\0 & *\end{matrix}\right) = 1$.
\end{lemma}
\begin{proof}
	Denote $D_2(k) = \left(\begin{matrix}* & 0\\0 & *\end{matrix}\right)$, and let $\zeta:D_2(k) \rightarrow SL_2(k)$ be the inclusion map. Since $D_2(k)$ is linearly reductive, $H^1(D_2(k), V)$ is trivial (Lemma \ref{lem:nonab_lin_red}), so $\mathrm{Ker}\left(H^1(\zeta)\right) = H^1(SL_2(k), V)$. Now apply Lemma \ref{kerh1}.
\end{proof}
