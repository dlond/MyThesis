%!TEX root = ../Thesis.tex
% Chapter 3

\chapter{The 1-Cohomology}
\label{Chapter3}
\lhead{Chapter 3. \emph{The 1-Cohomology}}

\section{Abelian 1-Cohomology}
The abelian 1-cohomology is standard (cf. \cite{brown1976cohomology}). We present this Section as motivation for the less-known non-abelian theory to appear in the next Section.

Let $K$ be an algebraic group and $V$ an abelian group on which $K$ acts by group automorphisms. We denote the action by $x \cdot v$, for $x \in K, v \in V$.

\begin{definition} We call a morphism $\alpha: K\rightarrow V$ a \emph{1-cocycle} if it satisfies
	\begin{align}
		\alpha(xy) = \alpha(x) + x\cdot\alpha(y),
		\label{eqn:a_z}
	\end{align}
	for all $x, y$ in $K$. Denote by $Z^1\left( K, V \right)$ the collection of all 1-cocycles from $K\rightarrow V$.

	We call Equation \ref{eqn:a_z} the \emph{1-cocycle condition}.
\end{definition}

For any $\alpha, \beta$ in $Z^1\left(K, V\right)$,
\begin{align*}
	\left(\alpha + \beta\right)(xy) &=  \alpha(xy) +  \beta(xy) \\
	&=  \alpha(x) + x\cdot\alpha(y) +  \beta(x) + x\cdot\beta(y)\\
	&=  \left( \alpha(x) + \beta(x) \right) + x\cdot\left(\alpha(y) + \beta(y)\right) \\
	&=  \left(\alpha+\beta\right)(x) + x\cdot\left(\alpha + \beta\right)(y),
\end{align*}
for all $x,y \in K$. Moreover, $\alpha + \beta$ is a morphism, so $Z^1(K, V)$ is closed under pointwise addition.

The trivial map from $K \rightarrow V$ that sends every $h \in K$ to the identity $0 \in V$ is a 1-cocycle. Let $1\in K$ be the identity, then for any $\alpha \in Z^1(K, V)$ we have
\begin{align*}
	\alpha(1)\, =\, \alpha(1\cdot 1) &=  \alpha(1) + 1\cdot \alpha(1) \\
	&=  \alpha(1) + \alpha(1) \\
	&=  2\,\alpha(1),
\end{align*}
so $\alpha(1) = 0$. Then, for all $x \in K$,
\begin{align*}
	0 = \alpha(1) = \alpha(xx^{-1}) = \alpha(x) + x \cdot \alpha(x^{-1})
\end{align*}
Hence each $\alpha$ has a negative defined by
\begin{align*}
	-\alpha(x) = x\cdot\alpha(x^{-1}),
\end{align*}
for all $x \in K$. 

Associativity of pointwise addition in $Z^1(K, V)$ holds because associativity holds in $V$. Therefore $Z^1\left(K, V\right)$ is a group, abelian since $V$ is, and hence a $\mathbb{Z}-$module under pointwise addition.

\begin{definition} Let $v \in V$. The morphism $\chi^K_v:K\rightarrow V$ defined by
\begin{align*}
	\chi^K_v (x) = v - x\cdot v,
\end{align*}
for all $x \in K$, is called a \emph{1-coboundary}. We denote by $B^1\left(K, V\right)$ the collection of all 1-coboundaries from $K \rightarrow V$. 
\end{definition}

For any $v \in V$
\begin{align*}
	\chi^K_v(xy) &=  v - (xy)\cdot v \\
	&=  v - x \cdot \left(y\cdot v \right)\\
	&=  v - x \cdot \left(v -v + y\cdot v \right)\\
	&=  v - x\cdot v + x\cdot \left( v - y\cdot v\right)\\
	&=  \chi^K_v(x) + x\cdot \chi^K_v(y),
\end{align*}
for all $x, y \in K$. Therefore $B^1(K, V) \subset Z^1(K, V)$. Furthermore, for any $v, w \in V$,
\begin{align*}
	(\chi^K_v - \chi^K_w)(x) &=  \chi^K_v(x) - \chi^K_w(x)\\
	&=  v - x \cdot v - w - x\cdot v \\
	&=  (v - w) - x\cdot (v - w) \\
	&=  \chi^K_{v - w} (x),
\end{align*}
for all $x \in K$. We see that $B^1\left(K, V\right)$ is a $\mathbb{Z}-$submodule of $Z^1(K, V)$, so we may form the quotient module.

\begin{definition} The \emph{1-cohomology} is the quotient module defined by
\begin{align*}
	H^1\left(K, V\right) = Z^1\left(K, V\right) / B^1\left(K, V\right).
\end{align*}
We denote by $\psi$ the canonical projection from $Z^1(K, V)\rightarrow H^1(K, V)$.
\end{definition}

\begin{lemma} \label{vspace} If $V$ is a vector space and $K$ acts linearly on $V$ then $Z^1(K, V)$ is a vector space and $B^1(K, V)\subset Z^1(K, V)$ is a vector subspace.
\end{lemma}
\begin{proof}
	Suppose $V$ is a vector space over $F$. We define scalar multiplication on $Z^1(K, V)$ and $B^1(K, V)$ as follows.

For $\lambda \in F, \alpha \in Z^1(K, V)$ define the map $\lambda\alpha$ by $(\lambda\alpha)(x) = \lambda\alpha(x)$ for all $x \in K$. Then $\lambda\alpha$ is a 1-cocycle, for
\begin{align*}
	\lambda \alpha(xy) &= \lambda\left( \alpha(x) + x \cdot \alpha(y)\right) \\
	&= \lambda\alpha(x) + \lambda(x\cdot \alpha(y)) \\
	&= \lambda\alpha(x) + x \cdot (\lambda\alpha(y)).
\end{align*}

Similarly, let $\chi^K_v \in B^1(K, V)$. Then for all $x \in K$
\begin{align*}
	\lambda\chi^K_v(x) &= \lambda\left(v - x \cdot v\right) \\
		&= \lambda v - \lambda(x \cdot v) \\
		&= \lambda v - x \cdot (\lambda v) \\
		&= \chi^K_{\lambda v}(x),
\end{align*}
so $\lambda\chi^K_v = \chi^K_{\lambda v} \in B^1(K, V)$ for all $v \in V$.

The vector space axioms follow.
\end{proof}

We conclude this Section with a useful Lemma \cite[Proposition 1]{kemper2000characterization}.
\begin{lemma} Suppose $K$ is linearly reductive. Then $H^1(K, V)$ is trivial.
  \label{lem:lin_red_h}
\end{lemma}

\section{Non-abelian 1-Cohomology}
	
We no longer require that $V$ is abelian, henceforth $V$ is an algebraic group on which $K$ acts by group automorphisms. Much of the preceding Section is a direct analogue; for the most part, formulas are just rewritten in multiplicative notation.
One main difference will see is that 1-coboundaries are less useful in the non-abelian setting, especially when defining the 1-cohomology.

\begin{definition} We call a morphism $\alpha:K\rightarrow V$ a \emph{1-cocycle} if it satisfies
\begin{align}
  \alpha(xy) = \alpha(x) (x\cdot\alpha(y)),
  \label{eqn:na_z}
\end{align}
for all $x, y \in K$. Denote by $Z^1\left( K, V \right)$ the collection of all 1-cocycles from $K\rightarrow V$.

We call Equation \ref{eqn:na_z} the \emph{1-cocycle condition}.
\end{definition}

\begin{remark} Unlike the previous Section there is no natural addition operation on $Z^1(K, V)$.
\end{remark}

\begin{definition} Let $v \in V$. The morphism $\chi^K_v:K\rightarrow V$ defined by
\begin{align*}
	\chi^K_v (x) = v (x\cdot v^{-1}),
\end{align*}
for all $x \in K$ is called a \emph{1-coboundary}. We denote by $B^1\left(K, V\right)$ the collection of all 1-coboundaries from $K \rightarrow V$.
\end{definition}

For any $v \in V$ and any $x, y \in K$,
\begin{align*}
	\chi^K_v(xy) &=  v \left((xy) \cdot v^{-1}\right) \\
	&=  v (x \cdot v^{-1}) (x \cdot v) ((xy) \cdot v^{-1}) \\
	&=  v \left(x \cdot v^{-1}\right) (x \cdot v) \left(x \cdot \left(y \cdot v^{-1}\right)\right) \\
	&=  v \left(x \cdot v^{-1}\right) \left(x \cdot v \left(y \cdot v^{-1}\right)\right) \\
	&=  \chi^K_v(x) (x \cdot \chi^K_v(y)),
\end{align*}
so $B^1(K, V) \subset Z^1(K, V)$.

In the abelian case we use the fact that we can take the quotient $Z^1(K, V)/B^1(K, V)$, as $B^1(K, V)$ is a $\mathbb{Z}$-submodule of $Z^1(K, V)$, but in the non-abelian they are just sets. Moreover, although we did not explicitly mention it, in the abelian case the following holds
\begin{align*}
	\psi(\alpha) = \psi(\beta) \Leftrightarrow \exists v \in V, \alpha = \beta + \chi^K_v.
\end{align*}
In the non-abelian case, we have the following.

\begin{lemma} Define the relation $\sim$ on $Z^1(K, V)$ by
	\begin{align}
		\alpha \sim \beta \Leftrightarrow \exists v \in V, \alpha = v\beta(x)(x \cdot v^{-1}), \forall x \in K.
		\label{eqn:h_equiv}
	\end{align}
Then $\sim$ is an equivalence relation.
\end{lemma}
\begin{proof}
The relation is symmetric, since $\alpha(x) = e\alpha(x)(e \cdot x^{-1})$ for all $\alpha\in Z^1(K, V)$ and all $x \in K$, where $e$ is the identity if $V$. It is reflexive since
\begin{align*}
	\alpha(x) = v \beta(x)(x \cdot v^{-1}) \Rightarrow \beta(x) = v^{-1}\alpha(x)(x\cdot v).
\end{align*}
We show the relation is transitive. Suppose
\begin{align*}
	\alpha(x) &= v\beta(x)(x\cdot v^{-1}) \\
	\beta(x) &= w\gamma(x)(x\cdot w^{-1}),
\end{align*}
Then
\begin{align*}
	\alpha(x) &= v\left(w \gamma(x)(x\cdot w^{-1})\right)\left(x\cdot v^{-1}\right) \\
		&= vw \gamma(x)\left(x\cdot w^{-1}v^{-1}\right) \\
		&= (vw) \gamma(x)\left(x\cdot (vw)^{-1}\right).
\end{align*}
Therefore, Equation \ref{eqn:h_equiv} defines an equivalence relation on $Z^1(K, V)$.
\end{proof}

\begin{definition}
Denote by $H^1(K, V)$ the \emph{1-cohomology}, defined to be the set of equivalence classes of $Z^1(K, V)$ under the relation in Equation \ref{eqn:h_equiv}, and denote by $\psi$ the canonical projection from $Z^1(K, V) \rightarrow H^1(K, V)$.
\end{definition}

R. Richardson \cite[Lemma 6.2.6]{richardson1982orbits} provides a result analogous to Lemma \ref{lem:lin_red_h}:
\begin{lemma}
  Suppose $K$ is linearly reductive and $V$ is unipotent. Then $H^1(K, V)$ is trivial.
  \label{lem:nonab_lin_red}
\end{lemma}

\section{Maps Between 1-Cohomologies}

The next few definition-like results (Lemma \ref{zeta}--Lemma \ref{h1maps}) could be stated in terms of the abelian 1-cohomology, but for reasons of efficiency we use multiplicative notation only. We discuss a consequence of the main Lemma in the abelian setting in Corollary \ref{zlinear}, and Lemma \ref{brown} requires that $V$ is a vector space. In such cases we revert to additive notation to emphasise the fact that $V$ is abelian.

As in the previous Section, $K, V$ are algebraic groups such that $K$ acts on $V$ by group automorphisms. We denote the action of $K$ on $V$ by $x \cdot v$, for $x \in K, v \in V$.

\begin{lemma} \label{zeta}
	Let $K'$ be an algebraic group that acts on $V$ by group automorphisms. Denote the action of $K'$ on $V$ by $x \ast v$, for $x \in K', v \in V$.

	Let $\zeta:K'\rightarrow K$ be a homomorphism and suppose that $x \ast v = \zeta(x) \cdot v$ for all $x \in K', v \in V$. Then for all $\alpha \in Z^1(K, V)$, $\alpha \circ \zeta \in Z^1(K', V)$ 
\end{lemma}
\begin{proof}
	Let $\alpha \in Z^1(K, V)$. Evidently $\alpha \circ \zeta$ is a morphism from $K' \rightarrow V$. Let $x, y \in K'$, then
	\begin{align*}
		(\alpha \circ \zeta)(xy) &= \alpha\left(\zeta(x)\zeta(y)\right) \\
			&= \alpha(\zeta(x))\left(\zeta(x) \cdot \alpha(\zeta(y))\right) \\
			&= \alpha(\zeta(x))\left(x \ast \alpha(\zeta(y))\right) \\
			&= (\alpha \circ \zeta)(x)\left(x \ast (\alpha \circ \zeta)(y)\right).
	\end{align*}
	Therefore, since $\alpha \circ \zeta$ is a morphism and satisfies the 1-cocycle condition, $\alpha \circ \zeta \in Z^1(K', V)$.
\end{proof}

\begin{lemma} \label{xi}
	Let $V'$ be an algebraic group on which $K$ acts by group automorphisms. Denote the action of $K$ on $V'$ by $x \wedge v$ for $x \in K, v \in V'$.

	Let $\xi: V \rightarrow V'$ be a $K$-equivariant homomorphism, that is, $x \wedge \xi(v) = \xi(x \cdot v)$ for all $x \in K, v \in V$. Then for all $\alpha \in Z^1(K, V)$, $\xi \circ \alpha \in Z^1(K, V')$. 
\end{lemma}
\begin{proof}
	Let $\alpha \in Z^1(K, V)$. Evidently $\xi \circ \alpha$ is a morphism from $K \rightarrow V'$. Let $x, y \in K$, then
	\begin{align*}
		(\xi \circ \alpha)(xy) &= \xi\left(\alpha(x)(x \cdot \alpha(y))\right) \\
			&= \xi(\alpha(x))\xi(x \cdot \alpha(y)) \\
			&= \xi(\alpha(x))\left(x \wedge \xi(\alpha(y))\right) \\
			&= (\xi \circ \alpha)(x)\left(x \wedge (\xi \circ \alpha)(y)\right).
	\end{align*}
	Therefore, since $\xi \circ \alpha$ is a morphism and satisfies the 1-cocycle condition, $\xi \circ \alpha \in Z^1(K, V')$.
\end{proof}

\begin{lemma}[Map of 1-Cohomologies] \label{h1maps} Let $K', V'$ be algebraic groups, such that
	\begin{itemize}
		\item[(a)] $K'$ acts on $V$ by group automorphisms, denoted by $x \ast v$, for $x \in K', v \in V$,
		\item[(b)] $K'$ acts on $V'$ by group automorphisms, denoted by $x \wedge v$, for $x \in K', v \in V'$.
	\end{itemize}
	Let $\zeta:K' \rightarrow K$ and $\xi: V \rightarrow V'$ be homomorphisms such that
	\begin{itemize}
		\item[(c)] $x \ast v = \zeta(x) \cdot v$ for all $x \in K', v \in V$,
		\item[(d)] $\xi(x \ast v) = x \wedge \xi(v)$ for all $x \in K', v \in V$.
	\end{itemize}
	Then the function $Z^1(\zeta, \xi)$ defined by
	\begin{align*}
		Z^1(\zeta, \xi)(\alpha) = \xi \circ \alpha \circ \zeta,
	\end{align*}
	maps $Z^1(K, V) \rightarrow Z^1(K', V')$.

	Furthermore, $Z^1(\zeta, \xi)$ descends to give a well-defined map
	\begin{align*}
		H^1(\zeta, \xi):H^1(K, V) \rightarrow H^1(K', V'),
	\end{align*}
	defined by
	\begin{align*}
		H^1(\zeta, \xi)(\psi(\alpha)) = \left(\psi' \circ Z^1(\zeta, \xi)\right)(\alpha),
	\end{align*}
	for all $\alpha \in Z^1(K, V)$, where $\psi'$ is the canonical projection from $Z^1(K', V') \rightarrow H^1(K', V')$.
	Moreover, the following diagram commutes:
	\begin{align*}
		\xymatrix{
			Z^1(K, V) \ar[r]^{Z^1(\zeta, \xi)} \ar[d]_{\psi} & Z^1(K', V') \ar[d]^{\psi'} \\
			H^1(K, V) \ar[r]^{H^1(\zeta, \xi)}               & H^1(K', V').
		}
	\end{align*}
\end{lemma}
\begin{proof}
	Let $\alpha \in Z^1(K, V)$. Evidently $\xi \circ \alpha \circ \zeta$ is a morphism. Let $x,y \in K'$, then
	\begin{align*}
		(\xi \circ \alpha \circ \zeta)(xy) &= \xi \left((\alpha \circ \zeta)(x)\right) \xi \left(x \ast (\alpha \circ \zeta)(y)\right) \quad\textrm{(Lemma \ref{zeta})} \\
			&= \xi \left((\alpha \circ \zeta)(x)\right) \left(x \wedge \xi\left((\alpha \circ \zeta)(y)\right)\right) \quad\textrm{(by (d))} \\
			&= (\xi \circ \alpha \circ \zeta)(x) \left(x \wedge (\xi \circ \alpha \circ \zeta)(y)\right).
	\end{align*}
	Therefore $(\xi \circ \alpha \circ \zeta) \in Z^1(K', V')$.

	It remains to show $H^1(\zeta, \xi)$ is well-defined. Let $\alpha,\beta \in Z^1(K, V)$ such that $\psi(\alpha) = \psi(\beta)$. Then there exists $v \in V$ such that
	\begin{align*}
		\beta(x) = v\alpha(x)(x \cdot v^{-1}),
	\end{align*}
	for all $x \in K$.

	Then, for all $x \in K'$
	\begin{align*}
		\left(Z^1(\zeta, \xi)(\beta)\right)(x) &= \xi\left(\beta(\zeta(x))\right) \\
			&= \xi \left( v \alpha(\zeta(x))\left(\zeta(x) \cdot v^{-1}\right) \right) \\
			&= \xi(v) \xi(\alpha(\zeta(x))) \xi\left( \zeta(x) \cdot v^{-1}\right) \\
			&= \xi(v) \xi(\alpha(\zeta(x))) \xi\left( x \ast v^{-1}\right) \\
			&= \xi(v) \xi(\alpha(\zeta(x)))\left(x \wedge \xi(v^{-1})\right) \\
			&= \xi(v) \left(\left(Z^1(\zeta, \xi)(\alpha)\right)(x)\right) \left(x \wedge \xi(v^{-1})\right).
	\end{align*}
	This shows that $\psi'\left(Z^1(\zeta, \xi)(\alpha)\right) = \psi'\left(Z^1(\zeta, \xi)(\beta)\right)$, hence $H^1(\zeta, \xi)$ is well-defined.
\end{proof} 

\begin{remark}
	The slightly unfortunate choice of notation ``$Z^1(K, V)$'' doesn't explicitly mention the action. A consequence is that given suitable homomorphisms
	\begin{align*}
		\zeta&:K \rightarrow K, \\
		\xi&: V \rightarrow V,
	\end{align*}
	the statement
	\begin{align*}
		H^1(\zeta, \xi):H^1(K, V) \rightarrow H^1(K, V)
	\end{align*}
	is misleading on its own: Is the 1-cohomology on the left of the arrow the same as the one on the right? If nothing is said about the action defining the 1-cocycles on the right, we take that to mean they are the same.

	Similarly, when $\zeta$ is the inclusion of some $K' < K$ in $K$ and $\xi$ is the identity map on $V$, it is implicit that the action of $K'$ on $V$ is defined by the action of $K$ on $V$ (cf. Example \ref{h1subgp}, Lemma \ref{brown}) unless specified otherwise.

	More satisfyingly, and by pure coincidence, later on in Chapter \ref{Chapter4} we adopt a modified notation for the 1-cocycles and the 1-cohomology which carries with it the action defining the 1-cocycles.
\end{remark}

\begin{example} \label{h1subgp}
	Let $K' < K$ and let $\zeta$ be the inclusion of $K'$ in $K$. Let $\xi$ be the identity map on $V$. Then by Lemma \ref{h1maps}, the map $Z^1(\zeta, \xi)$ defined by
\begin{align*}
	Z^1(\zeta, \xi)(\alpha) = \xi \circ \alpha \circ \zeta = \alpha \circ \zeta,
\end{align*}
maps $Z^1(K, V)$ into $Z^1(K', V)$, and the map
\begin{align*}
	H^1(\zeta, \xi)(\psi(\alpha)) = \psi' \circ Z^1(\zeta, \xi),
\end{align*}
is a well-defined map of 1-cohomologies from $H^1(K, V) \rightarrow H^1(K', V)$. Furthermore, we have the inclusions
\begin{align*}
	Z^1(K, V) &\subset Z^1(K', V),\\ H^1(K, V) &\subset H^1(K', V)
\end{align*}
\end{example}

\begin{corollary} \label{zlinear} Let $K,K',V,V',\zeta,\xi$ satisfy the requirements of Lemma \ref{h1maps} and suppose $V,V'$ are abelian (vector spaces). Then $Z^1(\zeta, \xi)$ is is a homomorphism (linear map).
\end{corollary}
\begin{proof}
	We prove the case where $V, V'$ are abelian. Let $\alpha, \beta \in Z^1(\zeta, \xi)$. Then
\begin{align*}
	\left(Z^1(\alpha + \beta)\right)(x) &= \xi\left((\alpha + \beta)(\zeta(x))\right) \\
		&= \xi\left(\alpha(\zeta(x)) + \beta(\zeta(x))\right) \\
		&= \xi\left(\alpha(\zeta(x))\right) + \xi\left(\beta(\zeta(x))\right) \\
		&= Z^1(\zeta, \xi)(\alpha) + Z^1(\zeta, \xi)(\beta).
\end{align*}
The case where $V, V'$ are vector spaces is left as an exercise.
\end{proof}

The following Lemma is useful when showing that $H^1(\zeta, \xi)$ is injective (cf. Lemma \ref{brown}, Example \ref{sl2_b_inj}).

\begin{lemma}\label{kerh1} Let $K, K', V, V', \zeta, \xi$ satisfy the requirements of Lemma \ref{h1maps} and suppose $\xi$ is surjective. 
Then there exists $\beta \in Z^1(K, V)$ such that $\psi(\beta) \in \mathrm{Ker}\left(H^1(\zeta, \xi)\right)$ and
\begin{align*}
	\left(Z^1(\zeta, \xi)(\beta)\right)(x) = e,
\end{align*}
for all $x \in K'$, where $e$ is the identity in $V'$.
\end{lemma}
\begin{proof}
	Let $\alpha \in Z^1(K, V)$ such that $\psi(\alpha) \in \mathrm{Ker}\left(H^1(\zeta, \xi)\right)$. Hence $Z^1(\zeta, \xi)(\alpha)$ is a 1-coboundary, so there exists $v \in V'$ such that
\begin{align*}
	Z^1(\zeta, \xi)(\alpha) = \chi^{K'}_v.
\end{align*}
Since $\xi$ is surjective there exists $w \in V$ such that $\xi(w) = v$.
Let $\beta \in Z^1(K, V)$ be defined by
\begin{align*}
	\beta(x) = w^{-1}\alpha(x)(x \cdot w),
\end{align*}
for all $x \in K$. Then $\psi(\beta) = \psi(\alpha)$, and for all $x \in K'$
\begin{align*}
	\left(Z^1(\zeta, \xi)(\beta)\right)(x) &= \xi\left(\beta(\zeta(x))\right) \\
		&= \xi\left(w^{-1}\alpha(\zeta(x))(\zeta(x)\cdot w )\right) \\
		&= v^{-1} \left(Z^1(\zeta, \xi)(\alpha)\right)(x) (\zeta(x) \wedge v ) \\
		&= v^{-1} \left(v(\zeta(x) \wedge v^{-1})\right) (\zeta(x) \wedge v ) \\
		&= (v^{-1} v) \left((\zeta(x) \wedge v)^{-1} (\zeta(x) \wedge v )\right) \\
		&= e.
\end{align*}
\end{proof}

The next Lemma is standard \cite[Theorem 10.3]{brown1976cohomology} but we give our own proof here. The Lemma deals with the abelian 1-cohomology of a finite group, so we alter our notation appropriately.

\begin{lemma} \label{brown}
Let $V$ be a vector space over $k$, $\mathrm{char}(k) = p$. Let $\Gamma$ be a finite group that acts linearly on $V$, and let $\Gamma_p$ be a \emph{Sylow $p$-subgroup} of $\Gamma$. Let $\zeta$ be the inclusion of $\Gamma_p$ in $\Gamma$, and $\xi$ the identity map on $V$. Then the map 
\begin{align*}
H^1(\zeta, \xi):H^1(\Gamma, V)\rightarrow H^1(\Gamma_p, V)
\end{align*}
is injective.
\end{lemma}
\begin{proof}
	By Lemma \ref{kerh1} there exists $\beta \in Z^1(\Gamma, V)$ such that $\psi(\beta) \in \mathrm{Ker}\left(H^1(\zeta, \xi)\right)$ and $\beta(\gamma) = e$ for all $\gamma \in \Gamma_p$.
	
	Choose a set of representatives $\{\gamma_1, \ldots, \gamma_l\} \subset \Gamma$ for the left coset space of $\Gamma_p$ in $\Gamma$.
	For any $\gamma \in \Gamma$ and $\gamma' \in \Gamma_p$,
\begin{align*}
	\beta(\gamma \gamma') = \beta(\gamma) + \gamma \cdot \beta(\gamma') = \beta(\gamma) +\gamma \cdot e = \beta(\gamma).
\end{align*} 
Hence $\beta$ is constant on the left $\Gamma_p$-cosets, and therefore
\begin{align}\label{betaconst}
	\sum_{i = 1}^l \beta(\gamma\gamma_i) = \sum_{i = 1}^l \beta(\gamma_i),
\end{align}
where the summands are possibly taken in a different order. 
Let $w = \sum_{i=1}^l \beta(\gamma_i)$ and consider the 1-coboundary $\chi^\Gamma_w \in B^1(\Gamma, V)$.
\begin{align*}
	\chi_{w}^\Gamma(\gamma) &=  w - \gamma\cdot w \\
	&=  \sum_{i = 1}^l\beta(\gamma_i) - \gamma\cdot \sum_{i = 1}^l\beta(\gamma_i) \\
	&=  \sum_{i = 1}^l\beta(\gamma_i) - \sum_{i = 1}^l\gamma \cdot \beta(\gamma_i) \\
	&=  \sum_{i = 1}^l\beta(\gamma_i) - \sum_{i = 1}^l\left(\beta(\gamma\gamma_i) - \beta(\gamma)\right) \quad(\textrm{Equation }\ref{eqn:a_z})\\
	&=  \sum_{i = 1}^l\beta(\gamma_i) - \sum_{i = 1}^l \beta(\gamma\gamma_i) +\sum_{i = 1}^l \beta(\gamma) \\
	&=  \sum_{i = 1}^l\beta(\gamma_i) - \sum_{i = 1}^l \beta(\gamma_i) +\sum_{i = 1}^l \beta(\gamma) \quad(\textrm{Equation }\ref{betaconst})\\
	&=  \sum_{i = 1}^l \beta(\gamma) \\
	&= l\beta(\gamma),
\end{align*}
for all $\gamma \in \Gamma$.

Since $\mathrm{gcd}(l, p) = \mathrm{gcd}\left([\Gamma_p:\Gamma], p\right) = 1$, $l \neq 0$. Therefore, by Lemma \ref{vspace}
\begin{align*}
	\beta = \chi^\Gamma_{l^{-1}w} \in B^1(\Gamma, V),
\end{align*}
which proves $\mathrm{Ker}\left(H^1(\zeta, \xi)\right)$ is trivial.
\end{proof}

\begin{example}
	Let $k = \overline{\mathbb{F}_p} = \bigcup_{r\in \mathbb{N}} \mathbb{F}_{p^r}$.
Let $V$ a be vector space over $k$ on which $SL_2(k)$ acts by group automorphisms, and let $U_2(k)$ be the subgroup of $SL_2(k)$ consisting of upper unitriangular matrices. Let $\zeta$ be the inclusion of $U_2(k)$ in $SL_2(k)$ and $\xi$ be the identity map on $V$.

Then the map
	\begin{align}
		H^1(\zeta, \xi): H^1(SL_2(k), V) \rightarrow H^1(U_2(k), V)
	\end{align}
	is injective.
\label{eg:sl2ab}
\end{example}
\begin{proof}
	Let $r \in \mathbb{N}$ and denote the inclusion maps
\begin{align*}
	\zeta_r&:U_2(\mathbb{F}_{p^r}) \hookrightarrow SL_2(\mathbb{F}_{p^r}), \\
	\iota_r&:SL_2(\mathbb{F}_{p^r}) \hookrightarrow SL_2(k), \\
	\iota'_r&:U_2(\mathbb{F}_{p^r}) \hookrightarrow U_2(k).
\end{align*}
Repeated application of Lemma \ref{h1maps} yields the following commutative diagram,
%\begin{align*}
%	\xymatrix{
%		Z^1(SL_2(k), V) \ar[r]^{Z^1(\zeta, \xi)} \ar[d]_{Z^1(\iota_r, \xi)} & Z^1(U_2(k), V) \ar[d]^{Z^1(\iota'_r, \xi)} \\
%		Z^1(SL^2(\mathbb{F}_{p^r}), V) \ar[r]_{Z^1(\zeta_r, \xi)} & Z^1(U_2(\mathbb{F}_{p^r}), V),
%	}
%\end{align*}
\begin{align*}
	\xymatrix@C=40pt{
		H^1(SL_2(k), V) \ar[r]^{H^1(\zeta, \xi)} \ar[d]_{H^1(\iota_r, \xi)} & H^1(U_2(k), V) \ar[d]^{H^1(\iota'_r, \xi)} \\
		H^1(SL_2(\mathbb{F}_{p^r}), V)\, \ar[r]_{H^1(\zeta_r, \xi)} &\, H^1(U_2(\mathbb{F}_{p^r}), V).
	}
\end{align*}

First we will show
\begin{itemize}
\item [(i)] $H^1(\zeta_r, \xi)$ is injective for each $r \in \mathbb{N}$, and
\item [(ii)] if $\beta \in Z^1(SL_2(k), V)$ such that for all $r \in \mathbb{N}$
\begin{align*} Z^1(\iota_{r!}, \xi)(\beta) \in B^1(SL_2(\mathbb{F}_{p^{r!}}), V),
\end{align*}
then $\beta \in B^1(SL_2(k), V)$.
\end{itemize}

The group $GL_2(\mathbb{F}_{p^r})$ has order $(p^{2r} - 1)(p^{2r} - p^r)$ since there are $p^{2r} - 1$ choices of vectors for the first column (all choices excluding the zero vector), and $p^{2r} - p^r$ choices of vectors for the second column (all choices excluding multiples of the first vector). The determinant is a homomorphism of groups
	\begin{align}
		\mathrm{det}:GL_2(\mathbb{F}_{p^r}) \rightarrow \mathbb{F}^*_{p^r},
	\end{align}
	with kernel $SL_2(\mathbb{F}_{p^r})$. Therefore, by the First Homomorphism Theorem for groups
	\begin{align}
		GL_2(\mathbb{F}_{p^r})\,/\,SL_2(\mathbb{F}_{p^r}) \simeq \mathrm{det}(GL_2(\mathbb{F}_{p^r})) = \mathbb{F}^*_{p^r},
	\end{align}
	and so
	\begin{align*}
		|SL_2(\mathbb{F}_{p^r})|
		&=  |GL_2(\mathbb{F}_{p^r})|\,/\,|\mathbb{F}^*_{p^r}|\\
		&=  (p^{2r} - 1)(p^{2r} - p^r)\,/\,(p^r - 1)\\
		&=  p^r(p^{2r} - 1).
	\end{align*}
	Since $|U(\mathbb{F}_{p^r})| = p^r$, $U(\mathbb{F}_{p^r})$ is a Sylow $p$-subgroup of $SL_2(\mathbb{F}_{p^r})$.
So by Lemma \ref{brown}, $H^1(\zeta_r, \xi)$ is injective. This proves (i).
	
Let $\beta\in Z^1(SL_2(k), V)$ such that $\beta \notin B^1(SL_2(k), V)$, that is,
\begin{align}\label{betanob1}
	\beta \neq \chi^{SL_2(k)}_v,
\end{align}
for any $v \in V$. For each $x\in SL_2(\mathbb{F}_{p^r})$ define the morphism $f_x:V\rightarrow V$ by
	\begin{align*}
		f_x(v) = \beta(x) - \chi^{SL_2(k)}_v(x).
	\end{align*}
Since $\mathbb{F}_{p^{r!}} \subset \mathbb{F}_{p^{(r+1)!}}$ we have $SL_2(\mathbb{F}_{p^{r!}}) \subset SL_2(\mathbb{F}_{p^{(r+1)!}})$.
Consider sequence $\left\{C_{r}\right\}_{r \in \mathbb{N}} \subset V$ defined by
	\begin{align*}
		\left\{C_{r}\right\} = \{v \in V \,|\,\forall x\in SL_2(\mathbb{F}_{p^{r}}),\, f_x(v) = 0\},
	\end{align*}
where $e$ is the identity in $V$.
	Each $C_{r}$ is closed, and the inclusion $SL_2(\mathbb{F}_{p^{r!}}) \subset SL_2(\mathbb{F}_{p^{(r+1)!}})$ induces the reverse inclusion for the subsequence $C_{r!} \supset C_{(r+1)!}$.
Then the Noetherian property for $V$ requires that the subsequence $\left\{C_{r!}\right\}_{r \in \mathbb{N}}$ becomes constant, and since
\begin{align*}
	&\{v \in V \,|\, \forall x \in SL_2(k), f_x(v) = 0\} \\
	= &\left\{v \in V \,|\, \forall x \in SL_2(k), \beta(x) = \chi^{SL_2(k)}_v(x)\right\} \\
	= \,&\emptyset \quad(\textrm{Equation }\ref{betanob1}), 
\end{align*}
the subsequence $\left\{C_{r!}\right\}_{r \in \mathbb{N}}$ is eventually empty.
That is, there exists $s\in\mathbb{N}$ such that
	\begin{align*}
		Z^1(\iota_s, \xi)(\beta) \neq \chi_v^{SL_2(\mathbb{F}_{p^{s}})},
	\end{align*}
	for any $v \in V$, which proves (ii).

Finally, let $\psi(\alpha) \in \mathrm{Ker}\left(H^1(\zeta, \xi)\right)$.
Then
\begin{align*}
	&\psi(\alpha) \in \mathrm{Ker}\left(H^1(\iota'_r, \xi) \circ H^1(\zeta, \xi)\right), \forall r \in \mathbb{N} \\
	\Rightarrow &\,\psi(\alpha) \in \mathrm{Ker}\left(H^1(\zeta_r, \xi) \circ H^1(\iota_r, \xi)\right), \forall r \in \mathbb{N}  \\
	\Rightarrow &\,H^1(\iota_r, \xi)(\psi(\alpha))\textrm{ is trivial }, \forall r \in \mathbb{N} \quad(\textrm{By (i)}) \\
	\Rightarrow &\,Z^1(\iota_r, \xi)(\alpha) \in B^1(SL_2(\mathbb{F}_{p^r}), V), \forall r \in \mathbb{N}  \\
	\Rightarrow &\,\alpha \in B^1(SL_2(k), V)\quad(\textrm{By (ii)}).
\end{align*}
This shows $H^1(\zeta, \xi)$ is injective.
\end{proof}

\begin{example} Now suppose that $V$ is an algebraic group, not necessarily a vector space. Let $B$ be a Borel subgroup of $SL_2(k)$, let $\zeta: B \rightarrow SL_2(k)$ be the inclusion map and $\xi: V \rightarrow V$ the identity map. Then $H^1(\zeta, \xi): H^1(SL_2(k), V)\rightarrow H^1(B, V)$ is injective.
  \label{sl2_b_inj}
\end{example}
\begin{proof}
By Lemma \ref{kerh1} there exists $\beta \in Z^1(SL_2(k), V)$ such that $\psi(\beta) \in \mathrm{Ker}\left(H^1(\zeta, \xi)\right)$ and $\beta(x) = e$ for all $x \in B$, where $e$ is the identity of $V$.

Let $\widehat{\beta}$ be the unique morphism such that the following diagram commutes,
\begin{align*}
	\xymatrix{
	SL_2(k) \ar[r]^{\beta} \ar[d]_{\pi} & V\\
	SL_2/B \ar[ru]_{\widehat{\beta}}
	},
\end{align*}
where $\pi$ is the canonical projection $\pi:SL_2 \rightarrow SL_2/B$. Hence $\widehat{\beta}(xB) = \beta(x)$ for all $x \in SL_2(k)$.

Since $SL_2(k)/B$ is an irreducible projective variety \cite[Theorem 21.3]{humphreys1975linear}, $\widehat{\beta}$ must be constant. Therefore, since $\beta(x) = e$ for all $x \in B$, $\beta(x) = \widehat{\beta}(xB) = e$ for all $x \in SL_2(k)$. Hence $\beta = \chi^{SL_2(k)}_e \in B^1(SL_2(k), V)$ and so $\mathrm{Ker}\left(H^1(\zeta, \xi)\right)$ is trivial.
\end{proof} 

TODO: Borel $B < SL_2$, $U = R_u(B)$, $SL_2$ acts on unipotent $V$ $\Rightarrow$ $H^1(B, V) \rightarrow H^1(U, V)$ is injective. Not sure if it's true, but would imply $H^1(SL_2, V) \rightarrow H^1(U, V)$ is injective. Then apply main theorem of CH4.
