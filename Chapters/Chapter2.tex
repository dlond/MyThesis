%!TEX root = ../Thesis.tex
% Chapter 2

\chapter{Mathematical Preliminaries}
\label{Chapter2}
\lhead{Chapter 2. \emph{Mathematical Preliminaries}}

We will assume the reader has an understanding of basic algebraic geometry and algebraic group theory which could be obtained by consulting Springer \cite{springer2008linear} or Humphreys \cite{humphreys1975linear}. In this Chapter we will recall some relevant parts of the theory of algebraic groups and establish some notation.

Throughout this Thesis, $k$ will denote an algebraically closed field of characteristic $p>0$, and all varieties and algebraic groups will be defined over $k$.

Many of our example calculations will involve the special linear group of $2\times 2$ matrices, $SL_2(k)$, and the following subgroups.
\begin{align*}
	&T_2(k): \textrm{ the subgroup of diagonal matrices, a maximal torus of }SL_2(k),\\
	&B_2(k): \textrm{ the subgroup of upper triangular matrices, a Borel subgroup of }SL_2(k),\\
	&B^-_2(k): \textrm{ the subgroup of lower triangular matrices },\\
	&U_2(k): \textrm{ the subgroup of upper unitriangular matrices},\\
	&U^-_2(k): \textrm{ the subgroup of lower unitriangular matrices}.
\end{align*}

$B_2^-(k)$ is the Borel subgroup opposite $B_2(k)$ with respect to $T_2(k)$.

Let $T$ be a maximal torus of a connected reductive linear algebraic group $G$. We write $\mathfrak{t}$ for the Lie algebra of $T$. We denote by $X(T)$ the character group of $T$, defined as the collection of all algebraic group homomorphisms from $T$ to $k^*$ with the addition law
\begin{align*}
	(x_1 + x_2)(t) = x_1(t)x_2(t),
\end{align*}
for all $x_1, x_2\in X(T), t\in T$.

The cocharacter group, $Y(T)$, is the collection of all algebraic group homomorphisms from $k^*$ to $T$ with the addition law
\begin{align*}
	(y_1 + y_2)(\lambda) = y_1(\lambda)y_2(\lambda),
\end{align*}
for all $y_1, y_2\in Y(T), \lambda \in k^*$.

If we compose $x\in X(T)$ with $y\in Y(T)$ we get a morphism from $k^*$ to $k^*$; that is, a morphism of the form $\lambda\mapsto \lambda^n$ for some $n\in \mathbb{Z}$. Hence there exists a pairing $\langle,\rangle:X(T)\times Y(T)\rightarrow \mathbb{Z}$ defined
\begin{align*}
	(x, y) \mapsto \langle x, y\rangle = n,
\end{align*}
where $x(y(\lambda)) = \lambda^n$. 

Let $\Phi$ denote the set of roots of $G$. If $B$ is a Borel subgroup of $G$ the positive roots with respect to $B$ will be denoted by $\Phi^+$ and the negative roots will be denoted by $\Phi^-$, the base denoted by $\Delta$. Borel subgroups of $G$ containing the maximal torus $T$ are in bijective correspondence with bases for $\Phi$: choosing a Borel of $G$ containing $T$ amounts to choosing a set of simple roots for $\Phi$.

We say that the characteristic $p$ is good for $G$ if $p$ does not divide any of the coefficients of any $\beta\in\Phi$ when $\beta$ is written as a linear combination of simple roots.

We interpret the roots of $G$ as living in the real vector space $X(T)\otimes \mathbb{R}$ (see the rank 2 root system diagrams in Appendix \ref{rank2diagrams}).

We define the root group $U_\alpha$ to be the unique connected $T$-stable subgroup of $G$ having Lie algebra $\mathfrak{g}_\alpha$, where the Lie algebra $\mathfrak{g}$ of $G$ has the decomposition $\mathfrak{g} = \mathfrak{t} \oplus \coprod_{\alpha\in\Phi} \mathfrak{g}_\alpha$. There exists an isomorphism $\epsilon_\alpha: k \rightarrow U_\alpha$ such that for all $t\in T$, $x\in k$, $t\epsilon_\alpha(x)t^{-1} = \epsilon_\alpha(\alpha(t)x)$.

The group generated by $\langle U_\alpha, U_{-\alpha}\rangle$ is isomorphic to $SL_2$ or $PGL_2$. If $U$ is a connected, $T$-stable unipotent subgroup of $G$ then $U=\prod_\alpha U_\alpha$, where the product is taken in any fixed order.

Define $s_\alpha(\beta)= \beta -\langle \beta,\alpha\rangle\alpha$. The Weyl Group $W$ is generated by $\{s_\alpha\,|\,\alpha\in \Delta\}$. If $\alpha,\beta$ are linearly independent, then there exist $\gamma,\delta\in \Delta$ and $w\in W$ such that $w(\alpha) = \gamma$, while $w(\beta)$ is a $\mathbb{Z}^+$-linear combination of $\gamma, \delta$.

