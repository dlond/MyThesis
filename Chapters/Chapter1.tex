%!TEX root = /Users/dan/Documents/Thesis/Thesis.tex
% Chapter 1

\chapter{Introduction}
\label{Chapter1}
\lhead{Chapter 1. \emph{Introduction}}

\begin{itemize}
	\item[] \textbf{What is the thesis about and what are the main results}
	\begin{itemize}
		\item \emph{What the thesis is about:}
	
		This thesis is a contribution to the study of the subgroup structure of simple algebraic groups, complementing some of the work done by M. Liebeck and G. Seitz (\cite{liebeck1996reductive}, \cite{liebeck2004maximal}). Let $G$ be a simple algebraic group over an algebraically closed field of characteristic $p$. For large enough characteristic ($p=0$ or $p>7$ covers all restrictions) Liebeck and Seitz determine explicitly the embeddings of arbitrary closed connected semisimple subgroups in $G$ where $G$ is of exceptional type. Using similar methods, we examine the subgroup structure of simple algebraic groups in low characteristic (usually $p=2$ or $p=3$) where less is known.
		
		
		The main motivation for the work carried out in this thesis is to investigate a question posed by B. K\"ulshammer \cite{weil1964remarks}:
		\begin{quote}
		Let $G$ be a linear algebraic group over an algebraically closed field of characteristic $p$. Let $\Gamma$ be a finite group and $\Gamma_p\subset\Gamma$ a Sylow $p$-subgroup. Fix a conjugacy class of representations $\Gamma_p\rightarrow G$. Are there, up to conjugation by $G$, only finitely many representations $\rho:\Gamma\rightarrow G$ whose restrictions to $\Gamma_p$ belong to the given class?
		\end{quote}
		
		
		We use Lemma \ref{kToHOne} to reduce K\"ulshammer's question to a 1-cohomology calculation in which $\Gamma$ acts on the unipotent radical $V$ of a parabolic subgroup $P \subset G$ via an irreducible map $\Gamma \rightarrow L$, $L \subset P$ a Levi subgroup of $P$. We show that we can answer a particular instance of K\"ulshammer's question by examining the restriction map of 1-cohomologies:
		\begin{displaymath}
			H^1(\Gamma, V) \rightarrow H^1(\Gamma_p, V)
		\end{displaymath}
		
		
		This is useful for finding a counterexample to K\"ulshammer's question. A counterexample is known for a non-reductive $G$ \cite{weil1964remarks} and we investigate the reductive case. 
		
		
		We broaden our search for a counterexample by defining an \emph{algebraic group} version of K\"ulshammer's question where instead of a finite group $\Gamma$ we use a reductive group $H$, and instead of a Sylow $p$-subgroup $\Gamma_p \subset \Gamma$ we use a unipotent subgroup $U \subset H$. Finding an algebraic counterexample may yield a finite counterexample for a finite subgroup of $H$. We pay special attention to $H = SL_2$.

		\item \emph{Main results}:
		

		???
	\end{itemize}

	\item[] \textbf{Context, history, literary review}
	\begin{itemize}
		\item \emph{K. II - motivation for this:}
		
		K\"ulshammer's question has it's roots Maschke's Theorem which asserts that any representation of a finite group over a field of characteristic not dividing the order of the group is completely reducible. 
		K\"ulshammer's first question ...
		Link
		
		\item \emph{Work of Liebeck \& Seitz, etc, on embedding reductive $H$ inside simple $G$:}
		
	\end{itemize}

	\item[] \textbf{Methods (can refer forward)}
	\begin{itemize}
		\item \emph{Use of 1-cohomology to (K. II):}
		\item \emph{Key results e.g. $H^1(SL_2, V)\rightarrow H^1(B, V)$:}
	\end{itemize}
	
	\item[] \textbf{Chapter Summary}
	\begin{itemize}
		\item \emph{Preliminaries:}
		\item \emph{1-Cohomology:}
		\item \emph{K. II:}
		\item \emph{Calculations:}
		\item \emph{Summary/Future work:}
	\end{itemize}
\end{itemize}

