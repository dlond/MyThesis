%!TEX root = ../Thesis.tex
% Chapter 1

\chapter{Introduction}
\label{Chapter1}
\lhead{Chapter 1. \emph{Introduction}}

A major motivation for the work carried out in this thesis is to investigate a question posed by B. K\"ulshammer to do with homomorphisms of finite groups into algebraic groups \cite{slodowy1997two}. We will call these homomorphisms \emph{representations} because of the obvious similarity with the usual kind of representations into $GL_n$. K\"ulshammer's second question reads as follows.
\begin{quote}
  Let $G$ be a linear algebraic group over an algebraically closed field of characteristic $p$. Let $\Gamma$ be a finite group and $\Gamma_p < \Gamma$ a Sylow $p$-subgroup of $\Gamma$. Fix a conjugacy class of representations $\Gamma_p\rightarrow G$. Are there, up to conjugation by $G$, only finitely many representations $\rho:\Gamma\rightarrow G$ whose restrictions to $\Gamma_p$ belong to the given class?
\end{quote}

So far only a non-reductive counterexample is known \cite[Appendix]{slodowy1997two}. We examine K\"ulshammer's second question for reductive $G$. 


The work in this thesis also contributes to the study of the subgroup structure of simple algebraic groups, complementing some of the work done by M. Liebeck and G. Seitz (\cite{liebeck1996reductive}, \cite{liebeck2004maximal}), and D. Stewart (\cite{stewart2010g}).
Let $G$ be a simple algebraic group over an algebraically closed field of characteristic $p$. For large enough characteristic ($p=0$ or $p>7$ covers all restrictions) Liebeck and Seitz determine explicitly the embeddings of arbitrary connected semisimple groups in $G$, where $G$ is of exceptional type.
On the other hand, Stewart's work concerns exceptional groups for the case $p<7$, so called \emph{low characteristic}.
Like Stewart, we examine the subgroup structure of simple algebraic groups in low characteristic (usually $p=2$ or $p=3$) where examples are exotic less is known. 

In the intersection of the above two topics of interest lies a variation of K\"ulshammer's second question which we call the \emph{algebraic} version of K\"ulshammer's second question. In this case we substitute the finite group $\Gamma$ for a connected reductive group $H$, instead of a Sylow $p$-subgroup $\Gamma_p < \Gamma$ we use a maximal unipotent subgroup $U < H$, and by the term representation we mean a homomorphism of algebraic groups.
The precise statement of the algebraic version of K\"ulshammer's second question reads as follows.
\begin{quote}
  Let $G,H$ be connected reductive linear algebraic groups over an algebraically closed field of characteristic $p$ and $U < H$ a maximal unipotent subgroup of $H$. Fix a conjugacy class of representations $U\rightarrow G$. Are there, up to conjugation by $G$, only finitely many representations $\rho:H\rightarrow G$ whose restrictions to $U$ belong to the given class?
\end{quote}

Note that in answering the algebraic version of K\"ulshammer's second question for $H,G$ we investigate embeddings of $H$ in $G$, thus contributing to the work of Liebeck and Seitz, and Stewart.

Furthermore, it may be the case that a counterexample to the algebraic version of K\"ulshammer's second question for some $H, G$ provides a counterexample to the original question for $\Gamma, G$ where $\Gamma$ is some finite subgroup of $H$. For instance, it would be encouraging if there existed a counterexample to the algebraic version of K\"ulshammer's question with $H = SL_2(k), G$, where $k$ is an algebraically closed field of prime characteristic $p$, as $SL_2(k)$ has many finite subgroups (e.g. $SL_2(\mathbb{F}_{p^r}$)) with which we can test the original question.

\section{$G$-Complete Reducibility}

\section{K\"ulshammer's Second Question}

The algebraic group version of K\"ulshammer's question a non-trivial pursuit in its own right as K\"ulshammer's question has it's roots Maschke's Theorem of representation theory. Maschke's Theorem asserts that any representation from a finite group $\Gamma \rightarrow GL_n$ over a field of characteristic not dividing the order of $\Gamma$ is completely reducible, and that there are only finitely many conjugacy classes of (completely reducible) representations $\Gamma \rightarrow GL_n$ [ref Lang].

Let $\Gamma$ be a finite group and let $G$ be a linear algebraic group over an algebraically closed field of characteristic $p$. K\"ulshammer's first question reads:
\begin{quote}
  Suppose $p$ does not divide the order of $\Gamma$. Are there only finitely many conjugacy classes of representations $\Gamma\rightarrow G$?
\end{quote}
The answer is positive and is essentially contained in a paper of A. Weil \cite{weil1964remarks}. K\"ulshammer's second question is a refinement of the first:
\begin{quote}
  Let $\Gamma_p < \Gamma$ be a Sylow p-subgroup and fix a conjugacy class of representations $\Gamma_p\rightarrow G$. Are there only finitely many conjugacy classes of representations $\Gamma\rightarrow G$ whose restrictions to $\Gamma_p$ belong to the fixed class?
\end{quote}
Note that the condition that $p$ does not divide $|\Gamma|$ is dropped from the hypothesis. If $p$ does not divide the order of $\Gamma$ then the answer is ``yes'', since $\Gamma_p$ is trivial and so all representations are equal when restricted to $\Gamma_p$.

If $\Gamma$ is a $p$-group then the answer is ``yes'', as $\Gamma_p = \Gamma$ so restricting to $\Gamma_p$ does nothing and therefore only representations that come from the fixed class will hit the class.

If $G=GL_n$ and $p$ does not divide $|\Gamma|$ the answer is also ``yes'', since by Maschke's Theorem there can only be finitely many conjugacy classes of representations $\Gamma\rightarrow GL_n$ anyway, regardless of whether or not their restrictions to $\Gamma_p$ hit the fixed class. If $p$ does divide $|\Gamma|$ the answer has again shown to be positive \cite[Theorem]{slodowy1997two}.

The following example shows infinitely many conjugacy classes of representations of a finite group into $SL_2(k)$.
\begin{example}
  Let $\Gamma = C_p \times C_p = \langle a, b \,|\, ab = ba, a^p = b^p = 1 \rangle$ and consider representations $\rho: \Gamma \rightarrow SL_2(k)$. In particular, for each $\lambda \in k$ define $\rho_\lambda: \Gamma \rightarrow SL_2(k)$ by
  \begin{align*}
    \rho_\lambda(a) = \left( \begin{matrix} 1 & 1 \\ 0 & 1 \end{matrix} \right) \\
    \rho_\lambda(b) = \left( \begin{matrix} 1 & \lambda \\ 0 & 1 \end{matrix} \right).
  \end{align*}
  It is straightforward to check that if $\lambda_1 \neq \lambda_2$ then $\rho_{\lambda_1}$ is not $SL_2(k)$-conjugate to $\rho_{\lambda_2}$. Hence there are infinitely many $SL_2(k)$-conjugacy classes of representations from $\Gamma \rightarrow SL_2(k)$.
\end{example}

\section{Subgroup Structure of Algebraic Groups}

We use similar methods to Liebeck and Seitz, calculating a certain 1-cohomology of $H$ with coefficients in $V$, the unipotent radical of a parabolic subgroup $P < G$.

The main difference in our calculations is that we deal with the so-called \emph{nonabelian} 1-cohomology directly where as Liebeck, Seitz and Stewart use results from Representation Theory to study \emph{abelian layers} of the 1-cohomology and then piece the layers back together. Our calculations in Chapter \ref{Chapter5} agree with Stewart's $G_2$ calculation, and we acknowledge Stewart's $F_4$ calculation which provided us with a good example to work with in Chapter \ref{Chapter6}.

\section{A Variation on K\"ulshammer's Second Question}

\section{Chapter Overview}

One of our main results is Theorem \ref{thm:k2_h1}. With this we are able to relate K\"ulshammer's question to a certain 1-cohomology calculation in which $\Gamma$ acts on the unipotent radical $V$ of a parabolic subgroup $P < G$ via a certain representation $\Gamma \rightarrow L$ into a Levi subgroup $L < P$. We show that we can reduce K\"ulshammer's question to another question: is the restriction map of 1-cohomologies
\begin{align}
  H^1(\Gamma, V) \rightarrow H^1(\Gamma_p, V)
\end{align}
injective for all parabolic subgroups $P<G$?

This approach might help settle K\"ulshammer's original question.

In Chapter \ref{Chapter2} we produce some basic facts to do with Linear Algebraic Groups which could be found in texts such as Humphreys \cite{humphreys1975linear} and will be well-known to readers with a background in this area. This is an attempt to standardize notation and provide some background for the results to come.

In Chapter \ref{Chapter3} we introduce the 1-cohomology, first the well-known abelian case and second the lesser-known nonabelian case. In Example \ref{eg:sl2ab} we show that the restriction map of 1-cohomologies $H^1(SL_2(k), V) \rightarrow H^1(U(k), V)$ is injective for $V$ a vector space, which is the kind of result we can apply Theorem \ref{thm:k2_h1} to, in answering K\"ulshammer's second question. We also show that $H^1(SL_2(k), V) \rightarrow H^1(B, V)$ is injective for $B$ a Borel subgroup of $SL_2$ and $V$ an algebraic group, not necessarily abelian, on which $SL_2$ acts. The final step in applying an argument similar to Theorem \ref{thm:k2_h1} in an algebraic setting is that $H^1(B, V) \rightarrow H^1(U, V)$ is injective, where $U$ is the unipotent radical of $B$. We were unable to prove this but have evidence for it in our calculations in Chapter \ref{Chapter6}.

Chapter \ref{Chapter4} introduces the approach of finding reductive subgroups in reductive $G$ via the 1-cohomology and finishes with Theorem \ref{thm:k2_h1} which relates K\"ulshammer's second question to a question of restriction maps of 1-cohomologies as above.

In Chapter \ref{Chapter5} we collect together the results of various 1-cohomology calculations for $SL_2$ in $G$, $G$ of rank 2. They allude to a perhaps startling conjecture: although the types of 1-cohomology calculations $H^1(SL_2, V)$ involve a nonabelian $V$, there is evidence that a 1-cocycle in $Z^1(SL_2(k), V)$ that is zero on a maximal torus $T$ has image lying in an abelian subgroup of $V$. For instance, in Corollary \ref{cor:im_ab} we show this is the case for rank 1 parabolic subgroups of $G$ not containing $G_2$ or $C_3$. Furthermore Examples \ref{eg:g2}, \ref{eg:c3} verify the conjecture for $G = G_2, C_3$ respectively.

In Chapter \ref{Chapter6} we provide two examples which demonstrate the theoretical results captured in Chapter \ref{Chapter5}.

The future directions are summarised in the final Chapter.
