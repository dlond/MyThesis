%!TEX root = /Users/dan/Documents/Thesis/Thesis.tex
% Chapter 1

\chapter{Introduction}
\label{Chapter1}
\lhead{Chapter 1. \emph{Introduction}}

\begin{itemize}
	\item[] \textbf{What is the thesis about and what are the main results}
	\begin{itemize}
		\item \emph{What the thesis is about:}
	
		This thesis is a contribution to the study of the subgroup structure of simple algebraic groups, complementing some of the work done by M. Liebeck and G. Seitz (\cite{liebeck1996reductive}, \cite{liebeck2004maximal}). Let $G$ be a simple algebraic group over an algebraically closed field of characteristic $p$. For large enough characteristic ($p=0$ or $p>7$ covers all restrictions) Liebeck and Seitz determine explicitly the embeddings of arbitrary closed connected semisimple subgroups in $G$ where $G$ is of exceptional type. We examine the subgroup structure of simple algebraic groups in low characteristic (usually $p=2$ or $p=3$) where less is known. We employ 1-cohomology calculations in a similar way to Liebeck and Seitz but they seek trivial 1-cohomology, which gives them their characteristic restrictions, while we seek non-trivial 1-cohomology. Liebeck and Seitz divide their calculations into layers of abelian 1-cohomology calculations where as we define and work with the non-abelian 1-cohomology directly.
		
		
		A further motivation for the work carried out in this thesis is to investigate a question posed by B. K\"ulshammer to do with homomorphisms of finite groups into algebraic groups \cite{weil1964remarks}. We will call these homomorphisms \emph{representations} because of the obvious similarity with the usual kind of representations into $GL_n$. K\"ulshammer's second question is as follows:
		\begin{quote}
		Let $G$ be a linear algebraic group over an algebraically closed field of characteristic $p$. Let $\Gamma$ be a finite group and $\Gamma_p\subset\Gamma$ a Sylow $p$-subgroup. Fix a conjugacy class of representations $\Gamma_p\rightarrow G$. Are there, up to conjugation by $G$, only finitely many representations $\rho:\Gamma\rightarrow G$ whose restrictions to $\Gamma_p$ belong to the given class?
		\end{quote}
		
		
		One of our main results is Lemma \ref{kToHOne}. With this we are able to relate K\"ulshammer's question to a 1-cohomology calculation in which $\Gamma$ acts on the unipotent radical $V$ of a parabolic subgroup $P \subset G$ via an irreducible map $\Gamma \rightarrow L$, $L \subset P$ a Levi subgroup of $P$. We show that we can reduce K\"ulshammer's question to another question regarding a restriction map of 1-cohomologies:
		\begin{displaymath}
			H^1(\Gamma, V) \rightarrow H^1(\Gamma_p, V)
		\end{displaymath}
		
		
		This approach might help settle K\"ulshammer's question. So far only a non-reductive counterexample is known \cite{weil1964remarks}. 
		
		
		We are able to define and examine a slight variation on the question more inline with the kind of work of Liebeck and Seitz, which we may call the \emph{algebraic group} version of K\"ulshammer's question. Instead of a finite group $\Gamma$ we use a reductive group $H$, and instead of a Sylow $p$-subgroup $\Gamma_p \subset \Gamma$ we use a maximal unipotent subgroup $U \subset H$. Although a well defined and non-trivial question in its own right, finding an algebraic counterexample may also yield a finite counterexample for a finite subgroup of $H$, settling K\"ulshammer's original question. In our example calculations we pay special attention to $H = SL_2$.

		\item \emph{Main results}:
		

		???
	\end{itemize}

	\item[] \textbf{Context, history, literary review}
	\begin{itemize}
		\item \emph{K. II - motivation for this:}
		
		K\"ulshammer's question has it's roots Maschke's Theorem which states that any representation of a finite group over a field of characteristic not dividing the order of the group is completely reducible. 
		K\"ulshammer's first question, the assertion of which is essentially contained in paper by A. Weil published three decades earlier, reads:
		\begin{quote}
			Let $\Gamma$ be a finite group and let $G$ be a linear algebraic group over an algebraically closed field of characteristic prime to the order of $\Gamma$. Are there only finitely many representations $\Gamma\rightarrow G$ up to conjugation by $G$?
		\end{quote}
		Link
		
		\item \emph{Work of Liebeck \& Seitz, etc, on embedding reductive $H$ inside simple $G$:}
		
	\end{itemize}

	\item[] \textbf{Methods (can refer forward)}
	\begin{itemize}
		\item \emph{Use of 1-cohomology to (K. II):}
		\item \emph{Key results e.g. $H^1(SL_2, V)\rightarrow H^1(B, V)$:}
	\end{itemize}
	
	\item[] \textbf{Chapter Summary}
	\begin{itemize}
		\item \emph{Preliminaries:}
		\item \emph{1-Cohomology:}
		\item \emph{K. II:}
		\item \emph{Calculations:}
		\item \emph{Summary/Future work:}
	\end{itemize}
\end{itemize}

