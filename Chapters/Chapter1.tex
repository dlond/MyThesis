%!TEX root = /Users/dan/Documents/Thesis/Thesis.tex
% Chapter 1

\chapter{Introduction}
\label{Chapter1}
\lhead{Chapter 1. \emph{Introduction}}

\begin{itemize}
	\item[] \textbf{What is the thesis about and what are the main results}
	\begin{itemize}
		\item \emph{What the thesis is about:}
	
		This thesis is a contribution to the study of the subgroup structure of simple algebraic groups, complementing some of the work done by M. Liebeck and G. Seitz. The main motivation for the work carried out in this thesis is to investigate a question posed by B. K\"ulshammer \cite{weil1964remarks}:
		\begin{quote}
		Let $G$ be a linear algebraic group over an algebraically closed field of characteristic $p$. Let $\Gamma$ be a finite group and $\Gamma_p\subset\Gamma$ a Sylow $p$-subgroup. Fix a conjugacy class of representations $\bar{\rho}:\Gamma_p\rightarrow G$. Are there, up to conjugation by $G$, only finitely many representations $\rho:\Gamma\rightarrow G$ whose restrictions to $\Gamma_p$ belong to the given class?
		\end{quote}
		A counterexample is known for a non-reductive $G$, and we investigate the reductive case.

		\item \emph{Main results}:

		???
	\end{itemize}

	\item[] \textbf{Context, history, literary review}
	\begin{itemize}
		\item \emph{K. II - motivation for this:}
		\item \emph{Work of Liebeck \& Seitz, etc, on embedding reductive $H$ inside simple $G$:}
		
		Let $G$ be a simple algebraic group of exceptional type over an algebraically closed field of characteristic $p$. 
	\end{itemize}

	\item[] \textbf{Methods (can refer forward)}
	\begin{itemize}
		\item \emph{Key results e.g. $H^1(SL_2, V)\rightarrow H^1(B, V)$:}
		\item \emph{Use of 1-cohomology to (K. II):}
		\item \emph{Working in low characteristic:}
	\end{itemize}
	
	\item[] \textbf{Chapter Summary}
	\begin{itemize}
		\item \emph{Preliminaries:}
		\item \emph{1-Cohomology:}
		\item \emph{K. II:}
		\item \emph{Calculations:}
		\item \emph{Summary/Future work:}
	\end{itemize}
\end{itemize}

