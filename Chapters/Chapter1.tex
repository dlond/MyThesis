%!TEX root = /Users/dan/Documents/Thesis/Thesis.tex
% Chapter 1

\chapter{Introduction}
\label{Chapter1}
\lhead{Chapter 1. \emph{Introduction}}

A major motivation for the work carried out in this thesis is to investigate a question posed by B. K\"ulshammer to do with homomorphisms of finite groups into algebraic groups \cite{slodowy1997two}. We will call these homomorphisms \emph{representations} because of the obvious similarity with the usual kind of representations into $GL_n$. K\"ulshammer's second question is as follows:
\begin{quote}
  Let $G$ be a linear algebraic group over an algebraically closed field of characteristic $p$. Let $\Gamma$ be a finite group and $\Gamma_p < \Gamma$ a Sylow $p$-subgroup of $\Gamma$. Fix a conjugacy class of representations $\Gamma_p\rightarrow G$. Are there, up to conjugation by $G$, only finitely many representations $\rho:\Gamma\rightarrow G$ whose restrictions to $\Gamma_p$ belong to the given class?
\end{quote}

So far only a non-reductive counterexample is known \cite[Appendix]{slodowy1997two}. We examine K\"ulshammer's second question for reductive $G$, and we also examine a slight variation on the question which we call the \emph{algebraic group} version of K\"ulshammer's question in which case we substitute a connected reductive group $H$ for the finite group $\Gamma$, and instead of a Sylow $p$-subgroup $\Gamma_p < \Gamma$ we use a maximal unipotent subgroup $U < H$:
\begin{quote}
  Let $G,H$ be connected reductive linear algebraic groups over an algebraically closed field of characteristic $p$ and $U < H$ a maximal unipotent subgroup of $H$. Fix a conjugacy class of algebraic group homomorphisms $U\rightarrow G$. Are there, up to conjugation by $G$, only finitely many algebraic group homomorphisms$\rho:H\rightarrow G$ whose restrictions to $U$ belong to the given class?
\end{quote}

We often overload the term \emph{representation} to also mean algebraic group homomorphism in this setting.

\section{K\"ulshammer's Questions and Maschke's Theorem}

The algebraic group version of K\"ulshammer's question a non-trivial pursuit in its own right as K\"ulshammer's question has it's roots Maschke's Theorem of representation theory. Maschke's Theorem asserts that any representation from a finte group $\Gamma \rightarrow GL_n$ over a field of characteristic not dividing the order of $\Gamma$ is completely reducible, and that there are only finitely many conjugacy classes of (completely reducible) representations $\Gamma \rightarrow GL_n$ [ref Lang].

Let $\Gamma$ be a finite group and let $G$ be a linear algebraic group over an algebraically closed field of characteristic $p$. K\"ulshammer's first question reads:
\begin{quote}
  Suppose $p$ does not divide the order of $\Gamma$. Are there only finitely many conjugacy classes of representations $\Gamma\rightarrow G$?
\end{quote}
The answer is positive and is essentially contained in a paper of A. Weil \cite{weil1964remarks}. K\"ulshammer's second question is a refinement of the first:
\begin{quote}
  Let $\Gamma_p < \Gamma$ be a Sylow p-subgroup and fix a conjugacy class of representations $\Gamma_p\rightarrow G$. Are there only finitely many conjugacy classes of representations $\Gamma\rightarrow G$ whose restrictions to $\Gamma_p$ belong to the fixed class?
\end{quote}
Note that the condition that $p$ does not divide $|\Gamma|$ is dropped from the hypothesis. If $p$ does not divide the order of $\Gamma$ then the answer is ``yes'', since $\Gamma_p$ is trivial and so all representations are equal when restricted to $\Gamma_p$.

If $\Gamma$ is a $p$-group then the answer is ``yes'', as $\Gamma_p = \Gamma$ so restricting to $\Gamma_p$ does nothing and therefore only representations that come from the fixed class will hit the class.

If $G=GL_n$ and $p$ does not divide $|\Gamma|$ the answer is also ``yes'', since by Maschke's Theorem there can only be finitely many conjugacy classes of representations $\Gamma\rightarrow GL_n$ anyway, regardless of whether or not their restrictions to $\Gamma_p$ hit the fixed class. If $p$ does divide $|\Gamma|$ the answer has again shown to be positive \cite[Theorem]{slodowy1997two}.

The following example shows infinitely many conjugacy classes of representations of a finite group into $SL_2(k)$.
\begin{example}
  Let $\Gamma = C_p \times C_p = \langle a, b \,|\, ab = ba, a^p = b^p = 1 \rangle$ and consider representations $\rho: \Gamma \rightarrow SL_2(k)$. In particular, for each $\lambda \in k$ define $\rho_\lambda: \Gamma \rightarrow SL_2(k)$ by
  \begin{eqnarray*}
    \rho_\lambda(a) = \left( \begin{matrix} 1 & 1 \\ 0 & 1 \end{matrix} \right) \\
    \rho_\lambda(b) = \left( \begin{matrix} 1 & \lambda \\ 0 & 1 \end{matrix} \right).
  \end{eqnarray*}
  It is straightforward to check that if $\lambda_1 \neq \lambda_2$ then $\rho_{\lambda_1}$ is not $SL_2(k)$-conjugate to $\rho_{\lambda_2}$. Hence there are infinitely many $SL_2(k)$-conjugacy classes of representations from $\Gamma \rightarrow SL_2(k)$.
\end{example}

\section{Connection with the Subgroup Structure of Algebraic Groups}

Our approach to K\"ulshammer's question also means that the work in this thesis contributes to the study of the subgroup structure of simple algebraic groups, complementing some of the work done by M. Liebeck and G. Seitz (\cite{liebeck1996reductive}, \cite{liebeck2004maximal}), and D. Stewart (\cite{stewart2010g}). Let $G$ be a simple algebraic group over an algebraically closed field of characteristic $p$. For large enough characteristic ($p=0$ or $p>7$ covers all restrictions) Liebeck and Seitz determine explicitly the embeddings of arbitrary connected semisimple groups in $G$, where $G$ is of exceptional type. We examine the subgroup structure of simple algebraic groups in low characteristic (usually $p=2$ or $p=3$) where examples are exotic less is known. We use similar methods to Liebeck and Seitz, calculating a certain 1-cohomology of $H$ with coefficients in $V$, the unipotent radical of a parabolic subgroup $P < G$.

The main difference in our calculations is that we deal with the so-called \emph{non-abelian} 1-cohomology directly where as Liebeck, Seitz and Stewart use results from Representation Theory to study \emph{abelian layers} of the 1-cohomology and then piece the layers back together. Our calculations in Chapter \ref{Chapter5} agree with Stewart's $G_2$ calcualtion, and we acknowledge Stewart's $F_4$ calculation which provided us with a good example to work with in Chapter \ref{Chapter6}.

\section{Chapter Overview}

One of our main results is Theorem \ref{thm:k2_h1}. With this we are able to relate K\"ulshammer's question to a certain 1-cohomology calculation in which $\Gamma$ acts on the unipotent radical $V$ of a parabolic subgroup $P < G$ via a certain representation $\Gamma \rightarrow L$ into a Levi subgroup $L < P$. We show that we can reduce K\"ulshammer's question to another question: is the restriction map of 1-cohomologies
\begin{displaymath}
  H^1(\Gamma, V) \rightarrow H^1(\Gamma_p, V)
\end{displaymath}
injective for all parabolics $P<G$?

This approach might help settle K\"ulshammer's original question.

In Chapter \ref{Chapter2} we produce some basic facts to do with Linear Algebraic Groups which could be found in texts such as Humphreys \cite{humphryes1975linear} and will be well-known to readers with a background in this area. This is an attempt to standardize notation and provide some background for the results to come.

In Chapter \ref{Chapter3} we introduce the 1-cohomology, first the well-known abelian case and second the lesser-known non-abelian case. In Example \ref{eg:sl2ab} we show that the restriction map of 1-cohomologies $H^1(SL_2(k), V) \rightarrow H^1(U(k), V)$ is injective for $V$ a vector space, which is the kind of result we can apply Theorem \ref{thm:k2_h1} to, in answering K\"ulshammer's second question. We also show that $H^1(SL_2(k), V) \rightarrow H^1(B, V)$ is injective for $B$ a Borel subgroup of $SL_2$ and $V$ an algebraic group, not necessarliy abelian, on which $SL_2$ acts. The final step in applying an argument similar to Theorem \ref{thm:k2_h1} in an algebraic setting is that $H^1(B, V) \rightarrow H^1(U, V)$ is injective, where $U$ is the unipotent radical of $B$. We were unable to prove this but have evidence for it in our calculations in Chapter \ref{Chapter6}.

Chapter \ref{Chapter4} introduces the approach of finding reductive subgroups in reductive $G$ via the 1-cohomology and finshes with Theorem \ref{thm:k2_h1} which relates K\"ulshammer's second question to a question of restriction maps of 1-cohomologies as above.

In Chapter \ref{Chapter5} we collect together the results of various 1-cohomology calculations for $SL_2$ in $G$, $G$ of rank 2. They allude to a perhaps startling conjecture: although the types of 1-cohomology calculations $H^1(SL_2, V)$ involve a nonabelian $V$, there is evidence that a 1-cocycle in $Z^1(SL_2(k), V)$ that is zero on a maximal torus $T$ has image lying in an abelian subgroup of $V$. For instance, in Corollary \ref{cor:im_ab} we show this is the case for rank 1 parabolics of $G$ not containing $G_2$ or $C_3$. Furthermore Examples \ref{eg:g2}, \ref{eg:c3} verify the conjecture for $G = G_2, C_3$ respectively.

In Chapter \ref{Chapter6} we provide two examples which demonstrate the theoretical results captured in Chapter \ref{Chapter5}.

The future directions are summarised in the final Chapter.
