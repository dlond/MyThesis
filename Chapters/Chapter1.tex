%!TEX root = /Users/dan/Documents/Thesis/Thesis.tex
% Chapter 1

\chapter{Introduction}
\label{Chapter1}
\lhead{Chapter 1. \emph{Introduction}}

\begin{itemize}
	\item[] \textbf{What is the thesis about and what are the main results}
	\begin{itemize}
		\item \emph{What the thesis is about:}
		
		
		It's about algebraic groups. Talk a little bit about them.
		
		A major motivation for the work carried out in this thesis is to investigate a question posed by B. K\"ulshammer to do with homomorphisms of finite groups into algebraic groups \cite{weil1964remarks}. We will call these homomorphisms \emph{representations} because of the obvious similarity with the usual kind of representations into $GL_n$. K\"ulshammer's second question is as follows:
		\begin{quote}
		Let $G$ be a linear algebraic group over an algebraically closed field of characteristic $p$. Let $\Gamma$ be a finite group and $\Gamma_p\subset\Gamma$ a Sylow $p$-subgroup of $\Gamma$. Fix a conjugacy class of representations $\Gamma_p\rightarrow G$. Are there, up to conjugation by $G$, only finitely many representations $\rho:\Gamma\rightarrow G$ whose restrictions to $\Gamma_p$ belong to the given class?
		\end{quote}
		
		
		So far only a non-reductive counterexample\cite{weil1964remarks} is known. 
		
		
		We examine a slight variation on the question which we call the \emph{algebraic group} version of K\"ulshammer's question. Instead of a finite group $\Gamma$ we use a reductive group $H$, and instead of a Sylow $p$-subgroup $\Gamma_p \subset \Gamma$ we use a maximal unipotent subgroup $U \subset H$:
		\begin{quote}
			Let $G$ be a reductive linear algebraic group over an algebraically closed field of characteristic $p$, $H$ an algebraic group and $U \subset H$ a maximal unipotent subgroup of $H$. Fix a conjugacy class of representations $U\rightarrow G$. Are there, up to conjugation by $G$, only finitely many representations $\rho:H\rightarrow G$ whose restrictions to $U$ belong to the given class?
		\end{quote}
		Not only is the algebraic group version of K\"ulshammer's question a well defined and non-trivial question in its own right, but finding an algebraic counterexample may also yield a finite counterexample to K\"ulshammer's original question for a finite subgroup of $H$ and a reductive $G$. In our example calculations we pay special attention to $H = SL_2$.
		
		
		Our approach to K\"ulshammer's question also means that the work in this thesis contributes to the study of the subgroup structure of simple algebraic groups, complementing some of the work done by M. Liebeck and G. Seitz (\cite{liebeck1996reductive}, \cite{liebeck2004maximal}). Let $G$ be a simple algebraic group over an algebraically closed field of characteristic $p$. For large enough characteristic ($p=0$ or $p>7$ covers all restrictions) Liebeck and Seitz determine explicitly the embeddings of arbitrary closed connected semisimple subgroups in $G$ where $G$ is of exceptional type. We examine the subgroup structure of simple algebraic groups in low characteristic (usually $p=2$ or $p=3$) where less is known. We use similar methods to Liebeck and Seitz, calculating a certain 1-cohomology of $H$ with coefficients in $V$.
		
		

		\item \emph{Main results}:
		
		
		One of our main results is Lemma \ref{kToHOne}. With this we are able to relate K\"ulshammer's question to a certain 1-cohomology calculation in which $\Gamma$ acts on the unipotent radical $V$ of a parabolic subgroup $P \subset G$ via a certain representation $\Gamma \rightarrow L$ into a Levi subgroup $L \subset P$. We show that we can reduce K\"ulshammer's question to another question regarding a restriction map of 1-cohomologies:
		\begin{displaymath}
			H^1(\Gamma, V) \rightarrow H^1(\Gamma_p, V)
		\end{displaymath}
		
		
		This approach might help settle K\"ulshammer's original question.
		
		In Lemma \ref{mapFromSylow} we show that if $V$ is abelian then the restriction map of 1-cohomologies
		\begin{eqnarray*}
			H^1(\Gamma, V)\rightarrow H^1(\Gamma_p, V)
		\end{eqnarray*}
		is injective for finite $\Gamma$ and $\Gamma_p\subset\Gamma$ a Sylow $p$-subgroup. As a consequence, in Example \ref{sl2ab} we show that the restriction map
		\begin{eqnarray*}
			H^1(SL_2(k), V) \rightarrow H^1(U_2(k), V)
		\end{eqnarray*}
		is injective, where $k = \bar{\mathbb{F}_p}$ and $U_2(k)\subset SL_2(k)$ is the maximal unipotent subgroup of upper unitriangular matrices. These results together with Lemma \ref{kToHOne} suggest investigating non-abelian $V$ which calls for a definition of the non-abelian 1-cohomology.

		Next we show that if $H = SL_2$ and $G$ is a linear algebraic group then although $V$ may be non-abelian, 1-cocycles $H\rightarrow V$ that are trivial on a fixed maximal torus $T\subset H$ have images in an abelian subgroup $W\subset V$.
	\end{itemize}

	\item[] \textbf{Context, history, literary review}
	\begin{itemize}
		\item \emph{K. II - motivation for this:}
		
		K\"ulshammer's question has it's roots Maschke's Theorem which states that any representation of a finite group over a field of characteristic not dividing the order of the group is completely reducible. 
		K\"ulshammer's first question, the assertion of which is essentially contained in paper by A. Weil published three decades earlier, reads:
		\begin{quote}
			Let $\Gamma$ be a finite group and let $G$ be a linear algebraic group over an algebraically closed field of characteristic prime to the order of $\Gamma$. Are there only finitely many representations $\Gamma\rightarrow G$ up to conjugation by $G$?
		\end{quote}
		Link
		
		\item \emph{Work of Liebeck \& Seitz, etc, on embedding reductive $H$ inside simple $G$:}
		
	\end{itemize}

	\item[] \textbf{Methods (can refer forward)}
	\begin{itemize}
		\item \emph{Use of 1-cohomology to (K. II):}
		\item \emph{Key results e.g. $H^1(SL_2, V)\rightarrow H^1(B, V)$:}
	\end{itemize}
	
	\item[] \textbf{Chapter Summary}
	\begin{itemize}
		\item \emph{Preliminaries:}
		\item \emph{1-Cohomology:}
		\item \emph{K. II:}
		\item \emph{Calculations:}
		\item \emph{Summary/Future work:}
	\end{itemize}
\end{itemize}

