Let $R = \{ \rho_\lambda : \Gamma \rightarrow G \,|\, \lambda \in \Lambda \}$ be a collection of representations indexed by the set $\Lambda$. For $\rho \in R$, we say that a parabolic subgroup $P$ of $G$ is $\rho$-minimal if $P$ is minimal among the parabolic subgroups of $G$ that contain $\rho(\Gamma)$. For a parabolic subgroup $P < G$ define
\begin{align*}
  R_P = \{ \rho \in R \,|\, P \textrm{ is } \rho\textrm{-minimal} \}.
\end{align*}

Since there are only finitely many $G$-conjugacy classes of parabolic subgroups of $G$ (a standard result, e.g. \cite[Theorem 30.1(a)]{humphreys1975linear}), we can choose a finite set of representative parabolic subgroups $\{Q_i\}_{i = 0}^n$ of $G$ such that every parabolic subgroup $P < G$ is $G$-conjugate to precisely one $Q_i$. Every $\rho \in R$ has a minimal parabolic subgroup and so there exists an element of each $G$-conjugacy class in $R$ with minimal parabolic $Q_i$ for some $i$, hence
\begin{align}
  G \cdot R = \bigcup_i G \cdot R_{Q_i}.
  \label{eqn:gr_gqi}
\end{align}

Furthermore, fix a particular $Q_i$ with Levi subgroup $M_i$. Since $Q_i$ is minimal for each $\rho \in R_{Q_i}$, $\rho_{M_i}$ (Equation \ref{eqn:proj_l}) is $M_i$-irreducible \cite[Lemma 6.2(ii)]{bate2005geometric}.

For an $M_i$-irreducible representation $\sigma : \Gamma \rightarrow M_i$ define
\begin{align*}
  R_{\sigma} = \{ \rho \in R_{Q_i} \,|\, \rho_{M_i} = \sigma \}.
\end{align*}

Since there are only finitely many $M_i$-conjugacy classes of $M_i$-irreducible representations $\Gamma \rightarrow M_i$ (Theorem \ref{thm:finiteGCR}), we can choose a finite set of representative $M_i$-irreducible representations $\{\sigma_i^j : \Gamma \rightarrow M_i \}_{j=0}^{n_i}$, such that every $M_i$-irreducible representation $\Gamma \rightarrow M_i$ is $M_i$-conjugate to precisely one $\sigma_i^j$. Every $M_i$-conjugacy class in $R_{Q_i}$ has an element $\rho$ such that $\rho_{M_i} = \sigma_i^j$ for some $j$, hence
\begin{align*}
  M_i \cdot R_{Q_i} = \bigcup_j M_i \cdot R_{\sigma_i^j},
\end{align*}
and therefore
\begin{align}
  G \cdot R = \bigcup_i G \cdot R_{Q_i} = \bigcup_i \bigcup_j G \cdot R_{\sigma_i^j}.
  \label{eqn:gr_grsigma}
\end{align}

Fix an $M_i$-irreducible representation $\sigma: \Gamma \rightarrow M_i$. We have a map from $Hom(\Gamma, P_i)_\sigma \rightarrow Hom(\Gamma, P_i)_\sigma / V_i Z(M_i)^\circ$ given by the canonical projection and a map $\tilde{h}: Hom(\Gamma, P_i)_\sigma / V_i Z(M_i)^\circ \rightarrow H^1(\Gamma, \sigma, V_i)/Z(M_i)^\circ$. We define the map
\begin{align*}
  \mathcal{H}: Hom(\Gamma, P_i)_\sigma \rightarrow H^1(\Gamma, \sigma, V_i) / Z(M_i)^\circ
\end{align*}
to be the composition of the above canonical projection with $\tilde{h}$. That is, $\mathcal{H}(\rho) = \tilde{h}(\tilde{\rho})$ for all $\rho \in Hom(\Gamma, P_i)_\sigma$, where $\tilde{\rho}$ is the projection of $\rho$ to $Hom(\Gamma, P_i)_\sigma / V_i Z(M_i)^\circ$.

We note that each subset $R_{\sigma_i^j} \subset R_{Q_i}$ is a $VZ(M_i)^\circ$-stable subset of $Hom(\Gamma, Q_i)_{\sigma_i^j}$, so it makes sense to calculate
\begin{align*}
  \mathcal{H}(R_{\sigma_i^j}) \subset H^1(\Gamma, \sigma_i^j, V_i) / Z(M_i)^\circ.
\end{align*}

\begin{lemma}
  Let $R_P = \{\rho_\lambda:\Gamma\rightarrow P\,|\,\lambda\in\Lambda\}$ be a collection of representations such that $P$ $\rho_\lambda$-minimal for each $\rho_\lambda$.
  
  The following statements are equivalent:
  \begin{itemize}
    \item[(i)] $R_P$ is contained in a finite union of $P$-conjugacy classes.
    \item[(ii)] For each irreducible representation $\sigma:\Gamma\rightarrow L$, $R_{\sigma}$ is contained in a finite union of $VZ(L)^\circ$-conjugacy classes.
    \item[(iii)] For each irreducible representation $\sigma:\Gamma\rightarrow L$,
      \begin{align*}
	\mathcal{H}(R_{\sigma}) \subset H^{1}(\Gamma,\sigma,V)/Z(L)^\circ
      \end{align*}
      is finite.
  \end{itemize}
  \label{lem:p_h1}
\end{lemma}
\begin{proof}\quad

  $(i) \Rightarrow (ii)$ Assume $R_P$ is contained in a finite union of $P$-conjugacy classes and fix an irreducible representation $\sigma : \Gamma \rightarrow L$. Then $R_{\sigma}$ is contained a finite union of $P$-conjugacy classes. Take $\rho \in R_{\sigma}$ and suppose that $p \cdot \rho \in R_{\sigma}$ for some $p \in P$. Writing $p = vl$ for some $v \in V$ and some $l \in L$, $(vl) \cdot \rho \in R_{\sigma}$ implies that in fact $l \in C_L(\sigma(\Gamma))$. Furthermore, since $\sigma$ is irreducible it follows that $C_L(\sigma(\Gamma))/Z(L)^\circ$ is finite \cite[Lemma 6.2]{martin2003reductive}, so we can choose a finite set $\{c_1, \ldots, c_m\}$ of coset representatives for $Z(L)^\circ\backslash C_L(\sigma(\Gamma))$. Therefore
  \begin{align*}
    R_{\sigma} \cap (P \cdot \rho) \subset \bigcup_{i = 1}^{m} VZ(L)^\circ \cdot \left( c_i \cdot \rho \right).
  \end{align*}
  Since $R_{\sigma}$ is contained in a finite number of $P$-conjugacy classes, we are done.

  $(ii) \Rightarrow (i)$ Assume that for each irreducible representation $\sigma : \Gamma \rightarrow L$, $R_{\sigma}$ is contained in a finite union of $VZ(L)^\circ$-conjugacy classes, so for each $\sigma$ there is a finite set $\Phi^\sigma \subset R_P$ such that
  \begin{align*}
    R_\sigma \subset \bigcup_{\phi \in \Phi^\sigma} VZ(L)^\circ \cdot \phi.
  \end{align*}
  We do no harm to assume that $R_P = P \cdot R_P$. Denote by $Hom(\Gamma, L)_{irr}$ the collection of all irreducible representations from $\Gamma \rightarrow L$. By Theorem \ref{thm:finiteGCR} we can choose a finite set $\Sigma \subset Hom(\Gamma, L)_{irr}$ such that
  \begin{align*}
    Hom(\Gamma, L)_{irr} = \bigcup_{\sigma \in \Sigma} L \cdot \sigma.
  \end{align*}
  For each $\sigma \in \Sigma$ define 
  \begin{align*}
    R_{L \cdot \sigma} = \bigcup_{l \in L} R_{l \cdot \sigma}.
  \end{align*}
  Since $P$ is minimal for each $\rho \in R_P$, $\rho_L$ is $L$-irreducible. Hence
  \begin{align*}
    R_P = \bigcup_{\sigma \in \Sigma} R_{L \cdot \sigma}.
  \end{align*}

  Suppose $\rho \in R_{L \cdot \sigma}$. Then there exists $l \in L$ such that $\rho_L = l \cdot \sigma$, so that $l^{-1} \cdot \rho_L = \sigma$. Since we assumed $R_P = P \cdot R_P$, $l^{-1} \cdot \rho \in R_P$ and therefore $l^{-1} \cdot \rho \in R_\sigma$. Conversely if $\rho \in L \cdot R_\sigma$ then $l \cdot \rho \in R_{L \cdot \sigma}$ for some $l \in L$. Hence
  \begin{align*}
    R_{L \cdot \sigma} = L \cdot R_\sigma.
  \end{align*}

  Therefore
  \begin{align*}
    R_P = \bigcup_{\sigma \in \Sigma} L \cdot R_\sigma \subset \bigcup_{\sigma \in \Sigma}\bigcup_{\phi \in \Phi^\sigma} LVZ(L)^\circ \cdot \phi = \bigcup_{\sigma \in \Sigma}\bigcup_{\phi \in \Phi^\sigma} P \cdot \phi.
  \end{align*}

  $(ii) \Leftrightarrow (iii)$ This follows directly from the fact that $\tilde{h}$ is a bijection (Lemma \ref{lem:vzl_h1zl}).

\end{proof}

We are now ready to state precisely the connection with $G$-conjugacy classes of representations and the 1-cohomology.

\begin{theorem}
  Let $R=\{\rho_\lambda:\Gamma\rightarrow G\,|\,\lambda \in \Lambda\}$ be a collection of representations indexed by the set $\Lambda$. 
  
  Suppose $R = G \cdot R$. Then $R$ is a finite union of $G$-conjugacy classes if and only if for each $i, j$ the subset $\mathcal{H}(R_{\sigma_i^j}) \subset H^1(\Gamma, \sigma_i^j, V_i) / Z(M_i)^\circ$ is finite.
  \label{thm:g_h1}
\end{theorem}
\begin{proof}
  Assume $R$ is a finite union of $G$-conjugacy classes. Then for each $Q_i$, $R_{Q_i}$ is contained in a finite union of $G$-conjugacy classes. By Lemma \ref{lem:GPconj} $R_{Q_i}$ is contained in a finite union of $Q_i$-conjugacy classes, and by Lemma \ref{lem:p_h1} $\tilde{h}(R_{\sigma_i^j}/VZ(M_i)^\circ)$ is finite for each $j$.

  Conversely, suppose that for each $i$, for each $j$, $\tilde{h}(R_{\sigma_i^j}/VZ(M_i)^\circ)$ is finite. Then by Lemma \ref{lem:p_h1}, each $R_{Q_i}$ is contained in a finite union of $Q_i$-conjugacy classes. Since
  \begin{align*}
    G \cdot R = \bigcup_i G \cdot R_{Q_i} \qquad (\textrm{Equation }\ref{eqn:gr_gqi})
  \end{align*}
  we are done.
\end{proof}

\begin{theorem}
  Let $G$ be an algebraic group over an algebraically closed field $k$ of characteristic $p$, $\Gamma$ a finite group and $\Gamma_p < \Gamma$ a Sylow $p$-subgroup. Define $M_i < Q_i < G$ and $\sigma_i^{j}:\Gamma \rightarrow_{irr} M_{i}$ as above. Let $\iota$ be the inclusion map $\iota : \Gamma_p \rightarrow \Gamma$ and for each $i,j$ let $Z^1(\iota), H^1(\iota)$ be the corresponding 1-cocycle and 1-cohomology restriction maps
  \begin{align*}
    Z^1(\iota) : Z^1(\Gamma, \sigma_i^j, V_i) \rightarrow Z^1(\Gamma_p, \sigma_i^j, V_i),
  \end{align*}
  and
  \begin{align*}
    H^1(\iota) : H^1(\Gamma, \sigma_i^j, V_i) \rightarrow H^1(\Gamma_p, \sigma_i^j, V_i).
  \end{align*}
  
  The answer to K\"ulshammer's second question for $\Gamma, G$ is positive only if $H^1(\iota)$ is injective for each $i,j$.
  \label{thm:k2_h1}
\end{theorem}
\begin{proof}
  Fix a representation $\rho_0: \Gamma_p \rightarrow G$ and let $X = G \cdot \rho_0$. For a map $\varphi$ from $\Gamma$ denote by $\varphi^p$ its restriction to $\Gamma_p$. Let
  \begin{align*}
    R &=  \{ \rho \in \textrm{Hom}(\Gamma, G) \,|\, \rho^p \in X\}, \\
    R_{Q_i} &=  \{ \rho \in R \,|\, Q_i \textrm{ is } \rho \textrm{-minimal} \}, \\
    R_{\sigma_i^j} &=  \{ \rho \in R_{Q_i} \,|\, \rho_{M_i} = \sigma_i^j \},
  \end{align*}
  as done previously.
  
  Fix $g \in G$ and consider the set $R_j^p$ defined
  \begin{align*}
    R_j^p = \{ \rho \in R_{\sigma_i^j} \,|\, \rho^p = g \cdot \rho_0 \}.
  \end{align*}

  Then
  \begin{align}
    R_{\sigma_i^j} \subset G \cdot R_j^p.
    \label{eqn:rs_grp}
  \end{align}

  Assume $H^1(\iota)$ is injective. Since $Z^1(\iota)(h(\rho))$ is equal to a fixed 1-cocycle for all $\rho \in R_j^p$ the image of $R_j^p$ in $H^1(\Gamma_p, \sigma_i^j, V_i)$ is finite, hence the image of $R_j^p$ in $H^1(\Gamma, \sigma_i^j, V_i)$ is finite. 
  
  Therefore, the image of $R_j^p$ in $H^1(\Gamma, \sigma_i^j, V_i)/Z(M_i)^\circ$ is finite. By Lemma \ref{lem:p_h1}, $R_j^p$ is contained in a finite union of $Q_i$-conjugacy classes, hence by Equation \ref{eqn:rs_grp}, $R_{\sigma_i^j}$ is contained in a finite union of $G$-conjugacy classes.

  So by Equation \ref{eqn:gr_grsigma}, $G \cdot R = R$ is contained in a finite union of $G$-conjugacy classes. Therefore the answer to K\"ulshammer's second question for $\Gamma, G$ is positive.
\end{proof}


